%%%This is a science homework template. Modify the preamble to suit your needs. 

\documentclass[12pt]{article}

\makeatother
%AMS-TeX packages
\usepackage{amsmath}
\usepackage{amssymb}
\usepackage{amsthm}
\usepackage{array}
\usepackage{amsfonts}
\usepackage[all,cmtip]{xy}%Commutative Diagrams
\usepackage[pdftex]{graphicx}
\usepackage{float}
%geometry (sets margin) and other useful packages
\usepackage[margin=1in]{geometry}
\usepackage{sidecap}
\usepackage{wrapfig}
\usepackage{verbatim}
\usepackage{mathrsfs}
\usepackage{marvosym}
\usepackage{hyperref}
\usepackage{graphicx,ctable,booktabs}

\newtheoremstyle{norm}
{3pt}
{3pt}
{}
{}
{\bf}
{:}
{.5em}
{}

\theoremstyle{norm}
\newtheorem{thm}{Theorem}[section]
\newtheorem{lem}[thm]{Lemma}
\newtheorem{df}{Definition}
\newtheorem{rem}{Remark}
\newtheorem{st}{Step}
\newtheorem{pr}[thm]{Proposition}
\newtheorem{cor}[thm]{Corollary}
\newtheorem{clm}[thm]{Claim}

%Math blackboard, fraktur, and script commonly used letters
\newcommand{\A}[0]{\mathbb{A}}
\newcommand{\C}[0]{\mathbb{C}}
\newcommand{\sC}[0]{\mathcal{C}}
\newcommand{\E}[0]{\mathbb{E}}
\newcommand{\cE}[0]{\mathscr{E}}
\newcommand{\F}[0]{\mathbb{F}}
\newcommand{\cF}[0]{\mathscr{F}}
\newcommand{\cG}[0]{\mathscr{G}}
\newcommand{\sH}[0]{\mathscr H}
\newcommand{\Hq}[0]{\mathbb{H}}
\newcommand{\cI}[0]{\mathscr{I}}%ideal sheaf
\newcommand{\N}[0]{\mathbb{N}}
\newcommand{\Pj}[0]{\mathbb{P}}
\newcommand{\sO}[0]{\mathcal{O}}
\newcommand{\cO}[0]{\mathscr{O}}
\newcommand{\Q}[0]{\mathbb{Q}}
\newcommand{\R}[0]{\mathbb{R}}
\newcommand{\Z}[0]{\mathbb{Z}}
%Lowercase
\newcommand{\ma}[0]{\mathfrak{a}}
\newcommand{\mb}[0]{\mathfrak{b}}
\newcommand{\fg}[0]{\mathfrak{g}}
\newcommand{\vi}[0]{\mathbf{i}}
\newcommand{\vj}[0]{\mathbf{j}}
\newcommand{\vk}[0]{\mathbf{k}}
\newcommand{\mm}[0]{\mathfrak{m}}
\newcommand{\mfp}[0]{\mathfrak{p}}
\newcommand{\mq}[0]{\mathfrak{q}}
\newcommand{\mr}[0]{\mathfrak{r}}
%Letter-related
%\newcommand{\cal}[1]{\mathcal{#1}}
\providecommand{\cal}[1]{\mathcal{#1}}
\renewcommand{\cal}[1]{\mathcal{#1}}
\newcommand{\bb}[1]{\mathbb{#1}}
%More sequences of letters
\newcommand{\chom}[0]{\mathscr{H}om}
\newcommand{\fq}[0]{\mathbb{F}_q}
\newcommand{\fqt}[0]{\mathbb{F}_q^{\times}}
\newcommand{\sll}[0]{\mathfrak{sl}}
%Shortcuts for symbols
\newcommand{\nin}[0]{\not\in}
\newcommand{\opl}[0]{\oplus}
\newcommand{\ot}[0]{\otimes}
\newcommand{\rc}[1]{\frac{1}{#1}}
\newcommand{\rra}[0]{\rightrightarrows}
\newcommand{\send}[0]{\mapsto}
\newcommand{\sub}[0]{\subset}
\newcommand{\subeq}[0]{\subseteq}
\newcommand{\supeq}[0]{\supseteq}
\newcommand{\nsubeq}[0]{\not\subseteq}
\newcommand{\nsupeq}[0]{\not\supseteq}
%Shortcuts for greek letters
\newcommand{\al}[0]{\alpha}
\newcommand{\be}[0]{\beta}
\newcommand{\ga}[0]{\gamma}
\newcommand{\Ga}[0]{\Gamma}
\newcommand{\de}[0]{\delta}
\newcommand{\De}[0]{\Delta}
\newcommand{\ep}[0]{\varepsilon}
\newcommand{\eph}[0]{\frac{\varepsilon}{2}}
\newcommand{\ept}[0]{\frac{\varepsilon}{3}}
\newcommand{\la}[0]{\lambda}
\newcommand{\La}[0]{\Lambda}
\newcommand{\ph}[0]{\varphi}
\newcommand{\rh}[0]{\rho}
\newcommand{\te}[0]{\theta}
\newcommand{\om}[0]{\omega}
\newcommand{\Om}[0]{\Omega}
\newcommand{\si}[0]{\sigma}
%Brackets
\newcommand{\ab}[1]{\left| {#1} \right|}
\newcommand{\ba}[1]{\left[ {#1} \right]}
\newcommand{\bc}[1]{\left\{ {#1} \right\}}
\newcommand{\pa}[1]{\left( {#1} \right)}
\newcommand{\an}[1]{\langle {#1}\rangle}
\newcommand{\fl}[1]{\left\lfloor {#1}\right\rfloor}
\newcommand{\ce}[1]{\left\lceil {#1}\right\rceil}
%Text
\newcommand{\btih}[1]{\text{ by the induction hypothesis{#1}}}
\newcommand{\bwoc}[0]{by way of contradiction}
\newcommand{\by}[1]{\text{by~(\ref{#1})}}
\newcommand{\ore}[0]{\text{ or }}
%Arrows
\newcommand{\hr}[0]{\hookrightarrow}
\newcommand{\xr}[1]{\xrightarrow{#1}}
%Formatting
\newcommand{\subprob}[1]{\noindent\textbf{#1}\\}
%Functions, etc.
\newcommand{\Ann}{\operatorname{Ann}}
\newcommand{\AP}{\operatorname{AP}}
\newcommand{\Ass}{\operatorname{Ass}}
\newcommand{\Aut}{\operatorname{Aut}}
\newcommand{\chr}{\operatorname{char}}
\newcommand{\cis}{\operatorname{cis}}
\newcommand{\Cl}{\operatorname{Cl}}
\newcommand{\Der}{\operatorname{Der}}
\newcommand{\End}{\operatorname{End}}
\newcommand{\Ext}{\operatorname{Ext}}
\newcommand{\Frac}{\operatorname{Frac}}
\newcommand{\FS}{\operatorname{FS}}
\newcommand{\GL}{\operatorname{GL}}
\newcommand{\Hom}{\operatorname{Hom}}
\newcommand{\Ind}[0]{\text{Ind}}
\newcommand{\im}[0]{\text{im}}
\newcommand{\nil}[0]{\operatorname{nil}}
\newcommand{\ord}[0]{\operatorname{ord}}
\newcommand{\Proj}{\operatorname{Proj}}
\newcommand{\Rad}{\operatorname{Rad}}
\newcommand{\rank}{\operatorname{rank}}
\newcommand{\Res}[0]{\text{Res}}
\newcommand{\sign}{\operatorname{sign}}
\newcommand{\SL}{\operatorname{SL}}
\newcommand{\Spec}{\operatorname{Spec}}
\newcommand{\Specf}[2]{\Spec\pa{\frac{k[{#1}]}{#2}}}
\newcommand{\spp}{\operatorname{sp}}
\newcommand{\spn}{\operatorname{span}}
\newcommand{\Supp}{\operatorname{Supp}}
\newcommand{\Tor}{\operatorname{Tor}}
\newcommand{\tr}[0]{\text{trace}}
\newcommand{\Var}{\operatorname{Var}}
%Commutative diagram shortcuts
\newcommand{\fiber}[3]{\xymatrix{#1\times_{#3} #2}\ar[r]\ar[d] #1\ar[d] \\ #2 \ar[r] & #3}
\newcommand{\commsq}[8]{\xymatrix{#1\ar[r]^{#6}\ar[d]^{#5} &#2\ar[d]^{#7} \\ #3 \ar[r]^{#8} & #4}}
%Makes a diagram like this
%1->2
%|    |
%3->4
%Arguments 5, 6, 7, 8 on arrows
%  6
%5  7
%  8
\newcommand{\pull}[9]{
#1\ar@/_/[ddr]_{#2} \ar@{.>}[rd]^{#3} \ar@/^/[rrd]^{#4} & &\\
& #5\ar[r]^{#6}\ar[d]^{#8} &#7\ar[d]^{#9} \\}
\newcommand{\back}[3]{& #1 \ar[r]^{#2} & #3}
%Syntax:\pull 123456789 \back ABC
%1=upper left-hand corner
%2,3,4=arrows from upper LH corner, going down, diagonal, right
%5,6,7=top row (6 on arrow)
%8,9=middle rows (on arrows)
%A,B,C=bottom row
%Other
%Other
\newcommand{\op}{^{\text{op}}}
\newcommand{\fp}[1]{^{\underline{#1}}}
\newcommand{\rp}[1]{^{\overline{#1}}}
\newcommand{\rd}[0]{_{\text{red}}}
\newcommand{\pre}[0]{^{\text{pre}}}
\newcommand{\pf}[2]{\pa{\frac{#1}{#2}}}
\newcommand{\pd}[2]{\frac{\partial #1}{\partial #2}}
\newcommand{\prc}[1]{\pa{\rc{#1}}}
\newcommand{\bs}[0]{\backslash}
\newcommand{\iy}[0]{\infty}
%Matrices
\newcommand{\coltwo}[2]{
\left[
\begin{matrix}
{#1}\\
{#2} 
\end{matrix}
\right]}
\newcommand{\matt}[4]{
\left[
\begin{matrix}
{#1}&{#2}\\
{#3}&{#4}
\end{matrix}
\right]}
\newcommand{\smatt}[4]{
\left[
\begin{smallmatrix}
{#1}&{#2}\\
{#3}&{#4}
\end{smallmatrix}
\right]}
\newcommand{\colthree}[3]{
\left[
\begin{matrix}
{#1}\\
{#2}\\
{#3}
\end{matrix}
\right]}
%
%Redefining sections as problems
%
\makeatletter
\newenvironment{problem}{\@startsection
       {section}
       {1}
       {-.2em}
       {-3.5ex plus -1ex minus -.2ex}
       {2.3ex plus .2ex}
       {\pagebreak[3]%forces pagebreak when space is small; use \eject for better results
       \large\bf\noindent{Problem }
       }
       }
       {%\vspace{1ex}\begin{center} \rule{0.3\linewidth}{.3pt}\end{center}}
       }
\makeatother


%
%Fancy-header package to modify header/page numbering 
%
\usepackage{fancyhdr}
\pagestyle{fancy}
%\addtolength{\headwidth}{\marginparsep} %these change header-rule width
%\addtolength{\headwidth}{\marginparwidth}
\lhead{Problem \thesection}
\chead{} 
\rhead{\thepage} 
\lfoot{\small\scshape 18.997 Probabilistic Method} 
\cfoot{} 
\rfoot{\footnotesize PS \# 5} % !! Remember to change the problem set number
\renewcommand{\headrulewidth}{.3pt} 
\renewcommand{\footrulewidth}{.3pt}
\setlength\voffset{-0.25in}
\setlength\textheight{648pt}


%%%%%%%%%%%%%%%%%%%%%%%%%%%%%%%%%%%%%%%%%%%%%%%
%
%Contents of problem set
%    
\begin{document}
\title{18.997 Probabilistic Method Problem Set \#5}% !! Remember to change the problem set number
\author{Holden Lee}
\date{4/18/11}% !! Remember to change the date
\maketitle
\thispagestyle{empty}

%Example problems
\begin{problem}{\it(6.1)}
Note $Q$ equals the probability that in a random subgraph $H$ of $G$ obtained by picking each edge of $G$ with probability $\rc2$, that both $H$ and $G\bs H$ are connected, where $G\bs H$ consists of the vertices of $G$ and the edges of $G$ not in $H$. (Just associate one color with ``being in $H$" and the  other color with ``not being in $H$.")

Note ``$H$ connected" is a monotone increasing graph property and ``$G\bs H$ connected" is a monotone decreasing graph property (with respect to $H$), because deleting an edge in $H$ corresponds to adding an edge in $G\bs H$. Thus by Theorem 6.3.2 applied to the set of edges of $G$, we get that
\begin{align*}
P(H\text{ connected  and }G\bs H\text{ connected})
&\le P(H\text{ connected})P(G\bs H\text{ connected})\\
&=P(H\text{ connected})^2.
\end{align*}
The last follows from the fact that the distribution for $H$ and $G\bs H$ is the same, since $H$ is equally likely to be any subgraph $H_0$ of $G$, and in particular, the probability that $H=H_0$ is the same as the probability that $H=G\bs H_0$.

Thus $Q\le P^2$.
\end{problem}
\begin{problem}{\it (6.3)}
Note for any vertex $v$, ``$v$ has degree at most $k-1$" is a monotone decreasing graph property. 

Label the vertices $v_1,\ldots, v_{2k}$. By repeated application of 6.3.3, (basically an induction; in the induction step noting that if $P_1$ and $P_2$ are monotone decreasing then so it $P_1\wedge P_2$),
\[
P(v_1,\ldots,v_{2k}\text{ all have degree }\le k-1)\ge
\prod_{i=1}^{2k} P(v_1\text{ has degree }\le k-1)=\pf12^{2k}=\rc{4^k}.
\]
The equality come from the fact that there are $2k-1$ edges coming out of $v$, each chosen independently with probability $\rc 2$, so the degree of $v$ gives a binomial distribution symmetric around $\frac{2k-1}{2}$. In particular, it is as likely to have degree at most $k-1$ as degree at least $k$, i.e. both probabilities are $\rc2$.
\end{problem}
\begin{problem} {\it (7.2)}
\begin{lem}
\[\E[\chi(H)]\ge 500.\]
\end{lem}
\begin{proof}
We show that given $U\subeq V$, $\chi(G[U])+\chi(G[U^c])\ge 1000$. Indeed, take proper colorings of $G[U]$ and $G[U^c]$ with $\chi(G[U])$ and $\chi(G[U^c])$ different colors, such that the colors used in $U$ are different from any color used in $U^c$. This gives a proper coloring of $G$, since the only edges in $G$ neither in $G[U]$ nor $G[U^c]$ are those between $U$ and $U^c$, which will be between different colors. Since $\chi(G)=1000$, there must be at least 1000 colors used.

Now summing $\chi(G[U])+\chi(G[U^c])\ge 1000$ over all $2^{|V|}$ subsets $U\subeq V$ and dividing by $2^{|V|+1}$ gives $\E[\chi(H)]\ge 500$.
\end{proof}

Color $G$ properly with 1000 colors, and let $A_1,\ldots, A_{1000}$ be the color classes. Note that each $A_i$ is an independent set. 
%Consider the martingale with
%\[
%X_j(U)=\E\ba{\chi(G[W])\left|\pa{\bigcup_{i=1}^j A_i}\cap W=\pa{\bigcup_{i=1}^j A_i}\cap U\right.},
%\]
%i.e. $X_j(U)$ is the expected value of the chromatic number of $G[W]$, where $W$ is a random subset of $V$ that matches $U$ on $A_1,\ldots, A_j$. Note $X_0(U)=E[\chi(H)]$, while $X_{1000}(U)=\chi(H)$. %It is a martingale since $X_j$ is the conditional expectation of $\chi(H)$ given information that includes the information at previous steps $0,1,\ldots, j-1$.
%Further note that $|X_{i+1}-X_i|\le 1$. Indeed, if 

%Let $U\cap \bigcup_{i=1}^j A_i=U'$. Then
%\begin{align*}
%X_j(U)=X_j(U')&=\rc{2^{|A_{j+1}|}}\sum_{W_{j+1}\subeq A_{j+1}} \E\ba{\chi(G[W])\left|\pa{\bigcup_{i=1}^{j+1} A_i}\cap W=U'\cup W_{j+1}\right.}\\
%&=\rc{2^{|A_{j+1}|}}\sum_{W_{j+1}\subeq A_{j+1}}X_j(U'\cup W_{j+1})
%\end{align*}
%Given the value of $X_j$ and given $U'$, $X_{j+1}$ has equal probability of being any of the terms in the sum above; averaging over all $U'$ with $X_j,\ldots, X_1,X_0$ fixed, we get $\E(X_{j+1}|X_j,\ldots,X_1,X_0)=X_j$ and hence 
% that $X_0,X_1,\ldots, X_{1000}$ is a martingale. 
%We claim that $|X_{j+1}(U)-X_j(U)|\le 1$. 
%Indeed, letting $U\cap \bigcup_{i=1}^j A_i=U'$ and $U\cap A_{j+1}=V_{j+1}$, we have $X_j$ as above and hence
%\begin{align*}
%&X_j(U)-X_{j+1}(U)\\
%&=\rc{2^{|A_{j+1}|}}\sum_{W_{j+1}\subeq A_{j+1}} \E\ba{\chi(G[W])\left|\pa{\bigcup_{i=1}^{j+1} A_i}\cap W=U'\cup W_{j+1}\right.}\\
%&\quad - \E\ba{\chi(G[W])\left|\pa{\bigcup_{i=1}^{j+1} A_i}\cap W=U'\cup U_{j+1}\right.}\\
%&=\rc{2^{|A_{j+1}|+\cdots +|A_{1000}|}}
%\sum_{\scriptsize\begin{array}{c} W_{j+1}\subeq A_{j+1}\\W''\subeq A_{j+2}\cup \cdots \cup A_{1000}\end{array}}
%\chi\pa{
%G\ba{
%U'\cup W_{j+1}\cup W''
%}
%}-\chi\pa{G\ba{U'\cup U_{j+1}\cup W''}}.
%\end{align*}
%But $\chi\pa{
%G\ba{
%U'\cup W_{j+1}\cup W''
%}
%}-\chi\pa{G\ba{U'\cup U_{j+1}\cup W''}}\le 1$ for each $W_{j+1},W''$ since at worst $U'\cup W_{j+1}\cup W''$ can be colored the same as $G\ba{U'\cup U_{j+1}\cup W''}$, except with each vertex of $W_{j+1}$ given a different color, and similarly for the opposite inequality. This shows that $|X_{j+1}-X_j|\le 1$.
Let $B_j=\bigcup_{i=1}^j A_j$; consider the gradation $B_0\sub B_1\sub\cdots\sub B_{1000}=V$. Let $L:\cal P(V)\to \Z$ be the function $L(W)=\chi(G[W])$. Let 
\[
X_j(U)=\E[L(W)|B_j\cap W=B_j\cap U].
\]
In other words, $X_j(U)$ is the expected value of the chromatic number of $G[W]$, where $W$ is a random subset of $V$ that matches $U$ on $A_1,\ldots, A_j$. Note $X_0(U)=\E[\chi(H)]$, while $X_{1000}(U)=\chi(H)$. (Here $H=G[U]$.)

We show $L$ satisfies the Lipschitz condition. Suppose $W,W'$ differ only on $B_{j+1}-B_j=A_{j+1}$. Color $G[W]$ as follows: color the vertices in $W\cap A_{j+1}^c$ the same as in $W'\cap A_{j+1}^c$, and then color the vertices in $W\cap A_{j+1}$ another color (which is okay since $A_{j+1}$ is an independent set). Then we have a proper coloring of $G[W]$ with at most $\chi(G[W'])+1$ colors. So $L(W)\le L(W')+1$. The other inequality similarly holds, so $|L(W)-L(W')|\le 1$.

Now apply Theorem 7.4.1 to $X_i$ (but stated in terms of subsets, rather than functions), to conclude $|X_{i+1}-X_i|\le 1$ for $0\le i<1000$. 

Let $\mu=\E[\chi(H)]$; by the lemma $\mu\ge 500$.
By Azuma's inequality with $m=1000$ and $\la=\sqrt{10}$,
\begin{align*}
P[\chi(H))\le 400]&\le P[\chi(H)-\mu\le -100]\\
&=P[X_{1000}-X_0<-\sqrt{10}\sqrt{1000}]\\
&=e^{-\frac{\sqrt{10}^2}{2}}\\
&=e^{-5}<\rc{100}.
\end{align*}
\end{problem}
\begin{problem} {\it (7.3)}
Let $\ep=\rc{300}$. Let $u=u(n,\ep)$ be the least integer so that $P(\chi(G)\le u)>\ep$. Define $Z(G)$ to be the maximal size of a set of vertices for which the induced graph can be $u$-colored, and let $Y=n-Z$. Note $Z$ (and hence $Y$) satisfies the vertex Lipschitz condition since if we add a vertex, $Z$ either stays the same or increases by 1. Let $\mu=\E[Y]$ and use Azuma's inequality on the vertex exposure martingale to get
\begin{align*}
P(Y\le \mu-\la\sqrt{n-1})&<e^{-\frac{\la^2}2}\\
P(Y\ge \mu-\la\sqrt{n-1})&<e^{-\frac{\la^2}2}.
\end{align*}
Let $\la=\sqrt{-2\ln \ep}$ so this becomes
\begin{align*}
P(Y\le \mu-\la\sqrt{n-1})&<\ep\\
P(Y\ge \mu-\la\sqrt{n-1})&<\ep.
\end{align*}
Now $P(Y=0)=P(Z=n)=P(\chi(G)\le u)>\ep$ so the first inequality forces $\mu\le \la\sqrt{n-1}$. The second inequality then gives
\[
P(Y\ge 2\la\sqrt{n})\le P(Y\ge \mu+\la\sqrt{n-1})< \ep.
\]
In other words, there is probability at least $1-\ep$ that there is a $u$-coloring of all but at most $2\la\sqrt n$ vertices. Call these set of uncolored vertices $U$.

Since $\chi(G)\sim \frac{n}{2\log_2 n}$ almost surely, there exists $c$ so that $P\pa{\chi(G)\le \frac{cn}{\log n}}\ge 1-\ep$ for all $n>1$. Assuming $|U|\le 2\la \sqrt n$, applying this to $G[U]$ we get that
\begin{align*}
1-\ep&\le 
P\pa{\chi(G[U])\le \frac{c(2\la\sqrt n)}{\log(2\la\sqrt n)}}\\
&=P\pa{\chi(G[U])\le \frac{c2\la\sqrt n}{\log (2\la)+\rc 2\log{ n}}}\\
&\le P\pa{\chi(G[U])\le \frac{c' \sqrt n}{\log n}}
\end{align*}
for some appropriate constant $c'$.

Given $|U|\le 2\la \sqrt n$, with probability at least $1-\ep$, $G[U]$ can be colored with at most $\frac{c'\sqrt n}{\log n}$ further colors, giving a coloring of $G$ with at most $u+\frac{c'\sqrt n}{\log n}$ colors. By minimality of $u$, there is probability at least $1-\ep$ that $u$ colors are needed for $G$. Hence
\[
P\pa{u\le \chi(G)\le u+\frac{c'\sqrt n}{\log n}}\ge 1-3\ep=.99
\]
\end{problem}
\end{document}