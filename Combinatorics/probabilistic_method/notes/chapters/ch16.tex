\lecture{Tue. 4/5/11}

\subsection{Four-function theorem}
\begin{thm}[Ahlswede-Daykin four function theorem]
Suppose $n\ge 1$ and set $N=\{1,\ldots, n\}$. Let $P(N)$ be the power set of $N$. Let $\R_{\ge 0}$ be the set of nonnegative reals. For a function $\rh:P(N)\to \R_{\ge 0}$ and $\cal A\subeq P(N)$, let
\[
\rh(\cal A)=\sum_{A\in \cal A} \rh(A).
\]
For $\cal A,\cal B\subeq P(N)$, define
\begin{align*}
\cal A \cup \cal B&=\{A\cup B:A\in \cal A,B\in \cal B\}\\
\cal A \cup \cal B&=\{A\cup B:A\in \cal A,B\in \cal B\}.
\end{align*}

Let $\al,\be,\ge, \de:P(N)\to \R_{\ge0}$. If for every $A,B\subeq N$,
\begin{equation}\label{fft}
\al(A)\be(B)\le \ga(A\cup B)\de(A\cap B),
\end{equation}
then, for every $\cal A,\cal B\subeq P(N)$,
\begin{equation}\label{fft2}
\al(\cal A)\be(\cal B)\le \ga(\cal A\cup \cal B)\de(\cal A\cap \cal B).
\end{equation}
\end{thm}
\begin{proof}
We may modify $\al, \be, \ga, \de$ by defining $\al(A)=0$ for every $A\nin \cal A$ and $\be(B)=0$ for every $B\nin \cal B$, $\ga(C)=0$ for every $C\nin \cal A\cup \cal B$, and $\de(D)=0$ for every $D\nin \cal A\cap \cal B$. Note~(\ref{fft}) still holds. (If $A\in \cal A,B\in \cal B$, neither the LHS and RHS changes. Else the LHS is 0.) Thus we may assume that $\cal A=\cal B=\cal A\cup \cal B=\cal A\cap \cal B=P(N)$.

We induct on $n$. For $n=1$, $P(N)=\{\phi,N\}$. For each $\rh\in \{\al,\be,\ga,\de\}$ let $\rh_0=\rh(\ph)$ and $\rh_1=\rh(N)$.~(\ref{fft}) gives
\begin{align*}
\al_0\be_0&\le \ga_0\de_0\\
\al_0\be_1&\le \ga_1\de_0\\
\al_1\be_0&\le \ga_1\de_0\\
\al_1\be_1&\le \ga_1\de_1.
\end{align*}
(Note the RHS do not range over all $\ga_i\de_j$.)
We need $(\al_0+\al_1)(\be_0+\be_1)\le (\ga_0+\ga_1)(\de_0+\de_1)$. If $\ga_1=0$ or $\de_0$ we're good. Otherwise
\begin{align*}
\ga_0&\ge \frac{\al_0\be_0}{\de_0}\\
\de_1&\ge \frac{\al_1\be_1}{\ga_1}
\end{align*}
so it suffices to show $\pa{\frac{\al_0\be_0}{\de_0}+\ga_1}\pa{\de_0+\frac{\al_1\be_1}{\ga_1}}\ge (\al_0+\al_1)(\be_0+\be_1)$. This is true as rearranging gives
\[
(\ga_1\de_0-\al_0\be_1)(\ga_1\de_0-\al_1\be_0)\ge 0.
\]
Suppose the theorem holds for $n-1$ ($n\ge 2$); we prove it for $n$. Put $N'=N\bs \{n\}$ and define for each $\rh\in \{\al,\be,\ga,\de\}, A\in P(N')$
\[
\rh'(A)=\rh(A)+\rh(A\cup \{n\})\]
since this makes
\[
\rh'(P(N'))=\rh(P(N)).
\]
Note~(\ref{fft}) would follow from applying the induction hypothesis to $\al',\be',\ga',\de'$. To use the induction hypothesis we need to check~(\ref{fft}) for these new functions: for all $A',B'\subeq N'$,
\[
\al'(A')\be'(B')\le \ga'(A'\cup B')\de'(A'\cap B').
\]
Now
\begin{align*}
\ol{\al}(\phi)&=\al(A')&\ol{\al}(T)&=\al(A'\cup \{n\})\\
\ol{\be}(\phi)&=\be(B')& \ol{\be}(T)&=\be(B'\cup \{n\})\\
\ol{\ga}(\phi)&=\ga(A'\cup B')&\ol{\ga}(T)&=\ga(A'\cup B'\cup \{n\}\\
\ol{\de}(\phi)&=\de(A'\cap B')&\ol{\de}(T)&=\de(A'\cap B'\cup \{n\}.
\end{align*}
By~(\ref{fft})
\[
\ol{\al}(S)\ol{\be}(R)\le\ol{\ga}(S\cup R)\ol{\de}(S\cap R)
\]
for all $S,R\subeq T$. Hence using the $n=1$ case,
\begin{align*}
\al'(A')\be'(B')&=\ol{\al}(P(T))\ol{\be}(P(T))\\
&\le \ol{\ga}(P(T))\ol{\de}(P(T))\\
&=\ga'(A'\cup B')\de'(A'\cap B').
\end{align*}
\end{proof}
\begin{df}
A \textbf{lattice} is a poset in which every two elements $x,y$ have a unique minimal upper bound (join) $x\vee y$ and a unique maximal lower bound $c\wedge y$. A lattice is distributive if for all $x,y,z\in L$, 
\[
x\wedge(y\vee z)=(x\wedge y)\vee (x\wedge z).
\]
Equivalently, $x\vee (y\wedge z)=(x\vee y)\wedge (x\vee z)$. For $X,Y\subeq L$, define
\begin{align*}
X\vee Y&=\{x\vee y:x\in X,y\in Y\}\\
X\wedge Y&=\{x\vee y:x\in X,y\in Y\}
\end{align*}
\end{df}
Any $L\subeq P(N)$ where $N=\{1,\ldots, n\}$, where posets are ordered by inclusion, is a finite distributive lattice.%, the joins and meets of $A$ are $A\cup N$ and $A\cap B$. it is closed under intersection and union. It is a finite distributive lattice. 
Conversely, every finite distributive lattice is of this form.

\begin{cor}[Four function theorem for distributive lattices]
Let $L$ be a finite distributive lattice and $\al,\be,\ga, \de:L\to \R_{\ge0}$. Then the same theorem holds.
\end{cor}
The simplest case is $\al,\be,\ga,\de=1$. Then we get the following.
\begin{cor}\label{xyineq}
Let $L$ be a finite distributive lattice and $X,Y\subeq L$. Then
\[
|X||Y|\le |X\vee Y|\cdot |X\wedge Y|.
\]
\end{cor}
\begin{cor}
Let $\cal A\subeq P(N)$ and $\cal A\bs\cal A=\{A\bs A':A,A'\subeq \cal A\}$. Then
\[
|\cal A\bs \cal A|\ge |\cal A|.
\]
\end{cor}
\begin{proof}
Let $L$ be a finite distributive lattice on $P(N)$. By Corollary~\ref{xyineq} with $\cal B=\{N\bs F:F\in \cal A\}$,
\[
|\cal A|^2=|\cal A|\cdot |\cal B|\le |\cal A\cup \cal B|\cdot |\cal A\cap \cal B|=|A\bs\cal A|^2.
\]
\end{proof}
\subsection{FKG inequality}
\begin{df}
Let $L$ be a finite distributive lattice. A function $\mu:L\to \R_{\ge0}$ is called \textbf{log-supermodular} if $\mu(x)\mu(y)\le \mu(x\wedge y)\mu(x\vee y)$, \textbf{increasing} if $\mu(x)\le \mu(y)$ for all $x\le y$ and \textbf{decreasing} if $\mu(x)\ge \mu(y)$ for all $x\le y$.
\end{df}
\begin{thm}[FKG inequality]
Let $L$ be a finite distributive lattice and $\mu:L\to \R_{\ge 0}$ %mu acts like measure.
be a log-supermodular function. Let $f,g:L\to\R_{\ge 0}$ be increasing functions. Then
\[
\pa{\sum_{x\in L} \mu(x)f(x)}
\pa{\sum_{x\in L} \mu(x)g(x)}
\le
\pa{\sum_{x\in L} \mu(x)f(x)g(x)}
\pa{\sum_{x\in L} \mu(x)}.
\]
\end{thm}
\begin{proof}

Define $\al,\be,\ga,\de$ by
\begin{align*}
\al(x)&=\mu(x)f(x)\\
\be(x)&=\mu(x)g(x)\\
\ga(x)&=\mu(x)f(x)g(x)\\
\de(x)&=\mu(x).
\end{align*}
We claim these functions satisfy~(\ref{fft}); then the conclusion follows from the four function theorem. % and hence~(\ref{fft2}).
Indeed,
\begin{align*}
\al(x)\be(y)&=\mu(x)f(x)\mu(y)g(y)\\
&\le \mu(x\wedge y)f(x)\mu(x\vee y)g(y)\\
&\le \mu(x\wedge y) f(x\vee y)\mu(x\vee y)g(x\vee y)\\
&=\de(x\wedge y)\ga(x\vee y).
\end{align*}
\end{proof}
Same holds if both $f,g$ decreasing; just reverse $\ga,\de$. If one is increasing and the other is decreasing, inequality reverses.

