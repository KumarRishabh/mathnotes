\lecture{Thu. 10/18/12}

Last time we considered isometric immersions $M^m\hra N^n$. This means that the map is an immersion and the metric is just induced by inclusion. Let $X,Y,Z\in \X(M)$. We proved that if $\ol X$ and $\ol Y$ are extensions of $X$ and $Y$ to a tubular neighborhood of $p$, then 
\[
(\ol{\nb}_{\ol X}\ol Y)^T=\nb_X Y.
\]
Last time we also defined the second fundamental form $B(X,Y)$ by
\[
B(X,Y)=(\ol{\nb}_{\ol X} \ol Y)^{\perp}.
\]
We have that $B$ is symmetric bilinear form
\[
B:T_pM\times T_pM\to (T_pM)^{\perp}\subeq T_p N.
\]
Suppose $E$ is a vector field perpendicular to $M$. Then 
\begin{align}
\nonumber
\an{B(X,Y),E}(p)&=\an{(\ol{\nb}_{\ol X}\ol Y)^{\perp},E}\\
\nonumber
&=\an{\ol{\nb}_{\ol X}\ol Y,E}&%\ol{\nb}_{\ol X}\ol Y
E\text{ is perpendicular}\\
&=\underbrace{\ol X\an{\ol Y,E}}_0-\an{\ol Y,\ol{\nb}_{\ol X}E}\llabel{eq:965-12-1}\\
\nonumber
&=-\an{\ol Y,\ol{\nb}_{\ol X}E}
\end{align}
In~\eqref{eq:965-12-1} we noted that $\ol Y$ is tangent to $M$ and $E$ is normal to $M$, so the first term is 0.

Since $B$ is symmetric, if we switch $X,Y$ we get the same thing:
\[
\an{Y,\nb_XE}=\an{X,\nb_YE}.
\]

\subsection{Weingarten map}
Suppose that $M^{n-1}\subeq N^n$ is a hypersurface. Then up to sign, locally there is a unique normal vector field $n$ to $M$. 
\begin{df}
The \textbf{Weingarten map} is the map $W:T_pM\to T_pM$ defined by
\[
W(v)=\nb_v n.
\]
\end{df}
It is clear that $W$ is linear. We need to show that $W(T_pM)\subeq T_pM$ is actally in $T_pM$, or equivalently, $\an{W(v),n}=0$ for all $v\in T_pM$. Let $X$ extend $v$. We have 
\[
\an{W(v),n}=\an{\nb_X n,n}=\rc 2X\underbrace{\an{n,n}}_1=0
\]
Next we show that $W$ is symmetric (self-adjoint):
\[
\an{W(v),u}=\an{v,W(u)}.
\]
We calculate
\begin{align*}
\an{W(v),u}&= \an{\nb_v n, u}\\
&=-\an{B(v,u),n}\\
&=-\an{B(u,v),n}\\%B only depends on vectors at points
%extend to vf on nbhd of point. 
&=\an{\nb_un,v}.
\end{align*}
%\fixme{check.}
This implies by the spectral theorem that $W$ has a basis of eigenvectors.
\begin{df}
The eigenvectors of $W$ are called \textbf{principal directions} and the eigenvalues are called \textbf{principal curvatures}.
\end{df}
Note it is easy to generalize the theory to arbitrary isometric immersions $M\hra N$; see the book.

We now relate the curvature of $M$ to the curvature of $N$ using the Gauss equations.
\subsection{Gauss equations}
%products of principal curvatures.
Let $E_1$, $E_2$, and $E_3$ be vector fields tangent to $M$. Then\footnote{We implicitly extend the vector fields to $N$. We may as well just work with vector fields on $M$, though, because the definition doesn't depend on the extension.}
\begin{align*}
\ol R(E_1,E_2)E_3=\ol{\nb}_{E_2}\ol{\nb}_{E_1}E_3-\ol{\nb}_{E_1} \ol{\nb}_{E_2} E_3+\ol{\nb}_{[E_1,E_2]} E_3,\\
R(E_1,E_2)E_3={\nb}_{E_2}{\nb}_{E_1}E_3-{\nb}_{E_1} {\nb}_{E_2} E_3+{\nb}_{[E_1,E_2]} E_3.
\end{align*}
Because we are working with a hypersurface, we can write
\begin{align*}
\ol{\nb}_{E_1}E_3 &= \nb_{E_1}E_3 + \an{B(E_1,E_3),n}n,\\
\ol{\nb}_{E_2}E_3 &= \nb_{E_2}E_3 + \an{B(E_2,E_3),n}n.
\end{align*}
%trivial things: vector fields tangent to $M$. Can extend (make sense even if don't extend). 
Then we have using the above and the Leibniz rule,
\begin{align*}
\ol{\nb}_{E_2}\ol{\nb}_{E_1}E_3
&=\ol{\nb}_{E_2}{\nb}_{E_1} E_3+\ol{\nb}_{E_2}(\an{B(E_1,E_3),n}n)\\
&=\ol{\nb}_{E_2}\nb_{E_1}{E_3}+E_2 (\an{B(E_1,E_3),n})n + \an{B(E_1,E_3),n} \onb_{{E_2}}n\\
&=\nb_{E_2}\nb_{E_1} E_3 +\an{B({\nb}_{E_1}E_3,E_2),n}n + E_2 (\an{B(E_1,E_3),n})n +\an{B(E_1,E_3),n}\onb_{{E_2}} n.
\end{align*}
Then (noting the normal terms don't contribute to the inner product),
\begin{align}
\nonumber
\an{\ol{\nb}_{E_2}\ol{\nb}_{E_1} E_1,E_2}
&=\an{\nb_{E_2}\nb_{E_1} E_1,E_2}
+\an{B(E_1,E_1),n}\an{\onb_{\ol{E_2}} n,E_2}\\
&=\an{\nb_{E_2}\nb_{E_1}E_1,E_2} - \an{B(E_1,E_1),n} \an{B(E_2,E_2),n}
\llabel{eq:965-12-2}
\end{align}
%curvature of N in 2-plane, hope relate to curvature of M
We similarly have %calculations omitted
\begin{equation}\llabel{eq:965-12-3}
%\an{\ol{\nb}_{E_2}\ol{\nb}_{E_1} E_1,E_2}
%=\an{\nb_{E_1}\nb_{E_2}E_2,E_1} + \an{B(E_2,E_2),n} \an{B(E_1,E_1),n}
\an{\ol{\nb}_{E_1}\ol{\nb}_{E_2} E_1,E_2}=\an{\nb_{E_1}\nb_{E_2}E_1,E_2} -\an{B(E_2,E_1),n}^2.
\end{equation}
%Then %tang and orth. orth doesn't matter
Finally, 
\begin{equation}\llabel{eq:965-12-4}
\an{\ol{\nb}_{[E_1,E_2]} E_1,E_2}
=\an{\nb_{[E_1,E_2]}E_1,E_2}
\end{equation}
From~\eqref{eq:965-12-2},~\eqref{eq:965-12-3}, and~\eqref{eq:965-12-4} we get
\[
\an{\ol R(E_1,E_2)E_1,E_2}
=\an{R(E_1,E_2)E_1,E_2}
-\an{B(E_1,E_1),n}\an{B(E_2,E_2),n}
+\an{B(E_1,E_2),n}^2
\]
%surface in R^3 0 curv--bad!
Choosing $E_1,E_2$ to be orthonormal, we get the \textbf{Gauss equation}
\beq{eq:965-gauss-eq}
\ol K=K-\an{B(E_1,E_1),n}\an{B(E_2,E_2),n}+  \an{B(E_1,E_2),n}^2;
\eeq
the curvature with respect to $M$ and $N$ differ by a correction term. 
Using $\an{B(E_1,E_2),n}=-\an{\nb_{E_1} n, E_2}$, we can write the Gauss equations in terms of the Weingarten map:
\[
\ol K=K-\an{\nb_{E_1}n,E_1} \an{\nb_{E_2}n,E_2} +\an{\nb_{E_1}n,E_2}^2.
\]
We can write this in terms of the eigenvalues of the Weingarten map, the principal curvatures.
%Principal curvatures, directions. $\nb_n:T_pM\to T_pM$. Eigenvectors eigenvalues. 
Consider the case $M^2\sub N^3$. Let $p\in M$. We compute the sectional curvature of $T_pM$. Take an orthonomal basis of eigenvectors (principal directions) $E_1,E_2$; suppose $\nb_{E_1}n=\kappa_1E_1$ and $\nb_{E_2}n=\kappa_2E_2$. Then we get
%small: just one curv. Bog: look at...
\begin{align*}
\ol K&=K-\an{\kappa_1E_1,E_1}\an{\kappa_2E_2,E_2}+\an{\kappa_1 E_1,E_2}^2\\
\ol K&=K-\kappa_1\kappa_2.
\end{align*}

\begin{ex}
We calculate the curvature of the unit sphere $S^2\sub \R^3$. We know $K_{\R^3}\equiv 0$. The curvature of $S^2$ is $K_{\R^3}$ minus the product of the two principal cuvatures:
\[
0=K_{\R^3} =K_{S^2}+\kappa_1\kappa_2.
\]
The unit normal is simply $x$. We have 
\[\nb_v x=v;\]
all directions are principal directions (the Weingarten map is the identity) and the principal cuvatures are 1. Hence $\kappa_1=\kappa_2=1$ and we get
\[
K_{S^2}=-1.
\]
\end{ex}
\cpbox{Use the Gauss equation $\ol K=K-\kappa_1\kappa_2$ to calculate the curvature of a submanifold.}
\vskip0.15in
\begin{ex}
We calculate the curvature of a round cylinder $S^1\times \R$ in $\R^3$. Let $n$ be the normal. Note that along the height of the cylinder, $n$ is constant. 
Hence
\[
\nb_v n=0.
\]
The height direction is a principal direction with principal value 0. It doesn't matter what the other principal value is; the product is 0. We get
\[
K_{S^1\times \R}=0.
\]

In fact his works for any cylinder $\ga\times \R$ where $\ga$ is a closed curve in the plane.
\end{ex}