\def\filepath{C:/Users/Owner/Dropbox/Math/templates}

\input{\filepath/packages_article.tex}
\input{\filepath/theorems_with_boxes.tex}
\input{\filepath/macros.tex}
\input{\filepath/formatting.tex}
%\input{\filepath/other.tex}

%\def\name{NAME}

%\input{\filepath/titlepage.tex}

\pagestyle{fancy}
%\addtolength{\headwidth}{\marginparsep} %these change header-rule width
%\addtolength{\headwidth}{\marginparwidth}
\lhead{Gems of theoretical computer science}
\chead{} 
\rhead{} 
\lfoot{} 
\cfoot{\thepage} 
\rfoot{} 
\renewcommand{\headrulewidth}{.3pt} 
%\renewcommand{\footrulewidth}{.3pt}
\setlength\voffset{0in}
\setlength\textheight{648pt}

\begin{document}
\tableofcontents
\section{List decoding of Reed-Solomon codes}
\subsection{Introduction}
An error-correcting code is $C\subeq \Si^n$, where $\Si,|\Si|=q$ is the alphabet and $n$ is the encoding length. 

Define the normalized Hamming distance
\[
d(x,y)=\rc n|\set{i}{x_i\ne y_i}|.
\]
We want 
\[
\de = \min_{x,y\in C,x\ne y} d(x,y)
\]
to be as large as possible (constant as $n\to \iy$). Imagine balls of radius $\de/2$ around each point.
We can send one of $|C|$ messages by mapping $[|C|]\to \Si^n$; they can withstand $\de/2$ errors. The rate of the code is $\fc{\lg|C|}{n}$ (how many bits of actual information are sent compared with encoding length); we want the rate and distance to be high, but there are fundamental limits to what can be achieved.

$\de/2$ is the unique decoding radius. Unique decoding is not possible beyond $\de/2$. %; they don't intersect. 

In many cases, even if there is not a unique codeword within a given radius ($\fc{\de}{2}+\ep$, say), there may be a only a small number of codewords.
\begin{df}
$C$ is $(\de,L)$-list decodable if for all $x\in \Si^n$, $|B(x,\de)\cap C|\le L$.
\end{df}
%random code, with high probability
If you choose your code randomly---that is, choose a codeword, removing a $\fc{\de}2$ ball around it, and repeat---with high probability it will have good list decodability, up to distance $(1-\ep)\de$. %; in fact, most of the time it can correct close to $\de$. %The code is good becuase most of the time it can correct $\de$.
List decoding is a better way to handle talk about how good the code is than stochastically, because it's worst-case.
%up to $(1-\ep)\de$
%People previously tried to model stochastically.
%This code is good becuase most of the time it can correct $\de$.
%List decoding is a better way to handle this than stochastically because it's worst-case.

The maximum $\de$ is the list-decoding radius. Here $L(n)$ is a function of $n$. The ideal setting is $L(n)=\poly(n)$.  Informally the list decoding radius is the maximum $\de$ such that $L(n)=\poly(n)$.

What does a random code give us; what is possible and not possible? What is the capacity of list decoding?
%minimum distance

\begin{thm}
Let $0<\de<1-\rc{q}$. (Above this error, the noisy word is essentially random, so we can say nothing.) Then there exists a $(\de,L)$-list decodable code with rate
\[
1-H_q(\de)-\rc{L}.
\]
Here $H_q$ is defined so that
\[
|B(0^n,\de)|\sim q^{H_q(\de)n}.
\]
\end{thm}
%\Si^n$.
\begin{proof}
Choose $M=q^{Rn}$ codewords at random. We want
\[
\Pj(C\text{ not }(\de,L)\text{-list decodable})<1.
\]
Do a union bound. The probability is at most, by the union bound,
\[
\binom{M}{L+1}q^n\pf{|B(x,\de)|}{q^n}^{L+1}<1.
\]
($q^n$ is the number of points; the probability of $L+1$ points in a ball centered at a point is $\le \binom{M}{L+1}\pf{|B(x,\de)|}{q^n}^{L+1}$.)
\end{proof}
This is almost tight. 
\begin{thm}
If a code is $(\de,L)$-list decodable with 
\[
R\ge 1-H_q(\de)+\ep
\]
then $L(n)\ge q^{\fc{\ep n}2}$.
\end{thm}
%1-H_q(\de)+\ep$
A random ball will contain too many points.
\begin{proof}
\bal
\E_{x\in_R \Si^n} [|B(x,\de)\cap C|]
&=
|C|\fc{|B(0^n,\de)|}{q^n}\\
&\ge 
2^{Rn}q^{n(H_q(\de)-1)}
%q^{n(1-H_q(\de)+\ep)}%%/\ep^n
\\
&\ge q^{\ep n/2}.
\end{align*}
\end{proof}
When $q\to \iy$, $H_q(\de)\to \de$. We can achieve rate $R\ge 1-\de-\ep$ with alphabet of size $2^{O\prc{\ep}}$  and list size $O\prc{\ep}$. 
%1/L
%all add up to at most $\rc \ep$. Plug in param.
%alphabet grows, can achieve any rate up to.

Graph: the achievable rates are below the line $\de+R=1$.

This is what is known existentially.

The current known explicit codes can achieve $R\ge 1-\de-\ep$ with alphabet of size $2^{\poly\prc{\ep}}$ and list size $n^{O\prc{\ep}}$.
%
The idea is a folded Reed-Solomon code concatenated with an asymptotically good list-decodable code.
%any code - at least this much list decodable.

\begin{thm}[Johnson bound]
Given $C\subeq \Si^n$ with min-distance $\de=1-\ep$, then $C$ is $(1-\sqrt{\ep},\poly(n))$ list-decodable.
%increase by $\sqrt{\ep}-\ep$.
\end{thm}
The Johnson bound says that if you slightly increase the radius, there cannot be an exponential number of codewords.

\begin{thm}[Singleton bound]
For any codewith distance $\de$, $R\le 1-\de$.
\end{thm}
\vocab{MDS (maximal distance separable) codes} are where $R+\de=1$, $\de=1-R$, R$=\ep$. MDS codes are $(1-\sqrt{R},\poly(N))$ list-decodable.
%if don't care about constant, simple. if want constant, need more careful calculation.

\subsection{Reed-Solomon Codes}
We introduce the Reed-Solomon code $RS_{k,n,\F}$, $k< n\le q$.

These are MDS codes. They achieve $R+\de=1$ so are $(1-\sqrt{R}, \poly(n))$-list decodable.

Let $\Si=\F_q=:\F$. Let
\[
C=\{\text{evaluations of degree $k$ polynomials on }\{\al_1,\ldots, \al_n\}\subeq \F\}
\]
where $\al_1,\ldots, \al_n$ are distinct and fixed.
Codewords correspond to degree $k$ polynomials in $\F[x]$.
2 distinct degree $k$ polynomials can only agree on $k$ points, so $\de=\fc{n-k}n=1-\fc kn$. 
%2 degree $k$ polynomials, distinct.
The rate is $\fc{\log_q(q^{k+1})}{n}=\fc{k+1}n$. 
The Reed-Solomon code is $(1-\sfc{k}n, \poly(n))$-list decodable. 

What is the algorithm? Given a set of points, we need to find all polynomials passing through enough of those points.
% Given a word $x$, give a list of the polynomials.
\begin{prb}
Given $(\al_1,y_1),\ldots, (\al_n,y_n)$, find all polynomials of degree $\le k$ such that $p(\al_i)=y_i$ for at least $\sqrt{nk}$ indices $i\in[n]$. 
\end{prb}
Madha Sudan (90's) showed that you can do this with $\sqrt{2nk}$. 
\begin{thm}
There is a polynomial-time algorithm (given in the proof) that given $n$ points as above, finds all degree $\le k$ polynomials agreeing on $\ge 2\sqrt{nk}$ points. 
\end{thm}
\begin{proof}
The algorithm is as follows.
Note this algorithm will work even if the $\al_i$ are not distinct.

Define the $(1,k)$-weighted degree of a polynomial $Q(x,y)$ as $\deg Q(X,Y^k)$.
The strategy is as follows.

We will find a low $(1,k)$-weighted degree polynomial $Q(x,y)$ of degree $D$ such that $Q(\al_i,y_i)=0$ for all $i$.

Look at $R(x)=Q(x,p(x))$ where $p\in L$. We have
\[
\deg R\le D.
\]
For all $i$ such that $p(\al_i)=y_i$, 
\[
R(\al_i)=Q(\al_i,p(\al_i))=Q(\al_i,y_i)=0.
\]
Suppose $R$ has at least $t$ roots and $\deg R\le D$.  If we arrange so that $t>D$, then $R(x)\equiv 0$ identically. Then $Q(x,p(x))\equiv 0$ implies $y-p(x)\mid Q(x,y)$.
%$Q(x,y+p(x))\equiv 0$, take $y=0$.
%t<q.
%cant have more roots than degree.
(There is a deterministic algorithm to factor bivariate polynomials.) Factor $R$ to find all factors in the form $y-p(x)$; then output those polynomials $p(x)$.

Now we just need to find the polynomial $R(x)$; this is polynomial interpolation. 

The number of coefficients in $Q(x,y)$ is $\rc{k}\binom{D+2}2$. We need to satisfy $n$ linear constraints; they are linear homogeneous equations in the coefficients. If $\fc{\binom{D+2}2}k>n$ then there is a nonzero solution: a nonzero $\ph(x,y)$ of $(1,k)$ weight degree $\le D$ with $\ph(\al_i)=y_i$ for all $i$.

If the number of roots is $t>D\approx \sqrt{2kn}$, then we can find all polynomials with agreement $t$.
%\item
%\end{enumerate}
\end{proof}
Guruswami and Sudan improved this to $\sqrt{kn}$.

\fixme{
Consider $k=1$, $\F=\R$, find all lines which pass through at least 3 points. If you can interpolate a degree 2 polynomial through all these points, get all lines as lines within the curve. The degree 2 curve will be 2 lines.

Let's look at a small example to see how to improve the bound.

Consider the following picture (see notebook). Here $n=10$ and $t=4$. Here we can choose $k=4$. However it can only have 4 linear factors. The maximum $D$ is 3. %D to small to contain all lines.
Peculiar: through each point there are 2 lines. An algebraic curve passing all points should vanish at the points with multiplicity 2. Fit a polynomial which vanishes with degree 2 at the points.}

First define multiplicity.
\begin{df}
$Q(x,y)$ vanishes with multiplicity $r$ at $(\al,\be)$ if $Q(x+\al,y+\be)$ doesn't have any monomial of degree $\le r$.
\end{df}
\begin{lem}
Let $Q(x,y)$ be with $(1,k)$ degree $\le D$, vanishing at $(\al_i,y_i)$ with multiplicity $r$ for all $i\in [n]$. Let $P$ be a degree $k$ polynomial with agreement $t>D/r$. Then $y-p(x)\mid Q(x,y)$.
%deg 5, should still contain line.
\end{lem}

\begin{proof}
Let $p\in L$, $P(\al_i)=y_i$. 
Define 
\[
Q^i(x,y) = Q(x+\al_i,y+y_i);
\]
it has no monomials of degree $<r$.
Then
\bal
R(x) &= Q(x,P(x))\\
&=Q^i (x-\al_i, P(x)-y_i)\\
&=Q^i (x-\al_i,\ub{P(x)-p(\al_i)}{x-\al_i\mid P(x)-P(\al_i)}).
\end{align*}
$Q^i$ has no monomials of degree $<r$. Thus $(x-\al_i)^r\mid R(x)$. The number of linear factors is $tr>D$. Thus $R(x)\equiv 0$ and $y-p(x)\mid Q(x,y)$. 
\end{proof}
The number of coefficients in $Q(x,y)$ of $(1,k)$-degree $D$ is $\rc k\binom{D+2}D$.
%no monom degree $r$. 
The number of homogeneous linear equations is $n\binom{r+1}2$. So a nonzero $Q$ exists if $\rc k\binom{D+2}2>n\binom{r+1}2$.

Choose $D=\sqrt{knr(r+1)}$. Let $t=\fc Dr = \sqrt{kn(1+\rc r)}$.

%find all polynomils. 
Make $r$ large enough, we approach the Johnson bound.

The number of polynomials in the list is at most $\fc Dk=\sfc{nr(r+1)}k$, the $y$-degree. 

Conclusion: 
\begin{enumerate}
\item
If $t>\sqrt{kn}$ the list size is $\le n^\ep$ and we can find all of them.
\item
The Reed Solomon code of rate $R$ can be list decoded up to $1-\sqrt{(1+\ep)R}$ errors with best size $\rc{\ep\sqrt R}$.
\end{enumerate}
%no more than $y$ degree

This method (the polynomial method) is very flexible. We give another application, the list recovering problem. Given $x\in C$, 
%replace each alphabet with small set
a noisy channel turns each coordinate into a set $s_i$ and spits out $s_1,\ldots, s_n$, $|s_i|\le \ell$. We have $|s_i|\le l $ such that for at least $1-\de$ fraction of $i$'s, $x_i\in s_i$. Find all $x$ such that $x_i\in s_i$ for at least $1-\de$ fraction of $i$'s. 
\[
\ab{
\pa{
\bigcup_{y\in S_1\times \cdots \times S_n} B(y,\de)
}\cap C
}\le L.
\]
Reed-Solomon codes are also good list-recoverable codes.

If $t>\sqrt{knl}$, where $t$ is the number of $i$ such that $x_i\in S_i$, then the Reed-Solomon code is $(1-\sqrt{kl},l,O(n^2l^2)=L)$ list recoverable.
%we want optimal paramters.

We only achieve $1-\sqrt R$. We want to attain $(1-R,L)$. Guruswami and Rudra came up with folded Reed-Solomon codes. 
Take a generator $\ga $ of $\F_q^{\times}$, $n=q-1$. $\al_1,\ldots, \al_n$ are $S=\{1,\ga,\ldots, \ga^{q-1}\}$. \fixme{Consider blocks. $\F^m$ by $m$. $P(1),\dots P(r),\ldots, P(r^{q-1})$.

It's an open problem to make $n^{O\prc{\ep}}$ independent of $n$. 
Use concatention to get length down. 

You can concatenate a list-recoverable code with a list-decodable codes to get a list-decodable code. $C_{\text{out}}\circ C_{\text{in}}$.}

%$C_{}$
%small list-decodable.
%folding: alphabet grow large

%S
%GS: RS (1-\srqt k)
%PVbetter than 1-R for small enough $R$
%FRS 1-R

%Extractors, expanders etc.

%to be list decodable don't need dep on n?
%codes achieve both poly n /poly n.

\section{$SL=L$}

%picture of Z.

Reingold, Vadhan, and Wigderson.

There are many stories in this problem.

One of the most fundamental questions in theoretical CS is the following.
\begin{clm}
Randomness is useless.
\end{clm}
This claim comes from very recent years. 20 years ago people actually believe it's useful (Papadimitriou gave $BPP\ne P$ as a homework exercise in his book).

There are 2 questions: is randomness useful in time and in space?
\begin{enumerate}
\item
$BPP\stackrel ? P$
\item
$RL\stackrel ? L$
\end{enumerate}
We focus on the second problem. The first problem is wide open: we don't know $BPP\stackrel?{\in}DTIME(2^{o(n)})$. We know more about the second question: Savitch proves
\[
BPP\subeq DSPACE(\ln^2n).
\]
We can do better: Saks and Zhou in 1999 show 
\[
RL\subeq DSPACE(\ln^{\fc 32}n).
\]
(They actual show this for BPP.)
Think of RL as (undirected) $s-t$ nonconnectivity problem. $coRL$ is interesting because it contains $s-t$ connectivity. (RL doesn't have a complete problem.) For $P\stackrel?=NP$, just look at a complete problem; here we don't have one.

Here we show $SL=L$. Reingold showed this in 2004. 

For simplicity, just think of SL as one problem, undirected $s-t$ connectivity.

Consider a graph. 


Jieming starts from a vertex and wants to find a way to Nanjing. Jieming has very little memory, and cannot remember the whole structure of the graph. He can't remember much more than the name of a city.

The standard way to solve this is to take a uniform random walk on the graph. If the graph has $|V|=n$, after $n^2\ln n$ steps, there is a high probability that he will have visited Nanjing.

But for a general graph, it's necessary to flip $n^2\ln n$ coins: Consider 2 complete graphs with a bridge: It takes order of $n$ time to hit the bridge vertex, and it has $O\prc n$ chance of crossing the bridge.

For which graphs can we do this randomized routing faster than the worst case? A natural family is the family of expaner graphs.

Expander graphs have rapid mixing under a random walk.
There are 2 definitions.
\begin{df}
A graph $G=(V,E)$ is a \vocab{$\la$-edge expander} if for all sets $S\subeq V$, $|S|\le \fc{|V|}2$, 
\[
\fc{E(S,\ol S)}{\Vol(S)}\ge \la.
\]
Here, $\Vol(S)=\sum_{v\in S}\deg(v)$. (We'll focus on the simple case when the graph is $d$-regular, i.e, for all $v\in V$< $\deg(v)=d$.) (Pictorially, a subset of $G=(V,E)$ is very spiky, like a sea urchin.)

A graph $G=(V,E)$ is a \vocab{$\la$-spectral expander}if $\la_2(L(G))\ge \la$. The Laplacian $L(G)$ is defined from the adjacency matrix $A_G$. $A_G$ is defined by $(A_G)_{ij}=1$ if there is a edge between $i$ and $j$. $M$ is the random walk matrix $\rc d A_G$, and 
\[
L=I-M.
\]
(Thinking of this matrix as an operator, $(Lx)_i=\rc d\sum_{(i,j)\in E}x_{ij}$, it flows from a vertex to adjacent vertices.)
\end{df}
The Laplacian originates in differential geometry, $L=\div \nb$.

These 2 definitions are quite close.
\begin{thm}
Let $h(G),\la(G)$ denote the edge and spectral expansions of $G$. Then
\[
\la(G)\le h(G)\le \sqrt{2\la(G)}.
\]
\end{thm}
The LHS is trivial, the RHS is Cheeger's inequality.

People in CS use a third definition that is more useful.
\begin{df}
A graph is a $\la$-expander if for all $x\perp u=\colthree{1}{\vdots}1,x\ne 0$, 
\[
\ab{\fc{x^TMx}{x^Tx}}\le \la.
\]
\end{df}
(For edge and spectral expansion we want $\la$ to be large. For random walk expansion we want $\la$ to be close to 0.) %For the other definitions we want $\la$ to be close to 1.)
$\la(L)$ range from 0 to 2, while $\la(M)$ range from 1 to $-1$.

Note that 
\[
\ab{\fc{x^T(I-L)x}{x^Tx}}=\ab{1-\fc{x^TLx}{x^Tx}}
\]
so spectral and random-walk expansion are not quite the same. For bipartite graphs, $\la(M)=1$: random walks are not mixing at all. The side you're on depends on the parity.

For the zig-zag product they consider the third definition.

\subsection{Zig-zag product}

Think of $G_1$ having large size and degree $(N_1,D_1)$, $G_2$ having small size and degree $(N_2,D_2)$, and suppose $D_1=N_2$. 
\[
G_3=G_1\zz G_2
\]
has large size $N_1N_2$ and small degree, $D_2^2$.

Let $\la_M(G_i)=\la_i$ denote the random walk expansion. Then  (a naive bound)
\[
\la_M(G_3)\le \la_1+\la_2+\la_2^2.
\]
It's easy to construct a large degree expander. It's easy to construct small-size expanders (try all possible graphs). It's easy to construct $G_1,G_2$, but an expander like $G_3$ is hard to construct. It takes as input 2 easy-to-construct expanders and outputs a harder to construct graph that's an expander.

Replace each vertex of the original graph with a copy of the second graph $G_2$. Its size is hence $N_1N_2$.

Each vertex in $G_3$ is labeled $(v,u)\in G_1\times G_2$. The $(i,j)$th neighbor is defined as follows. Take $u'$ the $i$th neighbor of $u$ in $G_2$, $u\xra i u'$. Take the $u'$th neighbor of $v$ in $G_1$; move to the cloud of $w$; $v\xra{u'} w$ in $G_2$. Now consider $\wt u$ such that $w\xra{\wt u}v$ in $G_1$; we move to $(w,\wt u)$. Now move $\wt u \xra j \wt u'$ in $G_2$. Define
\[
(v,u)\xra{(i,j)} (w,\wt u').
\]
%(Use $G_1$ inside the cloud, $G_2$ outside.)

What's the magic of the zig-zag product? We claim
\[
A_3:=A(G_3) = \wt{A_2}\wt{A_1}\wt{A_2}
\]
where $\wt{B_2}=I\ot A_2$ has $N_1$ blocks and each is a copy of $A_2$, and $\wt{A_1}$ is a permutation (actually matching) matrix where $(\wt{A_1})_{((v,u),(w,u'))}=1$ if 
\[
v\xra{u,G_1} w \xra{u',G_1}v.
\]

Consider when $(u,v)\xra{(i,j)} (w,\wt u')$:
\bal
u&\xra{i,G_2}u'\\
v&\xra{u',G_1}w\xra{\wt u,G_1} v\\
\wt u & \xra{j}\wt u'
\end{align*}
What is the action of $\wt{A_2}\wt{A_1}\wt{A_2}$ on $e_{(v,u)}$? It goes to $\sum_{uu'\in G_2}e_{(v,u')}=e_v\ot A_2e_u$. ``Stay in the block of $v$, move to all the neighbors of $u$."

In the second step, we move across blocks as given by 
$\wt{A_1}$.

%(If you sum over $u$, you get $A_1$.)
%want converge in log n steps
%replace tree - length log n.
%log n lg d.

We need to show
\[
%\la_m(C_{i_3}) = \max_{x\perp u} 
\forall x\in \R^{N_1\times N_2}, x\perp u \implies  \fc{x^TM_3x}{x^Tx}\le \la_1+\la_2+\la_2^2.
\]
Decompose $x$ as a vector that is uniform on each block, 
\[x=\ub{\al\ot u}{\al^{\parallel}} + \ub{x'}{\al^{\perp}}\] where $x'$ has blocks $\wt{\al_i}\perp u_i$. Then (noting $\wt{A_2}\al^{\parallel}=\al^{\parallel}$),
\bal
\an{x,\wt{A_2}\wt{A_1}\wt{A_2}x} 
&=\an{\wt{A_2}x, \wt{A_1}\wt{A_2}x}\\
&=\an{\wt{A_2}(\al^{\parallel}+\al^{\perp}), \wt{A_1}\wt{A_2}(\al^{\parallel}+\al^{\perp})}\\
&=\an{\al^{\parallel}+\wt{A_2}\al^{\perp}, \wt{A_1}(\al^{\parallel} + \wt{A_2} \al^{\perp})}\\
&=\an{\al^{\parallel}, \wt{A_1}\al^{\parallel}}
+\an{\al^{\parallel}, \wt{A_1}\wt{A_2}\al^{\perp}}
+\an{\wt{A_2}\al^{\perp}, \wt{A_1}\al^{\parallel}}
+\an{\wt{A_2}\al^{\perp}, \wt{A_1} \wt{A_2}\al^{\perp}}\\
&\le \la_1+2\la_2+\la_2^2
\end{align*}
%\wt{A_2}\al^{\parallel}=\al^{\parallel}
Because $\al_1+\cdots +\al_{N_1}=0$ we can use the expansion properties of $G_1$ to get the $\la_1$ bound for the first term.

The idea: 
\begin{enumerate}
\item
Suppose $G$ has parameters $(N,D)$ and expansion $\la$. (say $1-\rc{ND}$)
\item
$G^2$ has parameters $(N,D^2)$, expansion $\la^2$.
\item
Take $H$ with parameters $(D^2,\sqrt{D}), \la_2$. Then 
\[
G^2\zz H=G_{\text{new}}
\]
has parameters $(ND^2,D)$ expansion $\la(G_{\text{new}}) \le \la^2 + 2\la_2+\la_2^2 \le \la^2+\ep$.
\end{enumerate}
We go from $G$ with parameters $(N,D,\la)$ to $G_{\text{new}}$ with parameters $(ND, D,\la^2+\ep)$. Call this operator $Z$. Doing this operator $t$ times, $Z^tG=(ND^{2t}, D, \la^{2^t}+\ep')$. Then set $t=\lg \pf{ND}2$ to get $Z^tG$ with parameters $(N^2D, D, \rc e +\ep)$. (We're cheating a little, using the naive bound. We need a better bound to get $\ep$ small: $G_3$ has expansion $\rc2(1-\la_2^2)\la_1+\rc2 \sqrt{(1-\la_2^2)^2\la_1^2+t\la_2^2}$. 
%$\la_2=0$ this is 0
%if slightly bigger, constant $\la_1$ plus something. $(1-\la_2^2)\la_1+\la_2
%sum of norm squares equals 1. Just that problem. Just n
To go from our proof to the real proof, you just need to note $\al^{\parallel}\perp \al^{\perp}$, $\al^{\parallel}\perp \wt{A_2}\al^{\perp}$. Plug in this picture into the formula.

%($\ep$ should be $\rc{N}$).
%D^2 vertices and ... edges. 
%Start with 2 constant size expanders and blow up. The real constructor: start with arbitrary bound, but need better bound.

%\chapter{Introduction}


\section{Normed spaces}

\begin{df}
A \textbf{normed space} is a pair $(X,\ve{\cdot})$ where $X$ is a real or complex vector space and $\ved$ is a norm on $X$. Most of the time the choice of scalar field makes little difference; for convenience we'll use real scalars. A norm induces a metric: $d(x,y)=\ve{x-y}$. This induces a topology on $X$, called the \textbf{norm topology}. A \textbf{Banach space} is a complete normed space.
\end{df}
\begin{ex}
\begin{enumerate}
\item (sequences)
For $1\le p<\iy$, we have $\ell_p=\set{(x_n)\text{ scalar sequence}}{\sum_{n=1}^{\iy}|x_n|^p<\iy}$ with norm $\ve{x}_p=\pa{\sum_{n=1}^{\iy}|x_n|^p}^{\rc p}$. (Minkowski's inequality says that if $x,y\in \ell_p$ then $x+y\in \ell_p$, so $\ve{x+y}_p\le \ve{x}_p+\ve{y}_p$. Then $\ell_p$ is a Banach space.
\item (convergent sequences)
$\ell_{\iy}=\set{(x_n)\text{ scalar sequence}}{(x_n)\text{ is bounded}}$ with $\ve{x}_{\iy}=\sup_{n\in N}|x_n|$. Then $\ell_{\iy}$ is a Banach space.

$c_{00}=\set{(x_n)\text{ scalar sequence}}{\exists N\forall n>N, x_n=0}$. Let $e_n=(0,0,\ldots, 0,\ub{1}{n},0,\ldots)$; then $c_{00}=\spn\set{e_n}{n\in \N}$. Note that $c_{00}$ is a subspace of $\ell_{\iy}$ but it's not closed: In $\ell_p, 1\le p<\iy$, $\ell_p=\ol{\spn}\set{e_n}{n\in \N}$. 

$c_0=\set{(x_n)\in \ell_{\iy}}{\lim_{n\to \iy} x_n=0}$ is a closed subspace of $\ell_{\iy}$, $c_0=\ol{\spn}\set{e_n}{n\in \N}$ in $\ell_{\iy}$.

$c=\set{(x_n)\in \ell_{\iy}}{\lim_{n\to \iy} x_n \text{ exists}}$ is a closed subspace of $\ell_{\iy}$. $c_0$ and $c$ are Banach spaces.
\item (Euclidean space)
$\ell_p^n=(\R^n,\ved_p), 1\le p\le \iy$.
\item 
$K$ is any set, $\ell_{\iy}(K)=\set{f:K\to \R}{f\text{ is bounded}}$ with norm $\ve{f}_{\iy}=\sup_{x\in K}|f(x)|$. This is a Banach space, e.g. $\ell_{\iy}=\ell_\iy(\N)$.
\item 
$K$ compact topological space $C(K)=\set{f\in \ell_{\iy}(K)}{f\text{ continuous}}=\set{f:K\to \R}{f\text{ is continuous}}$. $C(K)$ is a closed subspace of $\ell_{\iy}(K)$ because any uniform limit of continuous functions is continuous, and hence it's a Banach space, e.g. $C[0,1]$.

We'll write $C^{\R}(K)$ and $C^{\C}(K)$ for the real and complex versions of $C(K)$, respectively.
\item 
Let $(\Om,\Si,\mu)$ be a measure space. Then for $1\le p<\iy$,  \[L_p(\mu)=\set{f:\Om\to \R}{f\text{ is measurable}, \int_{\Om}|f|^p\,d\mu<\iy}\] with norm $\ve{f}_p=\pa{\int_{\Om}|f|^p\,d\mu}^{\rc p}$ is a Banach space (after identifying functions that are equal almost everywhere. 

When $p=\iy$, $L_{\iy}(\mu)=\set{f:\Om\to \R}{f\text{ is measurable and essentially bounded}}$. (``Essentially bounded" means that there exists a null-set $N$ such that $f$ is bounded on $\Om\bs N$.)
\[
\ve{f}_{\iy}=\ess\sup|f|=\inf_N\sup_{\Om\bs N}|f|.
\]
\item
Hilbert spaces, e.g. $\ell_2$, $L_2(\mu)$. All Hilbert spaces are isomorphic, but some different representation may be more natural.
\end{enumerate}
\end{ex}

\begin{pr}
Let $X,Y$ be normed spaces, $T:X\to Y$ linear. Then the following are equivalent.
\begin{enumerate}
\item
$T$ is continuous.
\item
$T$ is bounded: $\exists C\ge 0$, $\ve{Tx}\le C\ve{x}$ for all $x\in X$.
\end{enumerate}
\end{pr}
\begin{proof}
To think about continuity at $a$, ``translate" to 0 using linearity.
\end{proof}

\begin{df}
Let $\cal B(X,Y)=\set{T:X\to Y}{T\text{ is linear and bounded}}$. This is a normed space with the \textbf{operator norm}: $\ve{T}=\sup\set{\ve{Tx}}{\ve{x}\le 1}$. $T$ is an \textbf{isomorphism} if $T$ is a linear bijection whose inverse is also continuous. (This is equivalent to $T$ being a linear bijection and there existing $a>0,b>0$, with $a\ve{x}\le \ve{Tx}\le b\ve{x}$ for all $x\in X$.)

If there exists such $T$, we say $X,Y$ are \text{isomorphic} and we write $X\sim Y$.

If $T:X\to Y$ is a linear bijection such that $\ve{Tx}=\ve{x}$ for all $x\in X$ (i.e. $a=b=1$), then $T$ is an \textbf{isometric isomorphism} and we say $X,Y$ are isometrically isomorphic and write $X\cong Y$\footnote{Some people use $\cong$ for isomorphism. We use it to mean isometric isomorphism.}.

$T:X\to Y$ is an \textbf{isomorphic embedding} if $T:X\to TX$ is an isomorphism. We write $X\hra Y$.
\end{df}
\begin{pr}
If $Y$ is complete, then $\cal B(X,Y)$ is complete. In particular, $X^*=\cal B(X,\R)$, the space of bounded linear functionals, called the \textbf{dual space} of $X$, is always complete.
\end{pr}
\begin{ex}
\begin{enumerate}
\item
For $1<p<\iy$, then $\ell^*\cong \ell_q$ where $\rc p+\rc q=1$. The proof uses H\"older's inequality: $x=(x_n)\in \ell_p$, $y=(y_n)\in \ell_q$ then $\sum|x_ny_n|\le \ve{x}_p\ve{y}_q$. This isomorphism is $\ph:\ell_q\to \ell_p^*,y\mapsto \ph_y, \ph_y(x)=\sum x_ny_n$.)
\item
$c_0^*\cong \ell_1$, $\ell_1^*\cong \ell_{\iy}$. 
(Later we will see that $c_0$ cannot be a dual space.)
%(This gives an alternate proof of completeness.)
\item 
If $H$ is a Hilbert space then $H^*\cong H$ (Riesz Representation Theorem).
\item
If $(\Om,\Si,\mu)$ is a measure space, $1<p<\iy$, then $L_p(\mu)^*\cong L_q(\mu)$ where $\rc{p}+\rc{q}=1$.

If $\mu$ is $\si$-finite then $L_1(\mu)^*\cong L_{\iy}(\mu)$.
(Else we only have $L_{\iy}(\mu)\hra L_1(\mu)^*$.)
\end{enumerate}
\end{ex}

{\color{blue}Lecture 2}

Recall that if $V$ is a finite-dimensional vector space, then any two norms on $V$ are equivalent. Specifically, if $\ved$ and $\ved'$ are two norms on $V$, then there exist $a,b>0$ such that 
\[
a\ve{x}\le\ve{x}'\le b\ve{x}\forall x\in V.
\]
In other words, $\Id:(V,\ved)\to (V,\ved')$ is an isomorphism. 

Some consequences are the following.
\begin{cor}
\begin{enumerate}
\item
If $X,Y$ are normed spaces, $\dim X<\iy$, $T:X\to Y$ is linear, then $T$ is bounded.
\item
If $\dim X<\iy$ then $X$ is complete. 
\item If $X$ is a normed space and $E$ a subspace with $\dim E<\iy$, then $E$ is closed.
\end{enumerate}
\end{cor}
\begin{proof}
\begin{enumerate}
\item
Set $\ve{x}'=\ve{x}+\ve{Tx}$. This is a norm on $X$, so there exists $b>0$, $\ve{x}'\le b\ve{x}$ for all $x$, so $\ve{Tx}\le b\ve{x}$ for all $x\in X$. 

%If $\dim X=\dim Y<\iy$ then $X\sim Y$.
\item
By (1), $X\sim \ell_2^n$ where $n=\dim X$.
\end{enumerate}•
\end{proof}
\section{Riesz's lemma and applications}
The unit ball is compact and this characterizes finite-dimensionality. We use the following.
\begin{lem}[Riesz's Lemma]
\llabel{lem:riesz}
Let $Y$ be a proper closed subspace of a normed space $X$. Then for every $\ep>0$ there exists $x\in X$, $\ve{x}=1$, such that $d(x,Y):=\inf_{y\in Y}\ve{x-y}>1-\ep$.
%null space, don't have notion as euclidean space, suggests proof that works. Intuition about euclidean space sometimes dangerous because may not work in normed space but in this case works
\end{lem}
We'd like to take the some sort of ``perpendicular" vector to $Y$, or the vector which minimizes the distance from a point not on $Y$ to $Y$. Note this is not in general possible since $X$ may not be complete, and hence the ``$\inf$." However, we can come arbitrarily close to that $\inf$, and get an ``almost perpendicular" vector.
\begin{proof}
Pick $z\in X\bs Y$ with $Y$ proper. Since $Y$ is closed, $d(z,Y)>0$. There exists $y\in Y$ such that $\ve{z-y}<\fc{d(z,Y)}{1-\ep}$ (WLOG $\ep<1$).

Set $x=\fc{z-y}{\ve{z-y}}$. Then
\[
d(x,Y)=d\pa{\fc{z-y}{\ve{z-y}},Y}=\rc{\ve{z-y}}d(z-y,Y)=\fc{d(z,Y)}{\ve{z-y}}> 1-\ep.
\]
\end{proof}
We give two applications. First, some notation. In a metrix space $(M,d)$, write 
\[
B(x,r):=\set{y\in M}{d(x,y)\le r},x\in M,r\ge 0
\]
for the closed ball of radius $r$ at $x$. In a normed space, 
\[
B_X:=B(0,1)=\set{x\in X}{\ve{x}\le 1}, \qquad B(x,r)=x+rB_X.
\]
Also, $S_X=\set{x\in X}{\ve{x}=1}$. 
\begin{thm}
Let $X$ be a normed space. Then $\dim X<\iy$ iff $B_X$ is compact.
\end{thm}
\begin{proof}
``$\Rightarrow$" We have $X\sim \ell_2^n$ where $n=\dim X$.

``$\Leftarrow$" By compactness there exist $x_1,\ldots, x_n\in B_X$ such that $B_X\subeq \bigcup_{i=1}^n B(x_i,\rc2)$.  Let $Y=\spn\{x_1,\ldots, x_n\}$. For all $x\in B_X$ there exists $y\in Y$ with $\ve{x-y}\le \rc 2$, so $d(x,Y)\le \rc 2$. Thus there do not exist ``almost orthogonal vectors" in the sense of  Riesz's Lemma~\ref{lem:riesz}. This means $Y$ is not a proper subspace of $X$, so $X$ is finite-dimensional. %$Y$ is dense in $X$, so $X=\ol Y=Y$.
%(We can't find an orthogonal vector to $Y$ in the sense of Riesz's Lemma.)
\end{proof}
\begin{rem}
We showed the following in the proof: If $Y$ is a subspace of a normed space $X$ and there exists $0\le \de<1$ such that for all $x\in B_X$ there exists $y\in Y$ with $\ve{x-y}\le \de$, then $Y$ is dense in $X$.
\end{rem}
If we let $\de= 1$ then this statement is trivial. The remark says that if we can do a little better than 1, the trivial estimate, then we can automatically approximate $x$ with much smaller $\ep$.

\begin{thm}[Stone-Weierstrass Theorem]
Let $K$ be a compact topological space and $A$ be a subalgebra of $C^{\R}(K)$. If $A$ separates the points of $K$ (i.e., for all $x\ne y$ in $K$, there exists $f\in A$, $f(x)\ne f(y)$), and $A$ contains the constant functions, then $A$ is dense in $C^{\R}(K)$.
\end{thm}
In this case it does matter whether the field of scalars is $\R$ or $\C$.

The following proof is due to T. J. Ransford.
\begin{proof}
First we show that if $E,F$ are disjoint closed subsets of $K$, then there exists $f\in A$ such that $-\rc2\le f\le \rc2$ on $K$ and $f\le -\rc 4$ on $E$ and $f\ge \rc 4$ on $F$.

Fix $x\in E$. Then for all $y\in F$, there exists $h\in A$ such that $h(x)=0$, $h(y)>0$, $h\ge 0$ on $K$. (This is since $A$ separates points, we can shift by a constant, and square the function.) Then there is an open neighborhood of $y$ on which $h>0$.  %finitely many cover $F$ by compactness.
An easy compactness argument gives that there exists $g=g_x\in A$ with $g(x)=0$, $g>0$ on $F$, 
%strictly positive on neighborhood, going to be strictly positive on all
%rescale it so that  
$0\le g\le 1$ on $K$. 
Pick $R=R_x\in \N$ such that $g>\fc2R$ on $F$, set $U=U_x=\set{y\in K}{g(y)<\rc{2R}}$. 
%compact, attain inf

Do this for all $x\in E$.
%finitely many will cover.
Compactness gives a finite cover: there exist $x_1,\ldots, x_m$ such that $E\subeq \bigcup_{i=1}^m U_{x_i}$. To simplify notation, set $g_i=g_{x_i}$, $R_i=R_{x_i}$, $U_i=U_{x_i}$, and $i=1,\ldots, m$. For $n\in \N$,  by Bernoulli's inequality,
\begin{align*}
\text{on }U_i&&
(1-g_i^n)^{R_i^n}&\ge 1-(g_iR_i)^n>1-2^{-n}\to 1\text{ as }n\to \iy\\
\text{on }F&&
(1-g_i^n)^{R_i^n}&\le \rc{(1+g_i^n)^{R_i^n}}\le \rc{(g_iR_i)^n}<\rc{2^n}\to 0\text{ as }n\to \iy
\end{align*}
There exists $n_i\in \N$ such that $h_i=1-(1-g_i^{n_i})^{R_i^{n_i}}$ satisties
\begin{itemize}
\item
on $U_i$, $h_i\le \rc 4$
\item
on $F$, $h_i\ge \pf 34^{\rc m}$
\item
on $K$, $0\le h_i\le 1$.
\end{itemize}
Set $h=h_1h_2\cdots h_m$. Then $h\le \rc 4$ on $E$, $h\ge \fc 34$ on $F$, and $0\le h\le 1$ on $K$. Set $f=h-\rc2$.
%corresp is at most a quarter, other at most 1, so at most 1/4
Given $g\in C^{\R}(K)$, $\ve{g}_{\iy}\le 1$, set
\[
E=\set{x\in K}{g(x)\le -\rc 4},\,F=\set{x\in K}{g(x)\ge \rc 4}.
\]
Let $f\in A$ be as above. Then $\ve{f-g}\le \fc 34$, i.e., $d(g,A)\le \fc 34$. By Riesz's Lemma~\ref{lem:riesz}, $A$ is dense in $C^{\R}(K)$.
\end{proof}
%reading math is a sort of computation, info decompression
\begin{rem}
The complex version says that if $A$ is a subalgebra of $C^{\C}(K)$ that separates points of $K$ contains the constant functions, and is closed under complex conjugation ($f\in A\implies \ol f\in A$), then $A$ is dense in $C^{\C}(K)$.
\end{rem}
\section{Open mapping lemma}
We'll assume the Baire category theorem and its consequences: principle of uniform boundedness, open mapping theorem (OMT), closed graph theorem (CGT). 
\begin{df}
Let $A,B$ be subsets of a metric space $(M,d)$ and let $\de\ge 0$. Say $A$ is \textbf{$\de$-dense} in $B$ if for all $b\in B$ there exists $a\in A$ with $d(a,b)\le \de$.
\end{df}
\begin{lem}[Open mapping lemma]\llabel{lem:oml}
Let $X,Y$ be normed spaces, $X$ complete, $T\in \cal B(X,Y)$. Assume for some $M\ge 0$ and $0\le \de<1$ that $T(MB_X)$ is $\de$-dense in $B_Y$. Then $T$ is surjective. 
More precisely, 
%important: have quantitative
for all $y\in Y$ there exists $x\in X$ such that $y=Tx$ and
\[
\ve x\le \fc{M}{1-\de}\ve y,
\]
i.e.,
\[
T\pa{\fc{M}{1-\de}B_X}\supeq B_Y.
\]
Moreover, $Y$ is complete.
\end{lem}
\begin{proof}
The proof involves successive approximations. 
Let $y\in B_Y$. There exists $x_1\in MB_X$ with $\ve{y-Tx_i}\le \de$. Then $\fc{y-Tx_i}{\de}\in B_Y$. There exists $x_2\in MB_X$, with $\ve{\fc{y-Tx_i}{\de}-Tx_2}\le \de$, i.e., $\ve{y-Tx_1-\de Tx_2}\le \de^2$, and so forth. Obtain $(x_n)$ in $MB_X$ such that 
\[
\ve{y-Tx_1-\de Tx_2-\cdots -\de^{n-1} Tx_n}\le \de^n
\]
for all $n$. Set $x=\sum_{n=1}^{\iy} \de^{n-1} x_n$. This converges since $\sum_{n=1}^{\iy}\ve{\de^{n-1}x_n}\le M\sum_{n=1}^{\iy}\de^{n-1}=\fc{M}{1-\de}$, and $X$ is complete.\footnote{This kind of geometric sum argument comes up a lot in functional analysis!} 
So $x\in \fc{M}{1-\de}B_X$ and  by continuity $Tx=\suo \de^{n-1} Tx_n=y$. For the ``moreover" part, let $\hat Y$ be the completion of $Y$, and view 
%unique banach space of which it is a dense subspace
$T$ as a map $X\to \hat Y$. Since $B_Y$ is dense in $B_{\hat Y}$, $T(MB_X)$ is $\de'$-dense in $B_{\hat Y}$ for $\de<\de'<1$. By the first part, $T(X)=\hat Y=Y$, so $Y$ is complete.
\end{proof}
\begin{rem}
Suppose $T\in\cal B(X,Y)$, $X$ is complete, and the image of the ball is dense: $\ol{T(B_X)}\supeq B_Y$. Suppose that for all $\ep>0$, $T((1+\ep)B_X)$ is $1$-dense in $B_Y$. Take $M>1,0< \de<1$ so that $1+\ep=\fc{M}{1-\de}$;  lemma~\ref{lem:oml} shows that  $T\pa{(1+\ep)B_X}\supeq B_Y$. It follows that $T(B_X^{\circ})\supeq B_Y^{\circ}$. (For a subset $A$ of a topological space, $A^{\circ}$ or $\text{int}(A)$ denotes the interior of $A$.)
\end{rem}

{\color{blue} Lecture 3}

\subsection{Applications of the open mapping lemma}
\begin{thm}[Open mapping theorem]\llabel{thm:omt}
Let $X,Y$ be Banach spaces, $T\in \cal B(X,Y)$ be onto. Then $T$ is an open map.
\end{thm}
\begin{proof}
Let $Y=T(X)=\bigcup_{n=1}^{\iy} T(nB_X)$. The Baire category theorem tells us that there exists $N$ with $\text{int}(\ol{T(NB_X)})\ne \phi$. Then there exists $r>0$ with
\[
\ol{T(NB_X)}\supeq rB_Y.
\]
By Lemma~\ref{lem:oml}, for $M=\fc{2N}{r}$, we have $T(MB_X)\supeq B_Y$. Therefore, $U$ is open and $T(U)$ is open.
\end{proof}
%say T is open, same as T^{-1} cont.
\begin{thm}[Banach isomorphism theorem]\llabel{thm-bit}
If in addition $T$ is injective, then $T^{-1}$ is continuous.
\end{thm}
\begin{proof}
An open map that is a bijection is a homeomorphism.
\end{proof}
\begin{thm}[Closed graph theorem]
Let $X,Y$ be Banach spaces and $T:X\to Y$ be linear. Assume that whenever $x_n\to 0$ in $X$, $Tx_n\to y$ in $Y$, then $y=0$. Then $T$ is continuous.
\end{thm}
This is a powerful result. Usually we have to show the sequence converges and show it converges to 0. This says that we only have to check the second part given the first part. %once we show it converges, it automatically converges to 0.
\begin{proof}
The assumption says that the graph of $T$
\[
\Ga(T)=\set{(x,Tx)}{x\in X}
\]
is closed in $X\opl Y=\set{(x,y)}{x\in X,y\in Y}$ with norm, e.g.
\[
\ve{(x,y)}=\ve{x}+\ve y.
\]
So $\Ga(T)$ is a Banach space. Consider $U:\Ga(T)\to X$, $U(x,y)=x$. $U$ is a linear bijection and $\ve{U}\le 1$. From the Banach isomorphism theorem~\ref{thm-bit}, $U^{-1}$ is continuous, i.e., $x\mapsto (x,Tx)$ is continuous.
\end{proof}
We'll give three more applications. The first one is to quotient spaces. Let $X$ be a normed space and $Y$ be a closed subspace. Then $X/Y=\set{x+Y}{x\in X}$ is a normed space with 
\[
\ve{x+Y}=d(x,Y)=d(0,x+Y)=\inf\set{\ve{x+y}}{y\in Y}.
\]
We need $Y$ closed to ensure that if $\ve{z}=0$ then $z=0$ for $z\in X/Y$.
\begin{pr}\llabel{pr:5}%5
Let $X,Y$ be as above. If $X$ is complete, then so is $X/Y$.
\end{pr}
%this can be proved directly too.
\begin{proof}
Consider the quotient map $q:X\to X/Y$, $q(x)=x+Y$. This is a bounded linear map, $q\in \cal B(X,X/Y)$, so
\[
\ve{q(x)}=d(x,Y)\le \ve{x}
\]
and $q(B_X^{\circ})\subeq B_{X/Y}^{\circ}$. If $\ve{x+Y}<1$ then there exists $y\in Y$ with $\ve{x+y}<1$ and $q(x+y)=q(x)=x+Y$ so $q(B_X^{\circ})=B_{X/Y}^{\circ}$. So for any $M>1$, $\ol{q(MB_X)}\supeq B_{X/Y}$. %image of some large ball contains ball. 
By Lemma~\ref{lem:oml}, $X/Y$ is complete.
\end{proof}
\begin{pr}
Every separable Banach space $X$ is (isometrically isomorphic to) a quotient of $\ell_1$.
\end{pr}
\begin{proof}
Let $\set{x_n}{n\in \N}$ be dense in $B_X$. Let $(e_n)$ be the standard basis of $\ell_1$. If $a=(a_n)\in \ell_1$, then $a=\sum_{n=1}^{\iy} a_ne_n$. Define
\[
T:\ell_1\to X,\qquad T\pa{\sum_{n=1}^{\iy} a_ne_n}=\sum_{n=1}^{\iy}a_nx_n.
\]
This is well defined because the sum converges: $\sum_{n=1}^{\iy} \ve{a_nx_n}\le \suo |a_n|=\ve{a}$. So $T\in \cal B(\ell_1,X)$, $\ve{T}\le 1$. Also $T(B_{\ell_1}^{\circ})\subeq B_X^{\circ}$. 
We have $T(B_{\ell_1})\supeq \set{x_n}{n\in \N}$, so $\ol{T(B_{\ell_1})}\supeq B_X$.

%checkme
%This means given $x\in B_X^{\circ}$, there exists a sequence $rx_n\to rx\in B_X$, $rx_n\in  $\ve{rx}=1$. Then $x_n\to x\in B_X^{\circ}$. 
So $T(B_{\ell_1}^{\circ})=B_X^{\circ}$. We have a unique $\wt T:\ell_1/\ker T\to X$ such that
\[
\xymatrix{
\ell_1\ar[rr]^T\ar[rd]_q & & X\\
& \ell_1/\ker T\ar[ru]_{\wt T}&
}
\]
commutes, i.e., $T=\wt Tq$ where $q$ is the quotient map.
%onto from lemma~\ref{lem:oml}
Moreover, $\wt T$ is a linear bijection
\[
\wt T(B_{\ell_1/\ker T}^{\circ})=\wt T(q(B_{\ell_1}^{\circ}))=T(B_{\ell_1}^{\circ})=B_X^{\circ}.
\]
So $\wt T$ is an isometric isomorphism, and $X\cong \ell_1/\ker T$.
\end{proof}
This suggests that to understand all Banach spaces we just have to understand $\ell_1$. But $\ell_1$ is quite complicated, not as innocent as it looks.

Recall the following definition.
\begin{df}
A topological space $K$ is \textbf{normal} if whenever $E,F$ are disjoint closed sets in $K$, there are disjoint open sets $U,V$ in $K$ such that $E\subeq U$, $F\subeq V$. 
\end{df}
\begin{lem}[Urysohn's lemma]\llabel{lem:urysohn}
If $K$ is normal and $E,F$ are disjoint closed subsets of $K$, then $\exists$ continuous $f:K\to [0,1]$ such that $f=0$ on $E$ and $f=1$ on $F$.
\end{lem}
This can be used to construct partitions of unity which we'll use in chapter 3.
\begin{thm}[Tietze's Extension Theorem]
If $K$ is normal and $L$ is a closed subset of $K$, $g:L\to \R$ is bounded and continuous, then there exists a bounded, continuous $f:K\to \R$ such that $f|_L=g$, $\ve{f}_{\iy}=\ve{g}_{\iy}$. 
\end{thm}
\begin{proof}
Let $C_b(K)=\set{h:K\to \R}{h\text{ bounded, continuous}}$, a closed subspace of $\ell_{\iy}(K)$ in $\ved_{\iy}$ (so it is a Banach space). Consider $R:C_b(K)\to C_b(L)$, $R(f)=f|_L$. We need $R(B_{C_b(K)})=B_{C_b(L)}$ (``$\subeq$" is clear, since $\ve{R}\le 1$.) Let $g\in B_{C_b(L)}$ so $-1\le g\le 1$. Let $E=\set{y\in L}{g(y)\le -\rc{3}}$, $F=\set{y\in L}{g(y)\ge \rc 3}$. Urysohn's lemma gives $\exists f\in C_b(K)$ such that $-\rc 3\le f\le \rc 3$, $f=-\rc 3$ on $E$, and $f=\rc 3$ on $F$.

We have
\[
\ve{R(f)-g}_{\iy} \le \fc23 
\]
and $\ve f\le \rc 3$. So $R(\rc 3B_{C_b(K)})$ is $\fc 23$-dense in $B_{C_b(L)}$. By the Open Mapping Lemma~\ref{lem:oml}, $R$ is surjective.
\end{proof}
\begin{rem}
The theorem holds in the complex case too.
\end{rem}

%%%%%%%%%%%%%%%%%%%%%%%%%%%


 
%\bibliographystyle{plain}
%\bibliography{refs}
\end{document}