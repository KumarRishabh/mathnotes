\subsection{Arithmetic circuits}

\begin{thm}
There is an explicit family of polynomials such that a representation
\[
f_d=\sum_{i=1}^s \prod_{j=1}^m Q_{ij}
\]
with $Q_{ij}$ degree 1 and involving $\le \sqrt d$ vars has $s\ge 2^{\Om(\sqrt d\ln d)}$.
\end{thm}
%For a random variable it's 
For a generic $f_d$, $s\ge 2^{1.5d\lg d}$.

The same statement also holds for $f_d=IMM_d$. 
This is bad news: our techniques can't give better lower bounds because there is a matching upper bound for IMM.
Answers another question on efficient parallel computation.

Arithmetic circuits are informally the most natural way to compute polynomials.

\begin{enumerate}
\item
Can explicit polys be efficiently computed? 
($VP\stackrel?= VNP$)
\item
Can computation be efficiently parallelized?
How efficiently can we simulate circuits of size $s$ by circuits of size $\De$.
\end{enumerate}

\begin{thm}
Any circuit of size $s$ and degree $d$ cn be simulated by a $\De$-depth circuit of size $s^{O(d^{\rc{\De-1}})}$.
\end{thm}
We can trade off size for structure (regular homogeneous circuit: $s^{O(d^{\fc 2{\De}})}$.

Is this optimal? For $\De=3$ yes, but with caveats. %bottom fanin $\sqrt d$.

Strong enough lower bounds for low-depth circuits imply VP$\ne$VNP. 

(What is the sweet spot? what depth? homogeoneus?)

Low-depth circuits are easy to analyze. Lots of work on lower bounds for low depth arithmetic circuits in recent years: hope to disvoer general patterns and technical ingredients.

Proof strategy: for $T_i=\prod_jQ_{ij}$,
\begin{enumerate}
\item
find a geometric property GP of the $T_i$'s. 
\item Express the property GP in terms of rank of big matrix $M$: if $T$ has property than $\rank(M(T))$ is small.
\item show that $\rank(M(f_d))$ is large.

If a matrix $M(f)$ has a large upper triangular submatrix, then it has large rank.

If the columns of $M(f)$ are almost orthogonal then $M(f)$ has large rank (Alon).
\end{enumerate}

$T$ is a product of low degree polynomials iff $V(T)$ is a union of low-degree hypersurfaces iff $V(T)$ has lots of higher-order singularities, iff $V(\pl^{=k}T)$ has lots of points.

How can we convert geometric propery to the rank of a matrix? $V=V(f_1,\ldots, f_m)$ is a variety. let $\cal G_{\ell}$ be the set of degree $\ell$ polynomials.

Hilbert's theorem: If $V$ is large then $G_{\ell}(V)$ has small dimension. If $V$ has dimension $r$ then 
\[
\cal I_l(V)=...
\]
has asymptotic dimension $\binom{n+l}{n}-\Te(l^r)$.

Hilbert's theorem can be generalized for algebraic restrictions of polynomials.

%alg restrictions of polynomials.

If $Q$ is a sparse polynomial then for a suitable random  algebraic restriction $Q\pmod{\cal I}$ is a low-degree polynoimal. If $T$ is a sum of product of low arith polynomials, this also hold. Yields lower bounds for homogeneous depth 5 with low bottom fanin.

Shpilka-Widgerson: a depth three circuit of size $s$ cn be converted to a homogeneous depth 5 circuit $C$ of size $s2^{\sqrt d}$. Preserves bottom fanin.

There is a meta-strategy common to any recent and older lower bounds. We don't understand the power or limitations of this meta-strategy.

Open: lower bounds for homogeneous depth 3 circuits for polynomials of degree $\gg n$.

