\subsection{Consistency in planted bisection model}

Connections to random $k$-SAT consistency. Physicists gave lots of conjectures.

Stochastic block model $G(n,p,q)$: two communities in a population, color randomly. There is probability $p$ of an edge between nodes within the same community, and $q<p$ between communities. More friends in your group on average.

%First introdced
%Statistics, computer science


When can you recover both communities exactly (exactly recoverable) with probability tending to 1?

If $|p_n-q_n|$ is large enough then there are efficient polynomial time exact recovery algorithms. To have exact recovery, we need $p_n=\Om\pf{\ln n}{n}$, otherwise we have vertices of degree 0.

Abbe, Bandeira, and Hall: recoverable if $a+b-\sqrt{ab}>2$, $p_n=a\ln n/n,q_n=b\ln n/n$.

Us: Exact recovery iff $\Pj(Y_n\ge X_n)=o\prc n$, $X_n=Bin(n/2,p_n)$ and $Y_n=Bin(n/2,q_n)$.
If $\rc n(\ln n)^3$, we get another characterization. 

%bad because easy to misclassify
Proof: Assume $\Pj(Y_n\ge X_n)=\Om\prc{n}$. Every vertex has $\Om\prc{n}$ chance of being bad (most neighbors are other color). $\Om(1)$ chance some vertex will be bad. We can't expect to get bad vertices right. Think of this as a ``local-to-global" result in random graphs: 

\begin{thm}[Erd\H os, Renyi]
$\cal G(n,p_n)$ is connected a.a.s. iff it has no isolated nodes a.a.s.
\end{thm}

Our result can be read as: $\cal G(n,p_n,q_n)$ aas has no bad vertices iff minimal bisection is planted bisection.

An exact recovery algorithm: one can recover labels with $o(n)$ mistakes iff $\fc{n(p_n-q_n)^2}{p_n+q_n}$ (fraction you get wrong tends to 0). We take this as a black box.
%whenever our condition on binom holds, so does this.

Algorithm
\begin{enumerate}
\item
Remove a vertex, apply accurate algorithm to the rest. 
\item 
Put vertex back in
\item
Color based on neighbors.
\end{enumerate}
Actually remove some small constant fraction at a time for efficiency.

Properties of this algorithm: when is it getting things wrong? 
\begin{df}
A node is \vocab{mediocre} if it has a little more than half of its neighbrs (as a function of $p,q$) have the same color.
\end{df}
Only mediocre nodes are labelled wrongly. There are few $n^{\rc 4}$ mediocre nodes (calculate using binomial). 
If the graph is sparse, they form an independent set. If the graph is dense, every mediocre node has $2n^{\rc 4}$ more neighbors of its same color.

\begin{cor}
Every node's true color except possible most mediocre nodes is the majority of the adjacent node's colors (?).
\end{cor}

Part 2: when a node disagrees with the majority of its neighbors, recolor it.

This is almost linear time $n\poly\log(n)$. 
%spectrla algorithms. 