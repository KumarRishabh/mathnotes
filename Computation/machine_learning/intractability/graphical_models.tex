\def\filepath{C:/Users/Owner/Dropbox/Math/templates}

\input{\filepath/packages_article.tex}
\input{\filepath/theorems_with_boxes.tex}
\input{\filepath/macros.tex}
\input{\filepath/formatting.tex}
\newcommand{\cmt}[1]{(#1)}
%\input{\filepath/other.tex}

%\def\name{NAME}

%\input{\filepath/titlepage.tex}

\pagestyle{fancy}
%\addtolength{\headwidth}{\marginparsep} %these change header-rule width
%\addtolength{\headwidth}{\marginparwidth}
\lhead{Learning graphical models}
\chead{} 
\rhead{} 
\lfoot{} 
\cfoot{\thepage} 
\rfoot{} 
\renewcommand{\headrulewidth}{.3pt} 
%\renewcommand{\footrulewidth}{.3pt}
\setlength\voffset{0in}
\setlength\textheight{648pt}

\addbibresource{gm.bib}

\begin{document}
\tableofcontents

\section{Introduction}
We summarize Guy Bresler's paper \autocite{Bre14}, ``Efficiently learning Ising models on arbitrary graphs." This is meant to be a reading guide for the paper itself.
\subsection{Graphical models}
For a graph $G$ and vertex $u$, let $\pl u$ denote the neighbors in $u$.
\begin{df}
Let $G=(V,E)$ be a graph.
A \vocab{Markov model} on $G$ is a distribution on $\{-1,1\}^V$ such that 
\[
\Pj[x_u | x_{V\bs \{u\}}] = \Pj[x_u|x_{\pl u}],
\]
i.e., $x_u$ is independent of all other coordinates, given the values on the neighborhood of $u$. The neighbors ``shield" $x_u$ from the rest of the graph.
\end{df}

Recall that we can characterize all Markov models as follows. 
\begin{df}[Gibbs distribution]
%Let $G$ be a hypergraph with vertex set 
Let $V$ be the set of coordinates, and $\cal S$ be a family of subsets of $V$. Let $x\in \{0,1\}^V$. 
For each $S\in \cal S$, let $\phi_S$ be a function $\R^S \to \R$. A \vocab{Gibbs distribution} supported on $\cal S$ is a distribution in the form
\[
\Pj(x) \propto \exp\pa{\sum_{S\in \cal S} \phi_S(x)}.
\]
\end{df}

\begin{thm}[Clifford-Hammersley]
If $\Pj(x_1,\ldots, x_n)>0$ for every configuration $x=(x_1,\ldots, x_n)\in \{0,1\}^V$, then there exists a family of subsets $\cal S$, each set of which is a clique in the dependency graph $G$, and such that the distribution is a Gibbs distribution supported on $\cal S$. 

Conversely, a Gibbs distribution supported on $\cal S$ is a Markov model whose edges are the union of the cliques $S\subeq \cal S$.
\end{thm}

\subsection{Ising model}

To make learning easier, we focus on the special case of the Ising model.
\begin{df}
Let $G=(V,E)$ be a graph. An \vocab{Ising model} is a probability distribution on $\{-1,1\}^V$ where for $x\in \R^V$,
\beq{eq:gm-1}
\Pj_\te(x) = \exp\pa{\sum_{ij\in E}\te_{ij} x_ix_j + \sum_{i\in V} \te_i x_i - \Phi(\te)}
\eeq
where $\te\in \R^{E\cup V}$ %with $\te_{ij}=0$ when $ij\nin E$ and $\te_{ij}\ne 0$ when $ij\nin E$, and 
and $\Phi(\te)$ is a factor to make the probabilities sum to 1.
\end{df}
%(Here $\Pj_\te$ doesn't mean $\te$ is a random variable; it means the distribution depends on $\te$.)
An Ising model is a Gibbs distribution where the cliques are just edges, and moreover the potential functions $\phi_{\{i,j\}}$ only depend on whether $x_i=x_j$.

Note that 
\begin{enumerate}
\item
if $\te_{ij}>0$ it is favorable for $x_i=x_j$,
\item
if $\te_{ij}<0$ it is favorable for $x_i\ne x_j$,
\item
if $\te_i>0$ it is favorable for $x_i=1$,
\item
if $\te_i<0$ it is favorable for $x_i=-1$.
\end{enumerate}•

We are concerned with the following problem. 
\begin{prb*}
Given $m$ samples for the Ising model, learn the graph $G$ exactly with probability $1-\ep$.
\end{prb*}
Learning the graph exactly is called \vocab{structure learning}.

We can't hope to have a uniform bound over $\te$ because $\te_{ij}\ne 0$ could be arbitrarily close to 0, and the difference in probabilities would be too small to determine if an edge exists or not. Thus we need to make the assumption that the $\te_{ij}$ are bounded away from 0. Neither do we want them to be too large: if some probabilities are much greater than others, the small ones won't be noticed until a lot of samples are take. 
\begin{enumerate}
\item
Thus we define
\[
\Om_{\al,\be,\ga}(G)=\set{\te\in \R^{E\cup V}}{\forall ij\nin E, \te_{ij}=0,\quad \forall ij\in E, |\te_{ij}|\in [a,b],\quad \forall i\in V, |\te_i|\le h},
\]
and ask what the dependence of $n$ is on $\al,\be,h,\ep$.
\item
We also restrict to $G$ having maximum degree $\le d$. This is reasonable because in real life, each variable typically doesn't ``depend" on more than a few other variables directly.
%(I think this restriction is more information-theoretic---otherwise there are too many possible graphs).
\end{enumerate}
Here is the modified problem. 
\begin{prb}
Suppose $\te \in \Om_{\al,\be,\ga}(G)$ where $G$ has $n$ vertices and maximum degree $\le d$.
Given $m$ samples from the distribution given by~\eqref{eq:gm-1}, find $G$ with probability $1-\ep$ in a reasonable time.

What is the right dependence of $m$ and the running time on $\al,\be,h,\ep,d$? 
\end{prb}
Bresler's main theorem is the following. 
\begin{thm}
%$\de=\rc 2 e^{-2(\be d+h)}$.
Using 
\[
m=e^{O(\al^{-O(1)}e^{O(\be d^2 + h d)}\ln \fc n{\ep})} 
%\pf{\be^{O(1)}}{\al^{O(1)}\de^{\fc{\be^{O(1)}}{\al^{O(1)}}\de^{O(d)}}}
\]
samples, Algorithm~\ref{alg:gm-b} learns $G$ exactly with probability $\ge 1-\ep$.

The running time of the algorithm is $\wt O(n^2 m)$.
\end{thm}
We remark the following.
\begin{enumerate}
\item
The dependence on $n$ is $O(\ln n)$ samples and $\wt O(n^2)$ running time (where the $O$ hides a constant depending on $\al,\be,d,\ep$). The previous best algorithm that works for all $G,\te$ as in the hypothesis takes time $\Om(n^d)$. It goes as follows: to find the neighborhood of a node, guess all subsets of size $d$. For each guess $S$, verify whether $u$ is independent of all other nodes given the coordinates in $S$ (up to some error parameter).

Thus in the regime of constant $d$ and $n\to \iy$, this algorithm is better.
\item
However, the dependence on $d$ is doubly exponential. We'll talk about why this is later.
\item
The dependence on $d$ is at least exponential by information theory (the argument is involved; see \autocite{SW12}). Thus there is a gap between this algorithm and the lower bound.
\item
There have been algorithms that work under stronger assumptions, for example, the \vocab{correlation decay property} (\autocite{Gam13}) which says that the correlation between $i,j\in V$ is at most $\rh^{d(i,j)}$ for some $\rh<1$, where $d(i,j)$ is the graph distance between $i$ and $j$. This allows quadratic time because to find all neighbors, we can choose some $D$ so that every vertex farther than $D$ will have small correlation with $u$. This restricts us to at most $d^D$ vertices. Now run the $O(n^d)$ algorithm on these vertices to get a running time of $O(n(d^D)^{d+1}\ln n)$ per node and $\wt O(n^2)$ running time overall.
\end{enumerate}
We omit the constants; see the original paper. Bresler defines the chain of variables in Theorem 4.1:
\bal
\de & = \rc2 e^{-2(\be d+h)}\\
\tau^* &= O\pf{\al^2 \de^{O(d)}}{d\be}\\
\ep^* & = O(\tau^*)\\
\ell^* &= O\prc{(\tau^*)^2}\\
m &= O\pf{\ell^*}{(\ep^*)^2 \de^{\ell^*}\ln \pf p{\ep}};
\end{align*}
unraveling these gives the bound for $m$ in the theorem.

We present the algorithm.
\begin{enumerate}
\item
Key Idea: Define the influence of $i$ on $u$ conditioned on $S$, $\nu_{u|i,S}$.
\item 
Key Theorem: Every vertex $u$ has a neighbor with large influence on $u$ (when conditioned on any $S\nsupeq \pl u$). 
\end{enumerate}

\begin{alg}\llabel{alg:gm-b}
The following finds the neighborhood $\pl u$ of a vertex $u$. (To find the graph, repeat for all vertices.)

The algorithm has 2 parts. Given a threshold parameter $\tau$ (we set it to $\tau^*$),
\begin{enumerate}
\item
Psueudo-neighborhood: Find $S\supeq \pl u$ as follows.
\begin{enumerate}
\item
Let $S=\phi$.
\item Repeat: While there exists $i$ with large influence $\wh{\nu}_{u|i,S}^{\text{avg}}$, 
\begin{enumerate}
\item 
choose $i$ with the largest influence and set $S\mapsfrom S\cup \{i\}$.
%Let $(i^*,\eta^*)=(\amax_i \wh{\nu}_{u|i;S}^{\text{avg}}, \max_i \wh{\nu}_{u|i;S}^{\text{avg}})$.
%\item Repeat until $\eta^*<\tau$. If $\eta^*\ge \tau$, then add $i^*$ to $S$ and repeat.
\end{enumerate}
\end{enumerate}
\item Pruning
\begin{enumerate}
\item
While there exists $i$ with $\wh{\nu}_{u|i,S\bs \{i\}}<\tau^*$,
\begin{enumerate}
\item
Remove $i$ from $S$, $S\mapsfrom S\bs \{i\}$.
\end{enumerate}
\end{enumerate}
\end{enumerate}
Output $S$. Note we only get a superset at first because a vertex that is not a neighbor, or in fact far away in graph distance, could have large influence on $u$ (``long-range correlation"). This happens exactly because we're not assuming the correlation decay property. There could, for example, be many paths with positive $\te_{jk}$'s from $i$ to $u$.
\end{alg}
We go through the following steps in the analysis.
\begin{enumerate}
\item
Bound the conditional probability of $u+,u-$ from below in terms of $\al,\be,h$.
\item
Define the notion of influence. 
\item 
Show that there always exists a neighbor of $u$ which has large influence on $u$ (also true if we condition on any subset not containing $\pl u$). To do this, show an anti-concentration bound.
\item 
Argue that $|S|$ is bounded indpendent of $n$ (it would be bad if $S$ grows to contain all the vertices!).
\item Use Chernoff's inequality to find how large $n$ has to be so that the estimated probabilities/influences are close to the actual. This gives the number of samples we need.
\end{enumerate}
Where does the double exponential (in $d$) come from? 
\begin{enumerate}
\item The first exponential comes from the fact that in order to estimate $\nu_{u|i}$ we need to estimate $\Pj(x_S)$, which is bounded below by $\de^{-|S|}$. Thus we need $\de^{-O(|S|)}$ samples.
\item
The second exponential comes from the fact that the maximum size of the set $S$ is bounded in terms of $\tau^*$, so $|S|=\de^{-O(d)}$.
\end{enumerate}


%\section{Other models; Prior work}
%
%Correlation decay property.
%
%Ising vs. Markov model.
First we justify that pruning works. 
\begin{enumerate}
\item
If $i\in \pl u$, then the key theorem will tell us that $\nu_{u|i,S}^{\text{avg}}$ is large, so we won't remove it.
\item
If $i\nin \pl u$, then $i$ is independent of $u$ given $\pl u\subeq S$, and this implies $\nu_{u|i,S}^{\text{avg}}=0$, so we will remove it.
\end{enumerate}
We don't get access to the actual influences, only to the estimated influence (from the sample), but the algorithm will work fine as long as the estimated influences are close enough $\ep^*=\fc{\tau^*}{2}$. We'll address this in the last step.

\section{Bounding conditional probabilities from below}

We will often write indices as shorthand for coordinates; for example, $\Pj[u+|x_S]$ means $\Pj[X_u=+1|X_S=x_S]$.

\begin{lem}\llabel{lem:gm-p-not-too-small}
For any $u\in V, S\subeq V\bs \{u\}, x_S\in \{-1,1\}^S$, 
\[
\Pj[u\pm |x_S]\ge \de=\rc2e^{-2(\be d+h)}
\]
For a set $U$,
\[
\Pj[x_{U}|x_S]\ge \de^{|U|}.
\]
\end{lem}

\begin{proof}
We just show the lemma for a singleton; the case for general $\cal U$ is similar.
\beq{eq:gm-2}
\Pj[u+ | x_{V\bs u}] = \fc{e^{\sum_{i\in \pl u} \te_{ui} x_i +\te_u}}{e^{\sum_{i\in \pl u} \te_{ui} x_i +\te_u}+ e^{-\sum_{i\in \pl u} \te_{ui} x_i -\te_u}} =\fc{1}{1+e^{-2(\sum_{i\in \pl u} \te_{ui} x_i +\te_u)}}\ge \de.
\eeq
\end{proof}

\section{Influence}

First, a little motivation. The original definition of influence is as follows: for a boolean function $f:\{0,1\}^n\to \{0,1\}$, the influence is the expected change in $f$ if we flip the $k$th bit, and the total influence is the sum of influences.
\bal
I_k(f) & = \E_x |\De_kf| = \E_x |f(x+e_k)-f(x)|\\
I(f) & = \sum_k I_k(f).
\end{align*}
This is well-understood (ex. KKL theorem: there is a coordinate with influence $\succsim \fc{\ln n}{n}$, etc.).

The definition of influence is different here, though it also measures the effect of a bit flip. It measures how a conditional probability of $u+$ changes when $x_i$ is flipped. We take the expectation with respect to a different measure, as well.
\begin{df}
Define
\bal
\nu_{u|i;x_S} &= \Pj[u+|i+, x_S]-\Pj[u+|i-, x_S]\\
\la_i(x_S) &= 2\Pj[i+|x_S]\Pj[i-|x_S]\\
\nu_{u|i;x_S}^{\text{avg}} &= \E[|\nu_{u|i;X_S}|\la_i(X_S)].
\end{align*}
\end{df}
Let's focus on the special case where $S=\phi$. We will only sketch proofs when $S=\phi$ (i.e., the first step, when we're trying to find a single neighbor of $u$). The proofs basically carry over by changing the probabilities to probabilities conditioned on $x_S$. In this case, 
\bal
\nu_{u|i} &= \Pj[u+|i+]-\Pj[u+|i-]\\
\la_i &= 2\Pj[i+]\Pj[i-]\\
\nu_{u|i}^{\text{avg}} &= |\nu_{u|i;X_S}|\la_i(X_S).
\end{align*}
Why are the weights are given by $\la_i$?
\begin{enumerate}
\item
We shouldn't weight the summands in the $\E$ equally, because some some of the events are more likely than others. 
We discount the events where $\Pj[i+]$ or $\Pj[i-]$ is very small because we rarely get data from them in sampling. Think of $i$ as ``almost fixed given $x_S$." Thus those events shouldn't affect the influence much.

(Note the definition of $I_k$ is an $\E$ assuming the uniform distribution on $\{0,1\}^n$, which we don't have here.)
\item
When we're trying to sum the $\nu_{u|i}$ over neighbors, the factor $\la_i$ is exactly what we multiply by to give the sum a nice interpretation as an expected value we can bound using anticoncentration inequalities.
\end{enumerate}
The key theorem that makes the algorithm work is the following.
\begin{thm}\llabel{thm:gm-v-large}
For $\te\in \Om_{\al,\be,\ga}(G)$, $\pl u\nsubeq S$, there is $i\in \pl u \bs S$ with $\nu_{u|i;S}^{\text{avg}}\ge 2\tau^*$.
\end{thm}
For simplicity, we'll just sketch the proof when $S=\phi$, but the proof carries over by conditioning on $x_S$.

To show there is $i\in \pl u$ with $\nu_{u|i}$ large, we show there is that the sum over $i\in \pl u$ is large. Let $\cal U=N(u)$. We will prove the following bound:
\[
\sum_{i\in \cal U} \te_{ui} \ub{\la_i \nu_{u|i}}{\nu_{u|i}^{\text{avg}}} \ge \ve{\te_{u\cal U}}_12\tau^*.
\]
The fact that one summand is large will follow directly.

Why the weights $\te_{ui}$? Recall that $\te_{ui}>0$ means (roughly) that $x_i=x_u$ is more likely, so we would expect $\nu_{u|i}$ to be positive, and if $\te_{ui}<0$ we would expect $\nu_{u|i}$ to be negative. Thus we multiply by the $\te_{ui}$ to try to make each summand positive (on the average).

We now calculate.
\bal
\nu_{u|i} & = \Pj[u+|i+] - \Pj[u+|i-]\\
&= \sum_{x_{\cal U\bs i}} \Pj[u+|i+, x_{\cal U\bs i}]
\Pj[x_{\cal U\bs i}|i+] 
+
 \sum_{x_{\cal U\bs i}} \Pj[u+|i-, x_{\cal U\bs i}]
\Pj[x_{\cal U\bs i}|i-]\\
&= \sum_{x_{\cal U\bs i}} \Pj[u+|i+, x_{\cal U\bs i}]
\fc{\Pj[x_{\cal U\bs i},i+]}{\Pj[i+]}
+
\sum_{x_{\cal U\bs i}} \Pj[u+|i-, x_{\cal U\bs i}]
\fc{\Pj[x_{\cal U\bs i},i-]}{\Pj[i-]}\\
&= \sum_{x_{\cal U}} \Pj[u+| x_{\cal U}] \fc{x_i}{\Pj[x_i]} \Pj[x_{\cal Y}]
\end{align*}
At this point, we notice that $\fc{x_i}{\Pj[x_i]}$ is ugly because it has $\Pj[x_i]$ in the denominator. It would be much nicer to work with simply the random variable $x_i$:
\bal
\fc{x_i}{\Pj(x_i)}&\in \bc{
\rc{\Pj(i+)},\rc{\Pj(i-)}
}\\
x_i&\in \{-1,1\}.
\end{align*}
Define the average and half-difference of $\fc{x_i}{\Pj(x_i)}$:
\bal
s_i&= \rc2 \pa{\rc{\Pj(i+)} - \rc{\Pj(i-)}}\\
t_i&= \rc2 \pa{\rc{\Pj(i+)} +\rc{\Pj(i-)}}=\rc{\la_i}.
\end{align*}
Note that $\fc{s_i}{t_i}=\Pj(i-)-\Pj(i+) = -\E X_i$.
Subtracting the average and dividing by the half-difference gives a $\pm1$ random variable:
\bal
\rc{t_i}\pa{\fc{x_i}{\Pj(x_i)}-s_i}&=x_i\\
&\implies \la_i \fc{x_i}{\Pj(x_i)} &= x_i + \fc{s_i}{t_i} = x_i-\E x_i.
\end{align*}
This suggests multiplying by $\la_i$. Note $\Pj[u+|X_{\pl u}]$ is given by~\eqref{eq:gm-2}; $2\Pj[u+|X_{\pl u}]$ is given by $\tanh$ which is why we make the transformation below. 
\bal
\la_i\nu_{u|i} &=  \E_{X_{\pl u}} \Pj[u+|X_{\pl u}] (X_i-\E X_i)\\
& = \rc2 \E(2\Pj[u+|X_{\pl u}](X_i-\E X_i))\\
&=\rc2 \E \tanh Z (X_i-\E X_i)\\
&\quad\text{where } Z=\sum_{i\in \pl u}\te_{ui}X_i + \te_u = 
\pl_{(u,\pl u)}\cdot X_{\pl u}+\te_u\\
\sum_{i\in \pl u} \te_{ui} \la_i \nu_{u|i}&=
\rc2 (\tanh Z)\te_{(u,\pl u)}\cdot (X_{\pl u}-\E X_{\pl u})\\
&=\rc 2\E(\tanh Z)(Z-\mu)
\end{align*}
where $\mu=\E Z$ and $\te_{(u,\pl u)}$ consists of $\te_{ui}$ for $i\in \pl u$.
We want this to be bounded $>0$.

%Let $Z=X_{\cal U}\cdot \te_{u\cal U}+\te_u$. (Note the $\te_u$ is a constant and doesn't really matter; we can throw it out if we want to with minor modifications.) Let $\mu=\E Z$. Then the above becomes
%\[\sum_{i\in \cal U} \te_{ui} \la_i \nu_{u|i}
%= \E(\tanh Z)(Z-\mu).\]
The first thing to note is that for any random variable with $\E Z=\mu$, this quantity is $\ge 0$. This makes sense because $(\tanh x)(x-\mu)$ is convex near $\mu$, and positive except in between 0 and $\mu$. 

Thus we'll just need to show some positive amount of mass (bounded away from 0) is bounded away from $\mu$. To do this we'll need an anticoncentration inequality.
\section{Anti-concentration inequality}
First, let's pretend that $Z$ is the uniform random variable $U_d$ on $\{-1,1\}^d$. Then $Z=\te_{(u,\pl u)}\cdot U+\te_u$. The entries of $U$ are $\pm 1$ so we might think of $Z$ as like a binomial random variable, which we know has some tail. The two issues are (1) $Z$ is not uniform, and (2) using the right anticoncentration result.

For (1), Lemma~\ref{lem:gm-p-not-too-small} gives us that the probability of $\Pj(x_{\pl u})$ is bounded from below by $\de^d$ (not depending on $n$). Thus we can write
\[
X_{\pl u} = c U_{\pl u}+(1-c)W
\]
where $c=(2\de)^d$.

\begin{thm}[Littlewood-Offord, Erd\H os]
\llabel{thm:gm-loe}
Let $w_i$ be reals with $|w_i|\ge 1$. Let $I$ be an open interval of length $k$. Let $\xi\in \{-1,1\}^d$ be iid Bernoulli. Then 
\[
\Pj(w\cdot \xi\in I)\le \text{sum of $k$th largest binomial coefficients }\binom d{\bullet}.
\]
In particular, for $d\ge O(1)$,
\[
\Pj(w\cdot \xi \in (t_0-1,t_0+1)\le 2\rc{2^r}\binom{r}{\ff r2}\le \rc2
\]
(and actually is $\precsim \rc{\sqrt d}$).
\end{thm}
\begin{lem}[Lemma 7.2 rephrased]
Let $\vec \te$ be a vector of $d\ge O(1)$ variables with each entry $|\te_i|\ge \al$. Let $\vec Y$ be iid $\pm 1$. Consider the variable 
\[
Z=c(\vec\te \cdot Y) + (1-c)W
\]
where $W$ is any random variable. 
Let $\mu = \E Z$. Then
\[
\E[(\mu-Z) \one_{Z\le \mu-\al \fc{c}{2}}] \ge \fc{\al c^2}8.
\]
\end{lem}
\begin{proof}
For simplicity, WLOG $\mu=0$. Use Theorem~\ref{thm:gm-loe}. Let $m_1$ be the probabilities that $Z\le -\al\fc c2$, $Z\in (-\al \fc c2, (2-\fc c2)\al)$, and $Z\ge (2-\fc c2)\al$. We just minimmize the desired quantity given the constraints
\bal
m_1+m_2+m_3 & = 1\\
m_2 &\le 1-\fc c2\\
m_1\E[-Z|Z\le -\fc{\al c}2] + m_2(-\al\fc{c}2) + m_3(2-\fc c2)\al&\ge \mu,
\end{align*} 
and also noting the desired quantity is $\ge m_1 (\fc{\al c}2)$.
\end{proof}
Apply the lemma with $Y=X_{\pl u}$ and $c=\rc 2(2\de)^{d}$ to get %This is since each vector $x_{\cal U}$ has $\ge \de^{|\cal U|}$ chance, from Lemma~\ref{lem:gm-p-not-too-small}.
\[
\E((\mu-Z)\one_{Z\le \mu - \al \fc c2})\ge \fc{\al c^2}{8} = \Om(\al \de^{O(d)}).
\]
Doing the calculation with bounding $\tanh$ (See lemma 7.3 in the paper), we get the bound
\[
\E [(\tanh Z)(Z-\mu)] 
\ge \fc{\al^2\de^{4d+1}}8 \ge d\be 2\tau^*.
\]
which gives the key theorem~\ref{thm:gm-v-large}.
\section{$|S|$ is bounded}
We bound the influence by a KL divergence, and then by mutual information. We can bound the sum of mutual informations by 1 (chain rule), so this gives an upper bound on how many times we have to grow our set.

Using Pinskler's inequality relating the total variation distance and KL divergence,
\[
%\ve{P-Q}_1, 2
d_{TV}(P,Q)\le \sqrt{\rc 2D_{KL}(P||Q)}
\]
we have
\bal
\rc2 v_{u|i}^{\text{avg}}
& =\Pj(i+)\Pj(i-) |\Pj(u+|i+)-\Pj[u+|i-]|\\
& = |\Pj(u+,i+)(1-\Pj(i+)) - \Pj(u+,i-)\Pj(i+)|\\
&=|\Pj(u+,i+)-\Pj(u+)\Pj(i+)|\\
&=d_{TV}(\Pj(u,i), \Pj(u)\Pj(i))\\
%&\le \rc 2\ve{\Pj(u,i)-\Pj(u)\Pj(i)}_1\\
&\le \sqrt{\rc2 D_KL(\Pj(u,i),\Pj(u)\Pj(i))}\\
&=\sqrt {\rc2 I(u:i)}.
\end{align*}
Exactly the same analysis goes through when conditioning on $x_{S_l}$ (the first $l$ nodes added to $S$).
Now suppose that the estimated $\wh{\nu}_{u|i;S_l}^{\text{avg}}$ is at most $\ep$ away from the true average. For each $i$ that is added, we then have
\bal
\sqrt{\rc2 I(X_u:X_i|S_l)}&\ge \rc2 (\wh{\nu}_{u|i;S_l}^{\text{avg}} - \ep)\\
I(X_u:X_i|S_l)&\ge \rc2(\tau-\ep)^2.
\end{align*}
Now use the chain rule for mutual information to obtain
\[
1\ge H(X_u)\ge I(X_u|X_S) = 
\sum_{k=1}^{|S|} I(X_u:X_{j_k}|X_{j_1},\ldots, X_{j_{k-1}}) \ge |S|\rc2(\tau-\ep)^2
\]
so the set grows to size
\[
|S|\le \fc{2}{(\tau-\ep)^2}.
\]
%By Theorem~\ref{thm:gm-v-large}, $S$ will contain $\pl u$ as needed.
%Now actually, if $i\nin \pl u$, then $x_u$ is independent of $x_i$ given $x_{\pl u}$ (the neighbors of $u$ shield it):
%\[
%\nu_{u|i,S\bs i}=0.
%\]
%Thus as long as the error in the $\wh{\nu}$ is  these nodes will be pruned from the neighbors. If the $\wh{\nu}$ have error $\ep<\fc{\tau}2$ then we will prune all the non-neighbors and keep the neighbors.
\section{Estimating $\nu$}
This is straightforward using Chernoff's inequality and getting your $\delta$s and $\ep$s in order. Let's just perform a sanity check---estimating how many samples we need. 
\begin{lem}[Lemma 3.2 in \autocite{Bre14}]
Let $l \le \fc n4-2$. If the number of samples is $\ge \fc{144(l+3)}{\ep^2\de^{2l}} \ln \fc{n}{\ep'}$, then 
\[
\Pj(\forall u,i\in V, S\subeq V\bs\{u,i\}, |S|\le l, |\nu_{u|i;S}^{\text{avg}} - \wh{\nu}_{u|i;S}^{\text{avg}}|\le \ep)\ge 1-\ep'.
\]
\end{lem}
We have to union bound over all subsets of size at most $l$, but this is negligible since it introduces a log factor. The key part is the error parameter we want for $\wh{\Pj}(x_S)$ as $S$ ranges over all subsets with $|S|\le l+2$. If we allow $\wh{\Pj}(x_S)$ to have $\ep''$ error, then because terms of the form $\wh{\Pj}(x_S)$ appears in the denominator of the sum of $\nu$'s, we get that the $\nu$'s have error $O\pf{\ep''}{\de^l}$ as $\de^l\le \de^{|S|}\le \Pj(x_S)$. So we need 
\[
O\pf{\ep''}{\de^l} \le \ep,
\]
and on the order of $O\prc{(\ep'')^2} = O(\de^{l}) = O(\de^{|S|})$ samples, where $|S|=O\prc{(\tau^*)^2}$. 
%; see the definition of $\de,\ep^*,\ell^*$.
This gives the doubly exponential dependence on $d$.
\section{Open questions}
\begin{enumerate}
\item
Can we find an algorithm with running time exponential rather than doubly exponential?
\item 
Extension to Markovian models: 
What if we have alphabet size $>2$? Given $x\in \Si^V$, and given $f_{ij}:\Si^2\to \R$ with $f_{ij}$ constant for $ij\nin E$ and with range in between $\al$ and $\be$ otherwise, and $f_i:\Si\to \R$ with range $\le \ga$, find $G$.
\item
Can we extend to hypergraphs? For example, in the Ising model, what if we had $x_ix_jx_k$ terms? What about the Markov model, where we allow functions $f_{ijk}$?
\item
What is the optimal algorithm if we are only required to learn the distribution up to some statistical distance, rather than give the exact graph?
\item 
What if we assume a random graph? (I.e., we have a generative model for the graph.) Is the problem easy ``on average"?
\end{enumerate}


%\chapter{Introduction}


\section{Normed spaces}

\begin{df}
A \textbf{normed space} is a pair $(X,\ve{\cdot})$ where $X$ is a real or complex vector space and $\ved$ is a norm on $X$. Most of the time the choice of scalar field makes little difference; for convenience we'll use real scalars. A norm induces a metric: $d(x,y)=\ve{x-y}$. This induces a topology on $X$, called the \textbf{norm topology}. A \textbf{Banach space} is a complete normed space.
\end{df}
\begin{ex}
\begin{enumerate}
\item (sequences)
For $1\le p<\iy$, we have $\ell_p=\set{(x_n)\text{ scalar sequence}}{\sum_{n=1}^{\iy}|x_n|^p<\iy}$ with norm $\ve{x}_p=\pa{\sum_{n=1}^{\iy}|x_n|^p}^{\rc p}$. (Minkowski's inequality says that if $x,y\in \ell_p$ then $x+y\in \ell_p$, so $\ve{x+y}_p\le \ve{x}_p+\ve{y}_p$. Then $\ell_p$ is a Banach space.
\item (convergent sequences)
$\ell_{\iy}=\set{(x_n)\text{ scalar sequence}}{(x_n)\text{ is bounded}}$ with $\ve{x}_{\iy}=\sup_{n\in N}|x_n|$. Then $\ell_{\iy}$ is a Banach space.

$c_{00}=\set{(x_n)\text{ scalar sequence}}{\exists N\forall n>N, x_n=0}$. Let $e_n=(0,0,\ldots, 0,\ub{1}{n},0,\ldots)$; then $c_{00}=\spn\set{e_n}{n\in \N}$. Note that $c_{00}$ is a subspace of $\ell_{\iy}$ but it's not closed: In $\ell_p, 1\le p<\iy$, $\ell_p=\ol{\spn}\set{e_n}{n\in \N}$. 

$c_0=\set{(x_n)\in \ell_{\iy}}{\lim_{n\to \iy} x_n=0}$ is a closed subspace of $\ell_{\iy}$, $c_0=\ol{\spn}\set{e_n}{n\in \N}$ in $\ell_{\iy}$.

$c=\set{(x_n)\in \ell_{\iy}}{\lim_{n\to \iy} x_n \text{ exists}}$ is a closed subspace of $\ell_{\iy}$. $c_0$ and $c$ are Banach spaces.
\item (Euclidean space)
$\ell_p^n=(\R^n,\ved_p), 1\le p\le \iy$.
\item 
$K$ is any set, $\ell_{\iy}(K)=\set{f:K\to \R}{f\text{ is bounded}}$ with norm $\ve{f}_{\iy}=\sup_{x\in K}|f(x)|$. This is a Banach space, e.g. $\ell_{\iy}=\ell_\iy(\N)$.
\item 
$K$ compact topological space $C(K)=\set{f\in \ell_{\iy}(K)}{f\text{ continuous}}=\set{f:K\to \R}{f\text{ is continuous}}$. $C(K)$ is a closed subspace of $\ell_{\iy}(K)$ because any uniform limit of continuous functions is continuous, and hence it's a Banach space, e.g. $C[0,1]$.

We'll write $C^{\R}(K)$ and $C^{\C}(K)$ for the real and complex versions of $C(K)$, respectively.
\item 
Let $(\Om,\Si,\mu)$ be a measure space. Then for $1\le p<\iy$,  \[L_p(\mu)=\set{f:\Om\to \R}{f\text{ is measurable}, \int_{\Om}|f|^p\,d\mu<\iy}\] with norm $\ve{f}_p=\pa{\int_{\Om}|f|^p\,d\mu}^{\rc p}$ is a Banach space (after identifying functions that are equal almost everywhere. 

When $p=\iy$, $L_{\iy}(\mu)=\set{f:\Om\to \R}{f\text{ is measurable and essentially bounded}}$. (``Essentially bounded" means that there exists a null-set $N$ such that $f$ is bounded on $\Om\bs N$.)
\[
\ve{f}_{\iy}=\ess\sup|f|=\inf_N\sup_{\Om\bs N}|f|.
\]
\item
Hilbert spaces, e.g. $\ell_2$, $L_2(\mu)$. All Hilbert spaces are isomorphic, but some different representation may be more natural.
\end{enumerate}
\end{ex}

\begin{pr}
Let $X,Y$ be normed spaces, $T:X\to Y$ linear. Then the following are equivalent.
\begin{enumerate}
\item
$T$ is continuous.
\item
$T$ is bounded: $\exists C\ge 0$, $\ve{Tx}\le C\ve{x}$ for all $x\in X$.
\end{enumerate}
\end{pr}
\begin{proof}
To think about continuity at $a$, ``translate" to 0 using linearity.
\end{proof}

\begin{df}
Let $\cal B(X,Y)=\set{T:X\to Y}{T\text{ is linear and bounded}}$. This is a normed space with the \textbf{operator norm}: $\ve{T}=\sup\set{\ve{Tx}}{\ve{x}\le 1}$. $T$ is an \textbf{isomorphism} if $T$ is a linear bijection whose inverse is also continuous. (This is equivalent to $T$ being a linear bijection and there existing $a>0,b>0$, with $a\ve{x}\le \ve{Tx}\le b\ve{x}$ for all $x\in X$.)

If there exists such $T$, we say $X,Y$ are \text{isomorphic} and we write $X\sim Y$.

If $T:X\to Y$ is a linear bijection such that $\ve{Tx}=\ve{x}$ for all $x\in X$ (i.e. $a=b=1$), then $T$ is an \textbf{isometric isomorphism} and we say $X,Y$ are isometrically isomorphic and write $X\cong Y$\footnote{Some people use $\cong$ for isomorphism. We use it to mean isometric isomorphism.}.

$T:X\to Y$ is an \textbf{isomorphic embedding} if $T:X\to TX$ is an isomorphism. We write $X\hra Y$.
\end{df}
\begin{pr}
If $Y$ is complete, then $\cal B(X,Y)$ is complete. In particular, $X^*=\cal B(X,\R)$, the space of bounded linear functionals, called the \textbf{dual space} of $X$, is always complete.
\end{pr}
\begin{ex}
\begin{enumerate}
\item
For $1<p<\iy$, then $\ell^*\cong \ell_q$ where $\rc p+\rc q=1$. The proof uses H\"older's inequality: $x=(x_n)\in \ell_p$, $y=(y_n)\in \ell_q$ then $\sum|x_ny_n|\le \ve{x}_p\ve{y}_q$. This isomorphism is $\ph:\ell_q\to \ell_p^*,y\mapsto \ph_y, \ph_y(x)=\sum x_ny_n$.)
\item
$c_0^*\cong \ell_1$, $\ell_1^*\cong \ell_{\iy}$. 
(Later we will see that $c_0$ cannot be a dual space.)
%(This gives an alternate proof of completeness.)
\item 
If $H$ is a Hilbert space then $H^*\cong H$ (Riesz Representation Theorem).
\item
If $(\Om,\Si,\mu)$ is a measure space, $1<p<\iy$, then $L_p(\mu)^*\cong L_q(\mu)$ where $\rc{p}+\rc{q}=1$.

If $\mu$ is $\si$-finite then $L_1(\mu)^*\cong L_{\iy}(\mu)$.
(Else we only have $L_{\iy}(\mu)\hra L_1(\mu)^*$.)
\end{enumerate}
\end{ex}

{\color{blue}Lecture 2}

Recall that if $V$ is a finite-dimensional vector space, then any two norms on $V$ are equivalent. Specifically, if $\ved$ and $\ved'$ are two norms on $V$, then there exist $a,b>0$ such that 
\[
a\ve{x}\le\ve{x}'\le b\ve{x}\forall x\in V.
\]
In other words, $\Id:(V,\ved)\to (V,\ved')$ is an isomorphism. 

Some consequences are the following.
\begin{cor}
\begin{enumerate}
\item
If $X,Y$ are normed spaces, $\dim X<\iy$, $T:X\to Y$ is linear, then $T$ is bounded.
\item
If $\dim X<\iy$ then $X$ is complete. 
\item If $X$ is a normed space and $E$ a subspace with $\dim E<\iy$, then $E$ is closed.
\end{enumerate}
\end{cor}
\begin{proof}
\begin{enumerate}
\item
Set $\ve{x}'=\ve{x}+\ve{Tx}$. This is a norm on $X$, so there exists $b>0$, $\ve{x}'\le b\ve{x}$ for all $x$, so $\ve{Tx}\le b\ve{x}$ for all $x\in X$. 

%If $\dim X=\dim Y<\iy$ then $X\sim Y$.
\item
By (1), $X\sim \ell_2^n$ where $n=\dim X$.
\end{enumerate}•
\end{proof}
\section{Riesz's lemma and applications}
The unit ball is compact and this characterizes finite-dimensionality. We use the following.
\begin{lem}[Riesz's Lemma]
\llabel{lem:riesz}
Let $Y$ be a proper closed subspace of a normed space $X$. Then for every $\ep>0$ there exists $x\in X$, $\ve{x}=1$, such that $d(x,Y):=\inf_{y\in Y}\ve{x-y}>1-\ep$.
%null space, don't have notion as euclidean space, suggests proof that works. Intuition about euclidean space sometimes dangerous because may not work in normed space but in this case works
\end{lem}
We'd like to take the some sort of ``perpendicular" vector to $Y$, or the vector which minimizes the distance from a point not on $Y$ to $Y$. Note this is not in general possible since $X$ may not be complete, and hence the ``$\inf$." However, we can come arbitrarily close to that $\inf$, and get an ``almost perpendicular" vector.
\begin{proof}
Pick $z\in X\bs Y$ with $Y$ proper. Since $Y$ is closed, $d(z,Y)>0$. There exists $y\in Y$ such that $\ve{z-y}<\fc{d(z,Y)}{1-\ep}$ (WLOG $\ep<1$).

Set $x=\fc{z-y}{\ve{z-y}}$. Then
\[
d(x,Y)=d\pa{\fc{z-y}{\ve{z-y}},Y}=\rc{\ve{z-y}}d(z-y,Y)=\fc{d(z,Y)}{\ve{z-y}}> 1-\ep.
\]
\end{proof}
We give two applications. First, some notation. In a metrix space $(M,d)$, write 
\[
B(x,r):=\set{y\in M}{d(x,y)\le r},x\in M,r\ge 0
\]
for the closed ball of radius $r$ at $x$. In a normed space, 
\[
B_X:=B(0,1)=\set{x\in X}{\ve{x}\le 1}, \qquad B(x,r)=x+rB_X.
\]
Also, $S_X=\set{x\in X}{\ve{x}=1}$. 
\begin{thm}
Let $X$ be a normed space. Then $\dim X<\iy$ iff $B_X$ is compact.
\end{thm}
\begin{proof}
``$\Rightarrow$" We have $X\sim \ell_2^n$ where $n=\dim X$.

``$\Leftarrow$" By compactness there exist $x_1,\ldots, x_n\in B_X$ such that $B_X\subeq \bigcup_{i=1}^n B(x_i,\rc2)$.  Let $Y=\spn\{x_1,\ldots, x_n\}$. For all $x\in B_X$ there exists $y\in Y$ with $\ve{x-y}\le \rc 2$, so $d(x,Y)\le \rc 2$. Thus there do not exist ``almost orthogonal vectors" in the sense of  Riesz's Lemma~\ref{lem:riesz}. This means $Y$ is not a proper subspace of $X$, so $X$ is finite-dimensional. %$Y$ is dense in $X$, so $X=\ol Y=Y$.
%(We can't find an orthogonal vector to $Y$ in the sense of Riesz's Lemma.)
\end{proof}
\begin{rem}
We showed the following in the proof: If $Y$ is a subspace of a normed space $X$ and there exists $0\le \de<1$ such that for all $x\in B_X$ there exists $y\in Y$ with $\ve{x-y}\le \de$, then $Y$ is dense in $X$.
\end{rem}
If we let $\de= 1$ then this statement is trivial. The remark says that if we can do a little better than 1, the trivial estimate, then we can automatically approximate $x$ with much smaller $\ep$.

\begin{thm}[Stone-Weierstrass Theorem]
Let $K$ be a compact topological space and $A$ be a subalgebra of $C^{\R}(K)$. If $A$ separates the points of $K$ (i.e., for all $x\ne y$ in $K$, there exists $f\in A$, $f(x)\ne f(y)$), and $A$ contains the constant functions, then $A$ is dense in $C^{\R}(K)$.
\end{thm}
In this case it does matter whether the field of scalars is $\R$ or $\C$.

The following proof is due to T. J. Ransford.
\begin{proof}
First we show that if $E,F$ are disjoint closed subsets of $K$, then there exists $f\in A$ such that $-\rc2\le f\le \rc2$ on $K$ and $f\le -\rc 4$ on $E$ and $f\ge \rc 4$ on $F$.

Fix $x\in E$. Then for all $y\in F$, there exists $h\in A$ such that $h(x)=0$, $h(y)>0$, $h\ge 0$ on $K$. (This is since $A$ separates points, we can shift by a constant, and square the function.) Then there is an open neighborhood of $y$ on which $h>0$.  %finitely many cover $F$ by compactness.
An easy compactness argument gives that there exists $g=g_x\in A$ with $g(x)=0$, $g>0$ on $F$, 
%strictly positive on neighborhood, going to be strictly positive on all
%rescale it so that  
$0\le g\le 1$ on $K$. 
Pick $R=R_x\in \N$ such that $g>\fc2R$ on $F$, set $U=U_x=\set{y\in K}{g(y)<\rc{2R}}$. 
%compact, attain inf

Do this for all $x\in E$.
%finitely many will cover.
Compactness gives a finite cover: there exist $x_1,\ldots, x_m$ such that $E\subeq \bigcup_{i=1}^m U_{x_i}$. To simplify notation, set $g_i=g_{x_i}$, $R_i=R_{x_i}$, $U_i=U_{x_i}$, and $i=1,\ldots, m$. For $n\in \N$,  by Bernoulli's inequality,
\begin{align*}
\text{on }U_i&&
(1-g_i^n)^{R_i^n}&\ge 1-(g_iR_i)^n>1-2^{-n}\to 1\text{ as }n\to \iy\\
\text{on }F&&
(1-g_i^n)^{R_i^n}&\le \rc{(1+g_i^n)^{R_i^n}}\le \rc{(g_iR_i)^n}<\rc{2^n}\to 0\text{ as }n\to \iy
\end{align*}
There exists $n_i\in \N$ such that $h_i=1-(1-g_i^{n_i})^{R_i^{n_i}}$ satisties
\begin{itemize}
\item
on $U_i$, $h_i\le \rc 4$
\item
on $F$, $h_i\ge \pf 34^{\rc m}$
\item
on $K$, $0\le h_i\le 1$.
\end{itemize}
Set $h=h_1h_2\cdots h_m$. Then $h\le \rc 4$ on $E$, $h\ge \fc 34$ on $F$, and $0\le h\le 1$ on $K$. Set $f=h-\rc2$.
%corresp is at most a quarter, other at most 1, so at most 1/4
Given $g\in C^{\R}(K)$, $\ve{g}_{\iy}\le 1$, set
\[
E=\set{x\in K}{g(x)\le -\rc 4},\,F=\set{x\in K}{g(x)\ge \rc 4}.
\]
Let $f\in A$ be as above. Then $\ve{f-g}\le \fc 34$, i.e., $d(g,A)\le \fc 34$. By Riesz's Lemma~\ref{lem:riesz}, $A$ is dense in $C^{\R}(K)$.
\end{proof}
%reading math is a sort of computation, info decompression
\begin{rem}
The complex version says that if $A$ is a subalgebra of $C^{\C}(K)$ that separates points of $K$ contains the constant functions, and is closed under complex conjugation ($f\in A\implies \ol f\in A$), then $A$ is dense in $C^{\C}(K)$.
\end{rem}
\section{Open mapping lemma}
We'll assume the Baire category theorem and its consequences: principle of uniform boundedness, open mapping theorem (OMT), closed graph theorem (CGT). 
\begin{df}
Let $A,B$ be subsets of a metric space $(M,d)$ and let $\de\ge 0$. Say $A$ is \textbf{$\de$-dense} in $B$ if for all $b\in B$ there exists $a\in A$ with $d(a,b)\le \de$.
\end{df}
\begin{lem}[Open mapping lemma]\llabel{lem:oml}
Let $X,Y$ be normed spaces, $X$ complete, $T\in \cal B(X,Y)$. Assume for some $M\ge 0$ and $0\le \de<1$ that $T(MB_X)$ is $\de$-dense in $B_Y$. Then $T$ is surjective. 
More precisely, 
%important: have quantitative
for all $y\in Y$ there exists $x\in X$ such that $y=Tx$ and
\[
\ve x\le \fc{M}{1-\de}\ve y,
\]
i.e.,
\[
T\pa{\fc{M}{1-\de}B_X}\supeq B_Y.
\]
Moreover, $Y$ is complete.
\end{lem}
\begin{proof}
The proof involves successive approximations. 
Let $y\in B_Y$. There exists $x_1\in MB_X$ with $\ve{y-Tx_i}\le \de$. Then $\fc{y-Tx_i}{\de}\in B_Y$. There exists $x_2\in MB_X$, with $\ve{\fc{y-Tx_i}{\de}-Tx_2}\le \de$, i.e., $\ve{y-Tx_1-\de Tx_2}\le \de^2$, and so forth. Obtain $(x_n)$ in $MB_X$ such that 
\[
\ve{y-Tx_1-\de Tx_2-\cdots -\de^{n-1} Tx_n}\le \de^n
\]
for all $n$. Set $x=\sum_{n=1}^{\iy} \de^{n-1} x_n$. This converges since $\sum_{n=1}^{\iy}\ve{\de^{n-1}x_n}\le M\sum_{n=1}^{\iy}\de^{n-1}=\fc{M}{1-\de}$, and $X$ is complete.\footnote{This kind of geometric sum argument comes up a lot in functional analysis!} 
So $x\in \fc{M}{1-\de}B_X$ and  by continuity $Tx=\suo \de^{n-1} Tx_n=y$. For the ``moreover" part, let $\hat Y$ be the completion of $Y$, and view 
%unique banach space of which it is a dense subspace
$T$ as a map $X\to \hat Y$. Since $B_Y$ is dense in $B_{\hat Y}$, $T(MB_X)$ is $\de'$-dense in $B_{\hat Y}$ for $\de<\de'<1$. By the first part, $T(X)=\hat Y=Y$, so $Y$ is complete.
\end{proof}
\begin{rem}
Suppose $T\in\cal B(X,Y)$, $X$ is complete, and the image of the ball is dense: $\ol{T(B_X)}\supeq B_Y$. Suppose that for all $\ep>0$, $T((1+\ep)B_X)$ is $1$-dense in $B_Y$. Take $M>1,0< \de<1$ so that $1+\ep=\fc{M}{1-\de}$;  lemma~\ref{lem:oml} shows that  $T\pa{(1+\ep)B_X}\supeq B_Y$. It follows that $T(B_X^{\circ})\supeq B_Y^{\circ}$. (For a subset $A$ of a topological space, $A^{\circ}$ or $\text{int}(A)$ denotes the interior of $A$.)
\end{rem}

{\color{blue} Lecture 3}

\subsection{Applications of the open mapping lemma}
\begin{thm}[Open mapping theorem]\llabel{thm:omt}
Let $X,Y$ be Banach spaces, $T\in \cal B(X,Y)$ be onto. Then $T$ is an open map.
\end{thm}
\begin{proof}
Let $Y=T(X)=\bigcup_{n=1}^{\iy} T(nB_X)$. The Baire category theorem tells us that there exists $N$ with $\text{int}(\ol{T(NB_X)})\ne \phi$. Then there exists $r>0$ with
\[
\ol{T(NB_X)}\supeq rB_Y.
\]
By Lemma~\ref{lem:oml}, for $M=\fc{2N}{r}$, we have $T(MB_X)\supeq B_Y$. Therefore, $U$ is open and $T(U)$ is open.
\end{proof}
%say T is open, same as T^{-1} cont.
\begin{thm}[Banach isomorphism theorem]\llabel{thm-bit}
If in addition $T$ is injective, then $T^{-1}$ is continuous.
\end{thm}
\begin{proof}
An open map that is a bijection is a homeomorphism.
\end{proof}
\begin{thm}[Closed graph theorem]
Let $X,Y$ be Banach spaces and $T:X\to Y$ be linear. Assume that whenever $x_n\to 0$ in $X$, $Tx_n\to y$ in $Y$, then $y=0$. Then $T$ is continuous.
\end{thm}
This is a powerful result. Usually we have to show the sequence converges and show it converges to 0. This says that we only have to check the second part given the first part. %once we show it converges, it automatically converges to 0.
\begin{proof}
The assumption says that the graph of $T$
\[
\Ga(T)=\set{(x,Tx)}{x\in X}
\]
is closed in $X\opl Y=\set{(x,y)}{x\in X,y\in Y}$ with norm, e.g.
\[
\ve{(x,y)}=\ve{x}+\ve y.
\]
So $\Ga(T)$ is a Banach space. Consider $U:\Ga(T)\to X$, $U(x,y)=x$. $U$ is a linear bijection and $\ve{U}\le 1$. From the Banach isomorphism theorem~\ref{thm-bit}, $U^{-1}$ is continuous, i.e., $x\mapsto (x,Tx)$ is continuous.
\end{proof}
We'll give three more applications. The first one is to quotient spaces. Let $X$ be a normed space and $Y$ be a closed subspace. Then $X/Y=\set{x+Y}{x\in X}$ is a normed space with 
\[
\ve{x+Y}=d(x,Y)=d(0,x+Y)=\inf\set{\ve{x+y}}{y\in Y}.
\]
We need $Y$ closed to ensure that if $\ve{z}=0$ then $z=0$ for $z\in X/Y$.
\begin{pr}\llabel{pr:5}%5
Let $X,Y$ be as above. If $X$ is complete, then so is $X/Y$.
\end{pr}
%this can be proved directly too.
\begin{proof}
Consider the quotient map $q:X\to X/Y$, $q(x)=x+Y$. This is a bounded linear map, $q\in \cal B(X,X/Y)$, so
\[
\ve{q(x)}=d(x,Y)\le \ve{x}
\]
and $q(B_X^{\circ})\subeq B_{X/Y}^{\circ}$. If $\ve{x+Y}<1$ then there exists $y\in Y$ with $\ve{x+y}<1$ and $q(x+y)=q(x)=x+Y$ so $q(B_X^{\circ})=B_{X/Y}^{\circ}$. So for any $M>1$, $\ol{q(MB_X)}\supeq B_{X/Y}$. %image of some large ball contains ball. 
By Lemma~\ref{lem:oml}, $X/Y$ is complete.
\end{proof}
\begin{pr}
Every separable Banach space $X$ is (isometrically isomorphic to) a quotient of $\ell_1$.
\end{pr}
\begin{proof}
Let $\set{x_n}{n\in \N}$ be dense in $B_X$. Let $(e_n)$ be the standard basis of $\ell_1$. If $a=(a_n)\in \ell_1$, then $a=\sum_{n=1}^{\iy} a_ne_n$. Define
\[
T:\ell_1\to X,\qquad T\pa{\sum_{n=1}^{\iy} a_ne_n}=\sum_{n=1}^{\iy}a_nx_n.
\]
This is well defined because the sum converges: $\sum_{n=1}^{\iy} \ve{a_nx_n}\le \suo |a_n|=\ve{a}$. So $T\in \cal B(\ell_1,X)$, $\ve{T}\le 1$. Also $T(B_{\ell_1}^{\circ})\subeq B_X^{\circ}$. 
We have $T(B_{\ell_1})\supeq \set{x_n}{n\in \N}$, so $\ol{T(B_{\ell_1})}\supeq B_X$.

%checkme
%This means given $x\in B_X^{\circ}$, there exists a sequence $rx_n\to rx\in B_X$, $rx_n\in  $\ve{rx}=1$. Then $x_n\to x\in B_X^{\circ}$. 
So $T(B_{\ell_1}^{\circ})=B_X^{\circ}$. We have a unique $\wt T:\ell_1/\ker T\to X$ such that
\[
\xymatrix{
\ell_1\ar[rr]^T\ar[rd]_q & & X\\
& \ell_1/\ker T\ar[ru]_{\wt T}&
}
\]
commutes, i.e., $T=\wt Tq$ where $q$ is the quotient map.
%onto from lemma~\ref{lem:oml}
Moreover, $\wt T$ is a linear bijection
\[
\wt T(B_{\ell_1/\ker T}^{\circ})=\wt T(q(B_{\ell_1}^{\circ}))=T(B_{\ell_1}^{\circ})=B_X^{\circ}.
\]
So $\wt T$ is an isometric isomorphism, and $X\cong \ell_1/\ker T$.
\end{proof}
This suggests that to understand all Banach spaces we just have to understand $\ell_1$. But $\ell_1$ is quite complicated, not as innocent as it looks.

Recall the following definition.
\begin{df}
A topological space $K$ is \textbf{normal} if whenever $E,F$ are disjoint closed sets in $K$, there are disjoint open sets $U,V$ in $K$ such that $E\subeq U$, $F\subeq V$. 
\end{df}
\begin{lem}[Urysohn's lemma]\llabel{lem:urysohn}
If $K$ is normal and $E,F$ are disjoint closed subsets of $K$, then $\exists$ continuous $f:K\to [0,1]$ such that $f=0$ on $E$ and $f=1$ on $F$.
\end{lem}
This can be used to construct partitions of unity which we'll use in chapter 3.
\begin{thm}[Tietze's Extension Theorem]
If $K$ is normal and $L$ is a closed subset of $K$, $g:L\to \R$ is bounded and continuous, then there exists a bounded, continuous $f:K\to \R$ such that $f|_L=g$, $\ve{f}_{\iy}=\ve{g}_{\iy}$. 
\end{thm}
\begin{proof}
Let $C_b(K)=\set{h:K\to \R}{h\text{ bounded, continuous}}$, a closed subspace of $\ell_{\iy}(K)$ in $\ved_{\iy}$ (so it is a Banach space). Consider $R:C_b(K)\to C_b(L)$, $R(f)=f|_L$. We need $R(B_{C_b(K)})=B_{C_b(L)}$ (``$\subeq$" is clear, since $\ve{R}\le 1$.) Let $g\in B_{C_b(L)}$ so $-1\le g\le 1$. Let $E=\set{y\in L}{g(y)\le -\rc{3}}$, $F=\set{y\in L}{g(y)\ge \rc 3}$. Urysohn's lemma gives $\exists f\in C_b(K)$ such that $-\rc 3\le f\le \rc 3$, $f=-\rc 3$ on $E$, and $f=\rc 3$ on $F$.

We have
\[
\ve{R(f)-g}_{\iy} \le \fc23 
\]
and $\ve f\le \rc 3$. So $R(\rc 3B_{C_b(K)})$ is $\fc 23$-dense in $B_{C_b(L)}$. By the Open Mapping Lemma~\ref{lem:oml}, $R$ is surjective.
\end{proof}
\begin{rem}
The theorem holds in the complex case too.
\end{rem}

%%%%%%%%%%%%%%%%%%%%%%%%%%%


\printbibliography
%\bibliographystyle{halpha}
%\bibliography{gm}
\end{document}