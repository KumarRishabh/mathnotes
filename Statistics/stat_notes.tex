\def\filepath{C:/Users/Owner/Dropbox/Math/templates}

\input{\filepath/packages_article.tex}
\input{\filepath/theorems_with_boxes.tex}
\input{\filepath/macros.tex}
\input{\filepath/formatting.tex}
%\input{\filepath/other.tex}

%\def\name{NAME}

%\input{\filepath/titlepage.tex}

\pagestyle{fancy}
%\addtolength{\headwidth}{\marginparsep} %these change header-rule width
%\addtolength{\headwidth}{\marginparwidth}
\lhead{Statistics}
\chead{} 
\rhead{} 
\lfoot{} 
\cfoot{\thepage} 
\rfoot{} 
\renewcommand{\headrulewidth}{.3pt} 
%\renewcommand{\footrulewidth}{.3pt}
\setlength\voffset{0in}
\setlength\textheight{648pt}

\begin{document}
\tableofcontents
\section{Introduction}
\subsection{Probability distributions}

\begin{tabular}{|c|c|c|c|c|c|}
\hline 
Name & Distribution & Mean & Variance & MGF & Notes\tabularnewline
\hline 
Normal & \textbf{$\frac{1}{\sqrt{2\pi}}e^{-\frac{(x-\mu)^{2}}{2\sigma^{2}}}$} & $\mu$ & $\sigma^{2}$ &  & \tabularnewline
\hline 
Exponential & $\lambda e^{-\lambda x}$ & $\frac{1}{\lambda}$ & $\frac{1}{\lambda^{2}}$ &  & \tabularnewline
\hline 
Gamma & $\frac{\beta^{\alpha}}{\Gamma(\alpha)}x^{\alpha-1}e^{-\beta x}$ & $\frac{\alpha}{\beta}$ & $\frac{\alpha}{\beta^{2}}$ & $(1-\frac{t}{\beta})^{-\alpha}$ & \tabularnewline
\hline 
Beta & $\frac{\Ga(\al)+\Ga(\be)}{\Ga(\al+\be)}{x^{\alpha-1}(1-x)^{\beta-1}}$ & $\frac{\alpha}{\alpha+\beta}$ & $\frac{\alpha\beta}{\alpha+\beta-2}$ &  & \tabularnewline
\hline 
Binomial & $\binom{n}{k}p^{k}(1-p)^{n-k}$ & $pn$ & ${p(1-p)n}$ &  & $\to N(pn,{p(1-p)n})$\tabularnewline
\hline 
Poisson & $\frac{\lambda^{k}e^{-\lambda}}{k!}$ & $\lambda$ & $\lambda$ &  & \tabularnewline
\hline 
 &  &  &  &  & \tabularnewline
\hline 
\end{tabular}

The moment generating function is defined as $\E(e^{tX})=\int e^tx\,d\mu$. Here is a sketch of uniqueness.
\begin{thm}[Billingsley, Theorem 30.1]
\begin{enumerate}
\item
Let $\mu$ be a probability measure
 with moments of all orders:
\[
\al_k=\int_{-\iy}^{\iy} x^k\,d\mu.
\]
If $\sum_{k\ge 0}\frac{\al_kr^k}{k!}$ has a positive radius of convergence, then $\mu$ is the only probability measure with those moments.
\item If the MGF of $\mu$ is $g$ and $g$ is analytic at 0, then $\mu$ is the unique probability measure with MGF $g$.
\end{enumerate}
\end{thm}
\begin{proof}[Proof sketch]
To see (1) from (2), note $g^{(k)}(0)=\al_k$.

\begin{enumerate}
\item We have uniqueness of characteristic functions (if we took $\E(e^{itX})$ instead) basically by the theorem for Fourier transforms.
\item
Let $\be_k = \int_{-\iy}^{\iy} |x|^k\,d\mu$ and suppose the series converges at radius $r$. Show that $\frac{\be_kr^k}{k!}=0$.
\item 
We want to show the moments determine the characteristic function. Letting $\ph$ be the characteristic function $\ph$ of $\mu$, we would like it to be true that
\[
\ph(t+h)=
\sum_{k=0}^\iy \frac{h^k}{k!}\int_{-\iy}^{\iy} (ix)^k e^{itx}\,d\mu ,\qquad |h|\le r.
\]
as the sum comes from just expanding $e^{itx}$ in power series. Show the partial sums converge by bounding by $\be_{n}$'s.

For $\ph^{(k)}(0)$ give the moments, and we get uniquenes son an interval $(-r,r)$. Repeat near the boundary of the interval. 
\end{enumerate}
\end{proof}

Like characteristic functions (Fourier transforms), MGF's turn convolution (addition of random variables) into multiplication.

Some notes about the distributions.
\begin{enumerate}
\item The normal distribution is ubiquitous because by the central limit theorem, the sum of many (nice) iid random variables is close to normal.
\item The exponential distribution is memoryless.
\item The sum of gamma random variables $(\al_n,\be)$ is $(\sum\al_n,\be)$ because MGF's turn addition of rv's into multiplication.
\item The Poisson distribution can be derived from the following axioms. It is the probability that $k$ events occur in an interval of length 1 when
\begin{enumerate}
\item
events occurring in disjoint intervals are independent and 
\item 
equal for intervals of the same length.
\item
the probability of an event occurring $I$ of length $\to 0$ goes $\to 0$.
\end{enumerate}
The probability of $k$ events in an interval of length 1 is then 
\[
\lim_{n\to \iy}\binom nk \pf{\la}{n}^k\pa{1-\frac{\la}{n}}^{n-k}= \frac{\la^k e^{-n}}{k!}.
\]
\end{enumerate}

\subsection{Statistical distributions}

Here are distributions common in statistics. They'll be explained later.

\begin{tabular}{|c|c|c|c|}
\hline 
Name & is distribution of... & Notes & \tabularnewline
\hline 
\textbf{$\chi_{n}^{2}=\Ga(\fc k2,\rc2)$} & $\sum_{i}Z_{i}^{2},\, Z_{i}\sim N(0,1)$ &  & \tabularnewline
\hline 
$t_{n}$ & $\frac{X}{\sqrt{\rc nS}},\,X\sim  N(0,1), \,S\sim \chi_{n}^{2}$ &  & \tabularnewline
\hline 
$F_{m,n}$ & $\frac{A/m}{B/n},\,A\sim \chi_{m}^{2},\,B \sim\chi_{m}^{2}$ &  & \tabularnewline
\hline 
 &  &  & \tabularnewline
\hline 
 &  &  & \tabularnewline
\hline 
\end{tabular}

\subsection{Multidimensional normal distribution}

For $A\in \Mat_{n\times n}(\R)$, if $x$ is iid standard normal, then $Ax$ is defined to have distribution $N(0,AA^T)$. Note that if the covariance of $x$ is $D$ ($D=I_n$ here), then the covariance of $Ax$ is $ADA^T$, so the covariance.

Why is this well-defined? (I.e., if $AA^T=BB^T$ then $Ax,Bx$ are identically distributed.) Assume $A$ is full rank (else we need to restrict to a subspace) We calculate the distribution. By change of variables it's
\[
\rc{(2\pi)^{\fc n2}|\det A|}e^{-\frac{|A^{-1}x|^2}{2}}\,dx = 
\rc{(2\pi)^{\fc n2}|\det A|}e^{-\frac{x^T(AA^T)^{-1}x}{2}}\,dx.
\]

Define the shifted distribution $N(\mu,A),\mu\in \R^n$ in the obvious way.

\subsection{Statistics}

In statistics, we want to estimate some function\footnote{``statistic." For instance, mean, standard deviation} of a distribution $F$\footnote{$F$ refers to cdf, $f$ refers to pdf} parameterized by $\te'\in \Te$, given that $F\in \cal F$ (some family), when we observe samples drawn from $F$.

\begin{enumerate}
\item
If $\cal F$ (i.e., $\Te$) is finite-dimensional, we're doing parametric statistics. (For example, we assume $F$ is normal---normal distributions are parametrized by $(\mu,\si)\in \R\times \R_{\ge0}$.)
\item
If $\cal F$ is infinite-dimensional, we're doing nonparametric statistics.
\end{enumerate}

The two kinds of statistics are\footnote{for simplicity, suppose we're dealing with $\cal F$ finite; for $\cal F$ infinite (as is usually the case), replace $\Pj$ with probability density and $\sum$ with $\int$.}
\begin{enumerate}
\item Bayesian: We assume a distribution on priors. Roughly speaking, given observed event (ex. the sample drawn) $B$, the likelihood that it came from distribution $F\in \cal F$ is 
\[
\Pj(F|B) = \frac{\blu{\Pj(F)} \Pj(B|F)}{\Pj(B)}.
\]
Our best guess for $F$ is $\arg\max_F \Pj(F|B)$. We can calculate the expected value for $\te(F)$; it will be $\sum_F \Pj(F|B) \te(F)$.
\item Frequentist: We assume no distribution on priors $\Pj(F)$ is known. In this case we simply maximize
\[
\Pj(B|F).
\]
\end{enumerate}

Frequentist vs. Bayesian axioms:

\noindent
\begin{tabular}{|c|c|c|}
\hline 
 & Frequentist & Bayesian\tabularnewline
\hline 
Probability is  & limiting relative frequency. & degree of belief.\tabularnewline
\hline 
Parameters  & are fixed unknown constants. & can be talk about probabilistically.\tabularnewline
\hline 
Statistical procedures & good long run frequency properties. & involve a distribution on $\theta$.\tabularnewline
\hline 
\end{tabular}

``To combine prior beliefs with data in a principled way, use Bayesian inference.
To construct procedures with guaranteed long run performance, such
as confidence intervals, use frequentist methods. Generally, Bayesian methods
run into problems when the parameter space is high dimensional."

\section{Basic inference}
Suppose $(X_1,\ldots, X_n)$ is the observed sample, and our estimate for the statistic $\te$ is $\wh{\te_n}=g(X_1,\ldots, X_n)$. (Example: $\te$ is one component of $\te'$. For example, $\te'=(\mu,\si)$ and $\te$ is just $\mu$ or $\si$.)\footnote{This is suboptimal notation, but $\te$ is used for both in the whole parameter and a function of it in the literature, and that's confusing.}


\subsection{Estimators}



\subsubsection{Error}

\begin{df}
Define the \textbf{standard error} by \[\se=\sqrt{\Var_\te(\wh{\te_n})}.\]
Note we can't find this directly because we don't know the actual distribution $F$ so don't know $\te(F)$. 
Suppose $\se=s(\te)$. The \textbf{estimated standard error} comes from estimating $F$ (i.e., $\te$) first and plugging that value of $\te$ into the formula for se: $\hse = s(\hten)$.

Define \textbf{bias} by\footnote{Warning: $\E_\te$ means average given $\te$, not over $\te$.}
\[
\bias(\wh{\te_n})=\E_\te(\wh{\te_n})-\te.
\]
(Note we must be given the actual value of $\te$ to calculate the bias, so this is a function of $\te$.)

The \textbf{mean standard error (MSE)} is 
\bal
\E_\te[ (\wh{\te_n}-\te)^2] 
&= (\ol{\wh{\te_n}}-\te)^2 + \E_\te (\hten-\ol{\te_n})^2\\
&= \bias(\hten)^2 + \Var_{\te}(\hten)
\end{align*}
\end{df}
Bias measures how much the average estimate is from the actual value, the second part measures how much the estimate is from the average estimate.

\begin{ex}
\begin{enumerate}
\item
Bernoulli($p$). $\se=\sfc{p(1-p)}{n}$ and $\hse = \sfc{\wh p (1-\wh p)}{n}$ where $\wh p = \rc n \sum_i X_i$.
\item
$N(\mu,\si)$. $\se=\sfc{n-1}{n}\si$ and $\hse = \sfc{n-1}n \hat{\si}=\sfc{n-1}n\sfc{n}{n-1}\sqrt{\E\Var_\te\{X_1,\ldots,X_n\}} =\sqrt{\E\Var_\te\{X_1,\ldots,X_n\}} = \fc{\sum(X_i-\ol X)^2}{n-1}$.

If we knew $\se$, then the distribution of $\wh\mu$ is
\[
\fc{\wh \mu - \mu}{\se}\sim N\pa{0,\fc{\si}{\sqrt n}}.
\]
However, if we use $\hse$ instead of $\se$ we get a $t$-distribution rather than a normal distribution. For $n$ large, the $t$-distribution is approximately the same.
\end{enumerate}
\end{ex}

\subsubsection{Properties}
What properties do we want for an estimator?
\begin{enumerate}
\item
Unbiased: For every $F\in \cal F$, $\E_{X_1,\ldots, X_n\sim \cal F}\wh{\te_n}=\te$. (This is actually not so important!)
\item Consistent: $\wh{\te_n}\xra{P}\te$ as $n\to \iy$.
\item Asymptotically normal (stronger than consistent): $\fc{\wh{\te_n}-\te}{\se}\leadsto N(0,1)$. \fixme{Warning: sometimes we care about this quantity with $\hse$!}
\item Asymptotic optimal/efficient: among all well-behaved estimators, the MLE has smallest variance.
%or \rightsquigarrow
\end{enumerate}

\begin{exr}
Explain why the sample variation is given by $\wh{\si}^2=\frac{\sum (X_i-\ol X)^2}{N-1}$.
\end{exr}
\begin{proof}[Solution.]
This is an unbiased estimator, while $\frac{\sum (X_i-\ol X)^2}{N}$ is a biased estimator.

For simplicity, consider the discrete case. Let $(X_i)_{i=1}^n$ denote the sample. We have (see exercise below)
\bal
\E \Var((X_i)) &= \E(X_i^2)-(\E(X_i))^2\\
&= \frac{\sum X_i^2}{n} - \frac{\sum X_i^2}{n^2} - \frac{\sum_{i\ne j} X_iX_j}{n^2}\\
&= \fc{n-1}{n} \pa{\frac{\sum X_i^2}{n}-\frac{\sum_{i\ne j} X_i X_j}{n(n-1)}}\\
&= \frac{n-1}{n} (\E (X_i^2)-(\E X_i)^2) = \frac{n-1}{n}\Var\{X_1,\ldots,X_n\}.
\end{align*}
\end{proof}

\subsection{Maximal likelihood estimators}

\begin{df}
Let $\te$ be the parameter for a distribution and $x^n=(x_1,\ldots,x_n)$ the sample. The \textbf{likelihood} of $x^n$ given $\te$ is
\[
\cal L(\te) :=\Pj(x^n|\te) =\prod_{i=1}^n f(x_i;\te)
\]
where $f(x;\te)$ is the probability density of $x$ given $\te$.
The \textbf{log-likelihood} is
\[
\ell(\te):= \ln \Pj(x^n|\te) = \sum_{i=1}^n \ln f(x_i;\te).
\]
(We usually aren't too careful with constants, and may drop them.)
The \textbf{maximal likelihood estimator} is
\[
\arg_{\te}\max \cal L(\te) = \arg_{\te}\max \ell(\te).
\]
\end{df}
\subsubsection{Examples}
We calculate the MLE for several distributions.
\begin{enumerate}
\item
Bernoulli. Here each $X_i$ is 0 or 1. We have
\[
\cal L(\te) = p^{\sum_i X_i} (1-p)^{n-\sum_i X_i}, \qquad \ell(\te) = (\sum X_i) \ln p + (n- \sum X_i)\ln (1-p).
\]
Setting $\pdt{p}=0$ gives $\wh p=\frac{\sum X_i}{n}$ as MLE.
\item
Normal. Finding the MLE for $\mu$ means maximizing $\ell(\mu,\si)=C-\sum \frac{(x_i-\mu)^2}{2\si^2}$, which is minimizing the sum of squares 
\[
\sum (x_i-\mu)^2.
\]
The MLE is $\wh{\mu} = \rc n \sum x_i$.\footnote{It does NOT make sense to find the MLE for $\si$, which is $\iy$.}
\item 
(Uniform distribution)
Given $\cal F=\{U_{[0,l]}\}$, the estimated $l$ is $\min\{X_1,\ldots, X_n\}$. (Suppose the buses in an unknown city are labeled 1 to $N$. Assuming no knowledge of a prior distribution on number of buses, and you see buses numbered $b_1,\ldots, b_n$, your best guess for the number of buses is $\max\{b_1,\ldots, b_n\}$.)
\item
Linear regression. We assume $x_i$ are fixed, and $y_i=\be_0+\be_1x_i+\ep$ where $\ep\sim N(0,\si^2)$ is error. Then we want to minimize
\[
\sum_i (y_i-\be_0-\be_1x_i)^2
\]
(minimize least squares). Setting $\pdt{\be_0}=\pdt{\be_1}=0$ gives the system $\matt{n}{\sum X_i}{\sum X_i}{\sum X_i^2}=\coltwo{\sum Y_i}{\sum Y_iX_i}$ which has solution
\[
\be_1 = \frac{\ol{XY}-\ol X\cdot\ol Y}{\ol{X^2}-\ol{X}^2}\qquad \be_0=\ol Y-\be_1\ol X.
\]
\item Multivariate linear regression. Here $Y=X\be + \ep$ where $X$ is $n\times p$. The MLE is again given by least squares, which is given by the projection $\wh \be = (X^TX)^{-1}X^TY$ (assume $X$ has full rank; this is necessary).
\end{enumerate} 


%\subsection{Examples}
%
%\begin{enumerate}
%\item
%(Uniform distribution)
%Given $\cal F=\{U_{[0,l]}\}$, the estimated $l$ is $\min\{X_1,\ldots, X_n\}$. (Suppose the buses in an unknown city are labeled 1 to $N$. Assuming no knowledge of a prior distribution on number of buses, and you see buses numbered $b_1,\ldots, b_n$, your best guess for the number of buses is $\max\{b_1,\ldots, b_n\}$.)
%\item 
%(estimate for normal distribution)
%Let $\wh\mu=\rc n\sum X_i$ and $\wh{\se}=\wh\si = \frac{\sum (X_i-\ol X)^2}{n-1}$. 
%\end{enumerate}•

\subsubsection{Properties of MLE}
Define some quantities first.
\begin{df}
\begin{enumerate}
\item
\textbf{KL distance}
\[
D(f,g)=\int f(x)\ln \pf{f}{g}\,dx.
\]
Why do we care about this? Maximizing $\ell_n(\te)$ is equivalent to maximizing 
\[
M_n(\te)=\rc n\sum_i\ln \fc{f(X_i;\te)}{f(X_i;\te_*)}
\]
which has the nice property that the maximum is 0. (Without the $\rc n$ it would blow up.) By LLN the expected value of this is exactly $-D(\te_*,\te)$.
\item 
\textbf{score function} $s(X;\te)=\pd{\ln f}{\te}$.

Important property: $\E s=\int_{-\iy}^{\iy} s(X;\te) f\,dx=(\int_{-\iy}^{\iy}f\,dx)_{\te}=0$. 
\item
\textbf{Fisher information} $I(\te)=\Var_\te(s(X;\te))$, $I_n(\te)=nI(\te)$.
I.e., $I(\te)=-\E((\ln f)_{\te\te})$.
\end{enumerate}
\end{df}

\begin{enumerate}
\item
\begin{thm}[Convergence of MLE]
Suppose 
\begin{enumerate}
\item
$\sup_{\te\in \Te}|M_n(\te)-M(\te)|\xra{P}0$,
\item
for all $\ep>0$, $\sup_{|\te-\te_*|\ge\ep} M(\te)<M(\te_*)$.
\end{enumerate}
Then the MLE $\wh{\te_n}\xra P\te_*$.
\end{thm}
\begin{proof}
First show that $M(\te_*)-M(\wh{\te_n})\xra P0$. Then use continuity of $M$.
\end{proof}
\item
\begin{thm}[Asymptotic normality of MLE]
\begin{enumerate}
\item
$\se\sim \sfc1{nI(\te)}$ and $\fc{\wh{\te_n}-\te}{\se}\to N(0,1)$.
\item \fixme{$\wh{\se}=\sfc{1}{nI(\wh{\te_n})}$: why are we redefining $\wh{\se}$? We defined it a different way before. Do these definitions coincide?}
$\fc{\wh{\te_n}-\te}{\wh{\se}}\to N(0,1)$.
\end{enumerate}•
\end{thm}
\begin{proof}
\begin{enumerate}
\item
Linearize to find that
\[
\ell'(\wh \te)-\ell'(\te)\approx (\wh \te-\te)(\ell''(\te))\implies -\fc{\ell'}{\ell''}(\te)\approx \wh{\te}-\te.
\]
Now
\[
\sqrt n(\wh{\te_n}-\te)=\fc{\rc{\sqrt n}\ell'(\te)}{-\rc n\ell''(\te)}\to \fc{N(0,I(\te))}{I(\te)}\to N(0,1),
\]
the top in distribution, the bottom in probability. (The top uses CLT on $\sum (\ln f)_\te$; the bottom uses LoLN on $\sum -(\ln f)_{\te\te}$.)
\item
Show that $\sfc{I(\wh{\te_n})}{I(\te)}\xra P 1$.
\end{enumerate}
\end{proof}
\item Think of this as a chain rule.
\begin{thm}
If $\tau=g(\te)$ and $g'(\te)\ne 0$, then $\fc{\wh{\tau_n}-\tau}{\wh{\se}(\wh{\tau})}\to N(0,1)$ where $\wh{\tau_n}=g(\wh{\te_n}),\wh{\se}(\wh{\tau_n})=|g'(\wh{\tau})|\wh{\se}(\wh{\tau_n})$.
\end{thm}
Proof: just expand $g$ using $g'$.
\item (Equivariance) If $\tau=g(\te)$ is 1-to-1, then $\wh{\tau_n}=g(\wh{\te_n})$. Follow definitions!
\end{enumerate}

Write $x^n\lra y^n$ if $f(x^n;\te)=cf(y^n;\te)$ as functions of $\te$. $T$ is sufficient if $T(x^n)=T(y^n)\implies x^n\lra y^n$.
$T$ is minimally sufficient if it is also a function of every other sufficient statistic. Factorization gives that $f(x^n;\te)=g(t(x^n);\te)h(x^n)$.

\fixme{Some stuff on exponential families at end of \S9}.
\section{Hypothesis testing}

%Let $\cal B$ be the space of possible $B$ (ex. samples). We look for a region $\cal A\subeq \cal B'$ for which 
%\[
%\Pj(B|\te\in\Te)\le \al,
%\]
%the probability that we observe $B$ given $\te \in \Te$ is small. This way, if we observe 

We are given hypothesis $H_0:\te\in \Te_0$ and $H_1:\te\in \Te_1=\Te_0^c$ and want to know which is true. (For example, $H_0=\{\te_0\}$ and $H_1=\R\bs \{\te_0\}$.) $H_0$ is the \textbf{null hypothesis}; we reject or fail to reject it. In other words, we have a decision function $\de$ that given $X^n$ gives 0 or 1.
There are two kinds of errors:
\begin{enumerate}
\item reject ($\de=1$) when $H_0$ is true
\item fail to reject ($\de = 0$) when $H_1$ is true.
\end{enumerate}

\begin{df}
The \textbf{power} is 
\[
\be(\te) = \Pj(\de=1|\te),
\]
the probability of rejecting given $\te$. We want to maximize power for $\te\in H_1$ (i.e., minimize $1-\be(\te)$, which is the type 2 error).

The \textbf{size} is
\[
\sup_{\te\in H_0}\be(\te) = \sup_{\te\in H_0}\Pj(\de=1|\te).
\]
The is the maximum probability of a false rejection (type 1 error). For example, when $H_0$ is $\{\te_0\}$ this is just $\Pj(\de=1|\te_0)$.
We say a test has level $\al$ if it has size $\ge \al$.

Given rejection regions $R_\al$ of size $\al$, the \textbf{$p$-value} is $\inf\set{\al}{T(x^n)\in R_{\al}}=\sup_{\te\in \Te_0}(\Pj_{X^n}(T(X^n)\ge T(x^n)|\te))$ where $T$ is the statistic used in the test. 
\end{df}

\subsection{Basic tests}

\prbox{
\begin{exr}
\begin{enumerate}
\item
Give a hypothesis test for the normal distribution (z and t test).
\item
Give a hypothesis test for comparing 2 means of normal variables when 
\begin{enumerate}
\item
2 groups are independent
\item 
the elements are paired.
\end{enumerate}
Do the same for Bernoulli.
\item Compare the variances of 2 normal distributions.
\end{enumerate}
\item Give a test comparing 2 variances of normal distributions.
\end{exr}
}

Let $z_{\al}$ be the value of $z$ such that $\int_z^{\iy}N(x)\,dx=\al$ where $N$ is the standard normal.

For the examples below we take $H_0$ consisting of a single point, so the tests are double-tailed. For $H_0=\{\te\le \te_0\}$ and $\{\te\ge \te_0\}$ we use single-tailed tests.
\begin{enumerate}
\item
For the normal distribution:
\begin{enumerate}
\item
We know that $\lim_{n\to \iy}\fc{\wh{\mu_n}-\mu_0}{\wh{\se}}\to N(0,1)$. Thus we can use the $z$-test which of size $\al$:\footnote{It really doesn't matter for these approximate tests whether we use $\wh{\si}$ or $\wh{\se}=\sfc{n-1}n\wt{\si}$ because $\sqrt{n-1}\sim \sqrt{n}$. For the exact tests we have to make the distinction.}
\[
\sqrt{n}\af{\wh\mu-\mu_0}{\wh{\si}}
>z_{\al/2}.
\]
The power is approximately $1-\Phi(\fc{\mu_0-\mu}{\wh{\si}}+z_{\al/2})+\Phi(\fc{\mu_0-\mu}{\wh{\si}}-z_{\al/2})$. 
% or n-1?
\item
That is only an approximation, good when $n$ large. The real distribution is the $t$-distribution with $n-1$ degrees of freedom. 

Now we use:
\begin{lem}
Let $X_1,\ldots, X_n\sim N(0,1)$. Then 
\[
\sqrt{n-1}\fc{\wt{\mu}}{\wh\si} \sim t_{n-1}.
\]
\end{lem}
\begin{proof}
$\sqrt n\wh{\mu}=(\rc{\sqrt n}\ldots \rc{\sqrt n})x$. Let $A$ be an orthogonal matrix with first row $A_1=(\rc{\sqrt n}\ldots \rc{\sqrt n})$. Let $y=Ax$. We have $y^Ty=x^Tx$ so 
\[n\wh{\si}=x_1^2+\cdots +x_n^2-\ub{y_1^2}{n\ol{X}^2}=y_2^2+\cdots +y_n^2.\]
This is distributed as $\chi^2_{n-1}$ and independent of $y_1$. Thus the distribution is $\sqrt{n-1}\fc{N(0,1)}{\chi_{n-1}^2}=t_{n-1}$.
\end{proof}
In summary we have
\[
\sqrt{n-1}\fc{\wh\mu-\mu_0}{\wh{\si}} = 
\sqrt{n-1} \cdot 
\ub{\fc{\sqrt n(\wh{\mu}-\mu_0)}{\si}}{\sim N(0,1)}
/ \ub{\fc{\sqrt n \wh{\si}}{\si}}{\sim \chi_{n-1}^2} \sim t_{n-1}.
\]

The more accurate test is
\[
\sqrt{n-1}\af{\wh\mu-\mu_0}{\wh{\si}}>t_{n-1,\al/2}.
\]
\end{enumerate}
\item Let $\mu_x=\E_{i=1}^m X_i$ and similarly for $y$.
\begin{enumerate}
\item
First, the normal approximation. We have $\fc{\wh{\mu_x}-\mu_x}{\wh{\si_x}}\approx N(0,1)$ and similarly for $y$; clear denominators, add, and divide by standard deviation (noting variances add) to get
\[
\fc{\wh{\mu_x}+\wh{\mu_y}-\mu_x-\mu_y}{\sqrt{n\wh{\si_x}^2+m\wh{\si_y}^2}} \approx N(0,1);
\]
do the $z$-test.

For an exact test, we find
\[
\fc{\wh{\mu_x}+\wh{\mu_y}-\mu_x-\mu_y}{\sqrt{n\wh{\si_x}^2+m\wh{\si_y}^2}} = \fc{N(0,\fc{\si_x^2}{m})+N(0,\fc{\si_y^2}{n})}{\sqrt{\si_x^2\chi_{m-1}^2+\si_y^2\chi_{n-1}^2}}
\]
which simplifies only when $\si_x=\si_y$ (assumption of equal variances) to get
\[
\sfc{m+n}{mn}\fc{N(0,1)}{\sqrt{\chi_{m+n-2}^2}}.
\]
Thus test
\[
\ab{\sfc{mn(m+n-2)}{m+n}\fc{\wh{\mu_x}+\wh{\mu_y}}{\sqrt{n\wh{\si_x}^2+m\wh{\si_y}^2}}}>t_{m+n-2,\al}
\]
For unequal variances, use the Satterthwaite approximation (see 18.443 lecture 7).
\item
Use the $t$-test on the pairs $x_i-y_i$.
\end{enumerate}
For Bernoulli, let $\wh{\si}=\sfc{\wh p (1-\wh p)}{m}$ and use the normal approximation to the binomial to test.
\item
Suppose we have $m$, $n$ samples respectively. Now if both variances equal $\si^2$,
\[
\fc{\wh{\si_m}^2}{\wh{\si_n}^2}\sim \fc{\rc m\chi_{m-1}^2}{\rc n\chi_{n-1}^2}.
\]
So take
\[
\frac{m(n-1)}{n(m-1)}\fc{\wh{\si_m}^2}{\wh{\si_n}^2}
\]
as the $F$-test statistic.
\end{enumerate}
\subsection{$\chi^2$ goodness of fit}

%\prbox{
%\begin{exr}
%\item
%Give Pearson's $\chi^2$ test.
%\item
%Give 
%\end{exr}
%}

\begin{enumerate}
\item
How to test multinomial distributions? Use the following
\[
\sum_{j=1}^r \fc{(X_j-E_j)^2}{E_j}
\]
as a test statistic for $\chi_{r-1}^2$, where $E_j=np_j$ is the expected number in category $j$.
\begin{proof}
The covariance matrix $A$ for the variables $\pf{X_i-np_i}{\sqrt{np_i}}_i$ has diagonal $(1-p_i)_i$ and off-diagonal entries $-\sqrt{p_ip_j}$. Note that $(\sqrt{p_i})_i$ spans the kernel. Note $A-I$ has rank 1 so all other eigenvectors have value 1. Let $U$ diagonalize $A$; let $y=Ux$. Then 
\[
\sum_j \fc{(X_j-E_j)^2}{E_j} = x^Tx = y^Ty.
\]
Because $y$ has covariance matrix $\matt 000{I_{r-1}}$, this is $\chi_{r-1}^2$.
\end{proof}
\item
For goodness-of-fit of a sample to a distribution $f$, split the domain of $f$ into intervals (where $\int f=\rc{n}$, say), and run Pearson's test on them.

Careful: here $\wh{p_j}$ should be the MLE for the grouped distribution, the maximum for $\arg_\te\max \Pj(I_1|\te)^{v_1}\cdots$, rather than the MLE for the distribution, then grouped. (See 18.443 L12.)

The degrees of freedom should be $r-\dim(\Te)-1$.

\item
Give tests for independence and homogeneity.

These are the same!

Suppose there are $N_{ij}$ observations for $(i,j)$. We want $\max \prod p_i^{N_{i\bullet}}\prod_j q_j^{N_{\bullet j}}$. Take logs; the gradient should be in the span of $(1,\ldots,1,0,\ldots,0)$ and $(0,\ldots,0,1,\ldots,1)$ (Lagrange multipliers) so all the $N_{i\bullet}/p_i$ are the same, and similarly for $q$.  df$=ab-(a-1)-(b-1)-1=(a-1)(b-1)$.

\end{enumerate}

\subsection{Nonparametric testing}

For any distribution, let $F_n(x)$ be the estimated distribution function after getting $n$ samples. By LLN, for a particular $x$, $F_n(x)\to F(x)$: $\sqrt{n}(F_n(x)-F(x))\xra d N(0,F(x)(1-F(x)))$.

The basis of nonparametric tests using $F_n$ is the following: 
\begin{thm}
$\sup|F_n-F|$ does not depend on $F$. 
\end{thm}
\begin{proof}
Let $y=F_n(x)$ so that $x=F_n^{-1}(y)$; putting things in terms of $y$ makes this into the problem for the uniform distribution. (Warning: some finesse is required because $F_n$ can have jumps.)
\end{proof}

The Kolmogorov-Smirnov test uses
\[
\Pj(\sqrt n\sup_{x\in \R} |F_n(x)-F(x)|\le t)\to H(t)=1-2\sum_{i=1}^{\iy} (-1)^{i-1} e^{-2i^2t}.
\]
There are variations; see L14.

\subsection{Multiple testing}

If we do $N$ tests at level $\al$, we can expect $pN$ false rejections. We can test at level $\fc{\al}N$ but often this is too stringent. A better way in practice is to order $p$-values from smallest to largest. If the $i$th $p$-value falls below $\fc{i\al}{m}$ then reject all tests below that.

\section{Bayesian statistics}

%\chapter{Introduction}


\section{Normed spaces}

\begin{df}
A \textbf{normed space} is a pair $(X,\ve{\cdot})$ where $X$ is a real or complex vector space and $\ved$ is a norm on $X$. Most of the time the choice of scalar field makes little difference; for convenience we'll use real scalars. A norm induces a metric: $d(x,y)=\ve{x-y}$. This induces a topology on $X$, called the \textbf{norm topology}. A \textbf{Banach space} is a complete normed space.
\end{df}
\begin{ex}
\begin{enumerate}
\item (sequences)
For $1\le p<\iy$, we have $\ell_p=\set{(x_n)\text{ scalar sequence}}{\sum_{n=1}^{\iy}|x_n|^p<\iy}$ with norm $\ve{x}_p=\pa{\sum_{n=1}^{\iy}|x_n|^p}^{\rc p}$. (Minkowski's inequality says that if $x,y\in \ell_p$ then $x+y\in \ell_p$, so $\ve{x+y}_p\le \ve{x}_p+\ve{y}_p$. Then $\ell_p$ is a Banach space.
\item (convergent sequences)
$\ell_{\iy}=\set{(x_n)\text{ scalar sequence}}{(x_n)\text{ is bounded}}$ with $\ve{x}_{\iy}=\sup_{n\in N}|x_n|$. Then $\ell_{\iy}$ is a Banach space.

$c_{00}=\set{(x_n)\text{ scalar sequence}}{\exists N\forall n>N, x_n=0}$. Let $e_n=(0,0,\ldots, 0,\ub{1}{n},0,\ldots)$; then $c_{00}=\spn\set{e_n}{n\in \N}$. Note that $c_{00}$ is a subspace of $\ell_{\iy}$ but it's not closed: In $\ell_p, 1\le p<\iy$, $\ell_p=\ol{\spn}\set{e_n}{n\in \N}$. 

$c_0=\set{(x_n)\in \ell_{\iy}}{\lim_{n\to \iy} x_n=0}$ is a closed subspace of $\ell_{\iy}$, $c_0=\ol{\spn}\set{e_n}{n\in \N}$ in $\ell_{\iy}$.

$c=\set{(x_n)\in \ell_{\iy}}{\lim_{n\to \iy} x_n \text{ exists}}$ is a closed subspace of $\ell_{\iy}$. $c_0$ and $c$ are Banach spaces.
\item (Euclidean space)
$\ell_p^n=(\R^n,\ved_p), 1\le p\le \iy$.
\item 
$K$ is any set, $\ell_{\iy}(K)=\set{f:K\to \R}{f\text{ is bounded}}$ with norm $\ve{f}_{\iy}=\sup_{x\in K}|f(x)|$. This is a Banach space, e.g. $\ell_{\iy}=\ell_\iy(\N)$.
\item 
$K$ compact topological space $C(K)=\set{f\in \ell_{\iy}(K)}{f\text{ continuous}}=\set{f:K\to \R}{f\text{ is continuous}}$. $C(K)$ is a closed subspace of $\ell_{\iy}(K)$ because any uniform limit of continuous functions is continuous, and hence it's a Banach space, e.g. $C[0,1]$.

We'll write $C^{\R}(K)$ and $C^{\C}(K)$ for the real and complex versions of $C(K)$, respectively.
\item 
Let $(\Om,\Si,\mu)$ be a measure space. Then for $1\le p<\iy$,  \[L_p(\mu)=\set{f:\Om\to \R}{f\text{ is measurable}, \int_{\Om}|f|^p\,d\mu<\iy}\] with norm $\ve{f}_p=\pa{\int_{\Om}|f|^p\,d\mu}^{\rc p}$ is a Banach space (after identifying functions that are equal almost everywhere. 

When $p=\iy$, $L_{\iy}(\mu)=\set{f:\Om\to \R}{f\text{ is measurable and essentially bounded}}$. (``Essentially bounded" means that there exists a null-set $N$ such that $f$ is bounded on $\Om\bs N$.)
\[
\ve{f}_{\iy}=\ess\sup|f|=\inf_N\sup_{\Om\bs N}|f|.
\]
\item
Hilbert spaces, e.g. $\ell_2$, $L_2(\mu)$. All Hilbert spaces are isomorphic, but some different representation may be more natural.
\end{enumerate}
\end{ex}

\begin{pr}
Let $X,Y$ be normed spaces, $T:X\to Y$ linear. Then the following are equivalent.
\begin{enumerate}
\item
$T$ is continuous.
\item
$T$ is bounded: $\exists C\ge 0$, $\ve{Tx}\le C\ve{x}$ for all $x\in X$.
\end{enumerate}
\end{pr}
\begin{proof}
To think about continuity at $a$, ``translate" to 0 using linearity.
\end{proof}

\begin{df}
Let $\cal B(X,Y)=\set{T:X\to Y}{T\text{ is linear and bounded}}$. This is a normed space with the \textbf{operator norm}: $\ve{T}=\sup\set{\ve{Tx}}{\ve{x}\le 1}$. $T$ is an \textbf{isomorphism} if $T$ is a linear bijection whose inverse is also continuous. (This is equivalent to $T$ being a linear bijection and there existing $a>0,b>0$, with $a\ve{x}\le \ve{Tx}\le b\ve{x}$ for all $x\in X$.)

If there exists such $T$, we say $X,Y$ are \text{isomorphic} and we write $X\sim Y$.

If $T:X\to Y$ is a linear bijection such that $\ve{Tx}=\ve{x}$ for all $x\in X$ (i.e. $a=b=1$), then $T$ is an \textbf{isometric isomorphism} and we say $X,Y$ are isometrically isomorphic and write $X\cong Y$\footnote{Some people use $\cong$ for isomorphism. We use it to mean isometric isomorphism.}.

$T:X\to Y$ is an \textbf{isomorphic embedding} if $T:X\to TX$ is an isomorphism. We write $X\hra Y$.
\end{df}
\begin{pr}
If $Y$ is complete, then $\cal B(X,Y)$ is complete. In particular, $X^*=\cal B(X,\R)$, the space of bounded linear functionals, called the \textbf{dual space} of $X$, is always complete.
\end{pr}
\begin{ex}
\begin{enumerate}
\item
For $1<p<\iy$, then $\ell^*\cong \ell_q$ where $\rc p+\rc q=1$. The proof uses H\"older's inequality: $x=(x_n)\in \ell_p$, $y=(y_n)\in \ell_q$ then $\sum|x_ny_n|\le \ve{x}_p\ve{y}_q$. This isomorphism is $\ph:\ell_q\to \ell_p^*,y\mapsto \ph_y, \ph_y(x)=\sum x_ny_n$.)
\item
$c_0^*\cong \ell_1$, $\ell_1^*\cong \ell_{\iy}$. 
(Later we will see that $c_0$ cannot be a dual space.)
%(This gives an alternate proof of completeness.)
\item 
If $H$ is a Hilbert space then $H^*\cong H$ (Riesz Representation Theorem).
\item
If $(\Om,\Si,\mu)$ is a measure space, $1<p<\iy$, then $L_p(\mu)^*\cong L_q(\mu)$ where $\rc{p}+\rc{q}=1$.

If $\mu$ is $\si$-finite then $L_1(\mu)^*\cong L_{\iy}(\mu)$.
(Else we only have $L_{\iy}(\mu)\hra L_1(\mu)^*$.)
\end{enumerate}
\end{ex}

{\color{blue}Lecture 2}

Recall that if $V$ is a finite-dimensional vector space, then any two norms on $V$ are equivalent. Specifically, if $\ved$ and $\ved'$ are two norms on $V$, then there exist $a,b>0$ such that 
\[
a\ve{x}\le\ve{x}'\le b\ve{x}\forall x\in V.
\]
In other words, $\Id:(V,\ved)\to (V,\ved')$ is an isomorphism. 

Some consequences are the following.
\begin{cor}
\begin{enumerate}
\item
If $X,Y$ are normed spaces, $\dim X<\iy$, $T:X\to Y$ is linear, then $T$ is bounded.
\item
If $\dim X<\iy$ then $X$ is complete. 
\item If $X$ is a normed space and $E$ a subspace with $\dim E<\iy$, then $E$ is closed.
\end{enumerate}
\end{cor}
\begin{proof}
\begin{enumerate}
\item
Set $\ve{x}'=\ve{x}+\ve{Tx}$. This is a norm on $X$, so there exists $b>0$, $\ve{x}'\le b\ve{x}$ for all $x$, so $\ve{Tx}\le b\ve{x}$ for all $x\in X$. 

%If $\dim X=\dim Y<\iy$ then $X\sim Y$.
\item
By (1), $X\sim \ell_2^n$ where $n=\dim X$.
\end{enumerate}•
\end{proof}
\section{Riesz's lemma and applications}
The unit ball is compact and this characterizes finite-dimensionality. We use the following.
\begin{lem}[Riesz's Lemma]
\llabel{lem:riesz}
Let $Y$ be a proper closed subspace of a normed space $X$. Then for every $\ep>0$ there exists $x\in X$, $\ve{x}=1$, such that $d(x,Y):=\inf_{y\in Y}\ve{x-y}>1-\ep$.
%null space, don't have notion as euclidean space, suggests proof that works. Intuition about euclidean space sometimes dangerous because may not work in normed space but in this case works
\end{lem}
We'd like to take the some sort of ``perpendicular" vector to $Y$, or the vector which minimizes the distance from a point not on $Y$ to $Y$. Note this is not in general possible since $X$ may not be complete, and hence the ``$\inf$." However, we can come arbitrarily close to that $\inf$, and get an ``almost perpendicular" vector.
\begin{proof}
Pick $z\in X\bs Y$ with $Y$ proper. Since $Y$ is closed, $d(z,Y)>0$. There exists $y\in Y$ such that $\ve{z-y}<\fc{d(z,Y)}{1-\ep}$ (WLOG $\ep<1$).

Set $x=\fc{z-y}{\ve{z-y}}$. Then
\[
d(x,Y)=d\pa{\fc{z-y}{\ve{z-y}},Y}=\rc{\ve{z-y}}d(z-y,Y)=\fc{d(z,Y)}{\ve{z-y}}> 1-\ep.
\]
\end{proof}
We give two applications. First, some notation. In a metrix space $(M,d)$, write 
\[
B(x,r):=\set{y\in M}{d(x,y)\le r},x\in M,r\ge 0
\]
for the closed ball of radius $r$ at $x$. In a normed space, 
\[
B_X:=B(0,1)=\set{x\in X}{\ve{x}\le 1}, \qquad B(x,r)=x+rB_X.
\]
Also, $S_X=\set{x\in X}{\ve{x}=1}$. 
\begin{thm}
Let $X$ be a normed space. Then $\dim X<\iy$ iff $B_X$ is compact.
\end{thm}
\begin{proof}
``$\Rightarrow$" We have $X\sim \ell_2^n$ where $n=\dim X$.

``$\Leftarrow$" By compactness there exist $x_1,\ldots, x_n\in B_X$ such that $B_X\subeq \bigcup_{i=1}^n B(x_i,\rc2)$.  Let $Y=\spn\{x_1,\ldots, x_n\}$. For all $x\in B_X$ there exists $y\in Y$ with $\ve{x-y}\le \rc 2$, so $d(x,Y)\le \rc 2$. Thus there do not exist ``almost orthogonal vectors" in the sense of  Riesz's Lemma~\ref{lem:riesz}. This means $Y$ is not a proper subspace of $X$, so $X$ is finite-dimensional. %$Y$ is dense in $X$, so $X=\ol Y=Y$.
%(We can't find an orthogonal vector to $Y$ in the sense of Riesz's Lemma.)
\end{proof}
\begin{rem}
We showed the following in the proof: If $Y$ is a subspace of a normed space $X$ and there exists $0\le \de<1$ such that for all $x\in B_X$ there exists $y\in Y$ with $\ve{x-y}\le \de$, then $Y$ is dense in $X$.
\end{rem}
If we let $\de= 1$ then this statement is trivial. The remark says that if we can do a little better than 1, the trivial estimate, then we can automatically approximate $x$ with much smaller $\ep$.

\begin{thm}[Stone-Weierstrass Theorem]
Let $K$ be a compact topological space and $A$ be a subalgebra of $C^{\R}(K)$. If $A$ separates the points of $K$ (i.e., for all $x\ne y$ in $K$, there exists $f\in A$, $f(x)\ne f(y)$), and $A$ contains the constant functions, then $A$ is dense in $C^{\R}(K)$.
\end{thm}
In this case it does matter whether the field of scalars is $\R$ or $\C$.

The following proof is due to T. J. Ransford.
\begin{proof}
First we show that if $E,F$ are disjoint closed subsets of $K$, then there exists $f\in A$ such that $-\rc2\le f\le \rc2$ on $K$ and $f\le -\rc 4$ on $E$ and $f\ge \rc 4$ on $F$.

Fix $x\in E$. Then for all $y\in F$, there exists $h\in A$ such that $h(x)=0$, $h(y)>0$, $h\ge 0$ on $K$. (This is since $A$ separates points, we can shift by a constant, and square the function.) Then there is an open neighborhood of $y$ on which $h>0$.  %finitely many cover $F$ by compactness.
An easy compactness argument gives that there exists $g=g_x\in A$ with $g(x)=0$, $g>0$ on $F$, 
%strictly positive on neighborhood, going to be strictly positive on all
%rescale it so that  
$0\le g\le 1$ on $K$. 
Pick $R=R_x\in \N$ such that $g>\fc2R$ on $F$, set $U=U_x=\set{y\in K}{g(y)<\rc{2R}}$. 
%compact, attain inf

Do this for all $x\in E$.
%finitely many will cover.
Compactness gives a finite cover: there exist $x_1,\ldots, x_m$ such that $E\subeq \bigcup_{i=1}^m U_{x_i}$. To simplify notation, set $g_i=g_{x_i}$, $R_i=R_{x_i}$, $U_i=U_{x_i}$, and $i=1,\ldots, m$. For $n\in \N$,  by Bernoulli's inequality,
\begin{align*}
\text{on }U_i&&
(1-g_i^n)^{R_i^n}&\ge 1-(g_iR_i)^n>1-2^{-n}\to 1\text{ as }n\to \iy\\
\text{on }F&&
(1-g_i^n)^{R_i^n}&\le \rc{(1+g_i^n)^{R_i^n}}\le \rc{(g_iR_i)^n}<\rc{2^n}\to 0\text{ as }n\to \iy
\end{align*}
There exists $n_i\in \N$ such that $h_i=1-(1-g_i^{n_i})^{R_i^{n_i}}$ satisties
\begin{itemize}
\item
on $U_i$, $h_i\le \rc 4$
\item
on $F$, $h_i\ge \pf 34^{\rc m}$
\item
on $K$, $0\le h_i\le 1$.
\end{itemize}
Set $h=h_1h_2\cdots h_m$. Then $h\le \rc 4$ on $E$, $h\ge \fc 34$ on $F$, and $0\le h\le 1$ on $K$. Set $f=h-\rc2$.
%corresp is at most a quarter, other at most 1, so at most 1/4
Given $g\in C^{\R}(K)$, $\ve{g}_{\iy}\le 1$, set
\[
E=\set{x\in K}{g(x)\le -\rc 4},\,F=\set{x\in K}{g(x)\ge \rc 4}.
\]
Let $f\in A$ be as above. Then $\ve{f-g}\le \fc 34$, i.e., $d(g,A)\le \fc 34$. By Riesz's Lemma~\ref{lem:riesz}, $A$ is dense in $C^{\R}(K)$.
\end{proof}
%reading math is a sort of computation, info decompression
\begin{rem}
The complex version says that if $A$ is a subalgebra of $C^{\C}(K)$ that separates points of $K$ contains the constant functions, and is closed under complex conjugation ($f\in A\implies \ol f\in A$), then $A$ is dense in $C^{\C}(K)$.
\end{rem}
\section{Open mapping lemma}
We'll assume the Baire category theorem and its consequences: principle of uniform boundedness, open mapping theorem (OMT), closed graph theorem (CGT). 
\begin{df}
Let $A,B$ be subsets of a metric space $(M,d)$ and let $\de\ge 0$. Say $A$ is \textbf{$\de$-dense} in $B$ if for all $b\in B$ there exists $a\in A$ with $d(a,b)\le \de$.
\end{df}
\begin{lem}[Open mapping lemma]\llabel{lem:oml}
Let $X,Y$ be normed spaces, $X$ complete, $T\in \cal B(X,Y)$. Assume for some $M\ge 0$ and $0\le \de<1$ that $T(MB_X)$ is $\de$-dense in $B_Y$. Then $T$ is surjective. 
More precisely, 
%important: have quantitative
for all $y\in Y$ there exists $x\in X$ such that $y=Tx$ and
\[
\ve x\le \fc{M}{1-\de}\ve y,
\]
i.e.,
\[
T\pa{\fc{M}{1-\de}B_X}\supeq B_Y.
\]
Moreover, $Y$ is complete.
\end{lem}
\begin{proof}
The proof involves successive approximations. 
Let $y\in B_Y$. There exists $x_1\in MB_X$ with $\ve{y-Tx_i}\le \de$. Then $\fc{y-Tx_i}{\de}\in B_Y$. There exists $x_2\in MB_X$, with $\ve{\fc{y-Tx_i}{\de}-Tx_2}\le \de$, i.e., $\ve{y-Tx_1-\de Tx_2}\le \de^2$, and so forth. Obtain $(x_n)$ in $MB_X$ such that 
\[
\ve{y-Tx_1-\de Tx_2-\cdots -\de^{n-1} Tx_n}\le \de^n
\]
for all $n$. Set $x=\sum_{n=1}^{\iy} \de^{n-1} x_n$. This converges since $\sum_{n=1}^{\iy}\ve{\de^{n-1}x_n}\le M\sum_{n=1}^{\iy}\de^{n-1}=\fc{M}{1-\de}$, and $X$ is complete.\footnote{This kind of geometric sum argument comes up a lot in functional analysis!} 
So $x\in \fc{M}{1-\de}B_X$ and  by continuity $Tx=\suo \de^{n-1} Tx_n=y$. For the ``moreover" part, let $\hat Y$ be the completion of $Y$, and view 
%unique banach space of which it is a dense subspace
$T$ as a map $X\to \hat Y$. Since $B_Y$ is dense in $B_{\hat Y}$, $T(MB_X)$ is $\de'$-dense in $B_{\hat Y}$ for $\de<\de'<1$. By the first part, $T(X)=\hat Y=Y$, so $Y$ is complete.
\end{proof}
\begin{rem}
Suppose $T\in\cal B(X,Y)$, $X$ is complete, and the image of the ball is dense: $\ol{T(B_X)}\supeq B_Y$. Suppose that for all $\ep>0$, $T((1+\ep)B_X)$ is $1$-dense in $B_Y$. Take $M>1,0< \de<1$ so that $1+\ep=\fc{M}{1-\de}$;  lemma~\ref{lem:oml} shows that  $T\pa{(1+\ep)B_X}\supeq B_Y$. It follows that $T(B_X^{\circ})\supeq B_Y^{\circ}$. (For a subset $A$ of a topological space, $A^{\circ}$ or $\text{int}(A)$ denotes the interior of $A$.)
\end{rem}

{\color{blue} Lecture 3}

\subsection{Applications of the open mapping lemma}
\begin{thm}[Open mapping theorem]\llabel{thm:omt}
Let $X,Y$ be Banach spaces, $T\in \cal B(X,Y)$ be onto. Then $T$ is an open map.
\end{thm}
\begin{proof}
Let $Y=T(X)=\bigcup_{n=1}^{\iy} T(nB_X)$. The Baire category theorem tells us that there exists $N$ with $\text{int}(\ol{T(NB_X)})\ne \phi$. Then there exists $r>0$ with
\[
\ol{T(NB_X)}\supeq rB_Y.
\]
By Lemma~\ref{lem:oml}, for $M=\fc{2N}{r}$, we have $T(MB_X)\supeq B_Y$. Therefore, $U$ is open and $T(U)$ is open.
\end{proof}
%say T is open, same as T^{-1} cont.
\begin{thm}[Banach isomorphism theorem]\llabel{thm-bit}
If in addition $T$ is injective, then $T^{-1}$ is continuous.
\end{thm}
\begin{proof}
An open map that is a bijection is a homeomorphism.
\end{proof}
\begin{thm}[Closed graph theorem]
Let $X,Y$ be Banach spaces and $T:X\to Y$ be linear. Assume that whenever $x_n\to 0$ in $X$, $Tx_n\to y$ in $Y$, then $y=0$. Then $T$ is continuous.
\end{thm}
This is a powerful result. Usually we have to show the sequence converges and show it converges to 0. This says that we only have to check the second part given the first part. %once we show it converges, it automatically converges to 0.
\begin{proof}
The assumption says that the graph of $T$
\[
\Ga(T)=\set{(x,Tx)}{x\in X}
\]
is closed in $X\opl Y=\set{(x,y)}{x\in X,y\in Y}$ with norm, e.g.
\[
\ve{(x,y)}=\ve{x}+\ve y.
\]
So $\Ga(T)$ is a Banach space. Consider $U:\Ga(T)\to X$, $U(x,y)=x$. $U$ is a linear bijection and $\ve{U}\le 1$. From the Banach isomorphism theorem~\ref{thm-bit}, $U^{-1}$ is continuous, i.e., $x\mapsto (x,Tx)$ is continuous.
\end{proof}
We'll give three more applications. The first one is to quotient spaces. Let $X$ be a normed space and $Y$ be a closed subspace. Then $X/Y=\set{x+Y}{x\in X}$ is a normed space with 
\[
\ve{x+Y}=d(x,Y)=d(0,x+Y)=\inf\set{\ve{x+y}}{y\in Y}.
\]
We need $Y$ closed to ensure that if $\ve{z}=0$ then $z=0$ for $z\in X/Y$.
\begin{pr}\llabel{pr:5}%5
Let $X,Y$ be as above. If $X$ is complete, then so is $X/Y$.
\end{pr}
%this can be proved directly too.
\begin{proof}
Consider the quotient map $q:X\to X/Y$, $q(x)=x+Y$. This is a bounded linear map, $q\in \cal B(X,X/Y)$, so
\[
\ve{q(x)}=d(x,Y)\le \ve{x}
\]
and $q(B_X^{\circ})\subeq B_{X/Y}^{\circ}$. If $\ve{x+Y}<1$ then there exists $y\in Y$ with $\ve{x+y}<1$ and $q(x+y)=q(x)=x+Y$ so $q(B_X^{\circ})=B_{X/Y}^{\circ}$. So for any $M>1$, $\ol{q(MB_X)}\supeq B_{X/Y}$. %image of some large ball contains ball. 
By Lemma~\ref{lem:oml}, $X/Y$ is complete.
\end{proof}
\begin{pr}
Every separable Banach space $X$ is (isometrically isomorphic to) a quotient of $\ell_1$.
\end{pr}
\begin{proof}
Let $\set{x_n}{n\in \N}$ be dense in $B_X$. Let $(e_n)$ be the standard basis of $\ell_1$. If $a=(a_n)\in \ell_1$, then $a=\sum_{n=1}^{\iy} a_ne_n$. Define
\[
T:\ell_1\to X,\qquad T\pa{\sum_{n=1}^{\iy} a_ne_n}=\sum_{n=1}^{\iy}a_nx_n.
\]
This is well defined because the sum converges: $\sum_{n=1}^{\iy} \ve{a_nx_n}\le \suo |a_n|=\ve{a}$. So $T\in \cal B(\ell_1,X)$, $\ve{T}\le 1$. Also $T(B_{\ell_1}^{\circ})\subeq B_X^{\circ}$. 
We have $T(B_{\ell_1})\supeq \set{x_n}{n\in \N}$, so $\ol{T(B_{\ell_1})}\supeq B_X$.

%checkme
%This means given $x\in B_X^{\circ}$, there exists a sequence $rx_n\to rx\in B_X$, $rx_n\in  $\ve{rx}=1$. Then $x_n\to x\in B_X^{\circ}$. 
So $T(B_{\ell_1}^{\circ})=B_X^{\circ}$. We have a unique $\wt T:\ell_1/\ker T\to X$ such that
\[
\xymatrix{
\ell_1\ar[rr]^T\ar[rd]_q & & X\\
& \ell_1/\ker T\ar[ru]_{\wt T}&
}
\]
commutes, i.e., $T=\wt Tq$ where $q$ is the quotient map.
%onto from lemma~\ref{lem:oml}
Moreover, $\wt T$ is a linear bijection
\[
\wt T(B_{\ell_1/\ker T}^{\circ})=\wt T(q(B_{\ell_1}^{\circ}))=T(B_{\ell_1}^{\circ})=B_X^{\circ}.
\]
So $\wt T$ is an isometric isomorphism, and $X\cong \ell_1/\ker T$.
\end{proof}
This suggests that to understand all Banach spaces we just have to understand $\ell_1$. But $\ell_1$ is quite complicated, not as innocent as it looks.

Recall the following definition.
\begin{df}
A topological space $K$ is \textbf{normal} if whenever $E,F$ are disjoint closed sets in $K$, there are disjoint open sets $U,V$ in $K$ such that $E\subeq U$, $F\subeq V$. 
\end{df}
\begin{lem}[Urysohn's lemma]\llabel{lem:urysohn}
If $K$ is normal and $E,F$ are disjoint closed subsets of $K$, then $\exists$ continuous $f:K\to [0,1]$ such that $f=0$ on $E$ and $f=1$ on $F$.
\end{lem}
This can be used to construct partitions of unity which we'll use in chapter 3.
\begin{thm}[Tietze's Extension Theorem]
If $K$ is normal and $L$ is a closed subset of $K$, $g:L\to \R$ is bounded and continuous, then there exists a bounded, continuous $f:K\to \R$ such that $f|_L=g$, $\ve{f}_{\iy}=\ve{g}_{\iy}$. 
\end{thm}
\begin{proof}
Let $C_b(K)=\set{h:K\to \R}{h\text{ bounded, continuous}}$, a closed subspace of $\ell_{\iy}(K)$ in $\ved_{\iy}$ (so it is a Banach space). Consider $R:C_b(K)\to C_b(L)$, $R(f)=f|_L$. We need $R(B_{C_b(K)})=B_{C_b(L)}$ (``$\subeq$" is clear, since $\ve{R}\le 1$.) Let $g\in B_{C_b(L)}$ so $-1\le g\le 1$. Let $E=\set{y\in L}{g(y)\le -\rc{3}}$, $F=\set{y\in L}{g(y)\ge \rc 3}$. Urysohn's lemma gives $\exists f\in C_b(K)$ such that $-\rc 3\le f\le \rc 3$, $f=-\rc 3$ on $E$, and $f=\rc 3$ on $F$.

We have
\[
\ve{R(f)-g}_{\iy} \le \fc23 
\]
and $\ve f\le \rc 3$. So $R(\rc 3B_{C_b(K)})$ is $\fc 23$-dense in $B_{C_b(L)}$. By the Open Mapping Lemma~\ref{lem:oml}, $R$ is surjective.
\end{proof}
\begin{rem}
The theorem holds in the complex case too.
\end{rem}

%%%%%%%%%%%%%%%%%%%%%%%%%%%


 
%\bibliographystyle{plain}
%\bibliography{refs}
\end{document}