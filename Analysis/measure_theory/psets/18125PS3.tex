%%%This is a science homework template. Modify the preamble to suit your needs. 

\documentclass[12pt]{article}

\makeatother
%AMS-TeX packages
\usepackage{amsmath}
\usepackage{amssymb}
\usepackage{amsthm}
\usepackage{array}
\usepackage{amsfonts}
\usepackage{cancel}
\usepackage[all,cmtip]{xy}%Commutative Diagrams
\usepackage[pdftex]{graphicx}
\usepackage{float}
%geometry (sets margin) and other useful packages
\usepackage[margin=1in]{geometry}
\usepackage{sidecap}
\usepackage{wrapfig}
\usepackage{verbatim}
\usepackage{mathrsfs}
\usepackage{marvosym}
\usepackage{stmaryrd}
\usepackage{hyperref}
\usepackage{graphicx,ctable,booktabs}

\newtheoremstyle{norm}
{3pt}
{3pt}
{}
{}
{\bf}
{:}
{.5em}
{}

\theoremstyle{norm}
\newtheorem{thm}{Theorem}[section]
\newtheorem{lem}[thm]{Lemma}
\newtheorem{df}{Definition}
\newtheorem{rem}{Remark}
\newtheorem{st}{Step}
\newtheorem{pr}[thm]{Proposition}
\newtheorem{cor}[thm]{Corollary}
\newtheorem{clm}[thm]{Claim}

%Math blackboard, fraktur, and script commonly used letters
\newcommand{\A}[0]{\mathbb{A}}
\newcommand{\C}[0]{\mathbb{C}}
\newcommand{\sC}[0]{\mathcal{C}}
\newcommand{\cE}[0]{\mathscr{E}}
\newcommand{\F}[0]{\mathbb{F}}
\newcommand{\cF}[0]{\mathscr{F}}
\newcommand{\cG}[0]{\mathscr{G}}
\newcommand{\sH}[0]{\mathscr H}
\newcommand{\Hq}[0]{\mathbb{H}}
\newcommand{\cI}[0]{\mathscr{I}}%ideal sheaf
\newcommand{\N}[0]{\mathbb{N}}
\newcommand{\Pj}[0]{\mathbb{P}}
\newcommand{\sO}[0]{\mathcal{O}}
\newcommand{\cO}[0]{\mathscr{O}}
\newcommand{\Q}[0]{\mathbb{Q}}
\newcommand{\R}[0]{\mathbb{R}}
\newcommand{\Z}[0]{\mathbb{Z}}
%Lowercase
\newcommand{\ma}[0]{\mathfrak{a}}
\newcommand{\mb}[0]{\mathfrak{b}}
\newcommand{\fg}[0]{\mathfrak{g}}
\newcommand{\vi}[0]{\mathbf{i}}
\newcommand{\vj}[0]{\mathbf{j}}
\newcommand{\vk}[0]{\mathbf{k}}
\newcommand{\mm}[0]{\mathfrak{m}}
\newcommand{\mfp}[0]{\mathfrak{p}}
\newcommand{\mq}[0]{\mathfrak{q}}
\newcommand{\mr}[0]{\mathfrak{r}}
%Letter-related
\providecommand{\cal}[1]{\mathcal{#1}}
\renewcommand{\cal}[1]{\mathcal{#1}}
\newcommand{\bb}[1]{\mathbb{#1}}
%More sequences of letters
\newcommand{\chom}[0]{\mathscr{H}om}
\newcommand{\fq}[0]{\mathbb{F}_q}
\newcommand{\fqt}[0]{\mathbb{F}_q^{\times}}
\newcommand{\sll}[0]{\mathfrak{sl}}
%Shortcuts for symbols
\newcommand{\nin}[0]{\not\in}
\newcommand{\opl}[0]{\oplus}
\newcommand{\ot}[0]{\otimes}
\newcommand{\rc}[1]{\frac{1}{#1}}
\newcommand{\rra}[0]{\rightrightarrows}
\newcommand{\send}[0]{\mapsto}
\newcommand{\sub}[0]{\subset}
\newcommand{\subeq}[0]{\subseteq}
\newcommand{\supeq}[0]{\supseteq}
\newcommand{\nsubeq}[0]{\not\subseteq}
\newcommand{\nsupeq}[0]{\not\supseteq}
%Shortcuts for greek letters
\newcommand{\al}[0]{\alpha}
\newcommand{\be}[0]{\beta}
\newcommand{\ga}[0]{\gamma}
\newcommand{\Ga}[0]{\Gamma}
\newcommand{\de}[0]{\delta}
\newcommand{\De}[0]{\Delta}
\newcommand{\ep}[0]{\varepsilon}
\newcommand{\eph}[0]{\frac{\varepsilon}{2}}
\newcommand{\ept}[0]{\frac{\varepsilon}{3}}
\newcommand{\la}[0]{\lambda}
\newcommand{\La}[0]{\Lambda}
\newcommand{\ph}[0]{\varphi}
\newcommand{\rh}[0]{\rho}
\newcommand{\te}[0]{\theta}
\newcommand{\om}[0]{\omega}
\newcommand{\Om}[0]{\Omega}
\newcommand{\si}[0]{\sigma}
%Brackets
\newcommand{\ab}[1]{\left| {#1} \right|}
\newcommand{\ba}[1]{\left[ {#1} \right]}
\newcommand{\bc}[1]{\left\{ {#1} \right\}}
\newcommand{\pa}[1]{\left( {#1} \right)}
\newcommand{\an}[1]{\langle {#1}\rangle}
\newcommand{\fl}[1]{\left\lfloor {#1}\right\rfloor}
\newcommand{\ce}[1]{\left\lceil {#1}\right\rceil}
%Text
\newcommand{\btih}[1]{\text{ by the induction hypothesis{#1}}}
\newcommand{\bwoc}[0]{by way of contradiction}
\newcommand{\by}[1]{\text{by~(\ref{#1})}}
\newcommand{\ore}[0]{\text{ or }}
%Arrows
\newcommand{\hr}[0]{\hookrightarrow}
\newcommand{\xr}[1]{\xrightarrow{#1}}
%Formatting
\newcommand{\subprob}[1]{\noindent\textbf{#1}\\}
%Functions, etc.
\newcommand{\Ann}{\operatorname{Ann}}
\newcommand{\Arc}{\operatorname{Arc}}
\newcommand{\Ass}{\operatorname{Ass}}
\newcommand{\Aut}{\operatorname{Aut}}
\newcommand{\chr}{\operatorname{char}}
\newcommand{\cis}{\operatorname{cis}}
\newcommand{\Cl}{\operatorname{Cl}}
\newcommand{\Der}{\operatorname{Der}}
\newcommand{\End}{\operatorname{End}}
\newcommand{\Ext}{\operatorname{Ext}}
\newcommand{\Frac}{\operatorname{Frac}}
\newcommand{\FS}{\operatorname{FS}}
\newcommand{\GL}{\operatorname{GL}}
\newcommand{\Hom}{\operatorname{Hom}}
\newcommand{\Ind}[0]{\text{Ind}}
\newcommand{\im}[0]{\text{im}}
\newcommand{\nil}[0]{\operatorname{nil}}
\newcommand{\ord}[0]{\operatorname{ord}}
\newcommand{\Proj}{\operatorname{Proj}}
\newcommand{\Rad}{\operatorname{Rad}}
\newcommand{\rank}{\operatorname{rank}}
\newcommand{\Res}[0]{\text{Res}}
\newcommand{\sign}{\operatorname{sign}}
\newcommand{\SL}{\operatorname{SL}}
\newcommand{\Spec}{\operatorname{Spec}}
\newcommand{\Specf}[2]{\Spec\pa{\frac{k[{#1}]}{#2}}}
\newcommand{\spp}{\operatorname{sp}}
\newcommand{\spn}{\operatorname{span}}
\newcommand{\Supp}{\operatorname{Supp}}
\newcommand{\Tor}{\operatorname{Tor}}
\newcommand{\tr}[0]{\text{trace}}
\newcommand{\Var}{\operatorname{Var}}
\newcommand{\vol}[0]{\operatorname{vol}}
%Commutative diagram shortcuts
\newcommand{\fiber}[3]{\xymatrix{#1\times_{#3} #2}\ar[r]\ar[d] #1\ar[d] \\ #2 \ar[r] & #3}
\newcommand{\commsq}[8]{\xymatrix{#1\ar[r]^{#6}\ar[d]^{#5} &#2\ar[d]^{#7} \\ #3 \ar[r]^{#8} & #4}}
%Makes a diagram like this
%1->2
%|    |
%3->4
%Arguments 5, 6, 7, 8 on arrows
%  6
%5  7
%  8
\newcommand{\pull}[9]{
#1\ar@/_/[ddr]_{#2} \ar@{.>}[rd]^{#3} \ar@/^/[rrd]^{#4} & &\\
& #5\ar[r]^{#6}\ar[d]^{#8} &#7\ar[d]^{#9} \\}
\newcommand{\back}[3]{& #1 \ar[r]^{#2} & #3}
%Syntax:\pull 123456789 \back ABC
%1=upper left-hand corner
%2,3,4=arrows from upper LH corner, going down, diagonal, right
%5,6,7=top row (6 on arrow)
%8,9=middle rows (on arrows)
%A,B,C=bottom row
%Other
%Other
\newcommand{\op}{^{\text{op}}}
\newcommand{\fp}[1]{^{\underline{#1}}}
\newcommand{\rp}[1]{^{\overline{#1}}}
\newcommand{\rd}[0]{_{\text{red}}}
\newcommand{\pre}[0]{^{\text{pre}}}
\newcommand{\pf}[2]{\pa{\frac{#1}{#2}}}
\newcommand{\pd}[2]{\frac{\partial #1}{\partial #2}}
\newcommand{\bs}[0]{\backslash}
\newcommand{\sia}[0]{ $\si$-algebra}
\newcommand{\ol}[1]{\overline{#1}}
\newcommand{\iy}[0]{\infty}
\newcommand{\nl}[1]{\left \Vert #1\right \Vert_{L^1}}
%Matrices
\newcommand{\coltwo}[2]{
\left[
\begin{matrix}
{#1}\\
{#2} 
\end{matrix}
\right]}
\newcommand{\matt}[4]{
\left[
\begin{matrix}
{#1}&{#2}\\
{#3}&{#4}
\end{matrix}
\right]}
\newcommand{\smatt}[4]{
\left[
\begin{smallmatrix}
{#1}&{#2}\\
{#3}&{#4}
\end{smallmatrix}
\right]}
\newcommand{\colthree}[3]{
\left[
\begin{matrix}
{#1}\\
{#2}\\
{#3}
\end{matrix}
\right]}
%
%Redefining sections as problems
%
\makeatletter
\newenvironment{problem}{\@startsection
       {section}
       {1}
       {-.2em}
       {-3.5ex plus -1ex minus -.2ex}
       {2.3ex plus .2ex}
       {\pagebreak[3]%forces pagebreak when space is small; use \eject for better results
       \large\bf\noindent{Problem }
       }
       }
       {%\vspace{1ex}\begin{center} \rule{0.3\linewidth}{.3pt}\end{center}}
       }
\makeatother


%
%Fancy-header package to modify header/page numbering 
%
\usepackage{fancyhdr}
\pagestyle{fancy}
%\addtolength{\headwidth}{\marginparsep} %these change header-rule width
%\addtolength{\headwidth}{\marginparwidth}
\lhead{Problem \thesection}
\chead{} 
\rhead{\thepage} 
\lfoot{\small\scshape 18.125 Real and Functional Analysis} 
\cfoot{} 
\rfoot{\scriptsize PS \# 3} % !! Remember to change the problem set number
\renewcommand{\headrulewidth}{.3pt} 
\renewcommand{\footrulewidth}{.3pt}
\setlength\voffset{-0.25in}
\setlength\textheight{648pt}



%%%%%%%%%%%%%%%%%%%%%%%%%%%%%%%%%%%%%%%%%%%%%%%
%
%Contents of problem set
%    
\begin{document}
\title{18.125 Real and Functional Analysis PSet \#3}% !! Remember to change the problem set number
\author{Holden Lee}
\date{3/10/11}% !! Remember to change the date
\maketitle
\thispagestyle{empty}
\begin{problem}{\it (2.2.36, Cauchy's Equation)}
\subprob{(i)}
Note $f(0x)=0$ (since $f(0)+f(0)=f(0+0)=f(0)$), $f(x)=x$, and if $f(nx)=nf(x)$, then %for all $x$, and a given $n\in \N$, then
\[
f((n+1)x)=f(nx)+f(x)=nf(x)+f(x)=(n+1)f(x).
\]
Hence by induction, $f(nx)=nf(x)$ for all $n\in \N_0$. Now
\[
f(-nx)+f(nx)=f(-nx+nx)=f(0)=0
\]
so $f(-nx)=-f(nx)=-nf(x)$, for all $n\in \N_0$. This shows $f(nx)=nf(x)$ for all integers $n$.

Now suppose $q\in \Q$. Write $q=\frac mn$ with $m,n\in \Z$. Then
\[
nf(qx)=f(nqx)=f(mx)=mf(x)\implies f(qx)=\frac mn f(x)=qf(x).
\]

%Now suppose $\frac{f(x)}{x}\neq \frac{f(y)}y$ for some $x,y\neq 0$. Since the rationals are dense in the reals, for any $\ep>0$ there exists $q\in \Q$ such that $|qx-y|<\ep$. Then
%\[
%f(qx-y)=f(qx)-f(y)=qf(x)-f(y)=qx\pf{f(x)}{x}-y\pf{f(y)}{y}.
%\]
%As $\ep\to 0$, this quantity approaches $y\pa{\pf{f(x)}{x}-\pf{f(y)}{y}}\ne 0$. However, $f(0)=0$, so $f$ is not continuous. Thus, if $f$ is continuous, then $\frac{f(x)}{x}$ is constant, i.e. $f$ is linear.\\
Now $f(q)=qf(1)$ for every $q\in \Q$. Since $f$ is continuous and $f(x)=xf(1)$ for $x$ in a dense set, we conclude $f(x)=xf(1)$.\\

\subprob{(ii)}
If $f$ is bounded on some open set, then it must be bounded on some open interval $(c-d,c+d)$. Suppose $|f(x)|<L$ on this interval. Given $\ep>0$, choose $N\in \N$ so that $N> \frac{2L}{\ep}$ and $N>\frac{2d}{L}$.
%let $\de=\min\pa{\frac{b-a}{2L},\frac{b-a}{2}}$. 
If $|x|<\frac{2d}{N}$, then $\frac{Nx}{2}<d$, so $c-Nx/2,c+Nx/2\in (c-d,c+d)$. Then
\begin{align*}
|f(x)|&=\ab{\rc N f(Nx)}\\
&=\rc N|f(c+Nx/2)-f(c-Nx/2)|\\
&\le \rc N\pa{|f(c+Nx/2)|+|f(c-Nx/2)|}\\
&\le \rc N(2L)<\ep.
\end{align*}
Since $f(0)=0$, $f$ is continuous at 0. Take any $x_0\in \R$. Since $f(x)=f(x_0)+f(x-x_0)$ and $f(x-x_0)$ is continuous at $x_0$, $f(x)$ is continuous at $x_0$.
Thus $f$ is continuous. By (i), $f$ is linear.\\

\subprob{(iii)}
Suppose $f$ is $\ol{\cal B_{\R}}^{\la_\R}$-measurable and additive. Now
\[
\bigcup_{N=1}^{\iy} f^{-1}([-N,N])=\R;
\]
$f^{-1}([-N,N])$ is a sequence of measurable sets increasing to $\R$. Thus
\[
\lim_{n\to \iy} f^{-1}([-N,N])=\la_{\R}(\R)=\iy.
\]
In particular, there exists $R$ so that $\Ga=f^{-1}([-R,R])$ has positive measure. 
By Vitalli's Lemma (2.2.16), there exists $\de>0$ so that $(-\de,\de)\subeq \Ga-\Ga$.
Given $|x|<\de$, write $x=y-z$ where $y,z\in \Ga$. Then
\[
|f(x)|=|f(y)-f(z)|\le |f(z)|+|f(y)|\leq 2R.
\]
Hence $f$ is bounded on $(-\de,\de)$. By (ii), $f$ is additive.
\end{problem}
\begin{problem}{\it (3.1.13, Measurable functions)}
If $f$ is measurable, then $f^{-1}((a,\iy])$ and $f^{-1}([a,\iy])$ are both measurable, as $(a,\iy]$ and $[a,\iy]$ are open and closed, respectively, and hence measurable.

For the converse, let $S$ be a dense set of $\R$. We claim that 
\begin{equation}\label{allgen}
\si(\{(a,\iy]:a\in S\})=\si(\{[a,\iy]:a\in S\})=\cal B_{\R}.
\end{equation}
Clearly $A:=\si(\{(a,\iy]:a\in S\})$ and $B:=\si(\{[a,\iy]:a\in S\})$ are subsets of $B_{\R}$.
It suffices to check that all open sets are in these two sets, as $\cal B_{\R}$ is generated by open sets. 
Since every open set is the union of a countable number of open intervals and possibly $\{\iy\},\{-\iy\}$, it suffices to show $A$ and $B$ contain open intervals $(a,b)$, $a,b\in \R$, and $\{\iy\},\{-\iy\}$.
Since $S$ is dense in $\R$, there exists a sequence of points of $S$ with $a_n\searrow a$ and a sequence of points of $S$ with $b_n\nearrow b$, with $a_n\neq a$ for any $n$ and $b_n\neq b$ for any $b$. 
Then
\[
(a,b)=(a,\iy]\bs [b,\iy]=\left.\bigcup_{n=1}^{\iy} (a_n,\iy]\right\backslash\bigcap_{n=1}^{\iy} (b_n,\iy]
=
\left.\bigcup_{n=1}^{\iy} [a_n,\iy]\right\backslash\bigcap_{n=1}^{\iy} [b_n,\iy]
\]
so $(a,b)\in A,B$. Note $\{\iy\}\in \cal A,B$ since choosing $a_n\to \iy$,
\[
\bigcap_{n=1}^{\iy} (a_n,\iy]=\{\iy\}.
\]
Similarly, choosing $a_n\to -\iy$,
\[
\pa{\bigcup_{n=1}^{\iy} (a_n,\iy]}^c=
\{-\iy\}
\]
and similarly with half-open replaced by closed intervals. 
This proves the claim. Because $f^{-1}$ preserves unions and complements,
\[
\{S\subeq \ol{\R}: f^{-1}(S)\text{ measurable}\}
\]
is a \sia. Since it contains either $A$ or $B$, by~(\ref{allgen}) it contains $\cal B_{\R}$, and $f$ is measurable. The proof for $(a,\iy]$ and $[a,\iy]$ replaced by $[-\iy,a)$, $[-\iy,a]$ is the same, but with all intervals reversed and $\iy$ replaced with $-\iy$.

Since $f$ is measurable, $f\times g:\cal B\to \R^2$ is measurable. It suffices to note
\begin{align*}
\{f<g\}&=(f\times g)^{-1}(\{(x,y):x<y\})\\
\{f\le g\}&=(f\times g)^{-1}(\{(x,y):x\le y\})\\
\{f= g\}&=(f\times g)^{-1}(\{(x,y):x= y\})\\
\{f\ne g\}&=(f\times g)^{-1}(\{(x,y):x\ne y\})
\end{align*}
and the sets in the arguments of the RHS are open, closed, closed, and open, respectively, so measurable.
%Similarly, choosing $a_n\nearrow a$ and $b_n\searrow b$
%\[
%(a,b)=(a,\iy]-\bs [b,\iy]=\left.\bigcap_{n=1}^{\iy} [a_n,\iy]\right\backslash\bigcup_{n=1}^{\iy} (b_n,\iy].
%\]
%\]
\end{problem}
\begin{problem}{\it (3.1.14, Measure defined by integral)}
Clearly, $\mu(\phi)=\phi$, so it suffices to check that if $\{\Ga_n:n\in \N\}$ is a sequence of pairwise disjoint measurable sets then
\[
\mu\pa{\bigcup_{n=1}^{\iy} \Ga_n}
=\sum_{n=1}^{\iy} \mu(\Ga_n).
\]
Note that
\[
\sum_{n=1}^m f1_{\Ga_n}\nearrow f1_{\Ga}
\]
pointwise: indeed, given $x\in \Ga$, $x\in \Ga_{n}$ for exactly one $n$, so $(f1_{\Ga_i})(x)=f(x)$ for $i=n$ and $0$ for $i\neq n$. Then by the Monotone Convergence Theorem,
\begin{align*}
\mu(\Ga)&=\int_{\Ga} f\,d\nu\\
&=\lim_{m\to \iy} \int_{\Ga}\sum_{n=1}^m f1_{\Ga_n}\,d\nu\\
&=\lim_{m\to \iy} \sum_{n=1}^m \int_{\Ga} f1_{\Ga_n}\,d\nu&\text{additivity}\\
&=\sum_{n=1}^{\iy} \int_{\Ga_n} f1_{\Ga_n}\,d\nu\\
&=\sum_{n=1}^{\iy} \int_{\Ga_n} f\,d\nu\\
&=\sum_{n=1}^{\iy} \mu(\Ga_n)
\end{align*}
as needed.

Suppose that $\nu(B)=0$. For any simple function $\ph:B\to \R$,
\[
\mu(B)=\int_B \ph\,d\nu=\sum_{y\in \ph(B)} \cancelto{0}{\mu(\ph^{-1}(y))}=0
\]
since $\ph^{-1}(y)\subeq B$ and $\nu(B)=0$.
Hence by definition of $\int_B f\,d\nu$ as a limit of integrals of simple functions, $\mu(B)=\int_Bf\,d\nu=0$. 
Thus $\mu$ is absolutely continuous with respect to $\nu$.
\end{problem}
\begin{problem} {\it (3.1.15, Integrable function determined by integrals)}
Suppose $\mu(f>g)=0$. Then
\[
\int_{\Ga}(f-g)\,d\mu
\le \int_{\{f\le g\}\cap \Ga} (f-g)\,d\mu +\int_{\{f>g\}\cap \Ga}  f-g\,d\mu\le 0
\]
(We showed $\{f>g\}$ is measurable in 3.1.13; its complement is also measurable.) Note the first integral is nonpositive since $f-g\le 0$ on that set, and the second integral is 0 because $\mu(\{f>g\}\cap \Ga)\le \mu(f>g)=0$. Hence $\int_{\Ga}f\,d\mu\leq \int_{\Ga} g\,d\mu$.

Conversely, suppose $\int_{\Ga}f\,d\mu\leq \int_{\Ga} g\,d\mu$ for every $\Ga\in \cal B$. Take $\Ga=\{f>g\}$. Then
\[
\int_{\{f>g\}}f\,d\mu\leq \int_{\{f>g\}} g\,d\mu
\implies
\int_{\{f>g\}}(f-g)\,d\mu\le 0.
\]
This implies $\mu(f-g>0)=0$ by Markov's inequality (3.1.6).

Thus $\mu(f>g)=0$ iff $\int_{\Ga}f\,d\mu\le \int_{\Ga}g\,d\mu$ for every $\Ga\in \cal B$. Reversing $f$ and $g$, $\mu(f<g)=0$ iff $\int_{\Ga}f\,d\mu\ge \int_{\Ga}g\,d\mu$ for every $\Ga\in \cal B$. Putting these two together, $\mu(f\ne g)=0$ iff $\mu(f<g)=\mu(f>g)=0$ iff $\int_{\Ga}f\,d\mu\ge \int_{\Ga}g\,d\mu$ for every $\Ga\in \cal B$.
\end{problem}
\begin{problem}{\it(3.1.16, Law of large numbers)}
\subprob{(i)}
Note there are $\binom nm$ binary sequences $a_1,\ldots, a_n$ with $m$ 1's and $n-m$ 0's. Hence
\[
\be_p\pa{\sum_{i=1}^n \om(i)=m}
=\sum_{\scriptsize
\begin{array}{c}
a_1,\ldots, a_n\in \{0,1\}\\
a_1+\cdots +a_n=m
\end{array}}
\be_p((\om(1),\ldots, \om(n))=(a_1,\ldots, a_n))=\binom nm p^mq^{n-m}.
\]
Hence, since $e^{\la\sum_{i=1}^n (\om(i)-p)}$ attains the value $e^{\la(m-pn)}$ when $\sum_{i=1}^n\om(i)=m$, and the set of $\om$ satisfying the latter has measure $\binom nm p^mq^{n-m}$,
\begin{align*}
\int e^{\la\sum_{i=1}^n (\om(i)-p)}\be_p(d\om)
&=\sum_{m=0}^n\binom nm p^mq^{n-m} e^{\la m-\la pn}\\
&=\sum_{m=0}^n \binom nm(e^{\la q}p)^m (qe^{-\la p})^{n-m}\\
&=(pe^{\la q}+qe^{-\la p})^n&\text{(Binomial theorem)}.
\end{align*}

Now
consider
\[f(\la)=\ln(pe^{\la q}+qe^{-\la p})=\ln((p^{\la}+q)e^{-p\la})=\ln(pe^{\la}+q)-p\la.\]
Differentiating,
\begin{align*}
f'(\la)&=\frac{pe^{\la}}{pe^{\la}+q}-p=\frac{-q}{pe^{\la}+q}+1-p\\
f''(\la)&=\frac{q}{(pe^{\la}+q)^2}\cdot pe^{\la}=\frac{pqe^{\la}}{(pe^{\la}+q)^2}.
\end{align*}
We calculate $f(0)=\ln(p+q)=\ln 1=0$ and $f'(0)=\frac{p}{1}-p=0$.
Now
\begin{align*}
f''(\la)=\frac{pqe^{\la}}{(pe^{\la}+q)^2}
&\le \rc 4\\
\iff
4pqe^{\la}&\le p^2e^{2\la}+2pqe^{\la}+q^2\\
\iff 0&\le (pe^{\la}-q)^2
\end{align*}
which is true. Hence $f''(\la)\le \rc 4$. For $\la\ge 0$, this gives $f'(\la)\le f'(0)+\rc 4\la=\rc 4 \la$ and $f(\la)\le f(0)+\rc 8 \la^2=\frac{\la^2}{8}$. Hence
\[(pe^{\la q}+qe^{-\la p})^n \le (e^{f(\la)})^n\le e^{\frac{n\la^2}{8}},\]
showing and the integral estimate follows.\\

\subprob{(ii)}
By Markov's inequality, since the integrand is nonnegative,
\[
\be_p\pa{e^{\la
\sum_{i=1}^n (\om(i)-p)}\ge e^{\la R}}
\le \frac{\int e^{\la\sum_{i=1}^n (\om(i)-p)}\be_p(d\om)}{e^{\la R}}
\le e^{-\la R+\frac{n\la^2}{8}}.
\]
The left-hand side is just $\be_p\pa{\sum_{i=1}^n (\om(i)-p)\ge R}$.

Now take $\la=\frac{4R}{n}$ to get $-\la R+\frac{n\la^2}{8}=\frac{-2R^2}{n}$, and
\begin{equation}\label{deviation}
\be_p\pa{
\sum_{i=1}^n (\om(i)-p)\ge R
}\le e^{-\frac{2R^2}{n}}.
\end{equation}

Note the map sending $\om$ to $\om'$ with $\om'(i)=1-\om(i)$ is a measure-preserving map from 
$(\Om, \cal B_{\Om}, \be_p)$ to $(\Om, \cal B_{\Om}, \be_q)$. Hence
\begin{align*}
\be_p\pa{\sum_{i=1}^n (\om(i)-p)\le -R}&=
\be_q\pa{\sum_{i=1}^n (\om'(i)-p)\le -R}\\
&=\be_q\pa{\sum_{i=1}^n((1-\om(i))-(1-q))\le -R}\\
&=\be_q\pa{\sum_{i=1}^n(\om(i)-q)\ge R}\\
&\le e^{-\frac{2R^2}{n}}.
\end{align*}
where in the last step we used~(\ref{deviation}) with $q$ instead of $p$.
Replacing $R$ with $nR$, we get
\begin{align}
\nonumber
\be_p\pa{
\ab{\rc n
\sum_{i=1}^n \om(i)-p
}\ge R
}
&=
\be_p\pa{
\ab{
\sum_{i=1}^n \om(i)-p
}\ge nR
}\\
\nonumber&= \be_p\pa{
\sum_{i=1}^n (\om(i)-p)\le -nR
}+\be_p\pa{\sum_{i=1}^n (\om(i)-p)\ge nR}\\
\label{devovn}&\le 2e^{-n2R^2}.
\end{align}

\subprob{(iii)}
Let $A_x=\bc{\om:
\ab{\rc{x^2}\sum_{i=1}^{x^2} \om(i)-p}\ge Cx^{-\rc 2}
}$.
Putting $n=x^2$ and $R=Cn^{-\rc 4}$ in~(\ref{devovn}) gives
\[
\be_p(A_x)\le 2e^{-2c^2x}
\]
Note
\[
\sum_{x=1}^{\iy} \be_p(A_x)\le \sum_{x=1}^{\iy}e^{-2c^2x}<\iy,
\] 
the sum being a convergent geometric series. Then by the Borel-Cantelli Lemma, $\be_p(\varlimsup_{x\to \iy} A_x)=0$, so $\be_p(\varliminf_{x\to \iy} A_x^c)=0$, i.e. 
\begin{equation}\label{p3-5-1}
\be_p\pa{
\exists x\,\forall y\ge x,\ab{\rc{y^2}\sum_{i=1}^{y^2}\om(i)-p}\le Cy^{-\rc{2}}
}=1.
\end{equation}
We claim that for appropriate choice of $C$,
\begin{equation}\label{p3-5-2}
\exists x\,\forall y\ge x,\ab{\rc{y^2}\sum_{i=1}^{y^2}\om(i)-p}\le Cy^{-\rc{2}}\implies 
\exists m\,\forall n\ge m,\ab{\rc{n}\sum_{i=1}^n\om(i)-p}\le Cn^{-\rc{4}}.
\end{equation}
This will show that the measure of the latter set is 1 (as it is squeezed between the measure of the former and the measure of $\Om$ which is also 1).

Indeed, suppose $x$ is such that the first statement is true. %We may suppose $x\ge 2$. 
Take $m=x^2$. Given $n\ge m$, let $a=\fl{\sqrt n}$. %Note $n-a^2<(\fl n+1)^2-\fl n^2=2\fl n+1\le 2n+1$, so $\sum_{i=a^2+1}^n (\om(i)-p)\le n-a^2-1< 2n$.
Note $n<(a+1)^2$. 
Then by assumption
\begin{align*}
\ab{\rc{a^2}\sum_{i=1}^{a^2} \om(i)-p}&\ge Ca^{-\rc 2}\\
\implies \ab{\sum_{i=1}^{a^2} (\om(i)-p)}&\ge Ca^{\frac 32}\\
\implies \ab{\sum_{i=1}^n (\om(i)-p)}&\ge Ca^{\frac 32}\\
\implies \ab{\rc{n}\sum_{i=1}^n \om(i)-p}&\ge \frac{Ca^{\frac 32}}{(a+1)^2}\geq \frac{C}{4}a^{-\rc2}\ge \frac C4 n^{-\rc4}
\end{align*}
where we used $\pf{a}{a+1}^2\geq \rc 4$. Taking $C=4$ gives this at most $n^{-\rc 4}$, as we wanted. This shows~(\ref{p3-5-2})
%; together with~(\ref{p3-5-1}) we get
%\[
%\be_p\pa{
%\exists m\,\forall n\ge m,\ab{\rc{n}\sum_{i=1}^n(\om(i)-p)}\le Cn^{-\rc{4}}
%}=1
%\]
as needed.
\end{problem}
\begin{problem}{\it(3.1.17)}
If $\ab{\rc n\sum_{i=1}^n \om(i)-p}\le n^{-\rc 4}$ for all $n\ge m$, then $\ab{\rc n\sum_{i=1}^n \om(i)}\to p$ as $n\to\iy$. Thus
\[
\bc{\om:
\exists m\,\forall n\ge m,\ab{\rc{n}\sum_{i=1}^n(\om(i)-p)}\le Cn^{-\rc{4}}}\subeq \Om_p\subeq \Om.
\]
Since the left-hand side and right-hand side both have $\be_p$-measure 1, so does $\Om_p$. Suppose $p\neq p'$. If $\rc{n}\sum_{i=1}^n \om(i)\to p$ as $n\to \iy$, then $\rc{n}\sum_{i=1}^n \om(i)\not\to p'$. % for $p'\neq p$. 
Hence $\Om_{p'}\subeq \Om_{p}^c$. But $\be_p(\Om_{p}^c)=1-\mu(\Om_p)=0$ so $\be_p(\Om_{p'})=0$. Now $\be_{p'}(\Om_p)=0$ and $\be_p(\Om_p^c)=1-\be_p(\Om_p)=0$ so $\be_p\perp \be_{p'}$.

By (2.2.28), $\Phi_*\be_{\rc 2}=\la_{[0,1]}$. Note 
\begin{enumerate}
\item
$\Omega_{p'}\subeq \hat{\Omega}$ since if $\lim_{n\to \iy} \rc{n}\sum_{i=1}^n \om(i)=p'<1$ then $\om$ must have infinitely many 0's.
%since a sequence ending in infinitely many 1's would have $\lim_{n\to \iy}\rc n\sum_{i=1}^n \om(i)=1\ne p'$. 
\item $B_{p'}=\Phi(\Om_{p'})$.
\item $\Phi^{-1}(\Phi(\Om_{p'}))=\Om_{p'}$ since the only numbers in $[0,1]$ which are the image of more than 1 element in $\Om$ are the ones whose binary expansion terminates (i.e. ends in all 0's), and $\om\in \Om_{p'}$ cannot end in all 0's as $\lim_{n\to \iy} \rc n\sum_{i=1}^n \om(i)=p'>0$.
\end{enumerate}
Then
\begin{align*}
\mu_p(B_{p'})&=\Phi_*\be_p(B_{p'})\\
&=\be_p(\Phi^{-1}(\Phi(\Om_{p'})))\\
&=\be_p(\Om_{p'})\\
%\end{align*}
%Since $\Phi$ is injective on $\hat{\Omega}$,
%\[
%\Om_{p'}\subeq
%\Phi^{-1}(\Phi(\Om_{p'}))
%\subeq \Om_{p'}\cup (\Om\bs \hat{\Om}).
%\]
%However, the last term is countable: For any $n$ there are a finite number of $\om$ such that $\om(n)=0$ and $1=\om(n+1)=\om(n+2)=\cdots$, and we are taking the union over a countable number of $n\in \N$. Hence 
%\[
%\be(\Phi^{-1}(\Phi(B_{p'})))
%=\be(\Om_{p'})
=\begin{cases}
1,&p=p'\\
0,&p\ne p'.
\end{cases}
\end{align*}
Thus for $p\ne p'$, $\mu_p(B_{p'})=0$. Also $\mu_{p'}(B_{p'}^c)=1-\mu_{p'}(B_{p'})=0$. Thus $\mu_p\perp \mu_{p'}$.

Note $F_p$ is the distribution function for $\mu_p$, if we extend $\mu_p$ to $\R$ by decreeing that $(-\iy,0)\cup (1,\iy)$ has measure 0. Since $(0,x^+\wedge 1]\subeq (0,y^+\wedge 1]$ for $x\le y$, $F_p(x)$ is nondecreasing. %Consider $F_p$ and $\la_{\R}$ as a measure on $[0,1]$.
 %By Problem, 2.2.39 $F_p=\mu_F$, so $F_p$ is singular to $\la_{\R}$ iff $F$ is a singular function. 
For $p\neq \rc 2$, $\mu_p$ is singular to $\mu_{\rc 2}=\la_{[0,1]}=\la_{\R}|_{[0,1]}$. Hence by Problem 2.2.39, $F_p=\mu_F$ is singular. 

Now we show $F$ is strictly increasing on $[0,1]$. Let $0\le x<y\le 1$ be given. There exists an interval inside $(x,y)$ in the form $(a2^{-n},(a+1)2^{-n})$ where $n\in \N$, $a\in \N_0$ and $0\le a< 2^n$. Now, since $F$ is nondecreasing,
\begin{align*}
F(y)-F(x)&\ge F_p((a+1)2^{-n})-F_p(a2^{-n})\\
&\ge \mu_p((0,(a+1)2^{-n}))-\mu((0,a2^{-n}))\\
&\ge \mu_p([a2^{-n},(a+1)2^{-n})).
\end{align*}
Now there exists exactly one binary sequence of $n$ terms so that
\begin{equation}\label{binex}
\sum_{m=1}^n\frac{a_m}{2^m}=a2^{-n}.
\end{equation}
Since $\sum_{m=n+1}^{\iy} a_{m}{2^{-m}}\in [0, 2^{-n})$ for any binary sequence $a_{n+1},\cdots$ (except the all 1's sequence which contributes 0 measure), we get that 
\begin{align*}
\mu_p([a2^{-n},(a+1)2^{-n}))&=\be_p(\Phi^{-1}([a2^{-n},(a+1)2^{-n})))\\
&=\be_p(\{\om:\om(m)=a_m\text{ for }1\le m\le n\})\\
&=p^{\sum_{m=1}^n a_m}q^{n-\sum_{m=1}^n a_m}>0.
\end{align*}
Hence $F(y)-F(x)>0$.%%%%SLEEPY

We show that $F$ is continuous. Suppose $y>x$ is close to $x$. Choose the smallest interval in the form $[a2^{-n},(a+1)2^{-n}]$ containing both $y$ and $x$. Again choosing the $a_m$ as in~(\ref{binex}),
\begin{align*}
F(y)-F(x)&\le F((a+1)2^{-n})-F(a2^{-n})\\
&=\mu_p([a2^{-n},(a+1)2^{-n}))\\
%&=\be_p(\Phi^{-1}([a2^{-n},(a+1)2^{-n}))\\
&=p^{\sum_{m=1}^n a_m}q^{n-\sum_{m=1}^n a_m}\\
&\le\max(p,q)^n.
\end{align*}
As $y\to x$, the intervals shrink and $n\to \iy$, so the RHS goes to 0. Since $F(y)\ge F(x)$, this shows that $F$ is right continuous. Letting $y\nearrow x$, we proceed in exactly the same way with $F(x)-F(y)$ to get $F$ is left continuous.
\end{problem}
\begin{problem}{\it(3.2.16, Convergence in $\mu$-measure)}
Suppose $f_n\to f$ in $\mu$-measure. 
%Then
%\[
%\mu(|f_n-f|>\ep)\to 0\text{ as }n\to \iy \text{ for every }\ep>0.
%\]
For $0<\ep\le1$,
\begin{align*}
\int|f_n-f|\wedge 1\,d\mu
&=\int_{|f_n-f|\ge \ep} |f_n-f|\wedge 1\,d\mu
+\int_{|f_n-f|<\ep} |f_n-f|\wedge 1 \,d\mu\\
&\le  \int_{|f_n-f|\ge \ep} 1\,d\mu
+\int_{|f_n-f|<\ep} \ep \,d\mu\\
&\le \mu(|f_n-f|\ge \ep)+\ep \mu(E).
\end{align*}
%Suppose $f_n\to f$ in $\mu$-measure.
Given $\ep'>0$ we can let $\ep=\frac{\ep'}{2\mu(E)}$ and choose $N$ so that $\mu(|f_n-f|\ge \ep)<\frac{\ep'}{2}$ for $n\ge N$ (since $f_n\to f$ in $\mu$-measure). Then for $n\ge N$, $\int |f_n-f|\wedge 1 \,d\mu <\ep'$.  
%As $\ep\to 0$, both terms go to zero. (The first term goes to 0 because $f_n\to f$ in $\mu$-measure.) 
Hence $\int |f_n-f|\wedge1\,d\mu\to 0$ as $n\to\iy$.

Conversely, suppose $\int |f_n-f|\wedge1\,d\mu\to 0$ as $n\to\iy$. For fixed $0<\ep\le1$, 
\begin{align*}
\int |f_n-f|\wedge 1\,d\mu
&\ge \int_{|f_n-f|\ge \ep} |f_n-f|\wedge 1\,d\mu\\
&\ge \int_{|f_n-f|\ge \ep} \ep\,d\mu\\
&\ge  \ep\mu(|f_n-f|\ge\ep).
\end{align*}
As $n\to \iy$, $\int|f_n-f|\wedge 1\,d\mu\to 0$, so we must have $\ep\mu(|f_n-f|\ge \ep)\to 0$, and $\mu(|f_n-f|\ge \ep)\to 0$. This holds for all $0<\ep\le1$ (so {\it a fortiori} for $\ep>1$), so $f_n\to f$ in $\mu$-measure.

\end{problem}
\begin{problem}{\it(3.2.18, Riemann vs. Lebesgue integral)}
%Let $S_n$ be the set of rectangles of the form
\subprob{(i)}
Let $J=[a_1,b_1]\times \cdots \times [a_N,b_N]$. 
Fix $n$. Let $\de_m=\frac{b_m-a_m}{2^n}$ and define
%\[
%R(k_1,\ldots, k_N)=I_1(k_1)\times I_N(k_N)
%%[a_1+k\de_1,a_1+(k+1)\de_1]\times \cdots \times [a_N+k\de_N,a_N+(k+1)\de_N]
%\]
%where 
\[
I_m(k)=[a_m+k\de_m,a_m+(k+1)\de_m].
\]
Let
\[
I_m'(k)=\begin{cases} [a_m+k\de_m,a_m+(k+1)\de_m),&0\le m< 2^n-1\\
[a_m+k\de_m,a_m+(k+1)\de_m],&m= 2^n-1.\end{cases}
\]
%where $\de_m=\frac{b_m-a_m}{2^n}$ and $1\leq k\leq 2^n$. For a rectangle of the above form, let $R'=[a_1+k\de_1,a_1+(k+1)\de_1)\times \cdots \times [a_N+k\de_N,a_N+(k+1)\de_N)$.
Let
\[
S_n=\{I_1(k_1)\times \cdots \times I_N(k_N):0\le k_m<2^n\}.
\]
For $R=I_1(k_1)\times \cdots \times I_N(k_N)\in S_n$, denote by $R'$ the rectangle $I_1'(k_1)\times \cdots \times I_N'(k_N)$. (We define the $R'$ so that they do not overlap.)
Let
\[g_n(x)=\inf_R f(x)\text{ for }x\in R'.\]
Note that $g_n$ are simple functions such that $g_n\nearrow g$. Indeed, fixing $x$, $g_n(x)$ is increasing, because 
if $x\in R_n',R_{n+1}'$ and $R_n'\in S_n$, $R'_{n+1}\in S_{n+1}$, then 
$R_n\supeq R_{n+1}$, so
%the rectangle $R_{n+1}'$ containing $x$ with $R_{n+1}\in S_{n+1}$  is contained inside the rectangle in $R_n',R_n\in S_n$ containing $x$, and 
the infimum of $f$ over $R_{n+1}$ is at least the infimum over $R_n$. The $g_n$ converge to $g$, because the diameter of the rectangles goes to 0 and $f$ is continuous.

Now the Riemann integral is the limit of the lower Riemann sum, so
\begin{align*}
(R)\int_J f(x)\,dx&=\lim_{\Vert \cal C\Vert\to 0} \sum_{R\in \cal C}\vol(R)\inf_R f(x)\\
&=\lim_{n\to \iy}\sum_{R\subeq S_n} \vol(R')\inf_R f(x)&\vol(R)=\vol(R')\\
&=\lim_{n\to \iy}\sum_{R\subeq S_n} \int_{R'} g_n(x)\,\la_{\R^N}(dx)\\
&=\lim_{n\to \iy}\int_J g_n(x)\,\la_{\R^N}(dx)\\
&=\int_J g(x)\,\la_{\R^N}(dx).
\end{align*}
In the last step we used the Monotone Convergence Theorem with $g_n\nearrow g$.

For the second part, let $R_n$ be a rectangle containing $J_{\ep}$. We may assume that $R_n$ is large enough to contain $[-n,n]^N$, so $R_n\nearrow \R^N$. By the first part,
\[
\int_{R_n} f(x)\,\la_{\R^N}(dx)=(R)\int_{R_n} f(x)\,dx.
\]
Taking $n\to \iy$, the right-hand side becomes $\lim_{J\nearrow \R^N} (R)\int_J f(x)\,dx$ and the left-hand side becomes $\int_{\R^N} f(x)\,\la_{\R^N}(dx)$, as needed. (The last assertion follows from the fact that $\mu(\Ga)=\int_{\Ga}f\,\la_{\R^N}(dx)$ is a measure by 3.1.14, and so $R_n\nearrow \R^N$ gives $\lim_{n\to \iy} \mu(R_n)=\mu(\R^N)$.)\\

\subprob{(ii)}
Let $\cal C_1,\cal C_2,\ldots$ be a sequence of partitions such that $\cal C_{n+1}$ refines $\cal C_n$ and $\Vert \cal C_n\Vert\to 0$ as $n\to \iy$. %, and no $\cal C_n$ contains a discontinuity point (possible since there are only a countable number of them).
Fix a choice function $\xi_n$ for $\cal C_n$, and let
\[
g_n(x)=\ph(\xi_n(I))\text{ when }I=[c,d]\text{ and }x\in (c,d].
\]
and let $g_n(x)=0$ on $(-\iy,a]\cup (b,\iy)$.
%, since the Riemann-Steltjes integral equals the limit of the lower

%Let $\hat{\psi}(x)=\psi((a\wedge x)\vee b)-\psi (a)$. 
Now
\begin{align*}
(R) \int_J \ph(x)\,d\psi(x) &= \lim_{n\to \iy}\sum_{I\in \cal C_n}\ph(\xi_n(I))\De_I \psi\\
&=\lim_{n\to \iy}\sum_{I=[c,d]\in \cal C_n}\ph(\xi_n(I))(\psi(d)-\psi(c))\\
&=\lim_{n\to \iy}\sum_{I=[c,d]\in \cal C_n}\ph(\xi_n(I))\mu_{\psi}((c,d])\\
%&=\lim_{n\to \iy}\sum_{I=[c,d]\in \cal C_n}\ph(\xi_n(I))\mu((c,d])\\
&=\lim_{n\to \iy}\sum_{I=[c,d]\in \cal C_n}\int_{(c,d]} g_n(x)\, d\mu_{\psi}\\
&=\lim_{n\to \iy}\int_{(a,b]}g_n(x)\,d\mu_{\psi}\\
&=\lim_{n\to \iy}\int_{\R}g_n(x)\,d\mu_{\psi}&(\mu_{\psi}((-\iy,a]\cup (b,\iy))=0)\\
&=\int_{\R} \ph(x)\,d\mu_{\psi}.
\end{align*}
We justify the last equality: %We have $g_n(x)=\ph(\xi_n(I))\to g(x)$ for all $x$: %such that $\psi$ is continuous at $x$: 
as $n\to \iy$ 
as the interval $I$ 
such that $x\in I\in S_n$ 
shrinks, and $g_n(x)=\ph(\xi_n(I))\to \ph(x)$ by continuity. %By~\cite[4.30]{rudin}, as a nondecreasing function $g$ has a countable number of discontinuities, so $\ph(\xi_I)\to \ph(x)$ almost everywhere. 
Also note 
%\[
%\int_{\R}g_n(x)\,d\mu_{\psi}
%<(\psi(b)-\psi(a))\max_{[a,b]}(\ph(x))<\iy.\]
\[
g_n(x)
\le \max_{t\in J} \ph(t)
\]
%on all of $\R$ except $(-\iy, a)\cup (b,\iy)$ (which has measure 0 in $\mu_{\psi}$), 
on $(a,b]$ (which is $\mu$-almost everywhere) 
and the constant function $\max_{t\in J} \ph(t)$ is integrable on $\mu_{\psi}$ (as it is a finite measure space).
Hence Lebesgue's Dominated Convergence Theorem
 applies.
\end{problem}
\begin{problem}{\it(3.2.24, Uniform $\mu$-absolute continuity and integrability)}
\subprob{(i)}
Suppose $\cal K$ is uniformly $\mu$-absolutely continuous and $L=\sup_{f\in \cal K} \nl{f}<\iy$. Let $\ep>0$ be given. Take $\de>0$ so that $\int_{\Ga} |f|\,d\mu\le \ep$ for all $f\in \cal K$ and $\Ga\in \cal B$ with $\mu(\Ga)<\de$. Let $R>\frac{L}{\de}$. Then 
by Markov's inequality,
\[
\mu(|f|\ge R)\le \frac{L}{R}<\de.
\]
Hence by choice of $\de$,
\[
\int_{\{|f|\ge R\}}f\,d\mu\le \ep
\]
for all $f\in \cal K$. This shows $\cal K$ is uniformly $\mu$-integrable.

Conversely, suppose $\cal K$ is uniformly $\mu$-integrable. Given $\ep>0$, 
choose $R$ so that $\int_{\{|f|\ge R\}}|f|\le \eph$ for any $f\in \cal K$. Then given $\Ga$ such that $\mu(\Ga)<\frac{\ep}{2R}$,
\begin{align*}
\int_{\Ga}|f|\,d\mu&=\int_{\Ga\cap \{|f|\ge R\}} |f|\,d\mu
+\int_{\Ga\cap \{|f|<R\}}|f|\,d\mu\\
&\le \eph+R\mu(\Ga)\le \ep.
\end{align*}
showing $\cal K$ is uniformly $\mu$-absolutely continuous.
Now if $\mu(E)<\iy$, then taking any $\ep>0$ and choosing $R$ so that $\int_{\{|f|\ge R\}}|f|\le \ep$ for all $f\in \cal K$,
\begin{align*}
 \int_E|f|\,d\mu &=\int_{\{|f|\ge R\}} |f|\,d\mu +\int_{\{|f|<R\}} |f|\,d\mu\\
&\le \ep+R\mu(E)
\end{align*}
which is a bound independent of $f\in \cal K$. So $\sup_{f\in \cal K}\nl{f}<\iy$.\\

\subprob{(ii)}
Let $L=\sup_{f\in \cal K}\int |f|^{1+\de}\,d\mu$. By Markov's inequality,
\begin{equation}\label{p3-9-1}
\mu(|f|\ge k)=\mu(|f|^{1+\de}\ge k^{1+\de})\le \frac{L}{k^{1+\de}}.
\end{equation}
Let $m\in \N$.
\begin{align*}
\int_{|f|\ge m}|f|\,d\mu&\le 
m\mu(|f|\ge m)+
\sum_{R=m}^{\iy}\int_{R\le |f|<R+1} (|f|-m)\,d\mu&\text{ by 3.1.14}\\
&\le m\mu(|f|\ge m)+\sum_{R=m}^{\iy}\mu(R\le |f|<R+1)(R+1-m)\\
&=m\mu(|f|\ge m)+\sum_{R=m}^{\iy}
\sum_{k=m+1}^{R+1}\mu(R\le |f|<R+1)
\\
&=m\mu(|f|\ge m)+\sum_{k=m+1}^{\iy}\sum_{R=k-1}^{\iy}\mu(R\le |f|<R+1)\\
&=m\mu(|f|\ge m)+\sum_{k=m+1}^{\iy}\mu(|f|\ge k-1)\\
&=\frac{mL}{m^{1+\de}}+\sum_{k=m}^{\iy}\frac{L}{R^{1+\de}}&\by{p3-9-1}\\
&=\frac{L}{m^{\de}}+\sum_{R=m}^{\infty} \frac{L}{k^{1+\de}}.
\end{align*}
Both terms go to 0 as $m\to \iy$, the second because $\sum_{R=1}^{\iy}\rc{R^{1+\de}}$ is convergent. Hence $\cal K$ is uniformly $\mu$-integrable.\\

\subprob{(iii)}
Note that if $g$ is $\mu$-integrable, then by 3.1.14, $\mu'(\Ga)=\int_{\Ga} g\,d\mu$ is absolutely continuous with respect to $\mu$. By 2.1.27, this means that for each $\ep>0$ there is a $\de>0$ such that $\int_{\Ga}f\,d\mu<\ep$ whenever $\mu(\Ga)<\de$, i.e. $\{g\}$ is uniformly $\mu$-absolutely continuous. Then any finite group of $\mu$-integrable functions is also uniformly $\mu$-absolutely continuous, since we can take $\de$ to be the minimum of the $\de$'s chosen for individual functions.

Since $f_n\to f$ in $L^1$, we can pick $N$ so that $\int|f_n-f|<\eph$ for $n>N$. By the above, we can choose $\de$ so that $\int_{\Ga}|f_n|\,d\mu<\eph<\ep$ whenever $n=1,\ldots, N$, or $\infty$, and $\mu(\Ga)<\de$. (We let $f_{\iy}=f$.) 
We would also like $\int_{\Ga}|f_n|<\ep$ for $n>N$ and $\mu(\Ga)<\de$. Indeed, for such $n$ and $\Ga$,
\[
\int_{\Ga}|f_n|\,d\mu\le \int_{\Ga}|f_n-f|\,d\mu+\int_{\Ga}|f|\,d\mu
\le \eph+\eph=\ep.
\]
Hence $\cal K$ is uniformly $\mu$-absolutely continuous. Note that $f_n\to f$ in $L^1(\mu;\R)$ implies $\int |f_n-f|\,d\mu\to 0$ as $n\to 0$, and hence there exists $C$ so that $\int |f_n-f|\,d\mu\le C$ for all $n$. 
Then
\[
\int|f_n|\,d\mu\le \int |f|\,d\mu+\int |f_n-f|\,d\mu
\le \int |f|\,d\mu+C
\]
so $\sup_{n\in \N}\nl{f_n}<\iy$, and $\sup_{n\in \N\text{ or }n=\iy} \nl{f_n}<\iy$. Thus by (i), $\cal K$ is uniformly $\mu$-integrable.

Converse: %Given $\ep>0$, by uniform $\mu$-integrability choose $L$ so that $\int_{|f|\ge L}|f|\,d\mu\le \ep$ for all $f\in \cal K$. Since $f_n\to f$ in $\mu$-measure, $\mu(|f_n-f|\ge L)\to 0$ as $n\to \iy$. 
For $\ep>0$,
\begin{equation}\label{threethings}
\int|f_n-f|\,d\mu\le \int_{|f_n-f|\ge L} |f_n-f|\,d\mu
+\int_{L>|f_n-f|\ge\ep}|f_n-f|\,d\mu
+\int_{|f_n-f|<\ep}|f_n-f|\,d\mu
\end{equation}
The first term of~(\ref{threethings}) is at most
\[
\int_{|f_n-f|\ge L}|f_n|+|f|\,d\mu
\le \int_{|f_n|\ge L/2}|f_n|+|f|\,d\mu
+\int_{|f|\ge L/2}|f_n|+|f|\,d\mu.
\]
By uniform $\mu$-integrability applied to the four integrals on the RHS, this goes to 0 as $L\to \iy$. 
The second term of~(\ref{threethings}) is at most
\[
L\mu(|f_n-f|\ge \ep)
\]
which goes to 0 as $n$ to $\iy$ since $f_n\to f$ in $\mu$-measure.
The third term of~(\ref{threethings}) is at most
\[
\ep\mu(E)
\]
which goes to 0 as $\ep\to 0$. Thus by choosing $L,\ep$ appropriately, the first and last term can be made arbitrary small (say, $<\ept$). Then choosing $n$ large enough makes the second term arbitrarily small (say, $<\ept$). We conclude $\int |f_n-f|\,d\mu\to 0$ as $n\to \iy$, as needed.\\

\subprob{(iv)}
We've already shown in (i) that $\cal K$ uniformly $\mu$-integrable implies $\cal K$ uniformly $\mu$-absolutely continuous, and that the reverse direction holds. Thus it suffices to show that if $\cal K$ is uniformly $\mu$-integrable and tight then $\sup_{f\in \cal K}\nl{f}<\iy$. Choose any $\ep>0$ and choose $R$ so that
\[
\int_{|f|\ge R}|f|\,d\mu \le \ep
\]
for all $f\in \cal K$, 
and by tightness choose $\Ga$ with finite measure such that 
\[
\sup_{f\in \cal K}\int_{\Ga^c} |f|\,d\mu\le \ep
\]
for all $f\in \cal K$.
Then
\begin{align*}
\int |f|\,d\mu&=\int_{\Ga\cap \{|f|\ge R\}} |f|\,d\mu
+\int_{\Ga\cap \{|f|<R\}} |f|\,d\mu
+\int_{\Ga^c}|f|\,d\mu\\
&\le \ep +R\mu(\Ga)+\ep
\end{align*}
independently of $f$.

For the second part, the forward direction holds by the first part of (iii). For the reverse direction, given $\ep>0$, choose $\Ga$ so that $\mu(\Ga)<\iy$ and $\int_{\Ga^c}|f|\,d\mu\le \eph$. Then
\[
\int|f-f_n|\,d\mu=\int_{\Ga}|f-f_n|\,d\mu+\int_{\Ga^c}|f-f_n|\,d\mu\le 
\int_{\Ga}|f-f_n|\,d\mu+\eph.
\]
Now by the converse of (iii) applied to the subspace $\Ga$, which has finite measure, $f\to f_n$ in $L^1(\mu; \Ga)$. Hence for large enough $n$ the first term is at most $\eph$, and $\int|f_n-f|\,d\mu<\ep$. This shows that $f_n\to f$ in $L^1$.
\end{problem}
%\begin{thebibliography}{9}
%\bibitem{rudin} Rudin, W.: "Principles of Mathematical Analysis," McGraw-Hill, CA, 1976.
%\end{thebibliography}
\end{document}
