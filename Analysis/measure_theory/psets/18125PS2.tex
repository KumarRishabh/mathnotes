%%%This is a science homework template. Modify the preamble to suit your needs. 

\documentclass[12pt]{article}

\makeatother
%AMS-TeX packages
\usepackage{amsmath}
\usepackage{amssymb}
\usepackage{amsthm}
\usepackage{array}
\usepackage{amsfonts}
\usepackage{cancel}
\usepackage[all,cmtip]{xy}%Commutative Diagrams
\usepackage[pdftex]{graphicx}
\usepackage{float}
%geometry (sets margin) and other useful packages
\usepackage[margin=1in]{geometry}
\usepackage{sidecap}
\usepackage{wrapfig}
\usepackage{verbatim}
\usepackage{mathrsfs}
\usepackage{marvosym}
\usepackage{stmaryrd}
\usepackage{hyperref}
\usepackage{graphicx,ctable,booktabs}

\newtheoremstyle{norm}
{3pt}
{3pt}
{}
{}
{\bf}
{:}
{.5em}
{}

\theoremstyle{norm}
\newtheorem{thm}{Theorem}[section]
\newtheorem{lem}[thm]{Lemma}
\newtheorem{df}{Definition}
\newtheorem{rem}{Remark}
\newtheorem{st}{Step}
\newtheorem{pr}[thm]{Proposition}
\newtheorem{cor}[thm]{Corollary}
\newtheorem{clm}[thm]{Claim}

%Math blackboard, fraktur, and script commonly used letters
\newcommand{\A}[0]{\mathbb{A}}
\newcommand{\C}[0]{\mathbb{C}}
\newcommand{\sC}[0]{\mathcal{C}}
\newcommand{\cE}[0]{\mathscr{E}}
\newcommand{\F}[0]{\mathbb{F}}
\newcommand{\cF}[0]{\mathscr{F}}
\newcommand{\cG}[0]{\mathscr{G}}
\newcommand{\sH}[0]{\mathscr H}
\newcommand{\Hq}[0]{\mathbb{H}}
\newcommand{\cI}[0]{\mathscr{I}}%ideal sheaf
\newcommand{\N}[0]{\mathbb{N}}
\newcommand{\Pj}[0]{\mathbb{P}}
\newcommand{\sO}[0]{\mathcal{O}}
\newcommand{\cO}[0]{\mathscr{O}}
\newcommand{\Q}[0]{\mathbb{Q}}
\newcommand{\R}[0]{\mathbb{R}}
\newcommand{\Z}[0]{\mathbb{Z}}
%Lowercase
\newcommand{\ma}[0]{\mathfrak{a}}
\newcommand{\mb}[0]{\mathfrak{b}}
\newcommand{\fg}[0]{\mathfrak{g}}
\newcommand{\vi}[0]{\mathbf{i}}
\newcommand{\vj}[0]{\mathbf{j}}
\newcommand{\vk}[0]{\mathbf{k}}
\newcommand{\mm}[0]{\mathfrak{m}}
\newcommand{\mfp}[0]{\mathfrak{p}}
\newcommand{\mq}[0]{\mathfrak{q}}
\newcommand{\mr}[0]{\mathfrak{r}}
%Letter-related
\providecommand{\cal}[1]{\mathcal{#1}}
\renewcommand{\cal}[1]{\mathcal{#1}}
\newcommand{\bb}[1]{\mathbb{#1}}
%More sequences of letters
\newcommand{\chom}[0]{\mathscr{H}om}
\newcommand{\fq}[0]{\mathbb{F}_q}
\newcommand{\fqt}[0]{\mathbb{F}_q^{\times}}
\newcommand{\sll}[0]{\mathfrak{sl}}
%Shortcuts for symbols
\newcommand{\nin}[0]{\not\in}
\newcommand{\opl}[0]{\oplus}
\newcommand{\ot}[0]{\otimes}
\newcommand{\rc}[1]{\frac{1}{#1}}
\newcommand{\rra}[0]{\rightrightarrows}
\newcommand{\send}[0]{\mapsto}
\newcommand{\sub}[0]{\subset}
\newcommand{\subeq}[0]{\subseteq}
\newcommand{\supeq}[0]{\supseteq}
\newcommand{\nsubeq}[0]{\not\subseteq}
\newcommand{\nsupeq}[0]{\not\supseteq}
%Shortcuts for greek letters
\newcommand{\al}[0]{\alpha}
\newcommand{\be}[0]{\beta}
\newcommand{\ga}[0]{\gamma}
\newcommand{\Ga}[0]{\Gamma}
\newcommand{\de}[0]{\delta}
\newcommand{\De}[0]{\Delta}
\newcommand{\ep}[0]{\varepsilon}
\newcommand{\eph}[0]{\frac{\varepsilon}{2}}
\newcommand{\ept}[0]{\frac{\varepsilon}{3}}
\newcommand{\la}[0]{\lambda}
\newcommand{\La}[0]{\Lambda}
\newcommand{\ph}[0]{\varphi}
\newcommand{\rh}[0]{\rho}
\newcommand{\te}[0]{\theta}
\newcommand{\om}[0]{\omega}
\newcommand{\Om}[0]{\Omega}
\newcommand{\si}[0]{\sigma}
%Brackets
\newcommand{\ab}[1]{\left| {#1} \right|}
\newcommand{\ba}[1]{\left[ {#1} \right]}
\newcommand{\bc}[1]{\left\{ {#1} \right\}}
\newcommand{\pa}[1]{\left( {#1} \right)}
\newcommand{\an}[1]{\langle {#1}\rangle}
\newcommand{\fl}[1]{\left\lfloor {#1}\right\rfloor}
\newcommand{\ce}[1]{\left\lceil {#1}\right\rceil}
%Text
\newcommand{\btih}[1]{\text{ by the induction hypothesis{#1}}}
\newcommand{\bwoc}[0]{by way of contradiction}
\newcommand{\by}[1]{\text{by~(\ref{#1})}}
\newcommand{\ore}[0]{\text{ or }}
%Arrows
\newcommand{\hr}[0]{\hookrightarrow}
\newcommand{\xr}[1]{\xrightarrow{#1}}
%Formatting
\newcommand{\subprob}[1]{\noindent\textbf{#1}\\}
%Functions, etc.
\newcommand{\Ann}{\operatorname{Ann}}
\newcommand{\Arc}{\operatorname{Arc}}
\newcommand{\Ass}{\operatorname{Ass}}
\newcommand{\Aut}{\operatorname{Aut}}
\newcommand{\chr}{\operatorname{char}}
\newcommand{\cis}{\operatorname{cis}}
\newcommand{\Cl}{\operatorname{Cl}}
\newcommand{\Der}{\operatorname{Der}}
\newcommand{\End}{\operatorname{End}}
\newcommand{\Ext}{\operatorname{Ext}}
\newcommand{\Frac}{\operatorname{Frac}}
\newcommand{\FS}{\operatorname{FS}}
\newcommand{\GL}{\operatorname{GL}}
\newcommand{\Hom}{\operatorname{Hom}}
\newcommand{\Ind}[0]{\text{Ind}}
\newcommand{\im}[0]{\text{im}}
\newcommand{\nil}[0]{\operatorname{nil}}
\newcommand{\ord}[0]{\operatorname{ord}}
\newcommand{\Proj}{\operatorname{Proj}}
\newcommand{\Rad}{\operatorname{Rad}}
\newcommand{\rank}{\operatorname{rank}}
\newcommand{\Res}[0]{\text{Res}}
\newcommand{\sign}{\operatorname{sign}}
\newcommand{\SL}{\operatorname{SL}}
\newcommand{\Spec}{\operatorname{Spec}}
\newcommand{\Specf}[2]{\Spec\pa{\frac{k[{#1}]}{#2}}}
\newcommand{\spp}{\operatorname{sp}}
\newcommand{\spn}{\operatorname{span}}
\newcommand{\Supp}{\operatorname{Supp}}
\newcommand{\Tor}{\operatorname{Tor}}
\newcommand{\tr}[0]{\text{trace}}
\newcommand{\Var}{\operatorname{Var}}
\newcommand{\vol}[0]{\operatorname{vol}}
%Commutative diagram shortcuts
\newcommand{\fiber}[3]{\xymatrix{#1\times_{#3} #2}\ar[r]\ar[d] #1\ar[d] \\ #2 \ar[r] & #3}
\newcommand{\commsq}[8]{\xymatrix{#1\ar[r]^{#6}\ar[d]^{#5} &#2\ar[d]^{#7} \\ #3 \ar[r]^{#8} & #4}}
%Makes a diagram like this
%1->2
%|    |
%3->4
%Arguments 5, 6, 7, 8 on arrows
%  6
%5  7
%  8
\newcommand{\pull}[9]{
#1\ar@/_/[ddr]_{#2} \ar@{.>}[rd]^{#3} \ar@/^/[rrd]^{#4} & &\\
& #5\ar[r]^{#6}\ar[d]^{#8} &#7\ar[d]^{#9} \\}
\newcommand{\back}[3]{& #1 \ar[r]^{#2} & #3}
%Syntax:\pull 123456789 \back ABC
%1=upper left-hand corner
%2,3,4=arrows from upper LH corner, going down, diagonal, right
%5,6,7=top row (6 on arrow)
%8,9=middle rows (on arrows)
%A,B,C=bottom row
%Other
%Other
\newcommand{\op}{^{\text{op}}}
\newcommand{\fp}[1]{^{\underline{#1}}}
\newcommand{\rp}[1]{^{\overline{#1}}}
\newcommand{\rd}[0]{_{\text{red}}}
\newcommand{\pre}[0]{^{\text{pre}}}
\newcommand{\pf}[2]{\pa{\frac{#1}{#2}}}
\newcommand{\pd}[2]{\frac{\partial #1}{\partial #2}}
\newcommand{\bs}[0]{\backslash}
\newcommand{\sia}[0]{ $\si$-algebra}
%Matrices
\newcommand{\coltwo}[2]{
\left[
\begin{matrix}
{#1}\\
{#2} 
\end{matrix}
\right]}
\newcommand{\matt}[4]{
\left[
\begin{matrix}
{#1}&{#2}\\
{#3}&{#4}
\end{matrix}
\right]}
\newcommand{\smatt}[4]{
\left[
\begin{smallmatrix}
{#1}&{#2}\\
{#3}&{#4}
\end{smallmatrix}
\right]}
\newcommand{\colthree}[3]{
\left[
\begin{matrix}
{#1}\\
{#2}\\
{#3}
\end{matrix}
\right]}
%
%Redefining sections as problems
%
\makeatletter
\newenvironment{problem}{\@startsection
       {section}
       {1}
       {-.2em}
       {-3.5ex plus -1ex minus -.2ex}
       {2.3ex plus .2ex}
       {\pagebreak[3]%forces pagebreak when space is small; use \eject for better results
       \large\bf\noindent{Problem }
       }
       }
       {%\vspace{1ex}\begin{center} \rule{0.3\linewidth}{.3pt}\end{center}}
       }
\makeatother


%
%Fancy-header package to modify header/page numbering 
%
\usepackage{fancyhdr}
\pagestyle{fancy}
%\addtolength{\headwidth}{\marginparsep} %these change header-rule width
%\addtolength{\headwidth}{\marginparwidth}
\lhead{Problem \thesection}
\chead{} 
\rhead{\thepage} 
\lfoot{\small\scshape 18.125 Real and Functional Analysis} 
\cfoot{} 
\rfoot{\scriptsize PS \# 2} % !! Remember to change the problem set number
\renewcommand{\headrulewidth}{.3pt} 
\renewcommand{\footrulewidth}{.3pt}
\setlength\voffset{-0.25in}
\setlength\textheight{648pt}



%%%%%%%%%%%%%%%%%%%%%%%%%%%%%%%%%%%%%%%%%%%%%%%
%
%Contents of problem set
%    
\begin{document}
\title{18.125 Real and Functional Analysis PSet \#2}% !! Remember to change the problem set number
\author{Holden Lee}
\date{2/19/11}% !! Remember to change the date
\maketitle
\thispagestyle{empty}
\begin{problem}{\it (2.1.19, Measurable functions)}
\subprob{(i)}
To see $\Phi^{-1}$ preserves unions, follow this string of equivalences.
\begin{enumerate}
\item
$x\in \Phi^{-1}\pa{\bigcup_{\al}B_{\al}}$.
\item
$\Phi(x)\in \bigcup_{\al}B_{\al}$.
\item $\Phi(x)$ is in $B_{\al}$ for some $\al$.
\item $x\in \Phi^{-1}(B_\al)$ for some $\al$.
\item $x\in \bigcup_{\al} \Phi^{-1}(B_{\al})$.
\end{enumerate}
To see $\Phi$ preserves unions, follow this string of equivalences.
\begin{enumerate}
\item
$y\in \Phi^{-1}\pa{\bigcup_{\al} B_{\al}}$.
\item
There exists $x\in \bigcup_{\al} B_{\al}$ such that $y=\Phi(x)$.
\item
$y=\Phi(x)$ for $x\in B_{\al}$, for some $\al$.
\item
$y\in \Phi(B_{\al})$ for some $\al$.
\item $y\in \bigcup_{\al}\Phi(B_{\al})$.
\end{enumerate}
To see $\Phi^{-1}$ preserves differences, follow this string of equivalences.
\begin{enumerate}
\item
$x\in \Phi^{-1}(B\bs A)$.
\item
$\Phi(x)\in B\bs A$.
\item
$\Phi(x)\in B$ and $\Phi(x)\nin A$.
\item $x\in \Phi^{-1}(B)$ and $x\nin \Phi^{-1}(A)$.
\item $x\in \Phi^{-1}(B)\bs\Phi^{-1}(B)$.
\end{enumerate}
To see $\Phi$ preserves differences when $\Phi$ is one-to-one,  follow this string of equivalences.
\begin{enumerate}
\item
$x\in \Phi(B\bs A)$.
\item
$x=\Phi(y),y\in B\bs A$.
\item 
$x=\Phi(y)$ for some $y\in B$ and $x\neq \Phi(y)$ for any $y\in A$.
\item
$x\in \Phi(B)\bs \Phi(A)$.
\end{enumerate}
Note $2\implies 3$ follows from the fact that $\Phi(y)\neq \Phi(y')$ if $y\in B\bs A$ and $y'\in A$, simply because $\Phi$ is one-to-one.

Finally, consider $\Phi:\{1,2\}\to \{1\}$ sending both elements to 1. Then
\[
\Phi^{-1}(\{1,2\}\bs\{1\})=\{1\}\neq \phi=\Phi(\{1,2\})\bs\Phi(\{1\}).
\]

\subprob{(ii)}
%\begin{lem}
%For a family of sets $\cal C\subeq \cal P(E)$, $\sigma(C)$ consists exactly of sets of the form
%\begin{equation}\label{sform}
%\bigcup_{i\geq 1} A_i\bs\bigcup_{i\geq 1} B_i
%\end{equation}
%where each $A_i,B_i$ is in $\cal C$ or equal to $E$.
%\end{lem}
%\begin{proof}
%Given $A_i,B_i$ in $\cal C\subeq \sigma(\cal C)$, since $\sigma(\cal C)$ is a $\sigma$-algebra, $\bigcup_{i\geq 1}A_i,\bigcup_{i\geq 1}B_i$ is also in $\sigma(\cal C)$. Hence their difference is also in $\sigma(\cal C)$.
%Note $E$ is in the form~(\ref{sform}). Now given $C_j=\bigcup_{i\geq 1} A_{i,j}\bs\bigcup_{i\geq 1} B_{i,j}$ of the form~(\ref{sform}) for $j\geq 1$, their union is of the above form, and $C_1\bs C_2$ is of the above form:
%\begin{align*}
%\bigcup_{j\geq 1} C_j&=
%\end{align*}
%\end{proof}
%Suppose $\cal B'=\sigma(\cal C')$ and $\Phi^{-1}(C)\in \cal B$ for every $C\in \cal C'$. We claim that
Note
\[
\cal S':=\{B\in \cal B'|\Phi^{-1}(B)\in \cal B\}
\]
is a $\sigma$-algebra because the following hold.
\begin{enumerate}
\item $E'\in \cal S'$: Indeed, $\Phi^{-1}(E')=E\in \cal B$.
\item If $A_i\in \cal S'$ for $i\in \N$ then $\bigcup_{i=1}^{\infty}A_i\in \cal S'$: From (i),
\[
\Phi^{-1}\pa{\bigcup_{i=1}^{\infty}A_i}=\bigcup_{i=1}^{\infty} \Phi^{-1}(A_i)\in \cal B
\]
since each $\Phi^{-1}(A_i)$ is in $\cal B$ and $\cal B$ is a \sia.
\item If $A\in \cal S'$, then $A^c\in \cal S$: Again from (i),
\[
\Phi^{-1}(E'\bs A)=\Phi^{-1}(E')\bs \Phi^{-1}(A)=E\bs \Phi^{-1}(A)\in \cal B,
\]
since $\Phi^{-1}(A)\in \cal B$ and $\cal B$ is a \sia.
\end{enumerate}
Given that $\cal B'=\sigma(\cal C')$ and $\Phi^{-1}(C)\in \cal B$ for every $C\in \cal C'$, we get $\cal C'\subeq \cal S'$. Since $\cal B'=\sigma(\cal C')$ is the smallest \sia{} containing $\cal C'$, we conclude $\cal B'= \cal S'$. In other words, for every $C\in \cal B'$, $\Phi^{-1}(C)\in \cal B$, i.e. $\Phi$ is measurable.\\

In particular, if $E$ and $E'$ are topological spaces and $\Phi$ is continuous, then letting $\cal C'$ be the open sets of $E'$, we see that $\Phi^{-1}(C)$ is open for every $C\in \cal C'$ and hence inside $\cal B_E$. The above applies to show that $\Phi$ is a measurable map from $(E,\cal B_E)$ to $(E',\cal B_{E'})$.\\

Now suppose $\Phi$ is one-to-one and $\Phi(E)\in \cal B'$. Then
%, $\cal B=\sigma(\cal C)$, and $\Phi(C)\in \cal B'$ for all $C\in \cal C$. We claim that
\[
\cal S:=\{C\in \cal B|\Phi(C)\in \cal B'\}
\]
is a $\sigma$-algebra because the following hold.
\begin{enumerate}
\item $\Phi(E)\in \cal S$: By assumption.
\item If $A_i\in \cal S$ for $i\in \N$ then $\bigcup_{i=1}^{\infty} A_i\in \cal S$: From (i),
\[
\Phi\pa{\bigcup_{i=1}^{\infty} A_i}=\bigcup_{i=1}^{\infty} \Phi(A_i)\in \cal B'
\]
since $\Phi(A_i)\in \cal B'$ and $\cal B'$ is a \sia.
\item If $A\in \cal S$ then $A^c\in \cal S$: From (i) (using injectivity),
\[
\Phi(E\bs A)=\Phi(E)\bs \Phi(A)\in \cal B'
\]
since $\Phi(E),\Phi(A)\in \cal B'$.
\end{enumerate}
By assumption ($\cal B=\sigma(\cal C)$, and $\Phi(C)\in \cal B'$ for all $C\in \cal C$), $\cal C\subeq \cal S$. Since $\cal B=\si(\cal C)$ is the smallest \sia{} containing $\cal C$, we conclude $\cal B=\cal S$, i.e. for all $B\in \cal B$, $\Phi(B)\in \cal B'$.\\

\subprob{(iii)}
We verify the properties of measure in each case. First, let $\mu'$ be the pushforward of $\mu$.
\begin{enumerate}
\item
$\mu'(\phi)=\mu(\Phi^{-1}(\phi))=\mu(\phi)=0$.
\item
Suppose $B_i'\in \cal B'$ for $i\in \N$, where the $B_i'$ are pairwise disjoint. Then the $\Phi^{-1}(B_i')$ are pairwise disjoint. Hence
\begin{align*}
\mu'\pa{\bigcup_{i=1}^{\infty} B_i}
&=
\mu\pa{\Phi^{-1}\pa{\bigcup_{i=1}^{\infty}B_i'}}\\
&=\mu\pa{\bigcup_{i=1}^{\infty} \Phi^{-1}(B_i')}\\
&=\sum_{i=1}^{\infty} \mu(\Phi^{-1}(B_i'))\\
&=\sum_{i=1}^{\infty}\mu'(B_i').
\end{align*}
\end{enumerate}
Now let $\mu$ be the pullback of $\mu'$ under $\Phi$, where $\Phi$ is one-to-one.
\begin{enumerate}
\item
$\mu(\phi)=\mu'(\Phi(\phi))=\mu'(\phi)=0$.
\item
Suppose $B_i\in \cal B$ for $i\in \N$, where the $B_i$ are pairwise disjoint. Then the $\Phi(B_i)$ are pairwise disjoint, since $\Phi$ is one-to-one. Hence
\begin{align*}
\mu\pa{\bigcup_{i=1}^{\infty} B_i}
&=
\mu'\pa{\Phi\pa{\bigcup_{i=1}^{\infty}B_i}}\\
&=\mu'\pa{\bigcup_{i=1}^{\infty} \Phi(B_i)}\\
&=\sum_{i=1}^{\infty} \mu'(\Phi(B_i))\\
&=\sum_{i=1}^{\infty}\mu(B_i).
\end{align*}
\end{enumerate}
\end{problem}
\begin{problem}{\it (2.1.21, Principle of inclusion and exclusion)}
The formula holds for $n=2$ by 2.1.6. Now suppose it holds for $n$; we prove it for $n+1$. Given $\Ga_1,\ldots, \Ga_{n+1}$, we first apply the formula for $n=2$ to $\Ga_1\cap \cdots \cap \Ga_n$ and $\Ga_{n+1}$. Then
we apply the induction hypothesis to the $n$ sets $\Ga_1,\ldots, \Ga_{n}$ as well as the $n$ sets $\Ga_1\cap \Ga_{n+1},\ldots, \Ga_n\cap \Ga_{n+1}$. (Note $\Ga_i$ and $\Ga_i\cap \Ga_{n+1}$ all have finite measure.)
We write $[k]$ as shorthand for $\{1,\ldots, k\}$.
\begin{align*}
\mu(\Ga_1\cup \cdots \cup \Ga_{n}\cup \Ga_{n+1})
&=
\mu(\Ga_1\cup \cdots \cup \Ga_n)+\mu(\Ga_n)-\mu((\Ga_1\cup \cdots \cup \Ga_n)\cap \Ga_{n+1})\\
&=
\mu(\Ga_1\cup \cdots \cup \Ga_n)+\mu(\Ga_n)-\mu((\Ga_1\cap \Ga_{n+1})\cup \cdots \cup (\Ga_n\cap \Ga_{n+1}))\\
&=-\pa{\sum_{\scriptsize\begin{array}{c}F\subeq [n]\\F\neq \phi\end{array}} (-1)^{|F|} \mu(\Ga_F)}
+\mu(\Ga_{n+1})
-\sum_{\scriptsize\begin{array}{c} F\subeq [n]\\F\neq \phi\end{array}} (-1)^{|F|+1} \mu(|\Ga_{F}\cap \Ga_{n+1}|)\\
%paren
&=
-\pa{\sum_{\scriptsize\begin{array}{c}n+1\nin F\subeq [n+1]\\F\neq \phi\end{array}} (-1)^{|F|} \mu(\Ga_F)}
-\pa{\sum_{\scriptsize\begin{array}{c}n+1\in F\subeq  [n+1]\\F\neq \phi\end{array}} (-1)^{|F|} \mu(\Ga_F)}\\
&=\sum_{\scriptsize\begin{array}{c}F\subeq [n+1]\\F\neq \phi\end{array}} (-1)^{|F|} \mu(\Ga_F)
\end{align*}
as needed.\\

The probability that a set of $m$ points is fixed under $\pi$ is $\frac{(n-m)!}{n!}$, since there are $(n-m)!$ possibilities for $\pi$---it is determined on those $m$ points but can permute the other $n-m$ points in any way---and each permutation has probability $\rc{n!}$ of being chosen. Hence $\mu(\Ga_F)=\frac{(n-m)!}{n!}$ if $|F|=m$. Then using PIE,
\begin{align*}
\mu(A)&=1-\mu(\Ga_1\cup \cdots \cup \Ga_n)\\
&=1+\sum_{\phi\neq F\subeq [n]} (-1)^{|F|} \mu(\Ga_F)\\
&=1+\sum_{m=1}^{n}\sum_{\scriptsize\begin{array}{c}F\subeq [n]\\|F|=m\end{array}}(-1)^m\frac{(n-m)!}{n!}\\
&=1+\sum_{m=1}^n (-1)^{m}\binom nm\frac{(n-m)!}{n!}\\
&=\sum_{m=0}^n \frac{(-1)^m}{m!}.
\end{align*}
By the Taylor expansion of $e^x$ this approaches $\rc e$ as $n\to \infty$.
\end{problem}
\begin{problem} {\it (2.1.22, Lim inf and lim sup)}
As mentioned, $\varliminf_{n\to \infty} B_n$ is the set of $x\in E$ that are in all but finitely many $B_n$ while $\varlimsup_{n\to \infty} B_n$ is the set of $x\in E$ that are in infinitely many $B_n$. An element in all but finitely many $B_n$ must be in infinitely many of them so $\varliminf_{n\to \infty} B_n\subeq \varlimsup_{n\to \infty}B_n$.\\

Since $\varliminf_{n\to \infty} B_n$ and $\varlimsup_{n\to \infty}B_n$ are both obtained from the $B_n$ by a sequence of countable unions and intersections, they are in $\cal B$.\\

Let $C_m=\bigcap_{n=m}^{\infty} B_n$. Note that $C_{m}\subeq C_{m+1}$ and $\bigcup_{m=1}^{\infty} C_m=\varliminf_{n\to \infty} B_n$. Hence $C_n\nearrow \varliminf_{n\to \infty} B_n$ and $\mu(C_n)\to \mu(\varliminf_{n\to \infty} B_n)$ as $n\to \infty$. Since $B_n\supeq C_n$, we get
\[
\mu(B_n)\geq \mu(C_n),
\]
%since $B_m\supeq \bigcap_{n=m}^{\infty}B_n$
and taking the lim inf of both sides gives
\begin{equation}\label{liminfineq}
\varliminf_{n\to \infty}\mu(B_n)\geq \mu(\varliminf_{n\to \infty} B_n).
\end{equation}

Let $D_m=\bigcup_{n=m}^{\infty} B_n$. Note that $D_{m}\supeq D_{m+1}$ and $\bigcap_{m=1}^{\infty} D_m=\varlimsup_{n\to \infty} B_n$. Hence $D_n\searrow \varlimsup_{n\to \infty} B_n$ and $\mu(D_n)\to \mu(\varliminf_{n\to \infty} B_n)$ as $n\to \infty$ (here we used the fact that $\mu(D_1)<\infty$). Since $B_n\subeq D_n$, we get
\[
\mu(B_n)\leq \mu(D_n),
\]
%since $B_m\supeq \bigcap_{n=m}^{\infty}B_n$
and taking the lim sup of both sides gives
\begin{equation}\label{limsupineq}
\varlimsup_{n\to \infty}\mu(B_n)\leq \mu(\varlimsup_{n\to \infty} B_n).
\end{equation}

If $\lim_{n\to \infty}B_n$ exists and $\mu\pa{\bigcup_{n=1}^{\infty} B_n}<\infty$ then combining~(\ref{liminfineq}) and~(\ref{limsupineq}) gives
\[
\varlimsup_{n\to \infty}\mu(B_n)\leq \mu(\lim_{n\to \infty} B_n)\leq \varliminf_{n\to \infty} \mu(B_n).
\]
But the LHS is clearly at least as large as the RHS, so equality holds everywhere, and $\mu(\lim_{n\to \infty} B_n)=\lim_{n\to \infty}\mu(B_n)$.
\end{problem}
\begin{problem} {\it (2.1.27, Absolute continuity)}
We prove the contrapositive. Suppose that there exists $\ep>0$ so that for every $\de>0$ there exists a set $B\in \cal B$ such that $\mu(B)>\ep$ but $\nu(B)<\de$. Let $B_n$ be such a set for $\de=2^{-n}$. 
Since $\mu$ is finite, we use~(\ref{limsupineq}) to get
\[
\mu\pa{\varlimsup_{n\to \infty} B_n}
\geq \varlimsup_{n\to \infty}\nu(B_n)\geq \ep.
\]
However, by countable subadditivity,
\[
\nu\pa{\varlimsup_{n\to \infty} B_n}\leq \nu\pa{\bigcup_{n=m}^{\infty} B_n}\leq \sum_{n=m}^{\infty} \nu(B_n)
<\sum_{n=m}\rc{2^n}
\leq \rc{2^{m-1}}.
\]
Hence $\nu\pa{\varlimsup_{n\to \infty} B_n}=0$, and $\mu$ is not absolutely continuous with respect to $\nu$.\\

For the second part, we've already shown the forward direction; we prove the reverse direction. Let $B$ be such that $\nu(B)=0$. Given $\ep>0$ take $\de>0$ such that $\mu(G)<\ep$ whenever $G$ is open and $\nu(G)<\de$. 
Since $\nu$ is a regular Borel measure there exist closed $F$ and open $G$ such that $\nu(G\bs F)<\de$ and $F\subeq B\subeq G$. Since $\nu(B)=0$ we get $\nu(F)=0$ and so $\nu(G)=\nu(G\bs F)+\nu(F)<\de$. Then $\mu(G)<\ep$. Since $B\subeq G$ we get $\mu(B)< \ep$. Since this is true for any $\ep>0$, we conclude $\mu(B)=0$, as needed.
\end{problem}
\begin{problem} {\it (2.1.28, Singular measures)}
%For the forward direction, 
Suppose $\mu(B)=0=\nu(B^c)$.
Given $\de>0$, since $\nu$ is a regular Borel measure, there exist closed $F$ and open $G$ so that $F\subeq  B^c\subeq G$ and $\nu(G\bs F)<\de$.
Since $\nu(B^c)=0$, we get $\nu(G)=\nu(G\bs F)+\cancelto{0}{\nu(F)}<\de$.
%As above, this means $\nu(B)\leq \de$. 
Since $G^c\subeq B$, we get $\mu(G^c)\leq \mu(B)=0$.

The reverse direction follows from this:
\begin{lem}\label{singcrit}
Suppose that $E$ is a metric space, $\nu,\mu$ are Borel measures on $E$, and %$\mu$ is finite. Further suppose 
for every $\de>0$ there is a set $A$ such that $\nu(A)<\de$ and $\mu(A^c)=0$. Then $\mu\perp \nu$.
\end{lem}
\begin{proof}
Take sets $A_n$ so that $\nu(A_n)<\rc{2^n}$ and $\mu(A_n^c)=0$. Let $B=\bigcup_{n=1}^{\infty} A_n^c$. Then
\[
\nu(B^c)=
\nu\pa{\bigcap_{n=1}^{\infty} A_n}
\leq \nu(A_n)< \rc{2^n}
\]
for any $n$ so $\nu(A^c)=0$. However, by countable subadditivity,
\[
\mu(B)\leq \sum_{n=1}^{\infty} \mu(A_n^c)=0.
\]
Hence $\mu\perp \nu$.
\end{proof}

\end{problem}
\begin{problem}{\it(2.2.33, Cantor set)}
\begin{lem}
For $a=(a_1,\ldots, a_n)\in \{0,1\}^n$, let $f(a)=\frac{2a_1}{3}+\cdots +\frac{2a_{n}}{3}$. (For $n=0$, and $a$ the empty string $(\,)$, $f(a)=0$.)
Then $C_0=[0,1]$ and for $n>0$,
\[
C_n=\bigcup_{a\in \{0,1\}^{n-1}}
\ba{f(a),f(a)+\rc{3^n}}\cup \ba{f(a)+\frac{2}{3^n},f(a)+\rc{3^{n-1}}}.
\]
Moreover, the $2^n$ intervals in the above union are disjoint.
\end{lem}
\begin{proof}
We proceed by induction on $n$. For $n=1$, the equation holds. Now suppose it holds for $n$, and that the intervals in the expression are disjoint. Now $C_{n+1}$ is obtained by deleting the middle third of each interval. Thus $\ba{f(a),f(a)+\rc{3^n}}$ becomes
\[
\ba{f(a),f(a)+\rc{3^{n+1}}}\cup \ba{f(a)+\frac{2}{3^{n+1}},f(a)+\rc{3^n}}
=
\ba{f(a'),f(a')+\rc{3^{n+1}}}\cup \ba{f(a')+\frac{2}{3^{n+1}},f(a')+\rc{3^n}}
\]
where $a'=(a_1,\ldots, a_{n-1},0)$.
Similarly $\ba{f(a)+\frac{2}{3^n},f(a)+\rc{3^{n-1}}}$ becomes
\begin{align*}
&\quad\ba{f(a)+\frac{2}{3^n},f(a)+\frac{2}{3^n}+\rc{3^{n+1}}}\cup \ba{f(a)+\frac{2}{3^{n}}+\frac{2}{3^{n+1}},f(a)+\rc{3^{n-1}}}\\
&=
\ba{f(a'),f(a')+\rc{3^{n+1}}}\cup \ba{f(a')+\frac{2}{3^{n+1}},f(a')+\rc{3^{n}}}
\end{align*}
where $a'=(a_1,\ldots, a_{n-1},1)$.
Hence
\begin{align*}
C_n&=\pa{\bigcup_{a\in \{0,1\}^{n},\,a_n=0} \ba{f(a),f(a)+\rc{3^{n+1}}}\cup \ba{f(a)+\frac{2}{3^{n+1}},f(a)+\rc{3^{n}}}}\\
&\quad
\cup
\pa{\bigcup_{a\in \{0,1\}^{n},\,a_n=1}
\ba{f(a),f(a)+\rc{3^{n+1}}}\cup \ba{f(a)+\frac{2}{3^{n+1}},f(a)+\rc{3^{n}}}
}\\
&=\bigcup_{a\in \{0,1\}^n} \ba{f(a),f(a)+\rc{3^{n+1}}}\cup
\ba{f(a)+\frac{2}{3^{n+1}},f(a)+\rc{3^{n}}}.
\end{align*}
Note the intervals are disjoint because they are disjoint parts of disjoint intervals from the previous step.
\end{proof}
We see the following.
\begin{enumerate}
\item
$C$ is closed: Each $C_n$ is the union of closed intervals so is closed; $C$ is the intersection of closed sets so is closed.
\item
$C$ is uncountable: Define $\Psi:\Om\to [0,1]$ by 
\[\Psi(\om)=\sum_{i=1}^{\infty} \frac{2\omega(i)}{3^i}.\]
Note that
\[\Psi(\om)=f(a)+\frac{2\omega(n)}{3^n}+t\]
where $a=(\omega(1),\ldots, \omega(n-1))$
\[
0\leq t\leq \sum_{m=n+1}^{\infty} \frac{2}{3^m}=\rc{3^n}.\]
Hence,
\[
\Psi(\om)\in \ba{f(a),f(a)+\rc{3^n}}\cup \ba{f(a)+\frac{2}{3^n},f(a)+\rc{3^{n-1}}}\subeq C_n.
\]
This shows $\Psi(\om)\in C$. Moreover, if $\om_1,\om_2$ are distinct, choosing $n$ so that $\om_1(n-1)\neq \om_2(n-1)$, the above shows that $\Psi(\om_1)$ and $\Psi(\om_2)$ are in disjoint intervals.  Since $\Om$ is uncountable and $\Psi$ is injective, we conclude that $C$ is uncountable.
\item
$C$ has measure 0. Note $C_n$ is a union of $2^{n}$ disjoint intervals of length $\rc{3^n}$, so $\mu(C_n)=\pf 23^n$, which goes to 0 as $n\to \infty$. As $C\subeq C_n$, $\mu(C)\leq \mu(C_n)$ for any $n$ and $\mu(C)=0$.
\end{enumerate}
\end{problem}
\begin{problem}{\it(2.2.36, Absolute continuity of distribution functions)}
Suppose $F$ is absolutely continuous. For $\ep>0$, take $\de>0$ such that $\sum_{n=1}^{\infty} (F(b_n)-F(a_n))<\ep$ whenever $\{(a_n,b_n):n\geq 1\}$ is a sequence of mutually disjoint open intervals with $\sum_{n=1}^{\infty} (b_n-a_n)<\de$. Now suppose $|x-y|<\de$. Without loss of generality $x<y$. Taking $a_1=x,b_1=y$ and all other intervals to be empty, we get $\sum_{n=1}^{\infty} (b_n-a_n)<\de$ so $0\leq F(y)-F(x)\leq \ep$. This shows $F$ is uniformly continuous.
(If we don't allow empty intervals, pick the other intervals so that they are disjoint and their lengths sum to less than $\de-(y-x)$.)\\

Note that if $\mu_F$ is absolutely continuous with respect to $\la_{\R}$, then, since $\la_{\R}(\{a\})=0$, 
\[F(a)-F(a-)=\mu_F(\{a\})=0\]
for every $a$, so $F$ is continuous (it was already assumed right continuous). If $F$ is absolutely continuous, $F$ is continuous as shown above. Thus in proving the second part we may assume $F$ is continuous.

We have that $F$ is absolutely continuous iff for any $\ep>0$, there exists $\de>0$ such that $\sum_{n=1}^{\infty} (F(b_n)-F(a_n))<\ep$ whenever $\{(a_n,b_n)|n\in \N\}$ is a sequence of mutually disjoint open intervals with $\sum_{n=1}^{\infty} (b_n-a_n)<\de$. By countable additivity,
\begin{equation}\label{distreq1}
\mu_F\pa{\bigcup_{n=1}^{\infty} (a_n,b_n)}=\sum_{n=1}^{\infty} (F(b_n-)-F(a_n)).
\end{equation}
and 
\begin{equation}\label{distreq2}
\la_{\R}\pa{
\bigcup_{n=1}^{\infty} (a_n,b_n)}=\sum_{n=1}^{\infty} (b_n-a_n)
.
\end{equation}
By continuity,~(\ref{distreq1}) gives
\begin{equation}\label{distreq3}
\mu_F\pa{\bigcup_{n=1}^{\infty} (a_n,b_n)}=\sum_{n=1}^{\infty} (F(b_n)-F(a_n)).
\end{equation}
\begin{lem}\label{openctun}
Any open set of $\R$ can be written as a countable union of disjoint open intervals.
\end{lem}
\begin{proof}
We can write $G$ as a union of (disjoint) connected components. The connected sets in $\R$ are intervals. Each of these intervals must be open since $G$ is open. $G$ is a countable union because each interval contains a rational number, and $\Q$ is countable.
\end{proof}
Thus 
\begin{equation}\label{bij}
\{(a_n,b_n)|n\in \N\}\leftrightarrow \bigcup_{n=1}^{\infty}(a_n,b_n)
\end{equation}
is a bijection between sequences of disjoint open intervals up to ordering, and open sets of $\R$.

Then~(\ref{distreq3}) and~(\ref{distreq2}) show the condition for absolute continuity is equivalent to: for every $\ep>0$ there exists $\de>0$ such that there exists an open set $G$ so that $\mu_{F}(G)<\ep$ whenever $\la_{\R}(G)<\de$. 
From Problem 4 (2.1.27), this is equivalent to $\mu_{F}$ being absolutely continuous with respect to $\la_{\R}$. (Note $\mu_F$ is finite as $F$ is bounded.)%, and hence $\mu_F(\R)=F(\infty)-F(-\infty)<\infty$.)
\end{problem}
\begin{problem}{\it(2.2.37, Singular distribution functions)}
First suppose $\mu_F\perp \la_{\R}$. By Problem 5 (2.1.28), for each $\de>0$ there exists an open set $G$ with $\la_{\R}(G)<\de$ and $\mu_F(G^c)=0$. The second condition is equivalent to $\mu_F(\R)=\mu_F(G)$. By~(\ref{distreq1}),~(\ref{distreq2}), and the bijection~(\ref{bij}), this is equivalent to: for each $\de>0$ there exists a sequence $\{(a_n,b_n)|n\in \N\}$ of mutually disjoint open intervals with $\sum_{n=1}^{\infty} (b_n-a_n)<\de$ and $F(\infty)-F(-\infty)=\sum_{n=1}^{\infty} (F(b_n-)-F(a_n))$. 
We claim that any choice of $\{(a_n,b_n)|n\in \N\}$ satisfying the above equation also satisfies $F(\infty)-F(-\infty)=\sum_{n=1}^{\infty} (F(b_n)-F(a_n))$.
Indeed,
since $F$ is increasing,
\[ F(\infty)-F(-\infty)=\sum_{n=1}^{\infty}(F(b_n-)-F(a_n))
\leq
\sum_{n=1}^{\infty}(F(b_n)-F(a_n))\leq \mu_F(\R)=F(\infty)-F(-\infty)\]
so equality holds everywhere. We can find appropriate disjoint intervals for every $\de>0$, so $F$ is singular.
%If $\{(a_n,b_n)\}$ is chosen to satisfy the second condition for some $\de$, then the above shows $\sum_{n=1}^{\infty}(F(b_n)-F(a_n))\leq F(\infty)-F(-\infty)$ so the same choice of intervals satisfies the first condition for $\de$.

Conversely, suppose $F$ is singular. For $\de>0$, take disjoint intervals $\{(a_n,b_n)|n\in \N\}$ such that $\sum_{n=1}^{\infty}(b_n-a_n)<\de$ and $\sum_{n=1}^{\infty}(F(b_n)-F(a_n))=F(\infty)-F(-\infty)$. Now let $A=\bigcup_{n=1}^{\infty}(a_n,b_n]$. Then by countable additivity
\[
\mu_F(A)=\sum_{n=1}^{\infty} \mu_F((a_n,b_n])=\sum_{n=1}^{\infty} F(b_n)-F(a_n)=F(\infty)-F(-\infty)=\mu_F(\R)
\]
and
\[
\la_{\R}(A)=\sum_{n=1}^{\infty} \la_{\R}((a_n,b_n])=\sum_{n=1}^{\infty} (b_n-a_n)<\de.
\]
%Since $\la_{\R}$ is regular there exists an open set $G$ containing $B$ such that $\la_{\R}(G)<\de$. (Choose $F\subeq G\subeq B$ so that $\la_{\R}(G\bs F)\leq \de-\sum_{n=1}^{\infty} (b_n-a_n)$.) Now $\mu_F(B)\leq \mu(G)\leq \mu_F(\R)$ so equality holds.

This shows that for every $\de>0$ there exists a set $A$ such that $\la_{\R}(A)<\de$ and $\mu_F(A)=\mu_F(\R)$ (i.e. $\mu_F(A^c)=0$). By Lemma~\ref{singcrit}, $\mu\perp \la_{\R}$.

%$F$ is singular iff for each $\de>0$ there exists a sequence $\{(a_n,b_n):n\geq 1\}$ of mutually disjoint open intervals with $\sum_{n=1}^{\infty} (b_n-a_n)<\de$ and $F(\infty)-F(-\infty)=\sum_{n=1}^{\infty} (F(b_n)-F(a_n))$.
%We claim this is equivalent to: for each $\de>0$ there exists a sequence $\{(a_n,b_n):n\geq 1\}$ of mutually disjoint open intervals with $\sum_{n=1}^{\infty} (b_n-a_n)<\de$ and $F(\infty)-F(-\infty)=\sum_{n=1}^{\infty} (F(b_n-)-F(a_n))$. 
%Indeed, note %if the latter is true than the first statement holds with the same choices of intervals, since the fact that $F$ is increasing gives 
%Since $F$ is increasing,
%\[ F(\infty)-F(-\infty)=\sum_{n=1}^{\infty}(F(b_n-)-F(a_n))
%\leq
%\sum_{n=1}^{\infty}(F(b_n)-F(a_n))\leq F(\infty)-F(-\infty).\]
%If $\{(a_n,b_n)\}$ is chosen to satisfy the second condition for some $\de$, then the above shows $\sum_{n=1}^{\infty}(F(b_n)-F(a_n))\leq F(\infty)-F(-\infty)$ so the same choice of intervals satisfies the first condition for $\de$.
%Now suppose $\{(a_n,b_n)\}$ is chosen to satisfy the first condition for $\de$. Consider $\{(a_n,b_n+\ep_n)\}$, for $\sum_{n=1}^{\infty}\ep_n<\de-(\sum_{n=1}^{\infty}(b_n-a_n))$. Now
%\[
%\sum_{n=1}^{\infty} (F(b_n+\ep_n)-F(a_n))\geq \sum_{n=1}^{\infty} (F(b_n)-F(a_n))
%\]
%
% since $F(b_n-\ep_n)<F(b_n)$.
%We claim that $F$ is left continuous at each $b_n$. Suppose $F$ is not left continuous at $b_n$; then $F(b_n-)<F(b_n)$. But 

%By~(\ref{distreq1}),~(\ref{distreq2}), and the bijection~(\ref{bij}), the second statement is equivalent to: for each $\de>0$ there exists an open set $G$ with $\la_{\R}(G)<\de$ and $\mu(\R)=\mu(G)$. Note $\mu(\R)=\mu(G)$ iff $\mu(G^c)=0$. By Problem 5 (2.1.28), the above statement is true iff $\mu\perp \la_{\R}$.
\end{problem}
\end{document}
