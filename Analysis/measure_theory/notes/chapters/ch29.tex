\lecture{Fri. 4/15/11}

\subsection{Transformations on Lebesgue Spaces}
\begin{pr}
If $f$ and $g$ are functions on $E_1\times E_2$ and $\rc{p}+\rc{p'}=\rc{q}+\rc{q'}=1$, then
\[
\ve{fg}_{L^1}\le \ve{f}_{(p,q)}\ve{g}_{(p',q')}.
\]
\end{pr}
\begin{proof}
Two applications of H\"older's inequality.
\end{proof}
We think of $K:E_2\times E_1\to \R$ as a ``matrix representation" of $\cal K$ if
Consider the map $\cal K:L(\mu_2)\to L(\mu_1)$ defined by 
\[(\cal Kf)(x_1)=\int K(x_1,x_2)f(x_2)\,\mu_2(dx_2).\]
Think of $K$ as a matrix representation of $\cal K$. $K$ is called the kernel. 

Any linear map $f:\R^m\to \R^n$ is continuous: Write $x=\sum_{j=1}^m \an{x,e_j}e_j$; then $f(x)=\sum_{j=1}^n \an{x,e_j}f(e_j)$. However this is not necessarily true for infinite-dimensional spaces.

To check a linear functions it suffices to check continuity at 0. %($\ve{Ax-Ay}=\ve{A(x-y)}$
Thus to check that $\cal K:L^p(\mu_2)\to L^p(\mu_1)$ is continuous, if suffices to show that
\begin{equation}\label{lfuncineq}
\ve{\cal K}_{L^r(\mu_1)}\le C\ve{f}_{L^p(\mu_2)}.
\end{equation}
\begin{thm}\label{lspacefuncineq}
Let $(E_1,\mu_1),(E_2,\mu_2)$ be $\si$-finite measure spaces, and let $p\in[1,\iy]$ and $q\in [1,\iy)$. Let $r$ be so that $\rc{r}=\rc{p}+\rc{q}-1\ge 0$. Define
\begin{align*}
M_1&=\sup_{x_2\in E_2} \ve{K(x_1,x_2)}_{L^q(\mu_1)}<\iy\\
M_2&=\sup_{x_1\in E_2} \ve{K(x_1,x_2)}_{L^q(\mu_2)}<\iy.
\end{align*}
Given $f\in L^{q'}(\mu_2)$, define
\[
(\cal K f)(x_1)=\int K(x_1,x_2)f(x_2)\,\mu_2(dx_2).
\]
(The integral is finite by H\"older.)
Then
\[
\ve{\cal K f}_{L^r(\mu_1)}\le M_1^{\frac qr}M_2^{1-\frac qr}\ve{f}_{L^p(\mu_2)}.
\]
\end{thm}
%Conservation of misery
\begin{proof}
First suppose $r=\iy$. Then $p=q'$. Therefore by H\"older's inequality,
\begin{align*}
\ab{\int K(x_1,x_2)f(x_2)\,\mu_2}&\le \ve{K(x_1,x_2)}_{L^q(\mu_2)}\ve{f}_{L^{q'}(\mu_2)}\\
&\le M_2\ve{f}_{L^{p}(\mu_2)}
%\le M_2^{\frac pq}M_1?
\end{align*}
Now take the sup.

Now suppose $p=1$. Then $q=r$. Then by Generalized Minkowski~(\ref{genmink}),
\begin{align*}
\ve{\cal K f}_{L^r(\mu_1)}&\le \ve{Kf}_{L^{(1,r)} (\mu_2,\mu_1)}\\
&\le\ve{Kf}_{L^{(r,1)}(\mu_1,\mu_2)}\\
&=\int\pa{\int |K(x_1,x_2)|^r |f(x_2)|^r\,\mu_1(dx_1)}^{\rc r}\,\mu_2(dx_2)\\
&= M_1 \ve{f}_{L^1(\mu_2)}.
\end{align*}
Switching order of integration with Minkowski allows $f(x_2)$ to come out as a constant.

Now suppose $r<\iy$, $p>1$. Then $q<r$ and $p\le r$. Let $\al=\frac qr\in (0,1)$. Then $(1-\al)p'=q$. From~(\ref{fpsup}), since $(E_1, \mu_1), (E_2\mu_2)$ are $\si$-finite,
\[
\ve{\xi}_{L^r}=\sup\{\ve{\xi \psi}:\psi\in L^{r'}, \ve{\psi}_{r'}\le 1\}.
\]
For $g\in L^{r'}(\mu_1)$,
\begin{align*}
\ve{g\cal Kf}_{L^1(\mu_1)}&\le 
\ve{\cal K}_{L^1(\mu_1\times \mu_2)}\\
& \le \ve{
(|f||K|^{\al})(|K|^{1-\al} |g|)
}_1\\
&\le \ve{|K|^{\al}f}_{%L^
(r,p)}%(\mu_1\times \mu_2)
\ve{g|K|^{1-\al}f}_{(r',p')}\\
&\le \ve{|K|^{\al}f}_{%L^
(r,p)}%(\mu_1\times \mu_2)
\ve{g|K|^{1-\al}f}_{L^{(p',r')}(\mu_2,\mu_1)}&\text{by Minkowski}
\end{align*}
Now
\begin{align*}
\ve{|K|^{\al}f}_{
(r,p)}&=
\ba{
\int
\pa{\int 
|f(x_2)|^r |K(x_1,x_2)|^q\,\mu(dx_1)
}^{\frac pr}\,\mu_2(dx_2)
}^{\rc p}.\\
&\le M_1^{\frac qr} \ve{f}_{L^p(\mu_2)}. 
\end{align*}
Similarly,
\[
\ve{g|K|^{1-\al}}_{(p',r')}\le M_2^{1-\al}\ve{g}_{L^{r'}(\mu_2)}.
\]
Combining these two gives the desired result.
%r>p, r'<p'
%Free up the variables!
\end{proof}
Note in the case of matrices, letting $A=[a_{i,j}]$, 
\begin{align*}
M_1&=\max_j \pa{ \sum_{i=1}^n|a_{ij}|^q}^{\rc q}\\
M_2&=\max_j \pa{ \sum_{j=1}^n|a_{ij}|^q}^{\rc q}
\end{align*}
Then
\[
M_1^{\frac qr}M_2^{1-\frac qr}\pa{\sum_j |x_j|^p}^{\rc p}.
\]

Theorem~\ref{lspacefuncineq} gives~(\ref{lfuncineq}), hence continuity of $\cal K$.
\begin{thm}
$\cal K$ is continuous.
\end{thm}
%\begin{thm}
%$L^p(\mu_2)\cap L^q(\mu_2)$ is dense in $L^p(\mu_2)$. Gives Lipschitz continuous, so unique continuous extension. ADD DETAILS.
%%Define on one space, can define on other space, transfer by continuity.
%\end{thm}