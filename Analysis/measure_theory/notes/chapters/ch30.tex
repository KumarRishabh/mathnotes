\lecture{Wed. 4/20/11}

\subsection{Convolution}
Theorem~\ref{lspacefuncineq} gives the following.
\begin{thm}
Define
\[\La_k(f)=\bc{x:\int |K(x_1,x_2)||f(x_2)|\,\mu_2(dx_2)<\iy}.\]
Then
\[
\mu_1(\La_k(f)^c)=0.
\]
In addition, defining 
\[
(\ol{\cal K}f)(x_1)
\begin{cases}
\int K(x_1,x_2)f(x_2)\,\mu_2(dx_2),&x_1\in \La_k(f)\\
0,&x_1\nin \La_k(f),
\end{cases}
\]
the inequality in~\ref{lspacefuncineq} holds for all $f\in L^p$. Furthermore $\cal \ol{\cal K}$ is the unique map $L^p(\mu_2)\to L^r(\mu_1)$ that equals equals $\cal K$ on $L^p\cap L^{q'}$ and is continuous.
\end{thm}
\begin{proof}
If integrate with respect to $\mu_1$, get something finite, so for almost every $x_1$ the integral is finite.

For the second part note $L^p\cap L^{q'}$ is a dense subset of $L^p$.
\end{proof}
\begin{df}
Suppose $g\in L^q(\la_{\R^N})$ and let $K(x,y)=g(x-y)$. (Then $M_1=M_2=\ve{g}_q$.) %translation invariance of Lebesgue measure
Let
\[
\La(f,g)=\bc{
\int|f(y)||g(x-y)|\,dy
}.
\]
(Note $\La(f,g)^c$ has measure 0.) Define the {\textbf{convolution}} of $f$ and $g$ to be
\[
(f* g)(x)=\begin{cases}
\int f(y)g(x-y)\,dy&x\in \La(f,g)\\
0,&x\nin\La(f,g).
\end{cases}
\]
\end{df}
Note $*$ is commutative ($f*g=g*f$) and bilinear.
\begin{thm}[Young's inequality]
For $\rc{r}=\rc{p}+\rc{q}-1$, $p,q\in [1,\iy]$,
\[
\ve{f*g}_r\le \ve{f}_p\ve{g}_q.
\]
\end{thm}
There are two cases where this is particularly useful.

One, $p=q'$ and $r=\iy$. The inequality says
\[
\ve{f*g}_{\iy}\le \ve{f}_p\ve{g}_{p'}
\]
which is H\"older's. %(In fact, the above is true pointwise)
Convolution gives a continuous function. Stroock: ``if you go to the aquarium they have spines and if they make love they have to rub off some of the spines. Convolution is like rubbing, scraping off the irregularities."

Take $p\in [1,\iy)$. Letting $T_h(y)=y+h$, and letting $T_fh:=f\circ T_h$.
\begin{lem}
%Let $f$ be continuous. NOT NEEDED
Then
\[
\lim_{h\to \iy} \ve{T_h f-f}_p=0.
\]
\end{lem}
%Using integration theory to measure size is rather crude. It's quite forgiving.
\begin{proof}
First think of a set of $f$'s which are dense in $L^p$ for which this set is true is trivial. Then show the set of $f$ such that the result holds is closed.

Thus it suffices to show this for continuous functions with compact support (dense in $L^p$). The desired set is closed because if $f_n\to f$ in $L^p$ then
\[
\ve{T_hf-f}_p\le \ve{T_h(f-f_n)}_p+\ve{T_hf_n-f_n}_p+\ve{f_n-f}_p
\]
by the triangle inequality and Minkowski's inequality. Taking the lim sup,
\begin{align*}
\varlimsup_{|h|\to 0}\ve{T_hf-f}_p & \le \varlimsup_{|h|\to 0}\ve{T_h(f-f_n)}_p+\varlimsup_{|h|\to 0}\ve{T_hf_n-f_n}_p+\varlimsup_{|h|\to 0}\ve{f_n-f}_p\\
&\le 2\ve{f-f_n}_p+0+\limsup \ve{f_n-f}_p.
\end{align*}
For $p<\iy$ %Take the uniform norm of $T_h(f*g)=\ve{f*g}_u$ to get
BLAH DARN YOU ERASED THIS.
If $p=\iy$, $p'=\iy$, do the same thing %(Solomon dance)
but put on $g$ instead of $n$.

Vitalle's Lemma says if $\la_{\R}(\Ga)\in (0,y)$, then $1_{\Ga}$? $\Ga-\Ga\supeq (-\de,\de)$. We know 
\[
u=1_{\Ga}\times 1_{-\Ga}
\]
If $u>0$ then $u\in \Ga-\Ga$.
\end{proof}
When we convolve we get smoother objects!

%Commutative product
The other common case is when $q=1, p=r$. Then $f\in L^p$ and $g\in L^1$ give $f*g\in L^p$.

\begin{thm}
Suppose $p=q'$ and $g\in C^1$, with $g\in L^q$ and $\pd{g}{x_1}\in L^q$. Take $f\in L^p$. Then $\partial_{x_i}(f*g)=f*(\partial_{x_i}g)$.
\end{thm}
\begin{proof}
Write out difference quotients...
\end{proof}
Taking $g\in C^{\iy}$, we can keep on passing derivatives through the convolution. Writing $\partial^{\al}=\partial_{x_1}^{\al_1}\cdots \partial_{x_n}^{\al_n}$ where $\al=(\al_1,\ldots, \al_n)\in \N^N$, we have $\partial^{\al}(f*g)=f*(\partial^{\al}g)$. %We have an ubiquitous procedure for smoothing functions.

\begin{thm}
For $g\in L^1(\la_N)$, define $g_t(x)=t^{-N}g(t^{-1}x)$. (We scale in a way preserving the total interval; for small $t$ it concentrates around the origin.)  Given $\int g=1$ and $f\in L^p$, $f*g_t\to f$ in $L^p$ as $t\searrow 0$.
\end{thm}
\begin{proof}
Making a change of variables,
\[
f*g_t(x)=\int(f(x-y)-f(x))g_t(y)\,dy
=\int (f(x-ty)-f(x))g(y)\,dy.
\]
Taking the $L_p$ norm as a function of $x$, %well look
%think of func as x, y fix t.
\[\ve{f*g_t(x)}_p
=\ve{\int (f(x-ty)-f(x))g(y)\,dy}_p
=\ve{(f(x-ty)-f(x))g(y)}_{(1,p)}.\]
But we know
\[
\ve{T_{-ty}f-f}_p
\]
from the previous lemma, so combining these two we get a way of approximating an arbitrary element of $L^p$ by a smooth function. 
%K.O. Freiedrichs mollification procedure.
\end{proof}
%dominates what i put here

We want $g\in C^{\iy}_{c}(B(0,1),[0,\iy))$ such that $\int g=1$. Define
\[
\Psi(t)=\begin{cases}
e^{-\rc t},&t\ge 0\\
0,&t< 0;
\end{cases}
\]
this is a $C^{\iy}$. %Polynomials times $e^{-\rc t}$, latter wins at 0, so infinitely hard zero at zero.
Now consider 
\[
\Psi(t)=\begin{cases}
e^{-\rc{1-|x|^2}},&|x|< 1\\
0,&t\ge 1;
\end{cases}
\]
and normalize it so it has integral 1.

We want $\eta\in C^{\iy}$ with range in $[0,1]$ such that $1_{\Ga}\le \eta\le 1\le 1_{\Ga^{(\ep)}}$. We can let $\eta=g_{\ep}*1_{\Ga}$.
%Replace jump with smoother neighborhood.
%Some guy in a pink shirt comes in, says Oh and leaves.
%Haughty son-of-a-bitch, Stroock says.