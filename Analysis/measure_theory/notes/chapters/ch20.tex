\lecture{Fri. 3/18/11}

\subsection{Isodiametric inequality}
\begin{thm}[Isodiametric inequality]
\[
\la_{\R^N}(\Ga)\le \Om_N \rad(\Ga)^N.
\]
\end{thm}
\begin{proof}
If $-x\in \Ga$ whenever $x\in \Ga$, then $\Ga\subeq B(0,\rad(\Ga))$. Indeed, $x\in \Ga$ implies $2|x|=|x-(-x)|\le 2\rad(\Ga)$. Thus the theorem holds when $\Ga$ is symmetric.

Now we use the technique of Steiner Symmetrization to reduce the general case to the symmetric case.
%Isoperim by steiner symmetrization.
%Systematically symmetrize, in one dimension at a time. Choose orthonormal basis for $\R^N$, symmetrize 1st, 2nd, ... coordinate.

\begin{thm}[Steiner symmetrization]
Let $e$ be a unit vector in $S^{N-1}$. Let
\[
P(e)=\{x\in \R^N:x\perp e\}.
\]
Given a point $\xi\in P(e)$, look at the straight line passing through $\xi$ in the direction $e$, and consider the intersection of it with $\Ga$. Let $l(\Ga,e,\xi)$ be the one-dimensional Lebesgue measure of that intersection:
\[l(\Ga,e,\xi)=\la_{\R}(\{t\in \R:\xi+te\in \Ga\}).
\]
Define the symmetrization of $\Ga$ with respect to $e$ by
\[
S(\Ga,e)=\{\xi+te:|t|<\rc 2l(\Ga,e,\xi)\}.
\]
(Think of the set as gooey stuff. Poke a stick through it in the direction $e$ and squoosh the gooey stuff to remove the holes and so that half of the stuff is on each side of $P(e)$.)

%We need to check that
Then
\begin{align}
\rad(S(\Ga,e))&\le \rad(\Ga)\\
\la_{\R_N}(S(\Ga,e))&=\la_{\R^N}(\Ga).
\end{align}
Furthermore, if $T_{\sO}$ is an orthogonal transformation such that  $T_{\sO}\Ga=\Ga$ and $T_{\sO}e=\pm e$ then $S(\Ga,e)=T_{\sO}S(\Ga,e)$.
The symmetrized object will be invariant under the orthogonal transformation.
\end{thm}
\begin{proof}
We may assume $e=(0,\ldots, 0,1)$; then $P(e)=\R^{N-1}\times \{0\}$. Let
\[
f(\xi)=l(\Ga,e,(\xi,0))=\int 1_{\Ga}((\xi,t))\,dt.
\]
(Note this is measurable by Lemma~\ref{meas2var}.) Now by Fubini's Theorem ~(\ref{fubini}),
\[
\la_{\R}(\Ga)=\int_{\R^{N-1}} f(\xi)\,d\xi=\int_{\R^{N-1}}\int_{\R} 1_{\Ga}((\xi,t))\,dt\,d\la_{\R^{N-1}}.
\]

For the second part, we may assume $\Ga$ is bounded, and $\Ga$ is closed (since $\ol{\Ga}=\Ga$), hence compact. Suppose $x,y\in S(\Ga,e)$ with $x=(\xi,s)$ and $y=(\eta,t)$. Let
\begin{align*}
M^+(x)&=\sup\{\al:(\xi,\al)\in \Ga\}\\
M^+(y)&=\sup\{\al:(\eta,\al)\in \Ga\}\\
M^-(x)&=\inf\{\al:(\xi,\al)\in \Ga\}\\
M^-(y)&=\inf\{\al:(\eta,\al)\in \Ga\}
\end{align*}
(The lowest and highest points in $\Ga$ along the direction $e$ from $x$ or $y$.) Note $M^+(x)-M^-(x)\ge 2|s|$. Similarly $M^+(y)-M^-(y)\ge 2|t|$. Now
\begin{align*}
\max(M^+(y)-M^-(x),M^+(x)-M^-(y))
&\ge \frac{M^+(y)-M^-(x)}{2}+\frac{M^+(x)-M^-(y)}{2}\\
&=\frac{M^+(x)-M^-(y)}{2}+\frac{M^+(y)-M^-(y)}{2}\\
&\ge |s|+|t|.
\end{align*}
Using the distance formula $|y-x|^2\le 4\rad(\Ga)^2$.

The last assertion is clear.
\end{proof}
Take an orthonormal basis $e_1,\ldots, e_N$. Let $\Ga_0=\Ga$ and $\Ga_{n+1}=S(\Ga_n,e_{n+1})$. 
We pick up more and more directions of symmetry until it's completely symmetric. 
($\Ga_1$ is invariant under reflection in the first coordinate, $\Ga_2$ is invariant under orthogonal transformations changing the first and second coordinate, and so on.)
Now
\[
\la_{\R^N}(\Ga)=\la_{\R^N}(\Ga_N)\le\Om_N \rad(\Ga)^N
\]
by the symmetric case, as needed.
%$
\end{proof}
\subsection{Hausdorff's construction}
\begin{df}
Given $\de>0$, define
\[
%ctable collection of sets
H^{N,\de} (\Ga)=\inf \bc{
\sum_{C\in \cal C} \Om_N\rad(C)^N:
\Ga\subeq \bigcup \cal C\text{ and }\Vert \cal C\Vert \le \de
}.
\]
%the infimum taken over all $C\in \cal C$, 
Note as a function of delta, this increases as delta decreases. The \textbf{Hausdorff measure} of $\Ga$ is
\[
H^N(\Ga)=\lim_{\de\searrow 0} H^{N,\de}(\Ga).
\]
\end{df}
\begin{thm}[Lebesgue and Hausdorff measure coincide for measurable sets]
If $\Ga\in \cal B_{\R^N}$ then $H^N(\Ga)=H^{N,\de}(\Ga)=\la_{\R^N}(\Ga)$.
\end{thm}
\begin{proof}
The inequality $\la_{\R^N}(\Ga)\le H^{N,\de}(\Ga)$ holds by the Isodiametric Inequality. The opposite inequality uses the following.
\begin{lem}[Covering lemma]
Given $G$ open, there exists a sequence $\{B_k:k\ge 1\}$ of closed mutually disjoint balls such that $G\supeq \bigcup_{n=1}^{\iy} B_n$ and
\[
\la_{\R^N}\pa{G\bs\bigcup_{k=1}^{\iy} B_k
}=0.
\]
\end{lem}
I.e. we can cover almost all of $G$ with a countable number of disjoint balls.
\begin{proof}
Recall we can write $G$ as a union of a countable number of nonoverlapping cubes $G=\bigcup_{n=1}^{\iy} Q_n$. Put a circle inside each cube; they occupy a volume that is a constant times the volume of the cube. Let these balls be $B_1,\ldots$; take enough of them (a finite number) so they've covered a fixed constant of the volume of $G$. Since we stopped with a finite number, $G\bs \bigcup_{n=1}^{n_1} B_n$ is open. Repeat with this set and keep going.
\end{proof}
Then $\la_{\R^N}\ge H^{N,\de}(G)$.
\end{proof}
Note we can also define $H^{s,\de}$ for $s\ne N$:
\[
%ctable collection of sets
H^{N,\de} (\Ga)=\inf \bc{
\sum_{C\in \cal C} \Om_s\rad(C)^s:
\Ga\subeq \bigcup \cal C\text{ and }\Vert \cal C\Vert \le \de
}.
\]
Unlike before there is dependence of $\de$. 
Then the $s$-dimensional Hausdorff measure is
\[
H^N(\Ga)=\lim_{\de\searrow 0} H^{N,\de}(\Ga).
\]
For $s>0$ the measure is identically 0.
For $s<N$ it's not a finite measure, it's $\iy$ for open sets; it's meaningful for ``$s$-dimensional" stuff.
They provide generalization of surface measure.
%H^{N-1}|M=\la_M surface measure