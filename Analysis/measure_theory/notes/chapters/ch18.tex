\lecture{Mon. 3/14/11}

\subsection{Lebesgue's differentiation theorem}

\begin{lem}\label{lipint}
Suppose $F(-\iy)=0$ and $F$ satisfies the Lipschitz condition 
\[0\le F(y)-F(x)\le L(y-x)\text{ for }x\le y.\]
Then $F$ is a definite integral.
\end{lem}

\begin{proof}
Let
\[
f_n(x)=2^n(F((k+1)2^{-n})-F(k2^{-n})), \,k2^{-n}\le x<(k+1)2^{-n}.
\]
Then $0\le f_n\le L$ and
\[
\int_{-\iy}^{\iy} f_n\,dx=F(\iy).
\]
Let $m<n$. %Note $f_m$ is constant on the interval $[k2^{-n}, (k+1)2^{-n})$. 
Consider
\begin{align*}
\int_{[k2^{-m}, (k+1)2^{-m})} f_m f_n\,d\la
&=f_m(k2^{-m})\int f_n\,d\la\\
&=2^{-m} f_m(k2^{-n})^2 \\
&=\int f_m^2\,d\la.
\end{align*}
Hence
\[
\int (f_n-f_m)^2\,d\la=\int f_n^2\,d\la-\int f_m^2\,d\la\to 0
\]
since $\int f_n^2\le LF(\iy)$. (A bounded monotonic sequence converges.) By Markov's inequality,
\[
\mu(|f_n-f_m|\ge\ep)=\mu(|f_n-f_m|^2\ge\ep^2)\le \rc{\ep^2}\int |f_n-f_m|^2\,d\la.
\]
Taking the sup over $n>m$, 
\[
\sup_{n>m}\mu(|f_n-f_m|\ge \ep)\le \rc{\ep^2}\sup_{n>m}\int |f_n-f_m|^2\to 0.
\]
By Cauchy's Criterion for convergence in measure~(\ref{cccom}), there exists $f$ so that $f_n\xrightarrow{\mu} f$. 
%(Subseq conv almost everywhere, true for each $f_n$, true for $f$) \
Changing $f$ on a set of measure 0 we may assume $0\le f\le L$. 

Consider $F(y)-F(x)$. For $n\ge m$,
\[
F(l2^m)-F(k2^{-m})=\int_{[k2^{-m}, l2^{-m})} f_n\,d\la
\to\int f\,d\la\text{ as }n\to\iy.
\]
by Lebesgue's Dominated Convergence Theorem for convergence in measure ($L$ is a Lebesgue dominant of the $f_m$). Let $F(y)=\int_{(-\iy,y]}f$. Hence for dyadics,
\[
F(y)-F(x)=\int_{(x,y]} f\,d\la.
\]
Both sides are continuous and the dyadics for a dense set so this is true for all $x$ and $y$.
%If Lipschitz continuous then indefinite integral.
\end{proof}

\begin{lem}\label{abint}
Suppose $F$ is absolutely continuous and nondecreasing. Then $F$ is a definite integral. (Hence, by Theorem~\ref{ftoc2}, it is differentiable a.e.)
\end{lem}
\begin{proof}
For every $\ep>0$ and $\de>0$ there exist intervals $I$ such that $\sum |I|<\de$ and
\[
\sum_I F(I^+)-F(I^-)<\ep.
\]
From a homework problem, $F$ is absolutely continuous iff $\mu_F$ is absolutely continuous with respect to $\la_R$ (i.e. $\la(\Ga)=0\implies \mu(\Ga)=0$). 

By Corollary~\ref{larlf}, $\mu_F(\cal LF=\iy)=0$. Let 
\[F_n(x)=\mu_F((-\iy, x]\cap \{\cal LF\le n\}).\]
It is nondecreasing, tends to 0 a $-\iy$, and increases to $F(x)$ as $n\to \iy$, because we've remove a set whose measure tends to 0. (The measure tends to $\mu((-\iy, x]\cap \{\cal L F<\iy\})=0$.)

We claim that $F_n$ is Lipschitz with constant $n$: $F_n(y)-F_n(x)\le n(y-x)$. Indeed, suppose $c\in(x,y]\cap \{\cal LF\leq n\}$. If $(x,y]$ does not intersect $\{\cal LF\le n\}$ then the difference is 0. Else, choose $c$ so that $c\in (x,y]\cap \{\cal LF\le n\}$. Then
\begin{align*}
F_n(y)-F_n(x)&=\mu_F((x,y]\cap \{\cal LF\leq n\})\\
&\le (F(y)-F(x))\\
&=(F(y)-F(x))-(F(c)-F(x))\\
&\le n(y-x).
\end{align*}
 Then by Lemma~\ref{lipint},
\begin{align*}
F_n(x)&=\int_{(-\iy, x]} f_n\\
F_{n+1}(x)-F_n(x)&=\int_{(-\iy, x]} f_{n+1}-f_n \,d\la\\
&=\mu_F((-\iy, x]\cap \{n<\cal L F\le n+1\}).
\end{align*}
Hence $F_n$ is nonnegative nonincreasing, and by the Monotone Convergence Theorem,
\[
F(x)=\lim_{n\to \iy}F_n(X)=\lim_{n\to \iy }\int_{(-\iy,x]} f_n\,d\la=\int_{(-\iy,x]}f\,d\la_{\R}.
\]
%For nondec cont.
\end{proof}

\begin{thm}\label{spns}
Let $F$ be a bounded, right-continuous, non-decreasing function with $F(-\iy)=0$.

Then $F$ can be decomposed into an absolutely continuous, non-decreasing $F_a$ and a singular, right-continuous, non-decreasing $F_b$:   %$F=F_{\la}+R-]k
$F=F_{a}+F_{b}$. 
\end{thm}
%\begin{lem}
%\end{lem}
\begin{proof}
Define the measures
\begin{align*}
\mu_a(\Ga)&=\mu_F(\Ga\cap \{\cal L F<\iy\})\\
\mu_b(\Ga)&=\mu_F(\Ga\cap \{\cal L F=\iy\})
\end{align*}
and let
\begin{align*}
\cal F_a(x)&=\mu_a((0,x])\\
\cal F_b(x)&=\mu_b((0,x])
\end{align*}
be the corresponding distribution functions.
%Now Hence $f_a$ absolutey continuous. By same reasoning $\la (\cal L F-\iy)=0$. Thus 
%\[
%\mu_{F_s}(\cal L<\iy)=0\implies \la(\cal L=\iy)=0
%\]
Since $\la_{\R}(\cal L F=\iy)=0$, $\mu_b$ is singular to $\la_{\R}$, so by Exercise 2.2.39, the distribution function $F_b$ is singular.

Since $\mu_a(\cal LF_a=\iy)\le \mu_F(\cal LF=\iy)=0$, $F$ is absolutely continuous by Exercise 2.2.38.
%By Exercise 2.2.28, since $F$ is absolutely continuous, $\mu_F$ is singular with respect to $\la_{\R}$. Since $\la_{\R}(\cal L=\iy)=0$, this gives $\mu_F(\cal LF=\iy)=0$. Hence $F_b$ is singular. 
%
%Note $F_a$ is singular since $\la_{\R}(\cal L F=\iy)$, giving $\mu_s(\Ga)=\mu_F(\Ga\cap \{\cal L F=\iy\})$ sincular to $\la_{\R}$.
%CLARIFY!
\end{proof}

\begin{lem}
If $F$ is singular, then $F'$ exists and equals 0 Lebesgue almost everywhere. %($\iy$ velocity at set $F$ can't see.)
\end{lem}
\begin{proof}
Since $F$ is singular, $\mu_F$ is singular to $\la_{\R}$ by Exercise 2.2.38. Thus there exists a set $B$ such that $\mu_F(B)=0$ and $\la_{\R}(B^c)=0$. Let $\{K_n\}$ be closed subsets of $B^c$ such that $K_n\subeq K_{n+1}$ and $\mu_F(B^c\bs K_n)\le \rc n$ for each $n\in \N$.
Set
\[F_n(x)=\mu_F((-\iy, x])\cap K_n.\]
%\mu_F$ regular. 
%But thm  $K_n\nearrow$ 
%such that
%\[\mu_F(B^c\bs K_n)<\rc n.\] 
For $x\in B$, $x$ is at positive distance away from the closed set $K_n$, so $F_n'(x)=0$. 
%
%n not elt of b, not $K_n$.
%Empty intersection.

%nondecreasing wrt x
For $x\in B$, $F_n'(x)=0$ for each $n$, so this is true for $F$ as well. 
%\[
%F_{n+1}((-\iy,x]\cap (K_{n+1}\bs K_n)).\]
Now
\begin{align*}
\lim_{n\to \iy} F_n(x)&=\mu_F((-\iy, x]\cap B^c).\\
&=\mu_F((-\iy, x])&(B \text{ has measure 0})\\
&=F(x).
\end{align*}
%Are there some more things to check here?
\end{proof}
Any $F$ satisfying the conditions of Lebesgue's differentiation theorem can be written as $F_a+F_b$ as in Theorem~\ref{spns}. Both $F_a$ and $F_b'$ have derivatives except on a set of measure 0, so the theorem follows.
%5 Change of variables (Jacobi), surface of measure
%6 Inequalities
%7 Fourier analysis
%8 Abstraction 