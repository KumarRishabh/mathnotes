\lecture{Mon. 2/7/2011}

\subsection{Integration by parts}
\begin{thm}[Integration by parts]
Suppose $\ph$ and $\psi$ are bounded functions on $J$. If $\ph$ is Riemann integrable, then $\psi$ is $\ph$-Riemann integrable, and
\[
(R)\int_J\psi(x)\,d\xi(x)=[\ph(x)\psi(x)]^b_a-(R)\int_J \ph(x)\,d\psi(x).
\]
\end{thm}
\begin{proof}
Let $\cal C=\{[\al_{m-1},\al_m]:1\leq leq n\}$ where $a=\al_0<\ldots<\al_n=b$. Take $\ph([\al_{m-1},\al_m])=\beta_m\in [\al_{m-1},\al_m]$. Now
\begin{align*}
\cal R(\ph|\psi;\cal C, \ph)&=\sum_{m=1}^n \psi(\be_m)(\ph(\al_m)-\ph(\al_{m-1}))\\
&=\sum_{m=1}^n \psi(\be_m)\ph(\al_m)-\sum_{m=0}^{n-1} \psi(\beta_{m+1})\ph(\al_m)\\
&=\psi(\be_n)\ph(b)-\psi(\be_1)\ph(a)-\sum_{m=1}^{n-1} \ph(\al_m)(\psi(\be_{m+1})-\psi(\be_m))
\end{align*}
Now we think of the $\be_m$ as the endpoints of intervals and $\al_m$ as the choices. Let $\be_0=a$ and $\be_{n+1}=b$. Rearranging gives
\[
\psi(b)\ph(b)-\psi(a)\ph(a)-\sum_{m=0}^n \ph(\al_m)(\psi(\be_{m+1})-\psi(\be_m)).
\]
The mesh size of the $\be$-partition is at most twice the mesh size of the $\al$-partition. 
Since $\ph$ is $\psi$-integrable, this last sum approaches $(R)\int_J \ph(x)\,d\psi(x)$, as needed. 
\end{proof}
\begin{cor}[Fundamental Theorem of Calculus]
Suppose $\ph$ is differentiable. Then
\[\int_J \frac{d\xi(x)}{dx}\,dx=[\ph(x)]^b_a\]
\end{cor}
\begin{proof}
Take $\psi=1$ and note $\int_J \frac{d\xi(x)}{dx}\,dx=\int_J d\xi(x)$.
\end{proof}
\subsection{Riemann-Stieltjes integrability}
\begin{thm}
If $\psi$ is increasing, then every $\ph\in C(J;\R)$ is $\psi$-Riemann integrable.
\end{thm}
\begin{proof}
Define $\cal U(\ph|\psi; \cal C)$ and $\cal L(\ph|\psi; \cal C)$. Use uniform continuity.
\end{proof}
\begin{pr}
If $\psi_1,\psi_2$ are increasing, then for every $\ph\in C(J;R)$,
\[
(R)\int_J \ph(x)\,d(\psi_1-\psi_2)=(R)\int_J \ph(x)\,d\psi_1-(R)\int_J \ph(x)\,d\psi_2.
\]
\end{pr}
\subsection{Variation}
\begin{pr}
%\[
%\ab{(R)\int_J \ph(x)\,d\psi(x)}\leq ||\ph||_u=\sup_J |\ph|.
%\]
If $\psi=\psi_2-\psi_1$,
\[
\ab{(R)\int_J \ph(x)\,d\psi(x)}\leq ||\ph||_u(\De_J\psi_1+\De_J\psi_2).
\]
\end{pr}
We can ask the following: for what functions $\psi$ does there exist $K_{\psi}$ such that for every $\ph\in C(J;R)$ $\psi$-integrable and 
\[\ab{(R)\int \ph\,d\psi}\leq K_{\psi}||\ph||_u?\]
\begin{thm}
Only functions that are the difference of two increasing functions.
\end{thm}
We give a better description of this criterion.
\begin{df}
Let
\[
S(\psi,\cal C)=\sum_{I\in\cal C} |\De_I \psi|.
\]
The \textbf{variation} of $\psi$ on $J$ is
\[
\text{Var}(\psi, J)=
\sup_{\cal C}
S(\psi;\cal C).
\]
\end{df}
This measures the amount of ``up-and-down" jiggliness of the function.
\begin{pr}[Basic properties]

\begin{enumerate}
\item By the Triangle Inequality, if $\cal C'$ is a refinement of $\cal C$, then 
$S(\psi;\cal C')\geq S(\psi;\cal C)$.
\item If $J=J_1\cup J_2$, $\Var(\psi,J)=\Var(\psi,J_1)+\Var(\psi,J_2)$.
\end{enumerate}
\end{pr}
\begin{df}
For $a\in \R$ let $a_+=\max(a,0)$ and $a_-=\max(-a,0)$. Define
\begin{align*}
S_+(\psi;\cal C)&=\sum_{I\in \cal C} (\De_I \psi)^+\\
S_-(\psi;\cal C)&=\sum_{I\in \cal C} (\De_I \psi)^-\\
\Var_{\pm} (\psi; \cal C)&=\sup_{\cal C} S_{\pm}(\psi; \cal C).
\end{align*}
\end{df}
Note $a_+-a_-=a$ and $a_++a_-=|a|$, so 
\begin{align*}
S_+(\psi,\cal C)-S_-(\psi, \cal C)&=\De_J \psi\\
S_+(\psi,\cal C)+S_-(\psi, \cal C)&=S(\psi;\cal C)\\
S_+(\psi,\cal C)&=\rc{2}(\De_J\psi-S(\psi; \cal C))\\
S_-(\psi,\cal C)&=\rc{2}(\De_J\psi+S(\psi; \cal C))
\end{align*}
If approach extreme values for one than for all of them. Statements pass to variations.
\begin{align*}
\Var_+(\psi;J)-\Var_-(\psi;J)&=\De_J \psi\\
\Var_+(\psi,J)+\Var_-(\psi;J)&=\Var(\psi;J)
%S_+(\psi,\cal C)&=\rc{2}(\De_J\psi-S(\psi; \cal C))\\
%S_-(\psi,\cal C)&=\rc{2}(\De_J\psi+S(\psi; \cal C))
\end{align*}
%Every cont func St integrable and we have this kind of control on the size?