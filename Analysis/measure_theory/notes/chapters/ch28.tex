\lecture{Wed. 4/13/11}

\subsection{Inequalities, continued}
\begin{thm}\label{fpsup}
For $p$ and $p'$ in $(1,\iy)$ such that $\rc p+\rc{p'}=1$,
\[\ve{fg}_1\le \ve{f}_p\ve g_{p'}.\]
For $f\in L^p$,
\[
\ve{f}_p=\sup\{
\ve{fg}_1:g\in L^{p'},\,\ve{g}_{p'}\le 1
\}
\]
\end{thm}
\begin{proof}
The first equality is just H\"older's inequality. 
For the second statement, ``$\le$" follows from the first equality. If $\ve{f}_p=0$ then $f$ vanishes except on a set of measure 0 so equality holds. Else, take $g=\frac{|f|^{p-1}}{\ve f_p^{p-1}}$. Then $|g|^{p'}=\frac{|f|^p}{\ve f^p}$ and $|fg|=\frac{|f|^p}{\ve f_p^{p-1}}$. 
(For $p=1$ take $g=1$.)
\end{proof}
The following gives a similar result when it is not {\it a priori} known that $f\in L^p$.
\begin{thm}
If $p=1$; or $p\in (1,\iy)$ and either $\mu(|f|>\de)<\iy$ or for every $\de>0$ or $\mu$ is $\si$-finite, then
\[
\ve{f}_p=\sup\{
\ve{fg}_1:g\in L^{p'},\,\ve{g}_{p'}\le 1
\}.
\]
\end{thm}
\subsection{Mixed Lebesgue spaces}
\begin{df}
let $(E_1,\cal B_1,\mu_1)$ and $(E_2,\cal B_2, \mu_2)$ be $\si$-finite.
Suppose $f$ is $\cal B_1\times \cal B_2$-measurable, $p_1,p_2\in [1,\iy)$. 
Then define
\begin{align*}
\ve{f}_{L^{(p_1,p_2)}(\mu_1\times \mu_2, R)} 
&=\ve{
\ve{f(x_1,x_2)}_{L^{p_1}(\mu_1;\R)}
}_{L^{p_2}(\mu_2;\R)}\\
&=\ba{
\int \pa{\int
|f(x_1,x_2)|^{p_1}\,\mu(dx_1)
}^{\frac{p_2}{p_1}}\,\mu(dx_2)
}^{\rc{p_2}}.
\end{align*}
\end{df}
($L^{(p_1,p_2)}$ denotes the space of functions for which the above norm is defined and finition.) Note for $p_1=p_2$ this defined just like in Fubini.
\begin{thm}[Generalized Minkowski's inequality]\label{genmink}
If $p_1\le p_2$, then
\[\ve{f}_{(p_1,p_2)}\le \ve{f}_{(p_2,p_1)}.\]
\end{thm}
For example, taking $E_1=\{1,2\}$, $p_1=1$ and $p_2=p$ we get out Minkowski's inequality.
This will give estimates for linear operators on $L^p$ spaces.
First we prove need the following lemma.
\begin{lem}\label{genminkhelper}
%On a finite measure
For $\mu_1$ and $\mu_2$ finite, for every $\ep>0$ there exists a function $\psi(x_1,x_2)=\sum_{m=1}^n 1_{\Ga_{1,m}}(x_1)\ph_m(x_2)$ where
\begin{enumerate}
\item the $\Ga_{1,m}$'s are mutually disjoint,
\item the $\ph_m$ are bounded, and
\item $\ve{f-\psi}_{(p_1,p_2)}<\ep$.
\end{enumerate}
\end{lem}
We want to approximate $f$ on $E_1\times E_2$ by these functions because the fact that the $\Ga_{1,m}$ are mutually disjoint makes it easy to raise $\psi$ to a power:
\[
\pa{\sum_{m=1}^n 1_{\Ga_{1,m}}(x_1)\ph_m(x_2)}^p=\sum_{m=1}^n 1_{\Ga_{1,m}}(x_1)\ph_m(x_2)^p
\]
\begin{proof}
By H\"older's inequality with $\frac qp$, $p\le q$, 
%\[
%\int |f|^p\,d\mu\le\pa{ \int \ve{f}^p_q\,d\mu}\pa{\mu(E)^{1-\frac pq}}.
%\]
\[
\int _E |f|^p\,d\mu\le \pa{\int_E |f|^{p\cdot \frac qp}\,d\mu}^{\frac pq} \pa{\int_E 1\,d\mu}^{1-\frac{p}{q}}=\pa{\int_E |f|^q\,d\mu}^{\frac pq}\mu(E)^{1-\frac pq}.
\]

Thus there exists $C$ so that
\begin{equation}\label{boundbyprod}
\ve{g}_{(p_1,p_2)}\le C\ve{g}_{L^q(\mu_1\times \mu_2)},
\end{equation}
and convergence of functions in $L^q(\mu_1\times \mu_2)$ implies convergence in $L^{(p_1,p_2)}$. 
We know functions of the form $\sum_{k=1}^l a_j1_{\Ga_{1,j}\times \Ga_{2,j}}$ are dense in $\mu_1\times \mu_2$ since sets of the form $\Ga_1\times\Ga_2$ generate the \sia{} (Theorem~\ref{lpbasic}(4)).
Let $\eta\in I:=\{0,1\}^l$ and let
\[
\Ga_{1,\eta}=\Ga_{1,1}^{(\eta_1)}\cap \cdots \cap \Ga_{1,l}^{(\eta_2)}.
\] 
(Define $\Ga^{(1)}=\Ga,\,\Ga^{(0)}=\Ga^c$.) Then
\begin{align*}
1_{\Ga_{1,j}}(x_1)&=\sum_{\eta\in I} \eta_j 1_{\Ga_{1,\eta}}(x_1)\\
\sum_{j=1}^l a_j1_{\Ga_{1,j}\times \Ga_{2,j}}&=\sum_{j=1}^l \sum_{\eta} a_j\eta_j1_{\Ga_{1,\eta}}(x_1) 1_{\Ga_{2,\eta}}(x_2) \\
&=\sum_{\eta\in I}1_{\Ga_{1,\eta}}(x_1)\underbrace{\sum_{j=1}^l \eta_ja_j1_{\Ga_{2,j}}(x_2)}_{\ph_{\eta}}
\end{align*}
\end{proof}
\begin{proof}(of Theorem~\ref{genmink})
It suffices to show the finite case since a $\si$-finite measure is the union of subsets of finite measure. By Lemma~\ref{genminkhelper} and~(\ref{boundbyprod}), it suffices to show the result for functions in the above form. 

Now (using Minkowski, noting $p_2\ge p_1$)
\begin{align*}
\ve{\psi}_{(p_1,p_2)}
&=
\ba{
\int_{E_2} \pa{
\sum_{m=1}^n
\mu_1(\Ga_{1,m})
|\ph_m(x_2)|^{p_1}\,\mu(dx_1)
}^{\frac{p_2}{p_1}}\,\mu(dx_2)
}^{\frac{1}{p_2}}.\\
&\le \ba{\sum_{m=1}^n \mu_1(\Ga_{1,m})  \ve{\ph_m}^{p_1}_{L^{p_2}}\mu_2(dx_2)}^{\rc{p_1}}\\
%Skip details.
%&=\sum_{m=1}^n \mu_1(\Ga_{(1,m)} \ph_m\ve{\xi_m}^{p_1}_{L^{p_2}(d\mu)}\\
%&\le \ba{
%\int
%\sum_{m=1}^n 1_{|Ga)k)1
%}
%\end{align*}
&=\ve{\psi}_{(p_2,p_1)}
\end{align*}
\end{proof}
We investigate linear maps $\cal K:L^p(\mu)\to  L^q(\nu)$. %On a finite dimensional space linear maps are continuous, but in infinite dimensions this isn't always true. 
%Apply 1-D infinitely many times, blows up in your face.  
We think of $K:E_2\times E_1\to \R$ as a ``matrix representation" of $\cal K$ if
\[\cal Kf(x_2)=\int K(x_2,x_1)f(x_1)\,\mu_1(dx_1).\]
We estimate this in terms of the function $K$ (kernel).
