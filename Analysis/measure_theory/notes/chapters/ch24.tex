\lecture{Mon. 4/4/11}

\subsection{Surface measure}
\begin{df}
%A \textbf{hypersurface} $M$ is a set in $\R^n$ so that for $p\in M$ and $r>0$, there exists $F$ with $|\nabla F|>0$.
%\[
%M\cap B(p,r)=\{x\in B(p,r):F(x)=0\}.
%\]
%%Define the tangent space $T_p(M)$ to be $\Ga\subeq M$ such that 
Define a cylindrical neighborhood of a hypersurface $M$ to be
\[
\Ga(\rh)=\{x:\exists p\in \Ga,|x-\rh|<\rh,\,x-p\perp T_p(M).\}
\]
%Growing perpendicular hair.
\end{df}

We will define a measure by
\[
\la_M(\Ga)=\lim_{\rh\searrow 0} \frac{\la_{\R^N}(\Ga(\rh))}{2\rh}.
\]
Now %for $\Ga\in \cal B_M$,
\[
%\Ga\cap B(p,r)
\Ga(\rh)=\bc{q+\xi\frac{\nabla F(q)}{|\nabla F(q)|}:q\in \Ga,\,|\xi|<\rh}.
\]
This is measurable because a Lipschitz function maps Lebesgue measurable sets to measurable sets (it maps compact to compact, closed to closed, countable unions of closed to closed; every Lebesgue measurable set differs by at most a set of measure 0 from such a set; by Lipschitz continuity a set of measure 0 goes to a set of measure 0), and $(q,\xi)\mapsto q+\xi\frac{\nabla F(q)}{|\nabla F(q)|}$ is Lipschitz.
%If we grow the hairs too long they might cross, $\Ga\cap \Ga'=\phi$, so $\Ga\cap \Ga'=\phi$ does not imply $\Ga(\rh)\cap \Ga'(\rh)=\phi$.

\begin{pr}
For every $p\in M$ there exists $U$ an open neighborhood of $O_{\R^{N-1}}$ and a $\Psi:U\to M$ such that $\Psi(O_{\R^{N-1}})=p$ and there exists $r>0$ such that $\Psi(U)=M\cap B_{\R^N}(p,r)$, and for every $u\in U$
\[
\{\partial_{u_1}\Psi(u),\ldots, \partial_{u_{N-1}}\Psi(u)\}
\]
is a basis for $T_{\Psi(u)}(M)$.
($(U,\Psi)$ is a coordinate chart.)
\end{pr}
\begin{proof}
By taking $r$ small enough we may assume without loss of generality that $|\partial_{x_N}F|>0$ on $B_{\R^N} (p,r)$. Define
\[
\Phi(x)=\ba{\begin{matrix}
x_1-p_1\\
\cdots \\
x_{N-1}-p_{N-1}\\
F(x)
\end{matrix}}.
\]
The Jacobian is
\[
\pd{\Ph}x=\matt{I_{N-1}}0{\partial_{u_1}F\; \cdots }{\partial_{u_N}F}.
\]
Its determinant doesn't vanish so by the Inverse Function Theorem, $F$ admits an inverse in a neighborhood of the origin. Define $\Psi(u)=\Phi^{-1}((u,0))$ on a neighborhood of 0. Note $\Phi^{-1}(O_{\R^{N-1}})=p$, and $\Psi(u)\in M$. Thus the derivative is in the tangent space; each $\partial_{u_i} \Psi(u)$ is in the tangent space. Since they are linearly independent and there are $N-1$ of them, they form a basis. (The $N-1$ column vectors are the first $N-1$ vectors in the Jacobian of $\Phi^{-1}$, which has nonzero determinant.)
\end{proof}
Let $(U,\Psi)$  be a coordinate chart at $p$. Define $\tilde{\Psi}$ on $U\times \R$ by
\[
\tilde{\Psi}(u,\xi)= \Psi(u)+\xi n(u)
\]
where $n(u)$ is the normal at $u$. 
It parameterize points near the hypersurface by points on the hypersurface and how far to go perperdicularly. %(along a hair). 
Note
\[
(\det A)^2=\det(A^TA)=\det\ba{(v_i,v_j)_{\R^N}}_{1\le i,j\le N}.
\]
The Jacobian of $\tilde{\Psi}$  at $(u,0)$ is 
\[
J\tilde{\Psi}(u,0)=\sqrt{
\ba{
\an{\partial_{u_i}\Psi, \partial_{u_j}\Psi}
}_{1\le i,j\le N}
}>0
\]
since ???
\[
J\tilde{\Psi}(u,0)=(\partial_{u_1}\Psi,\ldots, \partial_{u_{N-1}} \Psi, n \Psi(u)).
\]
Thus $\tilde{\Psi}$ is locally invertible.
FIXFIXFIXFIXFIXFIXFIXFIXFIXFIXFIXFIXFIXFIXFIXFIXFIXFIX

For $\Ga\subeq  M\cap B(p,r)$, 
by changes of variables, 
\begin{align*}
\la(\Ga(p))&=\int_{\tilde{\Psi}^{-1}(\Ga(\rh))} 1_{\Ga(\rh)}(\tilde{\Psi}(u,\xi))\,J\tilde{\Psi}(u,\xi).\\
&=\int_{\tilde{\Psi}^{-1}(\Ga)\times (-\rh,\rh)} J\tilde{\Psi}(u,\xi)\,d\xi du\\
&=\int_{\Psi^{-1}(\Ga)}
\pa{
J\tilde{\Psi}(u,\xi)\,d\xi
}\,du.
\end{align*}
Divide by $2\rh$ and take the limit as $\rh\to 0$ to get
\[
\int_{\Phi^{-1}(\Ga)} J\tilde{\Psi}(u,0)\,du.
\]
The limit exists and is given by a measure!

In differential geometry we go in the opposite direction: define a hypersurface (or manifold) to be something which has a coordinate chart at each point. Then define a measure on the hypersurface by using the formula above. We don't usually talk about the relationship to the measure in the ambient space because we use intrinsic descriptions. %we don't talk about geometric structures in an ambient space.

%squeezing!
If $\R^N\to \R^N$ is Lipschitz continuous $|V(y)-V(x)|\le L|y-x|$ then we can always solve $\frac{d\Phi}{dt}(t,x)=V(\Phi(t,x))$ and $\Phi(0,x)=x$.

Flow property: $\Phi(s+t,x)=\Phi(t,\Phi(s,x))$. (Use uniqueness of solutions.)

