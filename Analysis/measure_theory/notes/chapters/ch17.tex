\lecture{Fri. 3/11/11}
\subsection{Sunrise Lemma}
\begin{df}
A function $g$ is \textbf{upper semicontinuous} if
\[
g(x)\geq \varlimsup_{y\to\iy} g(y).
\]
\end{df}
%\begin{pr}
This is equivalent to $\{x:g(x)<a\}$ being open for all $a$.
%\end{pr}
%\begin{proof}
%
%\end{proof}
\begin{lem}[Sunrise lemma]\label{sunrise}
Suppose $g:\R\to \R$ is upper semicontinuous and right continuous, and that $\lim_{x\to \pm \iy}g(x)=\mp\iy$. Set 
\[
G=\{x:\exists y>x, g(y)>g(x)\}.
\]
Then $G$ is open, and any nonempty connected open component of $G$ is a bounded interval $(a,b)$ such that $G(b)=G(a)$.
\end{lem}
Think of $g$ as mountainous terrain, going up to infinity at the left and down infinitely at the right. The set $G$ is that portion of the terrain which will be in shadow the instant the sun comes over the horizon infinitely far at the right---those points where there's something to the right blocking the sun.
The Sunrise Lemma says if you look at region in shadow, then is valley surrounded by hills with equal height.
\begin{proof}
Openness follows by upper continuity.

Take $c\in (a,b)$. Let
\[
x=\sup\{x\ge c:g(x)\ge g(c)\},
\]
the point beyond which nothing is higher than $x$. Now $g(x)\ge g(c)$ by upper semicontinuity. (Take $y$'s in the set converging to $x$. Note $x$ is finite since $\lim_{n\to \iy} g(x)=-\iy$.) 
Then $x\nin G$, because any point to the right of $x$ higher than $g(x)$ would also be greater than $g(c)$, contradicting maximality. This means $b\le x$.

Also $g(c)\le g(b)$. (Draw a picture!) %Suppose the contrary, $g(c)>g(b)$. Now wtf?!

Now $g(x)\le g(b)$ for $x\in (a,b)$. In particular $a>-\iy$. By right continuity, $g(a)\le g(b)$. But if $g(a)<g(b)$, then $b>a$ while $g(b)>g(a)$, and actually $a\in G$, contradiction. Hence $g(a)=g(b)$.
%This means $a>-\iy$. Now $g(c)\le g(b)$ gives $a>-\iy$? By right continuity $g(a)=\lim_{c\to a^-} g(c)\le g(b)$. $a$ is not in the shaded region so nothing to the right is at a higher leve, and $g(a)\ge g(b)$.
\end{proof}
\subsection{Lebesgue's differentiation theorem}
\begin{df}
Define the left, right, and two-sided \textbf{Lipschitz constants} of $F$ at $x$ to be
\begin{align*}
\cal L_-F(x)&=\sup_{h<0} \frac{F(x+h)-F(x)}{h}\\
\cal L_+F(x)&=\sup_{h>0} \frac{F(x+h)-F(x)}{h}\\
\cal LF(x)&=\sup_{|h|>0} \frac{F(x+h)-F(x)}{h}=\max(\cal L_+F(x),\cal L_-F(x)).
\end{align*}
\end{df}
\begin{thm}\label{Lebesgue's differentiation theorem}
Let $F:\R\to \R$ be a nondecreasing function. %For $\la_{\R}$-almost ever
For almost all $x\in \R$ (with respect to $\la_{\R}$),
\[F'(x)=\lim_{h\to 0} \frac{F(x+h)-F(x)}{h}\]
exists in $\R$.
\end{thm}
We can replace monotonicity by bounded variation (since a function of bounded variation is the difference of two monotone functions).
%START OF PROOF
%We can just consider $F$ on an interval, so 
It suffices to prove this in an interval. Thus we can assume $F$ is bounded, and by shifting by a constant, $F(-\iy)=\lim_{x\to -\iy} F(x)=0$. 
A monotonous function is discontinuous at a countable number of points (a set of measure 0), so by redefining $F$ at these points we may assume $F$ is right continuous.
The main steps are as follows. We assume $F$ has the properties above.
\begin{enumerate}
\item The set of points at which $F$ has large Lipschitz constant is small (Lemma~\ref{hlineq} and Corollary~\ref{larlf}). This is proved with the Sunrise Lemma~(\ref{sunrise}).
\item If $F$ is a definite integral $\int_{-\iy}^x f(y)\,\la_{\R}(dy)$ then $F'$ exists and equals $f$ almost everywhere (Theorem~\ref{ftoc2}).
\item If $F$ is Lipschitz then $F$ is a definite integral (Lemma~\ref{lipint}).
\item Take the Lipschitz constant to $\iy$ to conclude, if $F$ is absolutely continuous then $F$ is a definite integral (Lemma~\ref{abint}), and hence differentiable a.e. by item 2.
\item If $F$ is singular then $F$ is differentiable a.e., in fact with $F'=0$.
\item Each $F$ can be written as the sum of a absolutely continuous function and a singular function. From items 4 and 5, $F$ is differentiable a.e.
\end{enumerate}

\begin{lem}\label{hlineq}
\[
\la_{\R}(\{x:\cal LF(x)>R\})\le \frac{2F(\iy)}{R}.
\]
%Points at which lip constant large is small.
\end{lem}
\begin{proof}
Consider $g(x)=F(x)-Rx$. It is right continuous and upper semicontinuous ($g$ is increasing so we have upper semicontinuity on the left). Let $G$ be as in the Sunrise Lemma~(\ref{sunrise}). Then
\[
G=\{x:\cal L_+F(x)>R\}.
\]
Write $G$ as union of disjoint components: $G=\bigcup_{n\ge 1}(a_n,b_n)$. By the Sunrise Lemma, $G(a_n)=G(b_n)$, giving $F(b_n)-F(a_n)=R(b_n-a_n)$. Now
\begin{align*}
\la_{\R}(G)&=\sum_{n=1}^{\iy} (b_n-a_n)=\rc R\sum_{n=1}^{\iy} (F(b_n)-F(a_n))\le \frac{F(\iy)}{R}.
\end{align*}
Similarly, $\la_{\R}(\{x:\cal L_+F(x)>R\})\le \frac{F(\iy)}{R}$ and the result follows. 
\end{proof}
\begin{cor}\label{larlf}
%$\la_{\R}(\cal LF>R)\le \frac{F(\iy)}{R}$, so 
$\la_{\R}(\cal L F=\iy)=0$.
\end{cor}
We next show a generalization of the fundamental theorem of calculus.
\begin{thm}\label{ftoc2}
Suppose that $F(x)=\int_{(-\iy,x]} f(y)\,\la_{\R}(dy)$ where $f\ge 0$ and $f\in L^1$. Then $F'(x)$ exists and equals $f(x)$ $\la_{\R}$-almost everywhere. 
\end{thm}
\begin{proof}
Let $\cal G$ %for good
be the set of all $f\in L^1(\la_{\R})$ such that 
\[\lim_{I\searrow \{x\}} \rc{|I|}\int_{I}|f(y)-f(x)|\la_{\R}(dy)=0\text{ almost everywhere.}\]
($I$ represents an interval.)
Note in any case the LHS is at least
\begin{equation}\label{hlineq2}
\lim_{I\searrow\{x\}} \ab{
\rc{|I|}\int_I f-f(x)}.
\end{equation}
%Contain all $L^1$ continous functions.
By Theorem~\ref{mufinitedense} the set of continuous functions with compact support in $L^1$ is dense in $L^1$.
Since continuous functions are in $\cal G$, it suffices to show $\cal G$ is closed under $L^1$-convergence.

Let $f_n\in \cal G$ with and $f_n\to f$ in $L'$. Then
\begin{align*}
\la_{\R}\pa{
\bc{x:\varlimsup_{I\searrow \{x\}} \rc{|I|} \int_I|f-f(x)|\,d\la_{\R} \ge \ep}}
&\le 
\la_{\R}\pa{\bc{
x:\varlimsup_{I\searrow \{x\}} \rc{|I|} \int_{I}|f-f_n|\,d\la_{\R}\ge \ept
}}\\
&\quad +
\la_{\R}\pa{\bc{
x:\varlimsup_{I\searrow \{x\}} \rc{|I|} \int_{I}|f_n-f_n(x)|\,d\la_{\R}\ge \ept
}}\\
&\quad +\la_{\R}(|f_n-f|\ge 1).
%\{:M(f-f_n)\ge \frac{\ept}{3})+\cancelto{0}\mu\{x:\lim_{I\searrow\{x\}}|f(x)-f_n(x)|\ge \ept\}
%+\la(\{x:|f_n(x)-f(x)|\ge \ept\}).
%bdded
\end{align*}
By~(\ref{hlineq}) and~(\ref{hlineq2}), the first and third terms are at most $\frac{2\nl{f-f_n}}{\ep}$ and $\frac{\nl{f-f_n}}{\ep}$ and by assumption the second term is 0, so the RHS is at most $\frac{\nl{f-f_n}}{\ep}$, as needed.

%True for every $n$.
%translation...
%$a/\ep\nl{f-f_m}$
%Thus if $F$ is an indefinite integral then $F'=f$.
%Class of $F$ in... blahs what what?
%absolutely continous Ff
%sum singular, absolutely ocntinous, this part, separate arguemtth wuth the sameparty
%leb gives disapppoints $F'=0$ a.e. ve.c
%collecting credit forcourse
%when they go to heaven they can say well i took 125.

%???????????????//
\end{proof}
For $f\in L^1$, define the \textbf{Hardy-Littlewood maximal function} of $f$ to be
\[
Mf(x)=\sup_{I\ni\{x\}}\rc{|I|} \int_I|f(y)|\la_R(dy)
\]
Now by~(\ref{hlineq}),
\[
\la_{\R} (Mf>R)\le \frac{2\nl{f}}{R}.
\]
(This is the Hardy-Littlewood inequality.)