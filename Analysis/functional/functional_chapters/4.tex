\chapter{Weak Topologies}
\section{Weak Topologies}
\begin{df}\llabel{df:weak-topology}
Let $X$ be a set. Let $\cal F$ be a family of functions, where each $f\in \cal F$ is a function $f:X\to Y_f$ and $Y_f$ is a topological space. The \textbf{weak topology} on $X$ generated by $\cal F$ denoted by $\si(X,\cal F)$, is the smallest topology on $X$ such that every $f\in \cal F$ is continuous. 
\end{df}
%the weak topology is the 
\begin{rem}
\begin{enumerate}
\item
In other words, $\set{f^{-1}(U)}{f\in \cal F, U\text{ open subset of }Y_f}$ is a subbase of $\si(X,\cal F)$. 

More generally, if $S_f$ is a subbase of the topology at $Y_f$, $f\in \cal F$, then 
\[
\set{f^{-1}(U)}{f\in \cal F,U\in S_f}
\]
is also a subbase of $\si(X,\cal F)$.
\item
For $V\subeq X$, we have $V\in \si(X,\cal F)$ iff for all $x\in V$, there exists $n\in \N$ and there exists $f_1,\ldots, f_n\in \cal F$ and open sets $U_i$ in $Y_{f_i}$ such that $x\in \bigcap_{i=1}^n f_i^{-1}(U_i)\subeq V$. 
\item (Universality Property) (This is a simple but very useful fact.) For any topological space $Z$ and any function $g:Z\to X$, $g$ is continuous iff $f\circ g:Z\to Y_f$ is continuous for all $f\in \cal F$. ($\implies$ is clear. For ``$\Leftarrow$," it's enough to check members of the subbase. We have $g^{-1}(f^{-1}(U))=(f\circ g)^{-1} (U)$ is open in $Z$ for all $f\in \cal F$ and all open $U\subeq Y_f$.)

As an exercise, if $\tau$ is a topology on $X$ with this property, then $\tau=\si(X,\cal F)$. 
\item
If $\tau$ is a topology on $X$ with this property, then $\tau=\si(X,\cal F)$. 
\item
If $Y_f$ is Hausdorff for all $f\in \cal F$, and $\cal F$ separates points of $X$, then $\si(X, \cal F)$ is Hausdorff. For all $x\ne y$ in $X$ there exists $f\in \cal F$, $f(x)\ne f(y)$. 
\end{enumerate}
\end{rem}
\begin{ex}
\begin{enumerate}
\item (Subspace topology)
Let $X$ be a topological space, $Y\subeq X$, and $i:Y\to X$ be the inclusion map. Then $\si(Y,\{i\})$ is the subspace topology.

Given a set $X$, $\cal F$ as above, given $Y\subeq X$, then $\si(X,\cal F)|_Y$ (the subspace topology on $Y$ induced by $\si(X,\cal F)$) is the same as $\si(Y, \set{f|_Y}{f\in \cal F})$. This is routine to verify and useful.
\item
(Product topology) Given topological spaces $X_{\ga},\ga\in \Ga$, let $X=\prod_{\ga\in \Ga}X_{\ga}$, i.e., $X=\set{x}{x\text{ is a function on }\Ga}$ with $x(\ga)\in X_{\ga}$ for all $\ga\in \Ga$.

Have $\pi_{\ga}:X\to X_{\ga}$, $x\mapsto x(\ga)$, $\ga\in \Ga$. The product topology on $X$ is $\si(X,\set{\pi_{\ga}}{\ga\in \Ga})$. 
\end{enumerate}
\end{ex}
%2 results about weak top
%
%Tychonoff prod cpt is cpt

\begin{pr}\llabel{pr:f4-1}
Let $X$ be a set. For each $n\in \N$, suppose we are given a metric space $(Y_n,d_n)$ and a function $f_n:X\to Y_n$. Assume the $f_n$ separate points of $X$, i.e., for all $x\ne y$ in $X$, there exists $n\in \N$, $f_n(x)\ne f_n(y)$. Then $\si(X,\set{f_n}{n\in \N})$ is metrizable. 
\end{pr}
\begin{proof}
WLOG $d_n\le 1$ (replace $d_n$ with the equivalent metric $\min(1,d_n)$). We define
\[
d(x,y)=\suo 2^{-n} d_n(f_n(x),f_n(y)),\qquad x,y\in X.
\]
It is easy to verify that $d$ is a metric on $X$. Then $\id:(X,d)\to (X,\si)$ is continuous by the universality property (here $\si=\si(X,\set{f_n}{n\in \N})$). (A map into a weak topology is continuous if, when composing with any of the defining maps it is continuous.)

We have $\id:(X,\si)\to (X,d)$ is continuous: check for all $x\in X$, $r>0$, $B(x,r)$ is a $\si$-neighborhood of $x$. This is an easy exercise.
\end{proof}
The separating condition is needed to get positivity in the metric. Without the separation property $d$ is a pseudo-metric.

\begin{thm}[Tychonov's Theorem]\llabel{thm:f4-2}
The product of compact topological spaces is compact.
\end{thm}
\begin{proof}
Let $X_{\ga},\ga\in \Ga$ be compact spaces, and $X=\prod_{\ga\in \Ga} X_{\ga}$ with the product topology. Let $\cal F$ be a nonempty family of closed subsets of $X$ with the FIP (finite intersection property): for all $n\in \N$ and all $F_1,\ldots, F_n\in \cal F$, $\bigcap_{i=1}^n F_i\ne \phi$. We need $\bigcap \cal F\ne \phi$.

Using Zorn, there exists a maximal (with respect to inclusion) family of (not necessarily closed) subsets of $X$, $\cal A$, such that $\cal A\supeq  F$ and $\cal A$ has FIP. It's enough to show that $\bigcap_{A\in \cal A}\ol{A}\ne \phi$. 
\begin{enumerate}
\item
For every $A,B\in \cal A$, $A\cap B\in \cal A$: If $B_1,\ldots, B_n\in \cal A$ then $A\cap B\cap B_1\cap \cdots \cap B_n\ne \phi$, so $\cal A\bigcup \{A\cap B\}$ has FIP, so it equals $\cal A$.
\item
For all $A\subeq X$, if $A\cap B\ne \phi$ for all $B\in \cal A$< then $A\in \cal A$: given $B_1,\ldots, B_n\in \cal A$, by (1), $\bigcap_{i=1}^n B_i\in \cal A$, so $A\cap \bigcap_{i=1}^n B_i\ne \phi$. So $\cal A\cup \{A\}$ has FIP, and $A\in \cal A$.
\item
Given $A\in \cal A,B\subeq X$, if $A\subeq B$ then $B\in \cal A$: for any $X\in \cal A$, $A\cap C\ne \phi$, so $B\cap C\ne \phi$. By (2), $B\in \cal A$.
\end{enumerate}
Let $\pi_y:X\to X_{\ga}$ be the projection $\pi_{\ga}(x)=x(\ga)$. For $n\in \N$, $A_1,\ldots, A_n\in \cal A$, $\bigcap_{i=1}^n \pi_{\ga}(A_i)\supeq \pi_{\ga}\pa{\bigcap_{i=1}^n A_i}\ne \phi$. So $\set{\pi_{\ga}(A)}{A\in \cal A}$ has FIP. $X_{\ga}$ is compact so $\bigcap_{A\in \cal A}\ol{\pi_{\ga}(A)}\ne \phi$. Pick $x_{\ga}\in \bigcap_{A\in \cal A}\ol{\pi_{\ga}(A)}$. Set $x=(x_{\ga})_{\ga\in \Ga}$. We claim $x\in \bigcap_{A\in \cal A}\ol A$. Let $V$ be a neighborhood of $x$. Then $\exists n\in \N$, $\ga_1,\ldots, \ga_n\in \Ga$, open sets $U_i$ on $X_{\ga_i}$, $1\le i\le n$, such that $x\in \bigcap_{i=1}^n \pi_{\ga_i}^{-1}(U_i)\subeq V$. For each $i$, $x_{\ga_i}\in U_i$, so $U_i\cap \pi_{\ga_i}(A)\ne \phi$ for all $A\in \cal A$.
%inverse of all open sets, and they generate
%use fact $x_{\ga}$ is in these closures.

So $\pi_{\ga_i}^{-1}(U_i)\cap A\ne \phi$ for all $A\in \cal A$. By (2) above $\pi_{\ga_i}^{-1}(U_i)\in \cal A$. Then by (1) and (3), $V\in \cal A$. So $V\cap A\ne \phi$ for all $A\in \cal A$. Since this holds for every neighborhood $V$ of $x$ we get $x\in \ol A$ for all $A\in \cal A$. 
\end{proof}

Let $E$ be a (real or complex) vector space, let $F$ be a subspace of the space of all linear functionals on $E$. Assume $F$ separates the points of $E$, i.e., for all $x\ne 0$ in $E$ there exists $f\in F$ such that $f(x)\ne 0$. We'll consider the weak topology $\si(E,F)$ on $E$. A subset $V$ of $E$ is open iff for all $x\in V$ there exists $n\in \N$, $f_1,\ldots, f_n\in F$, $\ep>0$ such that $\set{y\in E}{|f_i(y)-f_i(x)|<\ep\forall 1\le i\le n}\subeq V$.
%inverse images of open sets under functions of F, fin int arb union. Enough to take inverse image of subbase (open balls)

This is precisely a LCS given by the seminorms $x\mapsto |f(x)|:E\to \R$, $f\in F$.
%HB theorem valid?

In a moment we'll look at 2 particular examples, but first we'll give some basic results.

$\si(E,F)$ is Hausdorff, and addition and scalar multiplication are continuous.
\begin{lem}\llabel{lem:f4-3}
Let $E$ be a vector space, and let $f,g_1,\ldots, g_n$ be linear functionals on $E$ such that $\bigcap_{i=1}^n \ker g_i\subeq \ker f$. Then $f\in \spn\{g_1,\ldots, g_n\}$. 
\end{lem}
(There is a nice quantitative version of this; see the second example sheet.)
\begin{proof}
Let $K$ be the scalar field. We define $q:E\to K^n$ by taking $q(x)=(g_1(x),\ldots, g_n(x))$. Then $\ker q=\bigcap_{i=1}^n \ker g_i\subeq \ker f$. Hence $f$ factors through $q$: there exists a linear map $\wt f:K^n\to K$ such that
\[
\xymatrix{
E\ar[rr]^f \ar[rd]_q&& K\\
&K^n\ar[ru]_{\wt f}&
}
\]
commutes: $f=\wt f\circ q$. (Define $\wt f(q(x))=f(x)$, this is well-defined and linear on $\im q$, now just extend to the whole of $K^n$. There exist $a_1,\ldots, a_n\in K$ such tht $\wt f(y_1,\ldots, y_n)=\sui a_iy_i$, so $f(x)=\wt f(q(x))=\sui a_ig_i(x)$, $x\in E$. Thus $f=\sui a_ig_i$. 
\end{proof}
\begin{pr}\llabel{pr:f4-4}
Let $E$ be a (real or complex) vector space, $F$ be a subspace of all linear functionals on $E$ separating the points of $E$. let $f$ be a linear functional on $E$. Then $f$ is $\si(E,F)$-continuous iff $f\in F$.  So $(E,\si(E,F))^*=F$.
\end{pr}
The weak continuous was the smallest topology that makes all members of $F$ continuous, but the point was that there was nothing else.
\begin{proof}
``$\Leftarrow$" is clear by definition.

``$\Rightarrow$": Note $\set{x\in E}{|f(x)|<1}$ is a neighborhood of 0, so it contains a set of the form $\set{x\in E}{|g_i(x)|<\ep\text{ for all } 1\le i\le n}$ where $n\in \N$, $g_1,\ldots, g_n\in \cal F$, $\ep>0$. If $x\in \bigcap_{i=1}^{\iy} \ker g_i$ then so is $\la x$ for all scalar $\la$, so $|f(\la x)|<1$ for all $\la$. So $f(x)=0$. 

By Lemma~\ref{lem:f4-3}, $f\in \spn\{g_1,\ldots, g_n\}\subeq F$.
\end{proof}
We have now made all the preparations to define the weak and weak star topologies.

\section{$w$ and $w^*$ topologies}

Recall from Chapter 2 the canonical map $X\hra X^{**}$, $x\mapsto \hat x$ where $\an{f,\hat x}=\an{x,f}$ for all $f\in X^*$ is an isometric isomorphism of $X$ into $X^{**}$. The image $\hat{X}=\set{x}{x\in X}$ is closed in $X^{**}$ iff $X$ is complete. $X$ is reflexive if $\hat X=X^{**}$. 

\begin{df}
Let $X$ be a normed space. The \textbf{weak topology ($w$-topology)} on $X$ is $\si(X,X^*)$ (i.e., $E=X$, $F=X^*$, $F$ separates the points of $E$ by Hahn-Banach).
\end{df}

Note that $V\subeq X$ is in $\si(X,X^*)$ (say $V$ is \textbf{weak open ($w$-open)} iff for all $x\in V$, there exists $n\in \N$, $x_1^*,\ldots, x_n^*\in X^*,\ep>0$ such that $\set{y\in X}{|x_i^*(x-y)|<\ep,1\le i\le n}\subeq V$. %notation, x^*\in X^*.

\begin{df}
The \textbf{weak-star topology ($w^*$-topology)} on $X^*$ is the weak topology $\si(X^*,X)$, i.e., $E=X^*$, $F=X=\hat X\subeq X^{**}$. So $V\subeq X^*$ is in $\si(X^*,X)$ (say $V$ is \textbf{$w^*$-open}) iff for all $x^*\in V$, there exists $n\in \N$, $x_1,\ldots, x_n\in X$, $\ep>0$ such that 
\[
x^*\in\set{y^*}{|(x^*-y^*)(x_i)|<\ep \text{ for }1\le i\le n}\subeq V.
\]
\end{df}
These topologies look similar but we will see they are very different. Proposition~\ref{pr:f4-4} immediately gives the following---no more functionals are continuous than the ones we force to be.
\begin{pr}\llabel{pr:f4-5}
Let $X$ be a normed space, let $f$ be a linear functional on $X$, and $\ph$ a linear functional on $X^*$. Then we say
\begin{enumerate}
\item
$f$ is continuous in $\si(X,X^*)$ (say $f$ is \textbf{weakly continuous ($w$-continuous}) iff $f\in X^*$.
\item
$\ph$ is continuous in $\si(X^*,X)$ (say $\phi$ is \textbf{$w^*$-continuous}) iff $\ph\in X$, i.e., 
iff $\phi$ is an evalution: there exists $x\in X, \ph=\hat x$.
\end{enumerate}
It follows that $\si(X^*,X^{**})=\si(X^*,X)$ iff $X$ is reflexive.
\end{pr}
We have the following properties.
\begin{pr}
\begin{enumerate}
\item
$X$ with the $w$-topology and $X^*$ with the $w^*$-topology are LCSs. So they are Hausdorff, and addition and scalar multiplication are continuous.
\item
$\si(X,X^*)\subeq \ved$-topology. This is because in the $\ved$-topology, all the linear functionals are continuous, and the weak topology is the weakest such topology.
$\si(X^*,X)\subeq \si(X^*,X^{**})$ because in the LHS, we are making less functionals continuous (equality iff $X$ is reflexive).
\item
If $\dim X<\iy$, then these topologies coincide. (Exercise.
Show that anything open in norm topology will be open in the weak topology.)
\item
If $\dim X=\iy$, $U\ne \phi$ is an open set in $\si(X,X^*)$, then $U$ is unbounded. (Given a finite number of functionals, the intersection of their kernels is a nontrivial subspace.) 
So it follows that $\si(X,X^*)\subsetneq\ved$-topology. So $\si(X^*,X)\sub \si(X^*,X^{**})\subsetneq \ved$-topology. 
%Look at the functionals, taking the intersection of the kernels, %some translate contained in open set, so 
%open unit ball is not open.
\item
If $\dim X=\iy$, then $\si(X,X^*)$ is not metrizable (and not even first countable, i.e., does not have countable neigborhood base). 
\item In particular, if $\dim X$ is uncountable (e.g. $\dim X=\iy$, $X$ complete, since the
Baire category theorem gives that any complete normed space cannot have a countable Hamel basis), then $\si(X^*,X)$ is not metrizable, and not even first countable.
\item $\si(X^{**}, X^*)|_X=\si(X,X^*)$. (We use the fact that given a set $X$ and a family $\cal F$, if we restrict to subspace $Y$, we get exactly the weak topology induced by the restrictions of $\cal F$.)
\end{enumerate}
\end{pr}
We introduce some notation. Let $X$ be a normed space. Let $x_n\in X$, $n\in N$, $x\in X$. Say $x_n$ \textbf{converges weakly} to $x$, and write $x_n\xra{w}x$ if $x_n$ converges to $x$ in the $w$-topology.  This happens iff it converges pointwise at any functional.
\[
\an{x_n,x^*}\to \an{x,x^*}\qquad \forall x^*\in X^*.
\]
%Thinking of the $x_n$ as living in the second dual, this is saying $x_n\to x$ on the dual space.
Given $x_n^*$, $n\in \N$, $x^*\in X^*$, we say \textbf{$x_n^*$ converges $w^*$ to $x^*$} if $x_n^*$ converges to $x^*$ in the $w^*$-topology, and write $x_n^*\xra{w^*} x^*$. This happens iff 
\[
\an{x,x_n^*}\to \an{x,x^*}\qquad \forall x\in X.
\]
Note that the same applies to nets.

Recall the following.
\begin{thm}[Principle of Uniform Boundedness]\llabel{thm:pub}
Let $X$ be Banach and $Y$ be a normed space, and $ \cal T\subeq \cal B(X,Y)$. If $\cal T$ is pointwise bounded, i.e. $\sup_{T\in \cal T} \ve{Tx}<\iy$ for all $x\in X$, then $\sup_{T\in \cal T}\ve{T}<\iy$.
\end{thm}
A subset $A$ of a normed space $X$ is \textbf{weakly bounded ($w$-bounded)} if $\set{\an{x,x^*}}{x\in A}$ is bounded for all $x^*\in X^*$, i.e., $\set{\hat x}{x\in A}\subeq X^{**}=\cal B(X^*, \R\text{ or }\C)$ is pointwise bounded. A subset $B\subeq X^*$ is \textbf{$w^*$-bounded} if $\set{\an{x,x^*}}{x^*\in B}$ is bounded for all $x\in X$. We immediately get the following.
\begin{pr}\llabel{pr:f4-6}
Let $X$ be a normed space, $A\subeq X$, $B\subeq X^*$. Then
\begin{enumerate}
\item
$A$ is $w$-bounded iff $A$ is $\ved$-bounded.
%going from is complete, but the norm space is always complete?
\item
If $X$ is a Banach space and $B$ is $w^*$-bounded, then $B$ is $\ved$-bounded.
\end{enumerate}
\end{pr}
We use some facts of sequences. Since weak convergence of a sequence is defined as pointwise convergence, a weakly convergent sequence is pointwise bounded. 
Recall the following.
\begin{thm}[Banach-Steinhaus]\llabel{thm:bs}
Let $X$ be Banach, $Y$ be a normed space, and $T_n\in \cal B(X,Y)$, $n\in \N$. Assume $T_n$ converges pointwise to some map $T:X\to Y$. Then $T\in \cal B(X,Y)$ and $\ve{T}\le \liminf \ve{T_n}$.
%if don't know unif bounded, lim could be \iy
\end{thm}
\begin{pr}\llabel{pr:norm-bounded}
Let $X$ be a normed space, $x_n\xra{w} x$ in $X$, and $x_n^*\xra{w^*} x^*$ in $X^*$. Then 
\begin{enumerate}
\item
$\sup\ve{x_n}<\iy$, $\ve{x}\le \liminf\ve{x_n}$, and 
\item
if $X$ is complete, then $\sup\ve{x_n^*}<\iy$, and $\ve{x^*}\le \liminf \ve{x_n^*}$. 
\end{enumerate}
\end{pr}
%normed then LCS
\section{Hahn-Banach separation theorems}
\begin{df}
Let $(X, \cal P)$ be a LCS. Let $C$ be a convex subset of $X$ such that $0\in \text{int}(C)$. 

We define the \textbf{Minkowski functional} associated to $C$ by 
\[
\mu_C(x)=\inf\set{t>0}{x\in tC}.
\]
%We'll use the Minkowski functional associated with $C$.
\end{df}
For any $x\in X$, $0x\in C$, so since scalar multiplication is continuous, $\exists \de>0$ such that $\de x\in C$, so $x\in \rc{\de}C$ (so $\mu_C$ is well-defined). 

For example, if $X$ is a normed space, $C=B_X$, then $\mu_C=\ved$. 
\begin{lem}\llabel{lem:f4-8}
Let $(X,\cal P)$ be a LCS, $C$ a convex subset of $X$ with $0\in \text{int}(C)$. Then $\mu_C$ is a positive homogeneous subadditive function. Moreover, $\set{x\in X}{\mu_C(x)<1}\subeq C \subeq \set{x\in X}{\mu_C(x)\le 1}$.
Moreover, if $C$ is open, we have equality in the first case: $C=\set{x\in X}{\mu_C(x)<1}$.
\end{lem}
\begin{proof}
Positive homogeneity: Clearly for $x\in X$, $\al>0$, we have $x\in tC\iff\al x\in \al tC$, so $\mu_C(\al x)=\al \mu_C(x)$ and $\mu_C(0)=0$.

First observe that if $\mu_C(x)<t$, then $\exists s<t$ such that $x\in sC$. So $\fc{x}{s}\in C$ and hence $\fc{x}{t}=\fc st+\fc xs +\pa{1-\fc st}0\in C$ by convexity. So $x\in tC$. Now assume $x,y\in X$, pick $s>\mu_C(x)$, $t>\mu_C(y)$. Then $x\in sC$, $y\in tC$. So by convexity, $\fc{x+y}{s+t}=\fc{s}{s+t}\fc{x}{s}+\fc{t}{s+t}\fc{y}{t}\in C$. Thus $\mu_C(x+y)\le s+t$. Taking $\inf$ over $s,t$ we get $\mu_C(x+y)\le \mu_C(x)+\mu_C(y)$.

We observed that $\mu_C(x)<1$ then $x\in C$. Also if $x\in C$, then $\mu_C(x)\le 1$. If $C$ is open, then for $x\in C$, $1x=x\in C$ implies by continuity of scalar multiplication that there exists $\de>0$ such that $(1+\de)x\in C$, so $\mu_C(x)<1$.
\end{proof}

\fixme{6th Nov}

\begin{rem}
\begin{enumerate}
\item
The only place where we used the fact that $X$ is a LCS is to show $\mu_C$ is well-defined; we used continuity of scalar multiplication. All this makes sense in a vector space except that $\mu_C$ may take the value $\iy$.
\item
%base consisting of convex sets. 
%minkowski fun give seminorms
%there is a local base of convex neighborhoods 
If in the real case $C$ is symmetric ($x\in C\implies -x\in C$) or in the complex case, $C$ is balanced ($x\in C,|\la|=1\implies \la x\in C$), then $\mu_C$ is a seminorm.
\end{enumerate}
\end{rem}

The following is the most basic version of the separation theorem.
\begin{thm}[Hahn-Banach Separation Theorem]\llabel{thm:f4-9}
Let $(X,\cal P)$ be a LCS, $C$ an open convex subset of $X$, $0\in C$, and $x_0\nin C$. Then for all $x\in C$, there exists $f\in X^*$ such that $f(x)<f(x_0)$ for all $x\in C$. (In the complex case, $\Re f(x)<\Re f(x_0)$ for all $x\in C$.)
\end{thm}

%\begin{asy}
%line going through $x_0$, $\{f=f(x_0)\}$, $C$
%\end{asy}
\begin{proof}
Let $\mu_C$ be the Minkowski functional of $C$. $C$ is open, so $\set{x\in X}{\mu_C(x)<1}=C$. Let $Y=\spn\{x_0\}$. Define $g:Y\to \R$ by $g(\la x_0)=\la$ for all $\la\in \R$. So
\begin{itemize}
\item
if $\la\ge 0$ then since $x_0\nin C$, $g(\la x_0)=\la\le \la\mu_C(x_0)=\mu_C(\la x_0)$, and
\item
if $\la<0$ then $g(\la x_0)=\la\le 0\le \mu_C(\la x_0)$.
\end{itemize}
So $g$ is dominated by $\mu_C$ on $Y$. By the Hahn Banach Theorem~\ref{thm:hb1} $g$ extends to a linear map $f:X\to \R$ such that $f(x)\le \mu_C(x)$ for all $x\in X$. For $x\in C$, $f(x)\le \mu_C(x)<1=f(x_0)$. $C$ is open and $0\in C$, so there exists $n\in \N$, $p_1,\ldots, p_n\in \cal P$, $\ep>0$ such that $\set{x\in X}{p_i(x)\le \ep\forall 1\le i\le n}\subeq C$. 
So if $\max_{1\le i\le n} p_i(x)\le \ep$ then $f(x)<1$. It follows that $|f(x)|\le \rc{\ep}\max_{1\le i\le n}p_i(x)$ for all $x\in X$. By Lemma~\ref{lem:f2-9}, $f$ is continuous. %char of maps between LCS
%(Boundedness of norm of linear operator

In the complex case, we take the real part: View $X$ as a real space; we get a real-linear $f_1:X\to \R$ such that $f_1$ is continuous, $f_1(x)<f_1(x_0)$ for all $x\in C$. Define $f(x)=f_1(x)-if_1(ix)$ for $x\in X$. Then $f$ is complex, linear, continuous functional with $\Re f=f_1$.
\end{proof}
In the next result the statement is for the real case only. The complex case follows as above. 
\begin{thm}[Hahn-Banach separation theorem for convex sets]\llabel{thm:f4-10}
Let $(X,\cal P)$ be a LCS. Let $A,B$ be non-empty, disjoint convex sets.
\begin{enumerate}
\item
If $A$ is open, then there exists $f\in X^*$, $\al\in \R$ such that $f(a)<a\le f(b)$ for all $a\in A,b\in B$. 
\item
If $A$ is compact, $B$ is closed, then there exists $f\in X^*$ such that $\sup_{a\in A} f(a)<\inf_{b\in B} f(b)$.
\end{enumerate}
\end{thm}
In the first case, the hyperplane where $f=\al$ is allowed to touch $B$, and in the second, there is a hyperplane strictly separating the sets.
\begin{proof}
\begin{enumerate}
\item
Pick $a_0\in A,b_0\in B$. Set $C=A-B+b_0-a_0$, $x_0=b_0-a_0$. Now $C=\bigcup_{b\in B} (A-b+b_0-a_0)$, so $C$ is an open convex set with $0\in C$. $x_0\nin C$ since $A\cap B=\phi$. By Theorem~\ref{thm:f4-9}, there exists $f\in X^*$ such that $f(x)<f(x_0)$ for all $x\in C$. So $f(a)<f(b)$ for all $a\in A,b\in B$. Set $\al=\inf_{b\in B} f(b)$. Clearly, $f(b)\ge \al$ for all $b\in B$. Note that $f(x_0)>f(0)=0$; given $a\in A$ there exists $n\in \N$ such that $a+\rc nx_0\in A$ ($A$ is open). Then $f(a)<f(a)+\rc n f(x_0) = f\pa{a+\rc nx_0}\le \al$.
\item
Let $U$ be a neighborhood of 0: there exists $n\in \N$, $p_1,\ldots, p_n\in \cal P, \ep>0$ such that $U\supeq \set{x\in X}{p_i(x)<\ep \forall 1\le i\le n}$. Set $V=\set{x\in X}{p_i(x)<\fc{\ep}{2} \forall 1\le i\le n}$. Then $V$ is open, convex, and $V+V\subeq U$. 
%Since vector addition is continuous and $0\in U$, we get
For all $a\in A$ there exists an open neighborhood $U_a$ of 0 such that $a+U_a\cap B=\phi$. ($B$ is closed and $a\nin B$.) There exists an open convex neighborhood $V_a$ of 0 such that $V_a+V_a\subeq U_a$. (This is a standard trick.) Now $\set{a+V_a}{a\in A}$ is an open cover for $A$, so there exist $a_1,\ldots, a_n\in A$ such that $A\subeq\bigcup_{i=1}^n a_i+V_{a_i}$. Let $V=\bigcap_{i=1}^n V_{a_i}$. Then $V$ is an open convex neighborhood of 0, and $(A+V)\cap B=\phi$. Given $a\in A$, there exists $i$ such that $a\in \al+V_{a_i}$ so $a+V\subeq a+V_{a_i}+V_{a_i} \subeq a+Ua_i$ which is disjoint from $B$. Apply (1) to $A+V$, $B$ to get $f\in X^*$, $\al\in \R$ such that $f(a+v)<\al\le f(b)$ for all $a\in A,v\in V,b\in B$. Hence 
\[
\sup_A f<\inf_B f.
\]
Note the sup is attained as $A$ is compact.
\end{enumerate}
\end{proof}
These are all the separation theorems we will need. Now we return to weak topologies.
\section{Results on weak topologies}
\subsection{Closure and compactness, Banach-Alaoglu}
\begin{thm}[Mazur]
\llabel{thm:f4-11}
Let $X$ be a normed space and $C$ a convex subset of $X$. Then $C$ is $w$-closed iff $C$ is $\ved$-closed. (In general, $\ved$-closed may not imply $w$-closed.) In particular, the closure in the $w$-topology and $\ved$-topology are the same, $\ol{C}^{w}=\ol{C}^{\ved}$.
\end{thm}
\begin{proof}
``$\Leftarrow$": Assume $C$ is $\ved$-closed. Let $x_0\nin C$. By Hahn-Banach Separation Theorem~\ref{thm:f4-10} there exists $f\in X^*$ 
%same as cont func wrt weak topology, topologies same
such that $f(x_0)<\inf_{x\in C} f(x)$ (take $A=\{x_0\},B=C$ in Theorem~\ref{thm:f4-10}(2)). Then $\set{x\in X}{f(x)<\al}$ is a $w$-neighborhood of $x_0$ disjoint from $C$. 
%$s_0$ is singleton
So $X\bs C$ is $w$-open.
\end{proof}
\begin{cor}\llabel{cor:f4-12}
Suppose $x_n\xra{w} 0$ in a normed space $X$. Then for all $\ep>0$, there exists $n\in \N$, $t_1,\ldots, t_n\in [0,\iy)$ such that $\sui t_i=1$ and $\ve{\sui t_ix_i}<\ep$. 
\end{cor}
\begin{proof}
Let $C=\text{conv}\set{x_n}{n\in \N}=\set{\sui t_ix_i}{n\in \N,t_i\ge 0\forall i,\sui t_i=1}$. $C$ is convex. By assumption $0\in \ol{C}^w=\ol{C}^{\ved}$ by Mazur.
\end{proof}
\begin{rem}
%go far out in sequence
%tail seq tend 0
In particular we can find $p_1<q_1<p_2<q_2<\cdots$ and convex combinations $\sum_{i=p_n}^{q_n} t_ix_i\to 0$ as $n\to \iy$.
\end{rem}
The following is the most important theorem in this section.
\begin{thm}[Banach-Alaoglu Theorem]\llabel{thm:ba}
Let $X$ be a normed space. Then $(B_{X^*},w^*)$ is compact.
\end{thm}
Problem with $X^*$ with usual topology: it's an infinite dimensional space that is not compact. So this result is very good.
\begin{proof}
For $x\in X$, let $F_x=\set{\la\text{ scalar}}{|\la|\le \ve{x}}$. This is a bounded closed subset of scalars so by Theorem~\ref{thm:f4-2} (Tychonov), $F=\prod_{x\in X} F_x$ is compact in the product topology. 
We will realize $B_{X^*}$ as a closed subset of $F$.

Let $\pi_x:F\to F_x$ be the coordinate projection $\pi_x(f)=f(x)$. Define $\te:(B_{X^*}, w^*)\to F$ by $\te(x^*)=(x^*(x))_{x\in X}$.

It is clear $\te$ is injective; we show $\te$ is continuous. This comes from the universality property of the property: $\pi_x\circ \te(x^*)=x^*(x)$ is continuous in $x^*$, so $\te$ is continuous.%, i.e., $\pi_x\circ \te|_{B_{X^*}}$ is $w^*$-continuous. (We are implicitly using the fact that the weak topology restricted to a subset is the weak topology with the restricted functions.)

We have $\te^{-1}:\im(\te)\to (B_{X^*}, w^*)$ is continuous by the university property in the remark after Definition~\ref{df:weak-topology}: $\hat x\circ \te^{-1}(f)=f(x)=\pi_*(f)$. So $\hat x\circ \te^{-1} =\pi_x|_{\im(\te)}$ is continuous on $\im(\te)$. So $(B_{X^*}, w^*)$ is homeomorphic to $\im(\te)$.

It suffices to show $\im(\te)$ is closed, because a closed subset of a compact space is compact. We have 
\[\im(\te)=\set{f\in F}{f(\la x+\mu y)=\la f(x)+\mu f(y) \forall x,y\in X,\la,\mu\text{ scalars}}.\]
This is the inverse image of a closed set:
\[
\bigcap_{x,y\in X, \la,\mu\text{ scalars}} \set{f\in F}{(\pi_{\la x+\mu y}-\la \pi_x-\mu \pi_y)(f)=0}.
\]
This is closed in $F$ and hence compact.
\end{proof}

\fixme{Lecture 8-11}
\subsection{Separability and metrizability}
\begin{pr}\llabel{pr:f4-14}
Let $X$ be a normed space, and $K$ a compact Hausdorff space. Then 
\begin{enumerate}
\item
$X$ is separable iff $(B_{X^*},w^*)$ is metrizable.
\item
$C(K)$ is separable iff $K$ is metrizable.
\end{enumerate}
\end{pr}
\begin{proof}
\begin{enumerate}
\item
$\implies$: Let $\set{x_n}{x\in \N}$ be dense in $X$. Consider $\cal F=\set{\wh x_n|_{B_{X^*}}}{n\in \N}$. ($\hat x(x^*)=x^*(x), x\in X, x^*\in X^*$) Let $\si$ be the weak topology $\si(B_{X^*},\cal F)$ on $B_{X^*}$. Clearly $\si$ is in the $w^*$-topology. Also, since $\set{x_n}{n\in \N}$ is dense in $X$, $\cal F$ separates points of $B_{X^*}$ (if $x^*(x_n)$ for all $n$, then $x^*=0$), so by Proposition~\ref{pr:f4-1}, $\si$ is metrizable. The formal $\id:(B_{X^*},w^*)\to (B_{X^*},\si)$ is a continuous bijection from a compact space (Theorem~\ref{thm:ba}) to a Hausdorff space, so it is a homeomorphism.
\item
$\Leftarrow$: $K$ is a compact metric space, so it's separable; let $\set{t_n}{n\in \N}$ be dense in $K$. Let $f_n(t)=d(t,t_n)$, $t\in K$, $n\in \N$, where $d$ is the distance on $K$. Since $\set{t_n}{n\in \N}$ is dense in $K$, the $f_n$'s separate the points of $K$. Let $A$ be the algebra generated by $\set{f_n}{n\in \N}\cup \{1\}$. Then $A$ is a unital subalgebra of $C(K)$, separates the points of $K$ and, in the complex case, is closed under complex conjugation. So by Stone-Weierstrass, $\ol A=C(K)$. Since $A$ is countably generated, $A$ and hence $C(K)$ is separable. 
\item[1.] $\Leftarrow$: Consider $X\to C(K)$ where $K=(B_{X^*}, w^*)$ is a compact metric space given by $x\to \wh x|_K$. %what is the sup norm of $\wh x$?
By definition, 
\[
\ve{\hat x|_K}_{\iy}=\sup_{x^*\in B_{X^*}}\ve{x^*(x)}=\ve{x}
\]
so the embedding is isometric. By (ii)$\Leftarrow$, $C(K)$ is separable so $X$ is separable.
%countable collection which is dense. weaker than ... topology, by inverse func theorem, 
%given by countable many functions, so metrizable.
%same proof, or use part 1.
\item[2.] $\Rightarrow$: Note $\ved_{\iy}$ is a norm on $C(K)$ since $X$ is compact. $X=(C(K), \ved_{\iy})$ is separable, so $(B_{X^*},w^*)$ is metrizable by (i)$\implies$. Consider $\ph:K\to (B_{X^*},w^*)$, $\ph(k)=\de_k$, where $\de_k(f)=f(k)$. If $k\ne k'$ in $K$, then by Urysohn there exists $f\in C(K)$, $f(k)\ne f(k')$. So $\de_k(f)\ne \de_{k'}(f)$. So $\ph$ is injective.
%subspace of dual ball which is metrizable.

$\ph$ is continuous: $(\hat x\circ \ph)(k)=\hat x(\de_k)=\de_k(x)=x(k)$, $x\in C(K)$. So $\hat x\circ \ph\in C(K)$ for all $x\in C(K)$, so $\ph$ is continuous. So $\ph$ is a continuous bijection from compact $K$ to Hausdorff $\ph(K)$, so $K$ is homeomorphic to $\ph(K)\subeq (B_{X^*},w^*)$.
%BA by compact
%if know metrizable, sequentially compact
\end{enumerate}
\end{proof}
\begin{rem}
\begin{enumerate}
\item A compact metrizable set is sequentially compact.
\item $X$ is separable implies $X^*$ is $w^*$-separable, as $X^*=\bigcup_{n=1}^{\iy} nB_{X^*}$. For all $n$, $(nB_{X^*}, w^*)$ is a compact metric space, so separable. The converse is false in general, for example, for example, $X=\ell_{\iy}$. 
\item
$X$ is separable iff $X$ is $w$-separable. For $\implies$, note the norm topology is stronger, so it is harder to be separable in the norm topology. For $\Leftarrow$, if $X=\ol A^w$, and $A$ is countable, then $\ol{\spn}^{\ved}A=\ol{\spn}^w(A)\supeq \ol A^w=X$.
\end{enumerate}
\end{rem}
\begin{pr}\llabel{pr:f4-15}
$X^*$ is separable iff $(B_X,w)$ is metrizable.
\end{pr}
\begin{proof}
$\implies$: By Proposition~\ref{pr:f4-14}, $(B_{X^{**}}, w^*)$ is metrizable. Since the $w$-topology on $B_X$ is the restriction to $B_X$ of the $w^*$ topology on $B_{X^{**}}$, $(B_X,w)$ is metrizable.


It's not clear why metrizability goes the other way; we have to work a bit.

$\Leftarrow$: Let $d$ be a metric on $B_X$ inducing the $w$-topology. Then every open ball with respect to $d$ must contain a basis element of the $w$-topology: 
For all $n\in \N$, there exist $x_{n1}^*,\ldots, x_{n{k_n}}^*$ and $\ep_n>0$ such that 
\[
U_n=\set{y\in B_X}{|x_{ni}^*(y)|<\ep_n\forall 1\le i\le k_n}\subeq B\pa{0,\rc n} =\set{y\in B_X}{d(y,0)<\rc n}.
\]
%by Riesz lemma, get everything
Let $Y=\ol{\spn}\set{x_{ni}^*}{n\in \N,1\le i\le k_n}$. Let $x^*\in B_{X^*}$, there exists $n\in \N$ with $U_n\subeq \set{y\in B_X}{|x^*(y)|<\rc 2}$. If $y\in \bigcap_{i=1}^{k_n} \ker x_{ni}^*\cap B_X$ then $y\in U_n$, and hence $|x^*(y)|<\rc 2$. So \[\ve{x^*|_{\bigcap_{i=1}^{k_n}\ker x_{ni}^*}}\le \rc 2.\] Let $z^*$ be an extension of $x^*|\bigcap_{i=1}^{k_i}\ker x_{ni}^*$ to $X$ with $\ve{z^*}\le \rc 2$ (Hahn-Banach). 

We have $\ker (x^*-z^*)\supeq \bigcap_{i=1}^{k_n} \ker x_{ni}^*$. So by Lemma~\ref{lem:f4-3}, $x^*-z^*\in \spn\set{x_{ni}^*}{1\le i\le k_n}$. So $d(x^*,Y)\le\rc2$. By Riesz's lemma~\ref{lem:riesz}, $Y=X^*$. 
\end{proof}
We show weakly compact subsets are metrizable under certain conditions.
%dual of l^{\iy} also $w^*$ separable
\begin{pr}\llabel{pr:f4-16}
Let $K$ be a $w$-compact subset of a Banach space $X$. If $X^*$ is $w^*$-separable (e.g. if $X$ is separable), then $(K,w)$ is metrizable.
\end{pr}
\begin{proof}
Let $\set{x_n^*}{n\in \N}$ be $w^*$-dense in $X^*$. Then $\cal F=\set{x_n^*}{n\in \N}$ separates the points of $X$, and hence the points of $K$. So $\si=\si(K,\cal F)$ is metrizable and weaker than the $w$-topology (it's only making the $x_n^*$ continuous). So $\id:(K,w)\to (K,\si)$ is a continuous bijection from a compact to a Hausdorff space, so it's a homeomorphism.
\end{proof}

\subsection{Reflexivity, Goldstine's Theorem}
We aim to give a complete characterization of reflexivity. We'll give two proofs, the first relying on the following lemma.
\begin{lem}[Local reflexivity]
\llabel{lem:f4-17}
Let $X$ be a normed space, $ F$ be a finite-dimensional subspace of $X^*$, $x^{**}\in X^{**}$ and $\ep>0$. Then $\exists x\in X$, $\hat x|_{F}=x^{**}|_F$ and $\ve{x}\le (1+\ep)\ve{x^{**}}$. 
\end{lem}
\begin{proof}
The proof basically uses Hahn-Banach separation in finite dimensions. Let $x_1^*,\ldots, x_n^*$ be a basis of $F$. WLOG, $\ve{x^{**}}\le 1$. Consider $T:X\to \R^n$, $Tx=(x_1^*(x),\ldots, x_n^*(x))$. $T$ is linear, and onto: if it is not onto, then there exists $(\al_1,\ldots, \al_n)\in \R^n$ orthogonal to $T(X)$, so $\sum \al_ix_i^*(x)=0$ for all $x$, i.e., $\sum \al_ix_i^* = 0$.

So $T$ is an open map by the Open Mapping Theorem~\ref{thm:omt}\footnote{This is not a direct application as we don't have Banach spaces. We have linear $T:X\to \R^n$, $x\mapsto (x_i^*(x))_{i=1}^n$, $T$ onto. But because $\R^n$ is finite-dimensional there exists $E\subeq X$, $\dim E<\iy$ such that $T(E)=\R^n$. By the Open Mapping Theorem on $E\to \R^n$, $T(B_E)$, and hence $T(B_X)$ is a neighborhood of 0 in $\R^n$.}. It follows that $A=\set{Tx}{\ve{x}<1+\ep}$ is an open convex set in $\R^n$ and $0\in A$. We need $(x^{**}(x_1^*),\ldots, x^{**}(x_n^*))\in A$. If not, then by the Hahn-Banach Theorem~\ref{thm:f4-9}, there exist $(\al_1,\ldots, \al_n)\in \R^n$ such that \[\sum_{i=1}^n \al_ix_i^*(x)<\sum\al_ix^{**}(x_i^*)\]
for all $x\in X$, $\ve{x}<1+\ep$. 

%Taking the sup over $x$, $\ve{\sum \al_ix_i^*}(1+\ep)\le \sum\al_ix_i^*\le x^{**} (x_i^*)$ for all $x\in X$, $\ve{x}<1+\ep$. 
Taking the sup over $x$, 
\[%\ve{\al_ix_i}
\ve{\sum \al_ix_i^*}(1+\ep)\le x^{**}\pa{\sum \al_ix_i^*} \le \ve{\sum \al_ix_i^*}.
\]
This is a contradiction as the $x_i^*$ are linearly independent and $(\al_1,\ldots, \al_n)\ne 0$ (because of $<$ above). 
\end{proof}
\begin{thm}[Goldstine's Theorem]\llabel{thm:goldstine}
%as always imbed in second dual
$\ol{B_X}^{w^*}=B_{X^{**}}$ where $X$ is a normed space, $\ol{B_X}^{w^*}$ is the $w^*$-closure in $X^{**}$ of $B_X$.
\end{thm}
%
The dual of weak star separable does not imply that the space is separable. We can actually prove this now.
\begin{proof}
Let $K=\ol B_X^{w^*}$. Then $K$ is $w^*$-compact by the Banach-Alaoglu Theorem~\ref{thm:ba}. Assume $K\ne B_{X^{**}}$. Fix $x_0^{**}\in B_{x^{**}}\bs K$. 
We can finish in 2 ways.
\begin{enumerate}
\item
By Theorem~\ref{thm:f4-10}(2) (Hahn-Banach separation for convex sets) there exists $x^*$ such that $\sup_K x^*<x_0^{**}(x^*)$.
%LCs which is the second dual of the $w^*$ topology.
Since $K\supeq B_X$, $\ve{x^*}\le\sup_K x^*<x_0^{**}(x^*)\le \ve{x^*}$.
\item
$K$ is $w^*$-closed, so there exists a $w^*$-neighborhood $U$ of $x_0^{**}$ such that $U\cap K=\phi$. WLOG $U=\set{y^{**}\in X^{**}}{|y^{**}(x_i^*)-x_0^{**}(x_i^*)|<\ep\text{ for }1\le i\le n}$ for some $n\in \N$, $x_1^*,\ldots x_n^*$, $\ep>0$. Let $F=\spn\{x_1^*,\ldots, x_n^*\}$. By Lemma~\ref{lem:f4-17}, there exists $x\in X$ such that $\ve{x}\le 1+\ep$ , $\hat x|_F=x_0^{**}|_F$. Then $\fc{x}{1+\ep}\in B_X\subeq K$,
\[
\ab{\fc x{1+\ep}(x_i^*)-x_0^{**}(x_i^*)}=\ab{x_0^{**}(x_i^*)}\ab{\rc{1+\ep}-1}\le \ep \text{ for all }i,
\]
so $\fc{x}{1+\ep}\in U$.  
(We don't need the Hahn-Banach Theorem in full generality, we just needed it for normed spaces.)
%just need it for normed spaces, for $\R^n$ 
%good excuse introduce proper reflexivity.
%will be in ball if normalize
\end{enumerate}
\end{proof}
\begin{thm}\llabel{thm:4-19}
Let $X$ be a Banach space. TFAE:
\begin{enumerate}
\item
$X$ is reflexive.
\item
$(B_X, w)$ is compact.
\item
$X^*$ is reflexive.
%second dual is always reflexive.
%third dual to first dual, inverse of the canonical embedding.
\end{enumerate}
It follows that if $Y$ is a closed subspace of $X$ with $X$ reflexive, then $Y$ is reflexive.
\end{thm}
\begin{proof}
(1)$\implies $(2) If $X$ is reflexive, $(B_X,w)=(B_{X^{**}},w^*)$ and $(B_{X^{**}},w^*)$ is compact by Banach-Alaoglu~\ref{thm:ba}.

(2)$\implies$(1) Since the restriction to $X$ of the $w^*$-topology on $X^{**}$ is the $w$-topology, we get $B_X$ is $w^*$-compact on $X^{**}$, and so it's $w^*$-closed. By Goldstine's Theorem~\ref{thm:goldstine}, $B_X=\ol B_X^{w^{*}}=B_{X^{**}}$. So $X=X^{**}$.

(1)$\implies$(3) We have $\si(X^*,X)=\si(X^*, X^{**})$ since $X$ is reflexive, so $(B_{X^*},w)=(B_{X^*},w^*)$ is compact by Banach-Alaoglu~\ref{thm:ba}. By (2)$\implies$(1), $X^*$ is reflexive.

(3)$\implies$(1) By (1)$\implies$(3), $X^{**}$ is reflexive, so by (1)$\implies$(2), $(B_{X^{**}},w)$ is compact. $B_X$ is convex and $\ved$-closed in $X^{**}$, and hence $w$-closed by Mazur's Theorem~\ref{thm:f4-11}. So $(B_X,w)$ is $w$-compact. So by (2)$\implies $(1), $X$ is reflexive.
%being closed convex ,
%closed therefore weakly closed (in LCS...?) 
\end{proof}
\begin{rem}
If $X$ is separable and reflexive, then $(B_X, w)$ is a compact metric space (Proposition~\ref{pr:f4-16}).
%weak star separable implies metrizable. weak star is sep if original is metrizable
%take subsequences converge in weak topology. 
\end{rem}
\begin{thm}\llabel{thm:f4-20}
If $X$ is a separable Banach space, then $X\hra C[0,1]$ isometrically.
\end{thm}
We've used Hahn-Banach to show we can embed into $\ell^{\iy}$ (Theorem~\ref{thm:embed-liy}) but that's not separable, and $C[0,1]$ is.
\begin{lem}\llabel{lem:f4-21}
If $K$ is a compact metric space, then there exists a continuous surjection $\ph:\De\to K$, where $\De$ is the Cantor set. 
\end{lem}
\begin{proof}
$\De$ is homeomorphic to $\{0,1\}^{\N}$ with the product topology, with the homeomorphism given by $(\ep_i)_{i=1}^m \mapsto \sum_{i=1}^{\iy} \ep_i 2\cdot 3^{i}$. %continuous bijection.

We introduce some notation. For $m\le n$, $\ep=(\ep_i)_{i=1}^n\in \{0,1\}^n$, let $\ep|_{m}=(\ep_i)_{i=1}^m\in \{0,1\}^m$.

Our plan is to start with the compact set, %at the top, 
divide it into closed sets $K_0,K_1$ that are smaller, and then divide $K_{00}$ into $K_{00}\cup K_{01}$, and so forth. Any 0-1 sequence corresponds to a nested sequence of compact sets, whose intersection is a singleton $x$.

$K$ is a compact metric space, so there exist finitely many closed subsets with diameter less than 1 that cover $K$; we can assume the the number is a power of 2: There exists $n_1\in \N$ and nonempty closed subsets $K_{\ep}$ of $K$, $\ep\in \{0,1\}^{n_1}$ such that $K=\bigcup_{\ep\in \{0,1\}^{n_1}}K_{\ep}$, $\diam(K_{\ep})<1$.

There exists $n_2>n_1$ such that for each $\ep\in \{0,1\}^{n_1}$, there exist nonempty closed sets $K_{\de}\subeq K_{\ep}$, $\de\in \{0,1\}^{n_2}$, $\de|_{n_1}=\ep$ and $\bigcup_{\de\in \{0,1\}^{n_2},\, \de|_{n_1}=\ep} K_{\de}=K_{\ep}$, $\diam K_{\de}\le \rc 2$. Continue inductively to get $n_1<n_2<n_3<\cdots \ne \phi$, closed sets $K_{\ep}$, $\ep\in \{0,1\}^{n_t}$, $k\in \N$, $\bigcup_{\de\in \{0,1\}^{n_{k+1}},\de|_{n_k}=\ep} K_{\de}=K_{\ep}$, $\dim(K_{\de})<\rc{k+1}$ ($k\in \N, \ep\in \{0,1\}^{n_d}$). 
%nonempty close d, FIP intersection cannot contain 2 distinct points.

Define $\ph:\{0,1\}^{\N}\to K$ by $\ph((\ep_i)_{i=1}^{\iy})$, the unique point in $\bigcap_{k=1}^{\iy} K_{\ep_1,\ldots ,\ep_{n_k}}$. Given $\ep,\de\in \{0,1\}^{\N}$, if $\eta=\ep|_{n_k}=\de|_{n_k}$, then $\ph(\ep)$, $\ph(\de)\in K_{\eta}$ so $d(\ph(\ep),\ph(\de))<\rc{k}$. So $\ph$ is continuous.
Given $x\in K$ such that $x\in K_{\ep_1,\ldots, \ep_{n_1}}$, in turn there exist $\ep_{n_1+1},\ldots, \ep_{n_2}$ such that $x\in K_{\ep_{n_1},\ldots, \ep_{n_2}}$, etc. Then $x=\ph((\ep_i)_{i=1}^m)$, showing surjectivity.
%inf zero seq
%metri iff sepb
%rather than define all...
\end{proof}

\begin{proof}[Proof of Theorem~\ref{thm:f4-20}]
$X$ is separable, so $K=(B_{X^*},w^*)$ is a compact metric space (Banach-Alaoglu~\ref{thm:ba}, Proposition~\ref{pr:f4-14}). We saw in the proof of Proposition~\ref{pr:f4-14} that $X\hra C(K)$ isometrically ($x\mapsto \hat x|_K$). $C(K)\hra C(\De)$ isometrically by the map $f\mapsto f\circ \ph$ where $\ph:\De\to K$ is continuous surjective (Lemma~\ref{lem:f4-21}).
%onto, so sup norm is exactly sup norm
%linear extend to ball of 0,1

Finally define the map $C(\De)\hra C[0,1]$ by $f\mapsto \wt f$, where $\wt f$ is obtained by linearly extending $f$ to $[0,1]$; it is an isometry. Take the composition \[X\hra C(K)\hra C(\De)\hra C[0,1].\] 
\end{proof}

One reason why this is useful is that now we can think of the class of separable Banach spaces as the set of closed subspaces of $C[0,1]$. We can put a Borel structure on this set of subspaces and get a Polish space; there is a connection between Banach space properties and set theory.

Useful for us is the fact that our abstract spaces become concrete. (This also says $C[0,1]$ is a very hard space to understand.)

\section{Additional material (non-examinable)}
\begin{thm}[Eberlein-\v{S}mulian] %check
%Eberlein
Let $X$ be a Banach space and $K\subeq X$. Then $K$ is $w$-compact iff $K$ is $w$-sequentially compact.
\end{thm}
\begin{proof}
$\implies$: 
WLOG $X$ is separable. Then $(K,w)$ is metrizable (Proposition~\ref{pr:f4-16}).
%Take sequence in $K$, prove convergent sequence. Intersect with $K$, closed subset 
%live in separable space. Weakly conv seq in that space, conv in original space.

$\Leftarrow$: 
If $K$ is $w$-sequentially compact, then $K$ is bounded (as it's weakly bounded). So $\ol K^{w^*}$ is $w^*$ compact. We need $K\subeq X$.
%weakly bounded is norm bounded
%go to 2nd dual and take $w^*$ closure
%weak star closure didn't go outside, weakly compact because weak star... is top on X$

Let $x^{**}\in \ol K^{w^*}$. Pick $x_1^*\in S_{X^*}$. There exist $x_i\in K$ such that $|w^{**}(x_1^*)-x_1^*(x_1)|<1$. Let $F_1=\spn\{x^{**}, x_1\}$.
%Each vector has a norming functional. Take a $\de$-net. They will approximately norm everything else.
There exist $x_2^*,\ldots, x_{n_2}^*\in S_{X^*}$ such that 
\[
\ve{y^{**}}2\le \sup_{i\le n_2} |y^{**}(x_i)|.
\]
There exist $x_2\in K$ such that $|x^{**}(x_i^*)-x_i^*(x_2)|<\rc 2$ for all $i\le n_2$. Let $F_2=\spn(x^{**},x_1,x_2)$. There exist $x_{n_2+1}^*,\ldots, x_{n_3}^*\in S_{X^*}$ such that $\fc{\ve{y^{**}}}2\le \sup_{i\le n_3} |y^{**}(x_i^*)|$. There exists $x_3\in K$  with $|x^{**}(x_i^*)-x_i^*(x_3)|<\rc3$ for all $i\le n_3$, etc. ``The sequences chase each other, at $\iy$ they somehow meet." We obtain $(x_n)\sub K, (x_n^*)\sub S_{X^*}$. There exist $b_1<b_2<\cdots$ wuch that $x_{k_n}\xra w x\in K$. By Corollary~\ref{cor:f4-12}, $x^{**}-x\in \ol{\spn}\{x^{**}\}\cup\set{ x_n}{n\in \N}$. If $y^{**}\in {\spn}\{x^{**}\}\cup\set{ x_n}{n\in \N}$, $\fc{\ve{y^{**}}}{2}\le \sup_{n\in \N}|y^{**}(x_n^*)|$. The same holds for $y^{**}\in \ol{\spn}\{x^{**}\}\cup\set{ x_n}{n\in \N}$. We have
\[
|(x^{**}-x)(x_m^*)|\le \ub{|(x^{**}-x_{k_n})(x_m^*)|}{<\rc n\to 0\text{ as }n\to \iy}+\ab{(x-x_{k_n})(x_m^*)}{\to 0\text{ as }n\to \iy}
\]
\end{proof}
%%%%%%%%%%%%%%%%%%%%%%%
%HB in LCS
%char of refl

Lecture 13-11

\begin{thm}[Krein-\v{S}mulian Theorem]\llabel{thm:krein-smulian}
If $K$ is $w$-compact in a Banach space $X$, then $\ol{\text{conv}}(K)$ is $w$-compact. 
\end{thm}

(Note ``$K$ $\ved$-compact$\implies\ol{\text{conv}}(K)$ is $\ved$-compact" is easy.)

%Normally we define measurable without any reference to the measure, but here we need it.
We first need some background on vector-valued integration.
\subsection{Vector-valued integration}
\begin{df}
Let $(\Om,\cal F,\mu)$ be a finite measure space and $X$ be a Banach space.
\begin{enumerate}
\item
$f:\Om\to X$ is a \textbf{simple function} if $f=\sui x_i1_{A_i}$ where $x_i\in X$ and $A_i\in \cal F$.
\item
$f:\Om\to X$ is \textbf{$\mu$-measurable} if there exist simple functions $f_n,n\in \N$ such that $f_n\to f$ $\mu$-a.e. (i.e., $\mu(\set{\om\in \Om}{f_n(\om)\not\to f(\om)}=0)$). 
\end{enumerate}
\end{df}

\begin{thm}[Pettis measurability]
For $X$ separable, let $f:\Om\to X$. TFAE:
\begin{enumerate}
\item
$f$ is $\mu$-measurable. 
\item
$x^*\circ f:\Om\to \R$ is measurable foir all $x^*\in X^*$. 
\end{enumerate}
\end{thm}

\begin{df}
Let $f=\sui x_i1_{A_i}$ be a simple function. Define $\int_{\Om} f_1(\om) \,d\mu(\om)=\sui x_i\mu(A_i)\in X$. We say $f:\Om\to X$ is \textbf{Bochner integrable} if there exist simple functions $f_n,n\in \N$ such that $\int_{\Om}\ve{f(\om)-f_n(\om)}\,d\mu(\om)\to 0$ as $n\to \iy$. We set $\int_{\Om} f\,d\mu=\lim_{n\to \iy} \int_{\Om}f_n\,d\mu$. We say $f:\Om\to X$ is integrable iff $\int_{\Om}\ve{f(\om)}\,d\mu(\om)<\iy$.
\end{df}
%converges in measure, implies a.e. since finite measure space


The idea is that for $x_1,\ldots, x_n\in K$, $t_1,\ldots, t_n\ge 0$, $\sui t_i=1$, $\sui t_ix_i=\int_Kx\,d\mu$, $\mu=\sui t_i\de_{x_i}$. 

\begin{proof}
WLOG $X$ is separable (Eberlein-\v Smulian). Then $K$ is compact in the $w$-topology, $f:K\to X$, $f(x)=x$. For all $x^*\in X^*$, $x^*\circ f=x^*|_K$ is $w$-continuous, so measurable for the Borel $\si$-field of $(K,w)$. So $f$ is $\mu$-measurable for all $\mu\in C(K)^*=M(K)$. Also $K$ is $w$-compact implies $K$ is $\ved$-bounded, so $f$ is Bochner integrable for all $\mu\in M(K)$. Consider $T:C(K)^*\to X$, $T(\mu)=\int_K f\,d\mu$. Then $T$ is linear, $w^*$-$w$ continuous \fixme{???}: 
\[
x^*\circ T(\mu)=x^*\pa{
\int_K x\,d\mu
}
=\int_K x^*(x)\,d\mu
=\an{x^*(x),\mu}.
\]
(It's clear for simple functions; use the staandard argument to get it for all functions.)
%convex com of pointmasses
%hopefully include weak convex hull
So $x^*\circ T$ is $w^*$-continuous. $B_{C(K)^*}$ is $w^*$-compact, so $T(B_{C(K)^*})$ is $w$-compact. For all $x\in K$, $T(\de_x)=x$, so $T(B_{C(K)^*})\supeq K$. Hence $T(B_{C(K)^*})\supeq \ol{\text{conv}}(K)$.
\end{proof}


