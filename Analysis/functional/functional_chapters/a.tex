\chapter{Problems}
\fixme{I took these down very quickly during the examples class, so these are probably unreadable right now. Need to fix!}

\begin{enumerate}
\item
Show that for all $x\in K$ there exists $f_x\in A$ such that $f_x(x)\ne 0$. There exists a neighborhood $U_x$ of $x$, $f_x\ne 0$ on $U_n$. There exists $x_1,\ldots, x_n$, $\bigcup_{i=1}^n U_{x_i}=K$, $f=\sum_{i=1}^n f_{x_i}^2\in A$, $f>0$ on $K$. Write $B=A\opl \C 1$, $\ol B=C(K)$. Given $\ep>0$ there exists $a\in A$, $\la\in \C$, $\ve{(a+\la)-\rc f}<\ep$. (Replaced $f$ by $\fc{f}{\ve{f}}$. $\ve{(af+\la f)-1}<\ep$. So $\ol A=C(K)$.
\item
$q:X\to X/Y$ $q(B_{X}^{\circ})$, so $q$ is an open map.
%biggest that makes it continuous.
Assume $q^{-1}(U)$ is open in $X$, and $U=q(q^{-1}(U))$ is open in the norm topology. $q$ is continuous, so $\ved$-topology is weaker than quotient topology.
%cont map onto top space, open, top on image space is quotient top

Take a Cauchy sequence. $x_n+Y$ is Cauchy in $X/Y$, so $x_n+Y\to x+Y$. For all $n$, there exists $y_n\in Y$ such that $\ve{x_n-x+y_n}<\ve{x_n-x+Y}+2^{-n}$. The difference tends to 0, so $y_n$ is Cauchy, so $y_n\to y$, and $x_n\to x+y$.

\item
Factor $T:X\to Y$ with $X\to X/\ker T\xra{\wt T} Y$. $E\subeq X/\ker T$ is closed and bounded. Say $\ve{x}<C$ for all $x\in E$. Then $q(q^{-1}(E)\cap CB_X)=E$ by definition of the quotient norm. (lift by arbitrary small amount)
Then $\wt T(E)=T(q^{-1}(E)\cap CB_X)$. 
we've kept our property. So WLOG $T$ is injective.
%Interset with ball size bound on $E$.

Suppose $Tx_n\to y$. Then $y\in \ol{T(E)}$, $E=\ol{\set{x_n}{n\in \N}}$. If bounded set, image is closed, so $y\in T(E)$, and we're done. Assume $E$ is unbounded. Then $F:=\ol{\set{\fc{x_n}{\ve{x_n}}}{n\in \N}}\subeq S_X$ is closed. Then $0\in \ol{T(F)}=T(F)$.

(The injective step isn't essential, but instead of normalize you perturb by members of the kernel, it's slightly messier.)
\item 
(Grothendieck) We'll use this theorem when we study compact operators.

Pick a finite number of points around which balls of radius 1 cover $K$. 
%Find a sequence of coverings of translates of $K$. 
(There exist finite $F_1\subeq K$ such that $K\subeq F_1+B_X$.) 
Then compact $K_1=\bigcup_{x\in F_1} (K-x)\cap B_X\subeq F_z+\rc 4B_X$. (There exist $F_2\subeq K_1$.)
Let $K_n=\bigcup_{x\in F_n}[(K_{n-1}-x)\cap \rc{4^{n-1}}B_X]\subeq F_n+\rc{4^{n-1}}$.

%now in quarter ball. 

For $x\in K$, can keep approximating, $x-x_1-x_2-x_3-\cdots$, so $x=\suo x_n$ where $x_n\in F_n$.  $x=\suo \rc{2^n} (2^nx_n)$. Infinite convex combination of these points.

Then $y\in F_n$ implies $\ve{y}\le \rc{4^{n-2}}$ for $n\ge 2$. Then $2x,x\in F_1$, $2^nx,x\in F_{n}$, is a null sequence; its closed convex hull covers $K$.

\item Start with $\set{f_n}{n\in\N}$ dense in $Y$. For all $n$ there exists $(x_{n_i})_i\subeq S_X$ with $f_n(x_{n_i})\to \ve{f_n}$ as $n\to \iy$. Let $Z=\ol{\spn}\set{x_{n_i}}{n,i\in \N}$ separable. $Y\to Z^*$, $f\mapsto f|_Z$. Take $y^*\in Y$, $\ep>0$, there exists $n$ such that $\ve{y^*-f_n}<\fc{\ep}{3}$. There exists $k\in \N$ such that $f_n(x_{nk})>\ve{f_n}-\fc{\ep}{3}$. Use the triangle inequality to show $|y^*(x_{nk})|>\ve{y^*}-\ep$.
\item $T:X\to Y$, $T(X)=Y$. By the open mapping theorem there exists $C>0$ such that $T(CB_X)\supeq B_Y$. 
%If we come up with $Z$ such that when we restrict it's a bijection, then we're done. 
Choose dense $\{y_n\}\subeq B_Y$. Pick $x_n$ such that $Tx_n=y_n$. Let $Z=\ol{\spn\set{x_n}{n\in \N}}$. 
$T|_Z:Z\to Y$, $\ol{T|_Z(CB_Z)}\supeq B_Y$. By Open mapping lemma, $T|_Z$ is onto.
%We need the $x_n$ to lie in an open ball. 
%take the quotient, have isomorphism, can assume quotient.
%open mapping lemma you want to use
\item
 $f:X\to \R$ factors as $X\to X/\ker f\xra{\wt f} \R$. We have $\ve{\wt{f}}=\ve{f}=1$. We have $|f(x)|=|\wt f(x+\ker f)|=\ve{x+\ker f}=d(x,\ker f)$. %mod f(x) is norm in quotient space

Use fact $\ker f$ is 1-D space, so copy of reals, just multiplication by scalar. So for some $\la$, $\wt f(x+\ker f)=t\la$. There exist $\wt f\in S_{(X/Y)^*}$ such that $\wt f(x_0+Y)=\ve{x_0+Y}$ by Hahn-Banach. Let $f=\wt f\circ q$.

(You can do without quotient maps just by writing inequalities, but this is neater.)
\item
Unit sphere is compact in finite-dimensional space.
Consider $Y\cap Z$. Define... $d(S_F,Y\cap Z)=d(x,Y\cap Z)>0$ for some $x\in S_F$. Projection onto $Y\cap Z$ is continuous. In finite dimension everything is continuous. By triangle inequality we have the bound on the RHS.

Or let $f=0$ on $Y\cap Z$ and $f(y_i)=1$, $f(y_j)=0$ for all $j\ne 1$, $f(z_i), f(w_i)=0$. Check continuous functional by checking kernel is closed, you do the same work.

For example, to prove if $C[0,1]\sim \set{f\in C[0,1]}{f(1)=0}$.)
Sufficiently nasty Banch spaces are not isomorphic to their 1-codimensional subspaces.
\item  Hahn-Banach allows us to extend the functionals. 
Consider $Y$. Consider $a,b$ with $a|||y|||\le \ve y\le b|||y|||$. Check with the operator norms induced by this. We get 
\[
aB_Y^{\ved}\subeq B_Y^{|\ved|}\subeq bB_Y^{\ved}.
\]
so 
\[
a\ve{y^*}\le |\ve{y^*}|\le b\ve{y^*}.
\]
and we get
\[
bB_Y^{|\ved|}\subeq B_{Y^*}^{\ved}\subeq aB_Y^{|\ved|}.
\]
For all $f\in B_{Y^*}^{|\ved|}$ let $\wt f$ be a Hahn-Banach extension to $X$ such that $\ve{\wt f}=\ve{f}$. 
Define something that we'll show is a norm on the whole space.
\[
|\ve{x}|'=\sup\pa{
\set{|\wt f(x)|}{f\in B_{Y^*}^{|\ved|}}\cup 
\rc bB_{X^*}
}=\sup\set{|f(x)|}{f\in S}.
%bounded set because norm at most constant times triple bound norm
%equivalent because at most some const multiple of ball. 
\]
$y\in Y, |\ve{y}|'=|\ve{y}|$.
$\rc bB_{X^*} \subeq S\subeq \rc a B_{X^*}$. $\set{f|_Y}{f\in S}=B_{Y^*}^{|\ve{\cdot}|}$.

OR Define a convex symmetric set, using the Minkowski functional. Let $C=\set{conv}{B_Y^{|\ved|}\cup aB_X^{\ved}}$. We will define $\mu_C$. $C\cap Y=B_Y^{|\ved|}$. 

OR Zorn---a more ``helpful" version of Hahn-Banach, 1 dimension at a time.
\item 
\begin{enumerate}
\item
$\ol{\spn} A=(A^{\perp})_{\perp}$. 
\begin{enumerate}
\item
$\subeq$: From definition, $\subeq$ is clear.
\item
$\supeq$: 
 The RHS is a closed linear subspace because it is a intersection of kernels. ($B_{\perp}=\set{x\in X}{b(x)=0\forall b\in B}=\bigcap_{b\in B}\ker b$.) 

What might go wrong? There might not be enough functionals, so maybe if we take all $f$ such that $f(A)=0$, there might be stuff outside the $\ol{\spn}(A)$ such that $f(x)=0$ as well. But there {\it are} enough functionals, by Hahn-Banach. (Whenever we worry about lack of functionals, use Hahn-Banach!)

Pick $y\nin \ol{\spn}(A)$. By Hahn-Banach, there exists $f\in X^*$ such that $f\equiv 0$ on $\spn(A)$\footnote{Since $\spn(A)$ is a subspace, $f(\spn(A))$ is  a subspace of $\R$ and hence 0} and $f(y)=1$. Now $f\in A^{\perp}$ so $y\nin (A^{\perp})_{\perp}$. 
\end{enumerate}
\item $\ol{\spn}B\stackrel ?=(B_{\perp})^{\perp}$
%by HB can extend
\begin{enumerate}
\item
$\subeq$: 
We have $\ol{\spn}(B)\subeq (B_{\perp})^{\perp}$. 
\item
$\fixme{\supeq}$: NOT NECESSARILY. We try to emulate the above proof. What goes wrong? Take $f\nin \ol{\spn}(B)$. By Hahn-Banach, there exists $F\in X^{**}$ such that $F(f)=1$, $ F(B)=\{0\}$. Uh-oh: if $F=\wh x$ we get $f(x)=1,\set{f(y)}{y\in B}=\{0\}$ and $f\nin (B_{\perp})^{\perp}$.

{\it So what goes wrong is that we're trying to do the above with $X^*$ in place of $X$, but maybe $X\ne X^{**}$!}

\fixme{(?) So to disprove this, take $Y$ not reflexive, for instance, $Y=\ell_{\iy}$, $B=c_0$. Then $B_{\perp}=\ell_1$ but $(B_{\perp})^{\perp}=\ell_{\iy}$. }
\end{enumerate}
Question: How can we modify this statement to be true?

Take weak star closure: $\spn^{w^*} B=(B_{\perp})^{\perp}$. See Problem 7 in Example sheet 2.
%cf. Mazur's Theorem.
%weak closure same b/c Mazur's theorem
%Note that in the case where $B$
\end{enumerate}

\item
$X^*/Y^{\perp}\to Y^*$ by mapping $f+Y^{\perp}\mapsto f|_Y$. Use Hahn-Banach.

Start with $R:X^*\to Y^*$, $R(f)=f|_Y$. $R(B_{X^*})=B_{Y^*}$. 
Exact quotient map.
%Every extension has ... norm the same
$\ker R=Y^{\perp}$. So this factors through 
%when do things factor through
$X^*\to X^*/Y^{\perp}\xra{\wt R} Y^*$. 

$(X/Y)^*\to Y^{\perp}$ Take composite of $f$ and quotient map $f\mapsto f\circ q$ where $q:X\to X/Y$ is the quotient map. This is well defined because $q$ dies on $Y$. This is the fact that $q(B_X^{\circ})=B_{X/Y}^{\circ}$. Map open ball to same thing so same norm. Onto because everything dies on $Y$ factors through quotient map. $g:X\to \R$ and $\ker g\supeq Y$, then $X\xra q X/Y\xra{\exists g}\R$.
%$Y^{\perp}\to (X/Y)^*$. 
\item 
Suppose $x_n$ is continuous for all $n$. Form $y=\fc{x_n}{2^n\ve{x_n}}$. It's a finite sum $\sum_{y\in F}\la_yy$ where $F$ is finite. We get $\ep_{x_n}(y)=\rc{2^n\ve{x_n}}$ for all $n$, this is 0 for large $n$, get a contradiction.
%Suppose BWOC a sequence of $x_n$ with continuous coordinates, $\sum \fc{x

Using Zorn can construct continuous functional.

Kernels are closed, intersection of kernels $\bigcap_n \ker \ep_{x_n}$. The quotient is also complete. $X/Y$ has countable basis. $x_n+Y$ is a basis. There is no Banach space with countable basis.
\item
%Zorn's lemma $(C,D)$ convex dense disjoint elements. 
Maximal element must have $C\cup D$ to be the whole space. Can take discontinuous functional, kernel is dense.

$\{f>0\}$, $\{f\le 0\}$. $\ker f$ is dense. 1-codimensional. Kernel dense, translate of kernel dense.

Need Zorn to get discontinuous functional. 

``When you use Zorn in hidden way usually right."
\item
Take $Z=C[0,1]$. A finite-dimensional space imbeds isometrically into $\ell_{\iy}$. $T:F\hra \ell_{\iy}$. Embed almost isometrically? $P_N\circ T:H\hra c_{\iy}$, $1+\ep$ for all $\ep>0$. There exists an $\ep$-net in $S_F$ with $x_1,\ldots, x_n$ there exists $N$ such that $\ve{x_i-P_Nx_i}<\ep$ for all $i$. ($P_N(a_i)=(a_1,\ldots, a_N,\ldots)$)
$c_{00}$

There are countably many norms, so we can approximate every norm. 
Look at $M_n=\{(\R^n,\ved)\}$, $d$ Banach. %mazur
$M_n$ is compact, so separable. Take a countable dense set, do it for every $n$. Thus there exists a sequence $F_1,F_2,\ldots, $ of fintie dimensional spaces so that every finite-dimensional space is close to one of these. For all $\ep$ there exists $n$ such that $d(F,F_n)<1+\ep$. Take $\pa{\bigoplus f_n}_{\ell_2}$. Each is reflexive.
\end{enumerate}
