\chapter{Problems 3}
Progress: 

Done: 2, 3, 5, 7
\section{Problems}
\begin{enumerate}
\item ({\it The Swiss Cheese: a compact set $K\subeq \C$ such that $R(K)\ne A(K)$.}) Let $\De$ be the closed unit disc in $\C$. Show that one can choose a sequence $(D_n)$ of non-overlapping open discs in the interior of $\De$ such that $\sum_n\diam(D_n)<\iy$ and $\bigcup_n D_n$ is dense in $\De$. Set $K=\De\bs \bigcup_n D_n$. ($K$ is the Swiss cheese.) Show that the formula
\[
\te(f)=\int_{\pl \De} f(z)\,dz-\sum_{n=1}^{\iy} \int_{\pl D_n} f(z)\,dz
\]
defines a nonzero, bounded linear functional on $C(K)$. Deduce that $R(K)\ne A(K)$.
\item Let $p$ be an idempotent element of a Banach algebra $A$ (which means that $p^2=p$). Show that if $p$ is in the closure of an ideal $J$ of $A$, then $p$ does in fact belong to $J$.
\item Let $K$ be an non-empty compact subset of $\C$. Verify that the only characters of $R(K)$ are the point evaluations $\de_z$ with $z\in K$.
\item 
\item Give an example of $2\times 2$ matrices $x,y$ with $r(xy)>r(x)r(y)$ and $r(x+y)>r(x)+r(y)$.
\item
\item
Let $A$ be a Banach algebra such that every element of $A$ is nilpotent: for each $x\in A$ there exists $n(x)\in \N$ such that $x^{n(x)}=0$. Prove that there exists $N\in \N$ such that $x^N=0$ for all $x\in A$.
\end{enumerate}
\section{Solutions}
\begin{enumerate}
\item %\fixme{I don't know how to bound the sum of diameters. Maybe this will help~\url{http://en.wikipedia.org/wiki/Descartes'_theorem}. I tried to find a function $S$ such that $S(r_1,r_2,r_3)>r_1+r_2+r_3-\sqrt{r_1r_2+r_2r_3+r_3r_1}+
%\sum_{\text{cyc}} S(r_1,r_2,r_1+r_2+r_3-\sqrt{r_1r_2+r_2r_3+r_3r_1})$.}

\begin{enumerate}
\item
{\it Choosing the discs.} Enumerate the rational points $Q_j$ in the disc. For each $Q_j$, if $Q_j$ is not already covered  by (or on the boundary of) a disc, put a disc of radius $r_j<\rc{2^j}$ around $Q_j$ not intersecting any previous disc. For convenience, relabel so that the discs are $D_n$ with radius $r_n$.

We have $\sum_{n=1}^{\iy} r_n<1$. The discs are dense in $\De$ because $\De\cap \Q^2$ is dense in $\De$.
\item
{\it Non-zero, bounded linear functional.} The functional is clearly linear. To see it's nonzero, let $f(z)=\ol z$. Then for a disc $D$ with center $a$ and radius $r$,
\[
\int_{\pl D} f(z)\,dz\stackrel{z=a+e^{2\pi it}}{=}\int_{0}^1(\ol a+re^{-2\pi i t})2\pi ie^{2\pi i t}\,dt=2\pi i r.
\]
Thus
\[
\te(f)=\int_{\pl \De}f(z)\,dz-\sum_{n=1}^{\iy} \int_{\pl D_n} f(z)\,dz=2\pi i - 2\pi i\sum_{n=1}^{\iy} r_n>0.
\]

$\te$ is bounded because $\te(f)\le \ve{f}(2\pi + 2\pi \sum_{n=1}^{\iy}r_n)$.
\item
{\it $R(K)\ne A(K)$.} Because $K$ has empty interior, $A(K)=C(K)$. Hence $\te\in C(K)\bs R(K)=A(K)\bs R(K)$.
%\end{align*}
 
\end{enumerate}

\fixme{Pick rational points, one-by-one, if covered, done. If not, put in another disc. $r^2$ not rational. 
Choose sufficiently small.

Estimate of integrals. $z\mapsto \sum \ph(z_i)$.

Checking nonzero: $z\mapsto \ol z$.  Assume $\sum \ell(\pl D_n)<1$. Otherwise multiply by something. 

$R(K)\ne A(K)=C(K)$.}
\item
Idea: the set of units is open, and $p$ acts ``like" a unit.

\ul{Solution:} Consider the ideal $pAp$. This is an algebra, and $p=ppp$ is the identity element of $pAp$, since $p(pap)=pap$. Note $G(pAp)$ is open in $pAp$.

If $p\in \ol J$, then there is a sequence 
\[a_n\to p\text{ with }a_n\in J,\] 
and hence
\[pa_np\to p\text{ with }pa_np\in pJp.\]%\subeq pAp$. 
Since $p\in G(pAp)$ is open in $pAp$, we get that $pJp$ intersects $G(pAp)$, and hence $pJp=pAp$. Thus $p\in J$.
\item
Suppose $\ph$ is a character. Let $u(x)=x$. Suppose $\ph(u)=a$. If $a\in K$, then $\ph(u-a)=0$, contradicting the fact that $\rc{u-a}\in R(K)$. Hence $a\nin K$. For any $f(x)=\prod (x-a_i)\prod\rc{x-b_i}$ we have 
\[
\ph(f)=\prod (a-a_i)\prod\rc{a-b_i}=f(a).
\]
Since $\ph(f)=f(a)$ for any rational function $f$ and $\te_x$ is continuous, we get $\ph(f)=f(a)$ for any $f\in R(K)$, i.e., $\ph=\de_a$.
\item 
Let $f_n\in A$ and $f_n\to f\in C(\De)$. Let $g_n\in A(\De)$ with $g_n|_{\mathbb T}=f_n|_{\mathbb T}$. 

By the {\it maximum modulus principle}, $\ve{f_n-f_m}_{\mathbb T}=\ve{g_n-g_m}_{\mathbb T}=\ve{g_m-g_n}_{\De}$ so $g_n$ is Cauchy in $A(\De)$, and $g_n\to g$. So $g|_{\mathbb T}=f|_{\mathbb T}$. 

%more than contained, complemented
Define $\pi:A\to A(\De)\subeq A$ by letting $\pi(f)$ be the unique $g\in A(\De)$ with $g|_{\mathbb T}=f|_{\mathbb T}$. (It is unique by the maximum modulus principle.) If $f\in A(\De)$, then $\pi(f)=f$. $\pi$ is a projection and algebra homomorphism.

Now 
\[
A=A(\De)\opl B, \qquad B:=\set{f\in C(\De)}{f|_{\pi}=0}.
\]
For all $z\in \De$, $\de_z\circ \pi\in \Phi_A$, then $\de_z\circ \pi\ne \de_z$ if $z\nin \mathbb T$.
For all $z\in \De$, $\de_z\in \Phi_A$. 
We check there are no others. For $\ph\in \Phi_A$, 
\[
B\opl \C1=\set{f\in C(\De)}{f|_{\pi}\text{ is constant}}\cong C(\bS^2).
\]
%constant on boundary, identify it to single point. 
Thus $\ph|_{B\opl \C 1}=\de_z$, and $\ph|_{A(\De)}=\de_w$. If $z\in \mathbb T$, then $\ph=\de_w\circ \pi$. What if $z\ne w, z\nin \mathbb T$. This can't happen, because letting $f\in B$ with $f(z)=1$, 
\bal
\ph((u-f)^2)&=\ph(u^2-2uf+f^2)=w^2-2zf(z)+f^2(z)=w^2-2z+1\\
\ph(u-f)^2&=(w-1)^2=w^2-2w+1.
\end{align*}
Hence $w=z$.

We get $\De\sqcup \De \Ph_A$. 
Onto, continuous (weak): valuation at $f$. Agree on the boundary so it factors through
\[
\ctri{\De\sqcup \De}{\Phi_A}{\De\sqcup \De/\pi}{\te}{}{}{q}{}{\ol{\te}}.
\]
Continuous bijection from compact to Hausdorff. 

$\Phi_A$ consists of 2 discs glued together at the edge, i.e., a sphere.
%restrict to $B$ along with constant functions?

\item Let $x=\smatt2102$ and $y=\smatt 0220$. We have $r(x)=r(y)=2$, $r(x)r(y)=r(x)+r(y)=4$. But $xy=\smatt 2440$ and $r(xy)=1+\sqrt{17}>4$.

Easier: Let $x=\smatt 0100$ and $y=\smatt 0010$. Then $r(x)=r(y)=0$. But the sum is $\smatt 0110$ and the product is $\smatt 1000$, with $r(x+y)=r(xy)=1$.
\item Suppose $\la\ne 0$ and $\la\nin \si(xy)$. Then $\la-xy$ is invertible, we have to show $\la-yx$ is likewise invertible. Let $z=(\la-xy)^{-1}$. $(\la-yx)y=y(\la-xy)$. Take $y$ out on other side. $(\la-yx)yz=y$. $(\la-yx)yzx=yx-\la+\la$.
We get
\[
(\la -yx)(yzx+1)=\la\ne 0\implies (\la-yx)^{-1}=\rc{\la}(yzx+1).
\] 
so we're done.

If $xy-yx=\la 1$ for $\la \ne 0$, then $\si(xy)=\si(yx+\la 1)=\si(yx)+\la$. This contradicts the first part. %(not quite...). 
\item Let $C_n:=\set{x}{x^n=0}$. Because $A$ is nilpotent,
\[
\bigcup_{n\in \N} C_n=A.
\]
However, the Baire category theorem says that a complete metric space is not a countable union of nowhere dense sets. Hence $C_n$ (since it is closed) has nonempty interior for some $n$. %Since $C_n$ is a subspace, $C_n=A$.
There exists $x_0\in A$ such that $x_0+rB_A\subeq C_n$. In the commutative case we are done, because if $x\in C_n$ then $\la x\in C_n$, and if $xy=yx$ then $x+y\in C_{2n}$ (expand by the binomial theorem). Hence $C_{2n}=\spn C_n$, and hence $C_{2n}=A$.
%\rc r ((x_0+rx)-x_0)

Take the continuous path $p(t)=(x_0+t(x-x_0))^N$, there exists $\de$ such that for all $t$, $|t|<\de$, $p(t)=0$. Then $p\equiv 0$, $p(1)=x^N=0$. To apply this complex result to Banach algebras, apply a functional and use Hahn-Banach.
\item 
\begin{enumerate}
\item
This definition makes sense because $\ve{\fc{x^n}{n!}}\le \fc{\ve{x}^n}{n!}$, so the sum converges absolutely. 
\item
When $x,y$ commute, multiply $e^xe^y$ out, rearrange terms (legal since the series converge absolutely), commute, and collect terms using the binomial theorem to see that it equals $e^{x+y}$.
\item 
$\si(e^x)=e^{\si(x)}$ follows from holomorphic functional calculus applied to $e^x$. (Recall that if we take $C$ to be the maximual commutative subalgebra containing $x$, then the spectrum over $C$ is the same; we proved this using characters.) %\ph(e^x) 
\item
First note $e^x\in G_0$ as $t\mapsto e^{tx},0\le t\le 1$ is a path from 1 to $e^x$; $G_0$ is closed under multiplication since if $g,h\in G_0$, then there is a path $\ga(t)$ from 1 to $g$, and $\ga(t)h$ is a path from $h$ to $gh$. Hence $\an{e^x:x\in A}\subeq G_0$. 

Conversely, note $e^x$ is invertible on a neighborhood of identity, since its inverse is
\[
\ln(x-1)=x-\rc 2x^2+\rc 3x^3-\cdots 
\]

For $y\in H$ such that $\ve{y-z}<\rc{\ve{y^{-1}}}$, $\ve{zy^{-1}-1}<1$, spectrum contained in ball of radius 1, but $\si(zy^{-1})\subeq B(1,1)=\set{z\in \C}{|z-1|<1}$. Use 10(ii). (Analytic branch of log.) $zy^{-1}=e^x$, 
\[z=e^xy.\]  
Then $y\in H\iff z\in H$.
$G\bs H$ open is the same argument. 
%$\rc{y^{-1}}$. 
Then $H=\set{e^x}{x\in A}$ is connected and both open and closed, and so must be $G_0$.
%$\set{e^x}{x\in A}\subeq G_0$ since $
\end{enumerate}
\item We have
\bal
\Te_x(f)\Te_x(g)&=\prc{2\pi i}^2 \int_{\Ga}f(z)(z1-x)^{-1} \,dz\int_{\Ga'}g(u)(u1-z)^{-1}\,du\\
&=\prc{2\pi i}^2\int_{\Ga}\int_{\Ga'} \fc{f(z)g(w)}{z-w}(w1-x)^{-1}\,dw\,dz-\prc{2\pi i}^2\int_{\Ga}\int_{\Ga'} \fc{f(z)g(w)}{z-w}(z1-x)^{-1}\,dw\,dz\\
&=\int_{\Ga'}\pa{\int_{\Ga}\fc{f(z)}{z-w}\,dz} g(w)(w1-x)^{-1}
+\rc{2\pi i}\int_{\Ga}f(z)g(z)(z1-x)^{-1}\,dz. 
%analytic inside, 0 by cauchy
\end{align*}
using (in the second line)
\[
\fc{(z1-w)^{-1}(z-w)(w1-x)^{-1}}{z-w}=\fc{(w1-x)^{-1}-(z1-x)^{-1}}{z-w}.
\]
\item  
\begin{enumerate}
\item
Suppose $\si(x)=U\cup V$. Consider $\Te_x(1_U)$, we have $p^2=p$. Then $\si(p)=\{0,1\}$. 

nontrivial invariant subspaces.

%commutes with operator
Invariant subspace problem.
\item Take a branch of the logarithm $e^{\ln z}=z$, $z\in \C\bs(-\iy,0]$. $u=\Te_x(\ln)$ $\sum_{n=0}^{\mu}\fc{L(z)^n}{n!}\to z$. $\sum_{n=0}^{\mu} \fc{y^n}{n!}\to x$.

Power series converge locally uniformly.

\end{enumerate}
\item No. %Take a sequence of compact sets. %On each branch cut log. Approx by polynomial. 
On the unit disc, take the line at $\rc n$, take the function that is 0 $<\rc n$, 1 on $>\rc{2n}$, $P_n$ approximates within $\rc n$ on these compact sets. $P_n$ converges to the function 0 $\le 0$ and 1 $>0$. (Runge's Theorem)
%blow up at roots of unity.
\item %12
We have
\[
\ve{f(\la)}\le Kr(e^{-\la x}ye^{-\la x})=K.
\]
(amplification!)
Liouville: Bounded, so constant. Putting $\la=0$, constant must be $y$. $ye^{\la x}=e^{\la x}y$ for all $x$, so $yx=xy$. 
\item Take a closed ball $\ol B(z,r)\subeq U$, $\ve{f}=\sup_{z\in B(z,r)}|f(z)|$.
\item %The first part is standard: first look at the action on $u(x)=x$; suppose $\ph\in \Phi$ and $\ph(u)=a$. It follows that $\ph(p)=p(a)$ for all polynomials.
\begin{enumerate}
\item
$C(\R)$, $\Phi=\set{\de_x}{x\in \R}$, $\ph\in \Ph$. $a=\ph(u)$. 

For any $f\in C(\R)$, $\ph(f)\in \im f$. Else $g:=\ph(f)1-f$ is invertible, contradiction because $\ph(g)=0$. %$\ph(f)=f(b)$.

cf. restriction of analytic functions to boundary.
$g=(f-\ph(f))^2+(u-a)^2$. $\ph(g)=0$, so $g$ is not invertible. There exists $x$ such that $g(x)=0$.  In other words, $x=a$. $f(a)=\ph(f)$, so we're done.
%algebra!
%here we have all functions
\item
$A=C(\R)$. $(A,\ved)$, $B$ the completion. $\Phi_B=\Phi_C$. %set of evaluations.
We can map $\Phi_B=\Phi_C\to \R$ by $\ph:\mapsto \ph(u)$. This is a homeomorphism onto its image. Compact space, so compact image.
$\Phi_c=\set{\de_x}{x\in K}$.
\item
%nonvanishing on these points, but can elsewhere
Take a function $g$ that is 1 on $K$, and 0 away from $K$. Then $g$ is invertible in $B$. Now take any function $f$ which doesn't vanish but vanishs on $\Supp(g)$. Then $fg=0$, so $f=0$, contradiction.
%continuous extends uniformly, gives character.
\end{enumerate}
\item %15
Like multiplying out.
\begin{enumerate}
\item Use $D^n(ab)=\sum\binom nk D^k(a)D^{n-k}(b)$. 
\[
\sum_n\rc{n!}D^n(ab)=\sum_n\rc{n!} \sum \binom nk D^k(a)D^{n-k}(b)=\sum_k \rc{k!}D^k(a)\sum_{l}\rc{l!}D^{l}(b).
\]
(Careful: $e^D(a)\ne e^{D(a)}$.
$e^D(ab)=e^D(a)e^D(b)$. 
Isomorphism in the linear sense. 
%Aut in alg sense.
\item %Holomorphic functional calculus.
multiple also derivation.
\item
Fix $x,\ph$, $\la\mapsto \ph(e^{\la D}(x))$ bounded by $\ve{x}$ as $\ph(e^{\la D})$ is a character. By Liouville it's constant. 
\item %Differentiating is continuous?
Assume $\ved$ is a Banach algebra norm on $C^{\iy}[0,1]$. Look at $D(f)=f'$. If this was continuous, using the previous, maps into Jacobson radical, not possible because the derivative of $x$ is 1. 

Why is it continuous? 
eval at pt char%, eval func in spec
If $f_n\to 0$, $f_n\to g$ in $\ved$, in $\ved_{\iy}$ by the Closed Graph Theorem. (useful!)

Use $\ved_{\iy}\le \ved$. $\de x\in \Phi$ for all $x\in [0,1]$.  
\end{enumerate}
\item %16
Apply 20(b).
spec radius norm form for herm op Plug in 1

$\ve{x}^2=\ve{x^*x}=r(x^*x)=\ve{x^*x}'=\ve{x}'^2$. %(Radius doesn't care about the norm.)
\item %17
\item %18
\begin{enumerate}
\item
SWAP: $(L_1L_2,R_2R_1)$. 
We have
\bal
\ve{L(a)}^2&=\ve{L(a)^*L(a)}=\ve{R(L(a)^*)a}&
\le \ve{R}\ve{L(a)}\ve{a}\\
\ve{L(a)}&\le \ve{R}\ve{a}\\
\implies \ve{L}\le \ve{R}.
\end{align*}
\end{enumerate}
We get 
\[
\ve{L}=\sup_{\ve b=1}\ve{L(b)}=\sup_{\ve{a}=\ve{b}=1}\ve{aL(b)}.
\]
\item %19
$A^+\to A\opl \C$ in the sense of question 17. $x+\la\mapsto (x+\la,\la)$. LHS new unit, RHS old unit. If $A$ unital, $M(A)=A$, $(\id,\id)$. If unital $=(L_1,R_1)$. Unit belongs to $A$ thus $M(A)$. When unital doesn't help unitize. Sitting outside, add that. 

$A$ nonunital: $A^+\to M(A)$, $x+\la\mapsto x+\la$ (the respective units) $*$-homomorphism, pull back norm.
\item %20
This is a remarkable theorem becuase algebraic conditions imply continuity.
bollobas linear analysis, uniqueness of norm result. All operators on Banach space, unique. Map between certain banach with alg condition imposed.
Automatic continuity theory.
\begin{enumerate}
\item The characters of $A$ are evaluations, so $\Phi_K\cong K$. %we're interested in the continous ones. 
$(A,\ved_1)$. complete $B$.

$L=\set{y\in K}{\de_{\eta}\text{ is continuous w.r.t.}\ved_1}=\Phi_B$. Claim: $\ol L=K$. $\ve{f}_1\ge r_B(f)=\sup_{y\in L} |f(y)|=\ve{f}$. 

If not, take $x\nin \ol L$. Compact Hausdorff space normal, put a neighborhood around it: there exists $V$ of $x$, $\ol V\cap \ol L=\phi$. Let $g=1$ on $\ol L$ and 0 on $\ol V$. $g$ is invertible in $B$ because applying the character, we get something nonzero. Take $f=1$ at $x$, $0$ on $K\bs V$, $gf=0$, $f=0$, contradiction.
\item By unitization we can assume $A,B,\te$ are unital. (In 18 and 19 we checked that $C^*$ algebras have unitization. This is not trivial.) 
We have $\te_B(\te(x))\subeq \si_A(x)$. We have
\[
\ve{\te(x)}^2=\ve{\te(x)^*\te(x)}=\ve{\te(x^*x)}
=r_B(\te(x^*x))\subeq r_A(x^*x)=\ve{x^*x}=\ve{x}^2.
%image of te may not be closed
\]
%nondec auto continuous. inj auto cont
%restrict te on that, if isom for every x, isom.
Given $x$, let $C$ be the subalgebra generated by $1, x^*x$, a $C^*$-subalgebra. $B|_C:C\to B$: if $\te|_C$ isometric then $\ve{\te(x)}^2=\ve{\te(x^*x)}=\ve{x^*x}=\ve{x}^2$. WLOG $A$ is commutative.

$\te:C(K)\to B$ is injective, unital $*$-homomorphism, we can define $\ve{f}_1=\ve{\te(f)}_1\ge \ve{f}$ by Kaplansky.
\end{enumerate}
\end{enumerate}