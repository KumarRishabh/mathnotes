\def\filepath{C:/Users/Owner/Dropbox/Math/templates}

\input{\filepath/packages_article.tex}
\input{\filepath/theorems_with_boxes.tex}
\input{\filepath/macros.tex}
\input{\filepath/formatting.tex}
%\input{\filepath/other.tex}

%\def\name{NAME}

%\input{\filepath/titlepage.tex}

\pagestyle{fancy}
%\addtolength{\headwidth}{\marginparsep} %these change header-rule width
%\addtolength{\headwidth}{\marginparwidth}
\lhead{Harmonic analysis}
\chead{} 
\rhead{} 
\lfoot{} 
\cfoot{\thepage} 
\rfoot{} 
\renewcommand{\headrulewidth}{.3pt} 
%\renewcommand{\footrulewidth}{.3pt}
\setlength\voffset{0in}
\setlength\textheight{648pt}

\begin{document}
\section{Hardy-Littlewood Maximal Function}
%Real-variable theory
Throughout we work in $\R^n$.
\begin{df}
Given a function $f$, define the \vocab{maximal function} as the maximal average of absolute value over balls centered at $f$.
\[
Uf(x)=\sup_{r>0} \rc{m(B(x,r))} \int_{B(x,r)} |f(y)|\,dy.
\]
\end{df}
\begin{thm}[Maximal inequality]\label{thm:max-ineq}
\begin{enumerate}
\item (Weak $L^1$ bound)
For all $\al>0$, 
\[
m(\set{x}{Mf(x)>\al})\precsim \rc{\al}\int |f|\,dx.
\]
This essentially says that if $f\in L^1$, then $Mf\in L^{1,\iy}$. 
\item If $f\in L^p$ and $1<p\le \iy$, then $Mf\in L^p$, and 
\[
\ve{Mf}_{L^p}\precsim \ve{f}_{L^p}.
\]
\end{enumerate}
\end{thm}
To prove this we need the covering lemma. 
\begin{lem}[Vitali covering lemma]\label{lem:covering}
Let $E$ be a nonempty measurable set, covered by balls $B_1,\ldots, B_M$. There exists a subcollection $\{B_1,\ldots, B_N\}$ that is mutually disjoint and
\[
\sum_{k=1}^N m(B_k) \ge 3^n m(E).
\]
\end{lem}
%We can choose a disjoint subcollection that gives us some control over the original collection.

Suppose we have some weird set $E$. If we can cover it by a finite collection of balls, we can find a disjoint subcollection that still tells us something about the measure of $E$.

\begin{proof}
Use the greedy method (``analysis is being greedy").
Let $B_1\in \{B_\al\}$ have the largest radius. Let $B_2$ be the ball disjoint from $B_1$ which has largest radius, and so forth.

They are mutually disjoint by construction. Suppose $B\in \{B_\al\}$ which is not one of $B_1,\ldots, B_N$. Then for some $l$, $B\cap B_\ell\ne \phi$. Suppose $B_l$ is the largest (first) ball that it intersects. Then $r(B_l)>r(B)$ because otherwise $B$ would have been chosen instead of $B_l$ at that stage. Then the dilation of $B_l$ by a factor of 3 covers $B$: $B\subeq 3B_l$. 

Then $3B_1,\ldots, 3B_N$ cover $E$, so
\[
m(E)\le \sum_{k=1}^N m(3B_k)=3^n \sum_{k=1}^N m(B_k).
\]
\end{proof}
\begin{proof}[Proof of maximal inequality~\ref{thm:max-ineq}]
Define the \vocab{uncentered maximal function} by 
\[
\wt M f(x)=\sup_{B\ni x} \rc{m(B)} \int_B|f(y)|\,dy.
\]
It's clear that $Mf(x)\le \wt Mf(x)$.

It suffices to prove the theorem for $\wt Mf$.

Set
\[
E_\al = \set{x}{\wt Mf(x)>\al}.
\]
For all $x\in E_\al$, there exists $B_x\ni x$ such that 
\[
\rc{m(B_x)}\int_{B_x} |f(y)|\,dy>\al,
\]
i.e. $m(B_x)\le \rc{\al}\int_{B_x}|f(y)|\,dy$. Let measure of $E$ can be made arbitrarily close to that of $E_\al$ because of inner regularity.)
Let  $\{B_x\}_{x\in E_\al}$ be a collection of balls 
 that cover $E_\al$. %(It suffices to look at compact subsets because $E_\al$ is measurable.) 
By compactness we can choose a finite number $B_1,\ldots, B_M$ that cover $E$. By the Covering Lemma~\ref{lem:covering} we can choose $B_1,\ldots, B_N$ mutually disjoint and $\sum_{k=1}^N m(B_k)\ge 3^{-n} m(E)$. We have an upper bound on this already:
\bal
m(E)&\le 3^n \sum_{k=1}^N m(B_k)\\
&\le \fc{3^n}{\al} \sum_{k=1}^N\int_{B_f}|f(y)|\,dy\\
&\le \fc{3^n}{\al}\int |f|\,dx\\
\implies m(E_\al)&\precsim \rc{\al} \int|f|\,dx.
\end{align*}
%Inner regular (compact)
\end{proof}
The norm is independent of the dimension.

\begin{df}
Let $f$ be measurable. The \vocab{distribution function} of $f$ is a function $\la_f(\al):[0,\iy)\to [0,\iy)$ given by
\[
\la_f(\al)=m(\{|f(x)|>\al\}).
\]
\end{df}
This gives us useful information. For example, 
\[
\int|f|^p\,dx=p\int_0^{\iy} \al^{p-1}\la_f(\al)\,dx.
\]
``Cut the cake" horizontally.
\begin{proof}[Proof of part 2 of Theorem~\ref{thm:max-ineq}]
Let $f\in L^p$, $1<p<\iy$. Then
\[
\int|\wt Mf(x)|^p\,dx=
p\int_0^{\iy} \al^{p-1}m\pa{\set{x}{\wt Mf(x)>\al}}\,dx
\le 
p\int_0^{\iy} \al^{p-1}m\pa{\set{x}{\wt Mg(x)>\fc{\al}2}}\,dx.
\]
where
\[
g(x):=\begin{cases}
|f(x)|,&|f(x)|>\fc{\al}{2}\\
0,&\text{otherwise}.
\end{cases}
\]
We show the last inequality.
The new function $g$ contains all the information we need.
We know
\bal
|f(x)|&\le \max\{g(x),\fc{\al}{2}\}\\
&\le |g(x)|+\fc{\al}2.
\end{align*}
This translates to the maximal function.
\bal
\wt Mf(x)&\le \wt Mg(x)+\fc{\al}{2}\\
\wt Mf(x)>\al&\implies \wt Mg(x)>\fc{\al}2\\
\{\wt Mf(x)>\al\}&\subeq \{\wt M g(x)>\fc{\al}2\}.
\end{align*}
%only supported where $f$ is large, so we can use the first inequality. $g\in L^1$. The only it can fail is by blowing up somewhere. But $p>1$.
We have
\bal
\int|\wt Mf(x)|^p\,dx&\precsim \int_0^{\iy} \al^{p-1}\rc{\al}\pa{\int_{\R^n}|g(x)|\,dx}\,d\al\\
&=\int_0^{\iy} \al^{p-2} \pa{\int_{\set{x}{|f(x)|>\fc{\al}2}}|f(x)|\,dx}\,d\al\\
&=\int_{\R^n} |f(x)|\pa{\int_0^{2|f(x)|}\al^{p-1}\,d\al}\,dx \precsim \int|f|^p\,dx.
\end{align*}
\end{proof}
Let's build up what we need for the Calderon-Zygmund decomposition. First we need another (more annoying) covering lemma.

We can cover the complement of a closed set with cubes, where the diameter of the cube is proportion to the distance from the cube to the set. 
\begin{lem}[Whitney decomposition]\label{lem:whitney}
Let $F$ be a nonempty closed set. There exists a sequence of almost disjoint cubes (intersecting only on a set of measure 0, i.e., their boundary) $\{Q_k\}$ such that 
\[
F^c=\bigcup Q_k,
\]
and there exists $C>0$ such that 
\[
\diam(Q_k)\le d(Q_k,F)\le C\diam(Q_k).
\]
\end{lem}
\begin{proof}
%Consider cubes 
Let $M_0$ be cubes of unit side length with vertices in $\Z^n$.

Define $M_{k}$ by bisecting each cube in $M_{k-1}$, so that $M_k$ consists of cubes with side length $2^{-k}$, vertices $2^{-k}\Z^n$. (This is the ``dyadic decomposition.")

Let $\Om=F^c$; set
%bands around the original set
\[
\Om_k = \set{x\in \R^n}{C2^{-k}\le d(x,F)\le 2C2^{-k}},
\] 
where $C$ is a constant to be chosen. Think of them as bands around the original set.

For $Q\in M_k$, include $Q\in \cal F$ if $Q\cap \Om_k\ne \phi$.
We need a lower bound of the distance of the cube to the original set $F$. Let $x\in Q$. Then 
\bal
d(x,F)&\ge C2^{-k}-\ub{\diam(Q)}{\sqrt n2^{-k}}\\
&=(C-\sqrt n) 2^{-k} \ge \sqrt n 2^{-k} =\diam(Q).
\end{align*}
where we take $C=2\sqrt n$. This gives one direction.

For the other, 
\bal
d(Q,F)&\le 2C2^{-k}&\text{since it intersects the band}\\
&=2\cdot 2\sqrt{n}2^{-k}\\
&=4d(Q).
\end{align*}
For all $Q\in \cal F$, we have
\[
\diam (Q)\le d(Q,F)\le 4\diam(Q).
\]
We also have to remove redundant cubes. Proof omitted. %any cube in collection that contains it can only be so large. Get small cube by bisecting fixed number of times; get rid of smaller cubes.

\end{proof}

\begin{thm}[Calderon-Zygmund decomposition]
\llabel{thm:c-z}
Let $f\in L^1$. Fix $\al>0$. Then there exists a decomposition $f=g+\sum_{k}b_k$ and a sequence of almost disjoint cubes $\{Q_k\}$ such that 
\begin{enumerate}
\item
(the good part is bounded) $|g(x)|\le \al$,
\item $\Supp(b_k)\subeq Q_k$, $\int_{Q_k} b_k=0$, and $\int|b_k(x)|\precsim \al\cdot m(Q_k)$,
\item $\sum m(Q_k)\precsim \rc{\al}\int |f|\,dx$.
\end{enumerate}
\end{thm}
Bounds ($\precsim$) depend only on the dimension.

This is very useful. To construct $g,b_k$, it's tempting to just cut off $f$ at $\al$. But the correct thing to do is to cut off the maximal function at $\al$. %gives us more control.

Idea: The bad set will be covered by a bunch of cubes; if you dilate the cubes by a fixed factor it intersects the good set, so you have a bound on the maximal function.
\begin{proof}
%apply whitney decomp to compl?
We cutoff where $\wt Mf(x)>\al$. Let $E_{\al}=\set{x}{\wt Mf(x)>\al}$. $E_{\al}^c$ is closed (WLOG nonempty and not all of $\R^n$). Apply Whitney decomposition~\ref{lem:whitney} to cover $E_\al$ by $\{Q_k\}$. Set
\[
g(x)=\begin{cases}
|f(x)|,&x\in E_\al^c\\
\rc{m(Q_k)}\int_{Q_k} f(y)\,dy,&x\in Q_k\\
0,&\text{elsewhere.}%edges where intersect, for consistency.
\end{cases}
\]
%The remaining piece will have 0 average.
On $E_\al^c$ we have $|f(x)|\precsim |\wt Mf(x)|\le \al$. %Lebesgue differentiation theorem, have it almost everywhere.
\begin{clm*}
We have $|g(x)|\precsim \al$ almost everywhere.
\end{clm*}
If $x\in Q_k$ (This is a set completely contained in the bad region, but we can blow it up so that it overlaps the good region.),
\bal
|g(x)|&\le \rc{m(Q_k)}\int_{Q_k}|f(y)|\,dy \\
&\le \fc{4^n}{m(Q_k^*)}\int_{Q_k^*} |f(y)|\,dy&Q_k^*:=4Q_k\\
&\precsim\al.
\end{align*}
Let 
\[
b_k(x)=\chi_{Q_k}(x)\pa{
f(x)-\rc{m(Q_k)}\int_{Q_k} f(y)\,dy
};
\]
note that $\int b_k(x)=0$; we have $f=g+\sum b_k$.
%Notice that if you integrate over $Q_k$, this will give you 0.

Now we need to estimate the $L^1$ norm of each. We find (the ``$\le \al$" comes from taking $x\in Q_k^*\cap E_\al$ and noting $\rc{m(Q_k^*)}\int_{Q_k^*}|f(y)|\,dy\le \wt Mf(x)$  by definition of $\wt Mf$)
\bal
\int |b_k(x)|\,dx&\le 2\int_{Q_k} |f(y)|\,dy\\
& \le \int_{Q_k^*}|f(y)|\,dy \\
&=m(Q_k^*)\cdot \ub{\rc{m(Q_k^*)} \int_{Q_k^*} |f(y)|\,dy
}{\le \al}\\
&\precsim \al m(Q_k^*)\precsim \al m(Q_k)\\
\sum m(Q_k)
%cover bad set 
&=m(\set{x}{\wt Mf(x)>\al})\\
&\precsim \rc{\al}\int|f|\,dx.
\end{align*}

\end{proof}
What can you do with this decomposition? You can estimate in singular integrals. This comes up in $L^p$ elliptic regularity results for Laplace's equation. There's an integral kernel that can be estimated using Calderon-Zygmund.
%get all exponents in between for interpolation. Adjoint: get between 1 and $\iy$ for free.

\blu{2-17-15}

\section{Singular integrals}

\subsection{Approximation technique}
We first discuss the Lebesgue differentiation theorem. A closely related topic is approximation to the identity. They use an approximation theorem that is very important and will occur many times in proofs about singular integrals.

\begin{thm}[Lebesgue differentiation theorem]
Assume $f\in L^1_{\text{loc}}$ (locally integrable function, i.e., integrable when restricted to any ball). 
\begin{enumerate}
\item
Then
\[
\lim_{r\to 0}\rc{m(B(r))}\int_{B(X,r)} |f(y)-f(x)|\,dy=0
\]
for almost every $x$.
\item 
\[
\lim_{r\to 0}\rc{m(B(r))}\int_{B(X,r)} f(y)\,dy\to f(x)
\]
for almost every $x$.
\end{enumerate}•
\end{thm}
\begin{proof}
\begin{enumerate}
\item
Assume $f\in C_c$ (continuous and compactly supported). Then $f$ is uniformly continuous: 
\[\forall\ep>0\qquad\exists\de,\qquad\forall |x-y|<\de,\qquad|f(x)-f(y)|<\ep.\]
Then for $r<\de$, the integral is $<\ep$.
\item Assume $f\in L^1$. We approximate it with a continuous function
\[
\forall \ep>0,\qquad \exists g\in C_c \text{ such that }\ve{f-g}_{1}<\ep.
\]
%We want to bound $|f(y)-f(x)|$ by $|g(y)-g(x)|$ but there is an error:
By the triangle inequality,
\[
|f(y)-f(x)|\le |g(y)-g(x)|+|f(y)-g(y)|+|f(x)-g(x)|.
\]
Averaging over a ball, taking the limsup, and using (1),
\bal
\limsup \rc{m(B(r))} \int_{B(x,r)} |f(y)-f(x)| 
&\le \sup_{r\to 0}
\rc{m(B(r))}\int |f(y)-g(y)|\,dy + |f(x)-g(x)|\\
&\le M(f-g)(x)+|f-g|(x)\\
m\pa{\limsup_{r\to 0} \rc{m(B(r))}\int |f(y)-f(x)|\,dy \ge \al} &\le m(M(f-g)(x)>\fc{\al}2)+m(|f-g|(x)>\fc{\al}2)\\
&\precsim \rc{\al}\ve{f-g}_1 < \fc{\ep}{\al}.
\end{align*}
where we used the weak $L^1$ bound for the maximal function, Theorem~\ref{thm:max-ineq}(1).
%by chebyshev
Now take $\ep\to 0$, so we get the LHS is 0. 
Now take $\al=\rc{n},n\to \iy$.
%countable union of sets whose measure is 0.
\end{enumerate}
\end{proof}

\begin{thm}[Approximation to identity]
Let $|K(x)|<(1+|x|)^{-n-\ep}$. Let
%R(x)$ where $R$ is radial decreasing, $R\in L^1$ (for example $R=(1+|x|)^{-n-\ep}$). Let 
\[
K_t(x)=\rc{t^n} K\pf xt 
\]
(note this is normalized so $\int K_t=\int K$). Then the following hold.
\begin{enumerate}
\item
$\sup_{t>0}|f*K_t(x)|\precsim Mf(x)$
\item 
As $t\to 0$, 
$f*K_t(x)\to f(x)\int K$ for almost all $x$.
\end{enumerate}
\end{thm}
We use the same approximation argument.
\begin{proof}
WLOG assume $f>0,K>0$.
\begin{enumerate}
\item
First we do the easier case where $K=\sum_{j=1}^N c_j 1_{B_j(0,r_j)}$. (For example, something that looks like stacked cylinders.) 
By linearity you can consider $K$ to the characteristic function of a single ball, $K=1_{B(1)}, K_t=\rc{t^n} 1_{B(t)}$. Then $f*K_t$ is the maximal function, so this follows from the maximal inequality, Theorem~\ref{thm:max-ineq}.
Then
\bal
f*K_t(x)&=\int f(x-y)K_t(y)\,dy \\
&=\rc{t^n} \int_{B(x,t)} f(y)\,dy\\
&\precsim c_n Mf(x)\ve{K}_{\iy}.
\end{align*}
Now take $\sup_{t>0}$.
\item %We can assume 
For general $k$, choose $K_j>0$, simple $K^{(j)}\nearrow K$. By the Monotone Convergence Theorem, as $j\to \iy$,
\[
c*K_t^{(j)}(x) \to  f*K_t(x)
\]
%radial and decreasing.
(We check that $c*K_t^{(j)}(x) \le c_n Mf(x)\ve{K^{(j)}}_{\iy}\le c_n Mf(x)\ve{K}_{\iy}$, $c*K_t^{(j)}(x) \le c_n Mf(x)\ve{K}_{\iy}$.) 
\item For continuous functions we have that as $t\to 0, f*K_t(x)\to f(x)\int  K$ for almost all $x$. Now approximate $L^1$ functions by continuous functions.
\end{enumerate}
\end{proof}

\subsection{Singular integrals}

\begin{df}
\vocab{Singular integrals} are integrals of the form
\[
Tf(x)=\int K(x,y) f(y)\,dy
\]
satisfying the following.
\begin{enumerate}
\item
$T$ is bounded on $L^q$ for some $q>1$.
\item
(Regularity assumption $K$)
For some $c>1$, $\int_{|x-y|>c|y-y'|} |K(x,y)-K(x,y')|\,dx\le c$.\footnote{Commonly $K(x,y)\sim \rc{|x-y|^n}$; then $DK(x)\sim \rc{|x|^{n+1}}$. Then this condition is $\int_{C|y-y'|}^{\iy} \fc{|y-y'|}{r^{n+1}} r^{n-1}\,dr\le C$.
It's like an $L^1$ bound on the gradient of $K$.
``Away from the diagonal, you have good bounds." Ex. This shows up as the Newtonian potential when solving Laplace's equation.}
\item If $f\in L^q$ is compactly supported, then $Tf(x)=\int K(x,y)f(y)\,dy$ is absolutely convergent.
\end{enumerate}
%the usual kernel, actually 
\end{df}

\begin{pr}
The following are true. 
\begin{enumerate}
\item
$T$ is weak $(1,1)$:
\[
\forall f\in L^1,\al>0, \qquad \inf (Tf>\al) \precsim \rc{\al}\ve{f}_1.
\]
\item
For all $1<p<q$, $T$ is strong $(p,p)$ (i.e., $\ve{Tf}_p\precsim_{p,n}\ve{f}_p$).
\end{enumerate}•
\end{pr}
We prove the first claim using Calderon-Zygmund, and the second claim using the first and Marcikiewicz interpolation.
\begin{proof}
Use the Calderon-Zygmund decomposition~\ref{thm:c-z} to get
\bal
f&=g+\ub{\sum_k b_k}b\\
g&\precsim \al\\
\Supp b_k &\subeq Q_k,\qquad \int_{Q_k}|b_k|\precsim \al m(Q_k)\\
\int_{Q_k} b_k&=0\\
\sum_k m(Q_k)&\precsim \rc{\al}\ve{f}_1.
\end{align*}
Let $F=\R^n\bs \bigcup_k Q_k$. 
We bound
\[
m(Tf>\al) \le m\pa{Tg>\fc{\al}2}+m\pa{Tb>\fc{\al}2}.
\]
\begin{enumerate}
\item
For $m(Tg>\fc{\al}2)$ we use the $L^q$ bound for $T$. First we obtain a $L^q$ bound for $g$. To do this we use our $L^1$ bound for $g$.
\bal
\int|g|&=\int_F|g| + \sum_k \int_{Q_k}|g|\\
&\le \int_F |f| + \al \sum_k m(Q_k)\\
&\precsim \ve{f}_1\\
\int |g|^q&\precsim \al^{q-1}\ve{f}_1\\
\implies \int |Tg|^q &\precsim \al^{q-1} \ve{f}_{1}\\
\implies m\pa{Tg>\fc{\al}2}&\precsim \fc{\al^{q-1}\ve{f}_1}{\al^q}.
\end{align*}
the last step by Chebyshev.
%weak L^1 bounded by L^1
\item 
For the bad part, we have to get our hands dirty. 
Let the cube $cQ_k$ have the same center as $Q_k$ but be dilated by $c$. Let 
\[
F'=\R^n \bs \bigcup_k cQ_k.
\]
First look at the part bounded away from the cubes.
\begin{enumerate}
\item
Claim: 
\[
m(\{Tb > \fc{\al}2\}\cap F')\precsim \rc{\al}\ve{f}_1.
\]
Proof: 
\bal
|Tb(x)|&=|\sum_k Tb_k (x)|\\
&=\ab{\sum_k \int_{Q_k} K(x,y) b_k(y)\,dy}\\
%in hope that difference is small
\text{(fix $y_k\in Q_k$)}\quad & \le \sum_k \int_{Q_k} |K(x,y)-K(x,y_k)| |b_k(y)|\,dy.\\
&\quad\text{(interchange integrals)}\\
\int_{F'}|Tb(x)|&\le \sum_k \int_{Q_k} |b_k(y)|\,dy \int_{F'} |K(x,y)-K(x,y_k)|\,dx\\
&\quad \text{(used zero-mean condition, take absolute value last)}\\
&\quad x\nin c_n (1+c)Q_k, \quad y\in Q_k\\
\implies |x-y|& \succsim \diam(Q_k) \succsim |y-y_k|\\
%apply bund on regularity of $k$
|x-y| &\precsim \sum_k \int_{Q_k} |b_k(y)|\,dy \\
& \precsim \al \sum_k m(Q_k)\precsim \ve{f}_1.\\
%Bound L^1 of Tb by L^1 of f, implies weak.
m(\{Tb>\fc{\al}2\}\cap \bigcup_k  cQ_k) &\precsim_k m(Q_k)\precsim \rc{\al}\ve{f}_1.
\end{align*}
Put all these together. 
%integrate over $F'$.
%interchange the integral
\item We have to interpolate between a weak $(1,1)$ and strong $(p,p)$ inequality.
We use the identity (distribution formula for $L_p$ norm)
\beq{eq:cutcake}
\int |f|^p=p\int _0^{\iy} \al^{p-1} m(|Tf|>\al)\,d\al.
\eeq
We decompose $f$ into 1 pieces,
\[
f=f_\al+f^\al,\qquad f_\al=1_{|f|\le \al}f,\qquad f^\al = 1_{|f|>\al}f
\]
Use the $L^p$ bound to the first and the weak $L^1$ bound to the second. We have the inequalities
\bal
m(|g|>\al)&\precsim \fc{\ve{g}_1}{\al}\\
m(|g|>\al)& \precsim \fc{\ve{g}_1^p p}{\al^p}
\end{align*}
useful when $\al\precsim |g|$, $\al \succsim |g|$, respectively.
%useful when $\al\precsim |f|$.
%close to 1 cannot get large domain for integration.
%huge increase in small part.
Applying these 2 inequalities to the 2 parts,
\bal
m(|Tf|>\al) & \le m(|Tf_{\al}|>\fc{\al}2) + m(|Tf^{\al}|>\fc{\al}2)\\
&\precsim \rc{\al^q} \ve{Tf_\al}_q^q +\rc{\al}\ve{f^{\al}}_1\\%know L^q, want L^p.
&\precsim \rc{\al^q}\int_{|f|\le \al}|f|^q +\rc{\al}\int_{|f|>\al}|f|.
\end{align*}
Finally, plugging into~\eqref{eq:cutcake}
\bal
\int |Tf|^p & \precsim_p \int_0^{\iy} \pa{
\fc{\al^{p-1}}{\al^q}\int_{|f|\le \al} |f|^q
+ \fc{\al^{p-1}}{\al} \int_{|f|>\al} |f|
}\,d\al\\
& = \int\int_{|f|}^{\iy} \al^{p-q-1} |f|^q \,d\al\,dx + 
\int\int_{0}^{|f|} \al^{p-2} |f|\,d\al\\
&\precsim \int |f|^{p-q}|f|^q+\int |f|^{p-1}|f|\,dx\\
&\precsim \ve{f}_p^p.
\end{align*}
Bootstrap by getting $L^q$ norm. As long as you have 1 $L^q$ norm you can get all these results. How to get the $L^q$ norm of $T$ in the first case? Use Fourier analytic methods.
%translation invariant.

%1-2, then 2-\iy by duality.
Assume $K(x,y)=K(x-y)$. We have $\wh{f*K}=\wh f*\wh K$ so 
\[
\ve{f*K}_2=\ve{\wh f\wh K}_2\le \ve{f}_2\ve{\wh K}_{\iy}.
\]
If $\wh K$ is bounded then $Tf = f*K$ is bounded on $L^2$.
Given the inequality for $1<p<2$, we get the inequality for $2<p<\iy$ by a duality argument. 
Let $2<p<\iy$ and $p^*$ be such that $\rc p +\rc{p^*}=1$. Then
\bal
\ve{Tf}_{p}&=\sup_{\ve{g}_{p^*}=1} \int Tf(x)g(x)\,dx \\
&=\sup \iint f(y)K(x,y) g(x)\,dy\,dx\\
&=\sup \int f(y) T^* g(y)\\
&\le \sup_{\ve{g}_{p^*}=1} \ve{f}_p\ve{g}_{p^*}.
\end{align*}
%usually forced at 1 $\iy$.
\end{enumerate}

\end{enumerate}

\end{proof}
\section{Ch 2 continued}

\fixme{Everything after this is messy}
We'll discuss:
\begin{enumerate}
\item
Atomic decomposition
\item Averge of general collections of balls.
\item Singular approximations to the identity.
\end{enumerate}
\begin{df}
Define $F:\R_+^{n+1}\to \R$, 
\[
F^*(x) = \sup_{|x-y|<t}|F(y,t)|.
\]
\end{df}

Let $\ph=|B(0,r)|^{-1}$. Define $F(x,t)=f*\ph_t$, $\ph_t(x)=t^{-n}\ph\pf nt$. Then $F^*$ is the uncentered maximal function $\cal F f$.
Define $F^*(x)=\sup_{|y-x|<t}|F(y,t)|$. $F\in N$, tent space, if $F^*\in L^1(\R^n)$. $\ve{F}_N:=\ve{F^*}_{L^1}$.

split tent function into atoms which are easy to understand individually.

The closer a point is to the boundary, the smaller.

$T(0)=\bigcup_{x\in O}T(B(x,d(x,O^c)))$.

\subsection{Atomic decomposition}
\begin{df}
An \textbf{atom} is a function supported on $T(B)$ for some $B\subeq \R^n$ such that $|a|\le |B|^{-1}$.
%a^* supported on the ball, so the tent norm of $a$ satisfies $\ve{a^*}_N\le 1$.
\end{df}
In particular, $\ve{a^*}_N\le 1$.
\begin{thm}
If $F\in N$, then there exist atoms $\{a_k\}$ and $\{\la_k\}$ (scalars) such that 
\[
F=\sum_k \la_ka_k\in N
\]
and $\sum_k \la_k\le C\ve{F}_N$.
\end{thm}

Let $\cal B$ be a collection of balls in $\R^n$. Define the maximal function 
\[
(M_{\cal B}f)(x) = \sup_{B\in \cal B} \rc{|B|} \int_B |f(x-y)|\,dy.
\]
Does
\[
\rc{|B|}\int_B f(x-y)\,dy \to f(x)\qquad \text{a.e.}?
\]
%If I scale each ball by $B$, it will contain the origin.
E.g. if $|C(B)|\le k\diam(B)$, then $|M_{\cal B}(f)|\le k^nM(f)$.

The boundedness of the ordinary maximal function implies the boundedness of this maximal functions.

Consider the larger collection of balls 
\[
\ol{\cal B} = \set{B'}{B'\supeq B\text{ for some }B\in \cal B}.
\]
This is a  kind of completion process. If you can bound the maximal operator for the original collection, then you can bound it for the completed collection. Let
\[
\cal B(r)=\bigcup_{B\in \ol{\cal B},\text{radius}(B)=r} B.
\]

\begin{thm}
If $|\cal B(r)|\le cr^n$, then $M_{\cal B}$ is bounded on $L^p$ for all $1<p\le \iy$ and also on $L^{1,\iy}$.

Conversely, if $M_{\cal B}$ is bounded on some $L^{p_0}$, then $|\cal B(r)|\le cr^n$.
\end{thm}
%Union of $n$ balls that covers $r$, then this is true. Conversely, if you have this inequality, then you have a fixed $n$ so that you have balls of radius $O(r)$, such that it has a cover of that many balls of that radius. 

\begin{proof}
$\ve{M\ol{\cal B}}_{p_0}\le A\ve{M\cal B}_{p_0}$.
Define $m_B=|B|^{-1} \chi_B$. Let $B_{2r}$ be the bll of $2\text{radius}(\ol B)$ centered at 0. Claim: $2^{-n}m_{\ol B}\le m_{B_{2r}}*m_B$.

Then $M_{\ol{\cal B}}f\le M(M_{\cal B}f)$.
For $x\in \ol B$, $B(x,2r)\supeq \ol{B}\supeq B$.
\[
(m_{B_{2r}}* m_B) (x)=\rc{|B_{2r}|}=2^{-n} \rc{|B_r|} = 2^{-n}\rc{|\ol B|}.
\]
Using the boundedness of the ordinary maximal operator $M$,
\[
\ve{M_{\ol{\cal B}}f}_{p_0}\le A\ve{M_{\cal B}f}_B.
\]
Test function $\chi_{B_{2r}(0)}$. Claim: $\cal B(r)\subeq \set{x}{(M_{\cal B}f)(x)\ge 1}$. %If know have bound on $L^p$ norm 
This implies that $\vol(\ol{\cal B}(r))\le A\vol(B_{2r})$.

To show the converse, study the tent space. %If we want maximal functions with respect to general collection of balls
Given $F:\R_+^{n+1}\to \R$, define $F_{\ol{\cal B}}^*(x)=\sup_{B(y,t)\in \ol{\cal B}}|F(x-y,t)|$.

Claim: $\int_{\R^n}F_{\ol{\cal B}^*}(x)\,dx\le C\int_{\R^n} F^*(x) \,dx$.

Use atomic decomposition: Let $a$ be an atom supported on $B(0,r)$, assume $a_{\cal B}^*(x)\ne 0$. Then for some $(y,t)$, $B(y,t)\in \ol{\cal B}$, then $|x-y|\le r-t$. This implies $x\in B(y,r)\in \ol{\cal B}$. Thus $x\in \ol{B(r)}$.
%true for general F. To weak apply to char func of distribution function.

\[
\ve{a_{\cal B}}_{L^1} \le |B(0,r)|^{-1}\int_{\cal B(r)}\,dx\le C.
\]
We obtain that $M_{\ol{\cal B}}$ is bounded on $L^p$ and $L^{1,\iy}$ for $1< p\le \iy$. 
$F(x,t)=f*\ph_t$, $\ph_t$ distribution function.
\end{proof}

Recall the setup. Suppose $\ph\ge 0$ is bounded and integrable $\int \ph=1$. If $\ph$ has sufficiently uniform decay at $\iy$, then 
\[
(f*\ph_t)(x)=\int_{\R^n}f(x-ty)\ph(y)\,dy\to f(x)\text{ a.e.}
\]
where $\ph_t(x)=t^{-n}\ph\pf xt$. This is true if $|\ph(x)|\le (1+|x|)^{-n-\ep}$, $\ep>0$.
%closely connected to the study of the maximal operator.

We weaken the conditions; then the same techniques as before cannot be applied. 
Claim: If $\ph(rx)$ is decreasing in $r$, for all $0\ne x\in \R^n$ ($\ph$ bounded, $\ge 0$, integrable), then 
\[
(M\ph)(x)=\sup_{t>0}|(f*\ph_t)(x)|
\]
is bounded on $L^p$ for $1<p<\iy$. 
%a.e., need weak $1,1$ bound. It's open whether there is a weak $1,1$ bound.

The proof uses the method of rotations.
\begin{proof}
We want to bound (uniformly in $t>0$)
\[
\int_{|\xi|=1}\int_0^{\iy} f(x-rs) \ph_t(rs)r^{n-1}\,dr\,d\xi
\]
We claim that this is $\le \int_{|\xi|=1} (M^{\xi}f)(x)\int_0^{\iy} \ph_t(-s) r^{n-1}\,dr\,d\xi$.
Here $(M^{(\xi)})=\sup_{r>0}\rc{2r}\int_{-r}^r |f(x-t\xi)|\,dt$ is the maximal function in 1 direction.
Claim: $\ve{M^{(\xi)}f}_p\le A_p \ve{f}_p$
precompose with element of orthogonal group, can assume in $x_1$ direction, integrate with respet to $x_1$, then other coordinates.

Firstly, 
\[
\int_{|\xi|=1}\int_0^{\iy} f(x-r\xi)\ph(r\xi)|r|^{n-1} \,dr\,d\xi.
\]
We want to apply the result, majorizing $\ph(r\xi)|r|^{n-1}$ by a radially decreasing function
\[
\ph(r\xi) |r|^{n-1}\le \int_{|r|}^{\iy}s^{n-1} \,d\ph_S(rs)=\psi(r)
\]
by integration by parts.
We calculate
\[
\iiiy -\psi(r)\,dr=\int_0^{\iy} d_r\psi(r)\le A\int \ph.
\]
Then 
\[
\int_{|\xi|=1}\int_0^{\iy} f(x-r\xi)\psi(r)|r|^{n-1}\,dr\,d\xi\le \int_{|\xi|=1}(M^{(\xi)}f)(x)\iiy \ph(r\xi)r^{n-1}\,dr\,d\xi.
\]
Use the Minkowski inequality and the claim bounding $\ve{M^{(\xi)}(f)}_p$. Then $\ve{M\ph f}_p\le A\ve{f}_p$ for $1<p\le \iy$.
\end{proof}

Let $\eta$ be a Dini modulus of continuity. This means that
\begin{enumerate}
\item
\[
\int_0^1 \fc{\eta(s)}{s}\,ds<\iy,
\]
\item $\eta(0)=0$,
\item $\eta:[0,1]\to \R$,
\item $\eta$ is non-decreasing.
\end{enumerate}
If $\eta(s)=O(s^{\ep})$ for $\ep>0$ then $\eta$ is a Dini modulus of continuity.
(The inverse function of $e^{-\rc{x^2}}$ is not a Dini modulus of continuity.)
Without this condition, the following question is open. 
\begin{thm}
If $\ph:\R^n\to \R$ and if 
\[
\int_{\R^n}|\ph(x-y)-\ph(x)|\,dy\le \eta(|y|)
\]
and if 
\[
\int_{|x|\ge R} |\ph(x)|\le \eta(R),
\]
 then $M\ph$ is bounded $L^1\to L^{1,\iy}$.
\end{thm}
If $\ph$ is compactly supported, this is satisfied.
\begin{proof}
If $\ph(rx)$ is decreasing in $r$ for $x\ne 0$, then 
\[
M_\ph f=\sup_{t>0}|f*\ph_t| \le 2 \sup |f*\ph_{2j}|
\]
%radially decreasing
%2^j\ge t.
Claim: there exists $A$ such that
\bal
\int_{|x|\ge 2|y|} \sup|\ph_{2j}(x-y)-\ph_{2j}(x)|&\le A\\
&\le \ub{\sum_{2^j<|y|} \int_{|x|\ge 2|y|} |\ph_{2^j}(x-y)-\ph_{2^j}(x)|\,dx}{(1)} + \ub{\sum_{|y|\le 2j}\int_{|x|\ge 2|y|}|\ph_{2^j}(x-y)-\ph_{2^j}(x)|\,dx}{(2)}\\
(1)&\le 2\sum_{2^j<|y|}\int_{|x|\ge 2|y|}|\ph(x-2^{-j}y)-\ph(x)|\,dx\\
&\le 2\sum_{2^j<|y|}\eta(2^{-j}|y|)\\
&\le 4\sum_{2^j<|y|}\fc{\eta(2^{-j}|y|)}{2^{-j}|y|}2^{-j-1}|y|\\
&\le 4\int_0^1 \fc{\eta(s)}{s}\,ds<\iy\\
(2)&\le 4 \sum_{2^j<|y|}\int_{|x|\ge 2|y|}|\ph(2^{-j}|y|)|\,dx\\
&\le 4 \sum_{2^j<|y|} \eta(2^{-j}|y|)\\
&\le 4 \int_0^1 \fc{\eta(s)}{s}\,ds.
\end{align*}
%(1) using a change of variables, Riemann sum.

If $K=\{\ph_{2^j}\}_{j\in\Z}$, we want a bound $f\to f*K$. Using the Banach space version of boundedness of singular integrals, the result follows.
\end{proof}


%\chapter{Introduction}


\section{Normed spaces}

\begin{df}
A \textbf{normed space} is a pair $(X,\ve{\cdot})$ where $X$ is a real or complex vector space and $\ved$ is a norm on $X$. Most of the time the choice of scalar field makes little difference; for convenience we'll use real scalars. A norm induces a metric: $d(x,y)=\ve{x-y}$. This induces a topology on $X$, called the \textbf{norm topology}. A \textbf{Banach space} is a complete normed space.
\end{df}
\begin{ex}
\begin{enumerate}
\item (sequences)
For $1\le p<\iy$, we have $\ell_p=\set{(x_n)\text{ scalar sequence}}{\sum_{n=1}^{\iy}|x_n|^p<\iy}$ with norm $\ve{x}_p=\pa{\sum_{n=1}^{\iy}|x_n|^p}^{\rc p}$. (Minkowski's inequality says that if $x,y\in \ell_p$ then $x+y\in \ell_p$, so $\ve{x+y}_p\le \ve{x}_p+\ve{y}_p$. Then $\ell_p$ is a Banach space.
\item (convergent sequences)
$\ell_{\iy}=\set{(x_n)\text{ scalar sequence}}{(x_n)\text{ is bounded}}$ with $\ve{x}_{\iy}=\sup_{n\in N}|x_n|$. Then $\ell_{\iy}$ is a Banach space.

$c_{00}=\set{(x_n)\text{ scalar sequence}}{\exists N\forall n>N, x_n=0}$. Let $e_n=(0,0,\ldots, 0,\ub{1}{n},0,\ldots)$; then $c_{00}=\spn\set{e_n}{n\in \N}$. Note that $c_{00}$ is a subspace of $\ell_{\iy}$ but it's not closed: In $\ell_p, 1\le p<\iy$, $\ell_p=\ol{\spn}\set{e_n}{n\in \N}$. 

$c_0=\set{(x_n)\in \ell_{\iy}}{\lim_{n\to \iy} x_n=0}$ is a closed subspace of $\ell_{\iy}$, $c_0=\ol{\spn}\set{e_n}{n\in \N}$ in $\ell_{\iy}$.

$c=\set{(x_n)\in \ell_{\iy}}{\lim_{n\to \iy} x_n \text{ exists}}$ is a closed subspace of $\ell_{\iy}$. $c_0$ and $c$ are Banach spaces.
\item (Euclidean space)
$\ell_p^n=(\R^n,\ved_p), 1\le p\le \iy$.
\item 
$K$ is any set, $\ell_{\iy}(K)=\set{f:K\to \R}{f\text{ is bounded}}$ with norm $\ve{f}_{\iy}=\sup_{x\in K}|f(x)|$. This is a Banach space, e.g. $\ell_{\iy}=\ell_\iy(\N)$.
\item 
$K$ compact topological space $C(K)=\set{f\in \ell_{\iy}(K)}{f\text{ continuous}}=\set{f:K\to \R}{f\text{ is continuous}}$. $C(K)$ is a closed subspace of $\ell_{\iy}(K)$ because any uniform limit of continuous functions is continuous, and hence it's a Banach space, e.g. $C[0,1]$.

We'll write $C^{\R}(K)$ and $C^{\C}(K)$ for the real and complex versions of $C(K)$, respectively.
\item 
Let $(\Om,\Si,\mu)$ be a measure space. Then for $1\le p<\iy$,  \[L_p(\mu)=\set{f:\Om\to \R}{f\text{ is measurable}, \int_{\Om}|f|^p\,d\mu<\iy}\] with norm $\ve{f}_p=\pa{\int_{\Om}|f|^p\,d\mu}^{\rc p}$ is a Banach space (after identifying functions that are equal almost everywhere. 

When $p=\iy$, $L_{\iy}(\mu)=\set{f:\Om\to \R}{f\text{ is measurable and essentially bounded}}$. (``Essentially bounded" means that there exists a null-set $N$ such that $f$ is bounded on $\Om\bs N$.)
\[
\ve{f}_{\iy}=\ess\sup|f|=\inf_N\sup_{\Om\bs N}|f|.
\]
\item
Hilbert spaces, e.g. $\ell_2$, $L_2(\mu)$. All Hilbert spaces are isomorphic, but some different representation may be more natural.
\end{enumerate}
\end{ex}

\begin{pr}
Let $X,Y$ be normed spaces, $T:X\to Y$ linear. Then the following are equivalent.
\begin{enumerate}
\item
$T$ is continuous.
\item
$T$ is bounded: $\exists C\ge 0$, $\ve{Tx}\le C\ve{x}$ for all $x\in X$.
\end{enumerate}
\end{pr}
\begin{proof}
To think about continuity at $a$, ``translate" to 0 using linearity.
\end{proof}

\begin{df}
Let $\cal B(X,Y)=\set{T:X\to Y}{T\text{ is linear and bounded}}$. This is a normed space with the \textbf{operator norm}: $\ve{T}=\sup\set{\ve{Tx}}{\ve{x}\le 1}$. $T$ is an \textbf{isomorphism} if $T$ is a linear bijection whose inverse is also continuous. (This is equivalent to $T$ being a linear bijection and there existing $a>0,b>0$, with $a\ve{x}\le \ve{Tx}\le b\ve{x}$ for all $x\in X$.)

If there exists such $T$, we say $X,Y$ are \text{isomorphic} and we write $X\sim Y$.

If $T:X\to Y$ is a linear bijection such that $\ve{Tx}=\ve{x}$ for all $x\in X$ (i.e. $a=b=1$), then $T$ is an \textbf{isometric isomorphism} and we say $X,Y$ are isometrically isomorphic and write $X\cong Y$\footnote{Some people use $\cong$ for isomorphism. We use it to mean isometric isomorphism.}.

$T:X\to Y$ is an \textbf{isomorphic embedding} if $T:X\to TX$ is an isomorphism. We write $X\hra Y$.
\end{df}
\begin{pr}
If $Y$ is complete, then $\cal B(X,Y)$ is complete. In particular, $X^*=\cal B(X,\R)$, the space of bounded linear functionals, called the \textbf{dual space} of $X$, is always complete.
\end{pr}
\begin{ex}
\begin{enumerate}
\item
For $1<p<\iy$, then $\ell^*\cong \ell_q$ where $\rc p+\rc q=1$. The proof uses H\"older's inequality: $x=(x_n)\in \ell_p$, $y=(y_n)\in \ell_q$ then $\sum|x_ny_n|\le \ve{x}_p\ve{y}_q$. This isomorphism is $\ph:\ell_q\to \ell_p^*,y\mapsto \ph_y, \ph_y(x)=\sum x_ny_n$.)
\item
$c_0^*\cong \ell_1$, $\ell_1^*\cong \ell_{\iy}$. 
(Later we will see that $c_0$ cannot be a dual space.)
%(This gives an alternate proof of completeness.)
\item 
If $H$ is a Hilbert space then $H^*\cong H$ (Riesz Representation Theorem).
\item
If $(\Om,\Si,\mu)$ is a measure space, $1<p<\iy$, then $L_p(\mu)^*\cong L_q(\mu)$ where $\rc{p}+\rc{q}=1$.

If $\mu$ is $\si$-finite then $L_1(\mu)^*\cong L_{\iy}(\mu)$.
(Else we only have $L_{\iy}(\mu)\hra L_1(\mu)^*$.)
\end{enumerate}
\end{ex}

{\color{blue}Lecture 2}

Recall that if $V$ is a finite-dimensional vector space, then any two norms on $V$ are equivalent. Specifically, if $\ved$ and $\ved'$ are two norms on $V$, then there exist $a,b>0$ such that 
\[
a\ve{x}\le\ve{x}'\le b\ve{x}\forall x\in V.
\]
In other words, $\Id:(V,\ved)\to (V,\ved')$ is an isomorphism. 

Some consequences are the following.
\begin{cor}
\begin{enumerate}
\item
If $X,Y$ are normed spaces, $\dim X<\iy$, $T:X\to Y$ is linear, then $T$ is bounded.
\item
If $\dim X<\iy$ then $X$ is complete. 
\item If $X$ is a normed space and $E$ a subspace with $\dim E<\iy$, then $E$ is closed.
\end{enumerate}
\end{cor}
\begin{proof}
\begin{enumerate}
\item
Set $\ve{x}'=\ve{x}+\ve{Tx}$. This is a norm on $X$, so there exists $b>0$, $\ve{x}'\le b\ve{x}$ for all $x$, so $\ve{Tx}\le b\ve{x}$ for all $x\in X$. 

%If $\dim X=\dim Y<\iy$ then $X\sim Y$.
\item
By (1), $X\sim \ell_2^n$ where $n=\dim X$.
\end{enumerate}•
\end{proof}
\section{Riesz's lemma and applications}
The unit ball is compact and this characterizes finite-dimensionality. We use the following.
\begin{lem}[Riesz's Lemma]
\llabel{lem:riesz}
Let $Y$ be a proper closed subspace of a normed space $X$. Then for every $\ep>0$ there exists $x\in X$, $\ve{x}=1$, such that $d(x,Y):=\inf_{y\in Y}\ve{x-y}>1-\ep$.
%null space, don't have notion as euclidean space, suggests proof that works. Intuition about euclidean space sometimes dangerous because may not work in normed space but in this case works
\end{lem}
We'd like to take the some sort of ``perpendicular" vector to $Y$, or the vector which minimizes the distance from a point not on $Y$ to $Y$. Note this is not in general possible since $X$ may not be complete, and hence the ``$\inf$." However, we can come arbitrarily close to that $\inf$, and get an ``almost perpendicular" vector.
\begin{proof}
Pick $z\in X\bs Y$ with $Y$ proper. Since $Y$ is closed, $d(z,Y)>0$. There exists $y\in Y$ such that $\ve{z-y}<\fc{d(z,Y)}{1-\ep}$ (WLOG $\ep<1$).

Set $x=\fc{z-y}{\ve{z-y}}$. Then
\[
d(x,Y)=d\pa{\fc{z-y}{\ve{z-y}},Y}=\rc{\ve{z-y}}d(z-y,Y)=\fc{d(z,Y)}{\ve{z-y}}> 1-\ep.
\]
\end{proof}
We give two applications. First, some notation. In a metrix space $(M,d)$, write 
\[
B(x,r):=\set{y\in M}{d(x,y)\le r},x\in M,r\ge 0
\]
for the closed ball of radius $r$ at $x$. In a normed space, 
\[
B_X:=B(0,1)=\set{x\in X}{\ve{x}\le 1}, \qquad B(x,r)=x+rB_X.
\]
Also, $S_X=\set{x\in X}{\ve{x}=1}$. 
\begin{thm}
Let $X$ be a normed space. Then $\dim X<\iy$ iff $B_X$ is compact.
\end{thm}
\begin{proof}
``$\Rightarrow$" We have $X\sim \ell_2^n$ where $n=\dim X$.

``$\Leftarrow$" By compactness there exist $x_1,\ldots, x_n\in B_X$ such that $B_X\subeq \bigcup_{i=1}^n B(x_i,\rc2)$.  Let $Y=\spn\{x_1,\ldots, x_n\}$. For all $x\in B_X$ there exists $y\in Y$ with $\ve{x-y}\le \rc 2$, so $d(x,Y)\le \rc 2$. Thus there do not exist ``almost orthogonal vectors" in the sense of  Riesz's Lemma~\ref{lem:riesz}. This means $Y$ is not a proper subspace of $X$, so $X$ is finite-dimensional. %$Y$ is dense in $X$, so $X=\ol Y=Y$.
%(We can't find an orthogonal vector to $Y$ in the sense of Riesz's Lemma.)
\end{proof}
\begin{rem}
We showed the following in the proof: If $Y$ is a subspace of a normed space $X$ and there exists $0\le \de<1$ such that for all $x\in B_X$ there exists $y\in Y$ with $\ve{x-y}\le \de$, then $Y$ is dense in $X$.
\end{rem}
If we let $\de= 1$ then this statement is trivial. The remark says that if we can do a little better than 1, the trivial estimate, then we can automatically approximate $x$ with much smaller $\ep$.

\begin{thm}[Stone-Weierstrass Theorem]
Let $K$ be a compact topological space and $A$ be a subalgebra of $C^{\R}(K)$. If $A$ separates the points of $K$ (i.e., for all $x\ne y$ in $K$, there exists $f\in A$, $f(x)\ne f(y)$), and $A$ contains the constant functions, then $A$ is dense in $C^{\R}(K)$.
\end{thm}
In this case it does matter whether the field of scalars is $\R$ or $\C$.

The following proof is due to T. J. Ransford.
\begin{proof}
First we show that if $E,F$ are disjoint closed subsets of $K$, then there exists $f\in A$ such that $-\rc2\le f\le \rc2$ on $K$ and $f\le -\rc 4$ on $E$ and $f\ge \rc 4$ on $F$.

Fix $x\in E$. Then for all $y\in F$, there exists $h\in A$ such that $h(x)=0$, $h(y)>0$, $h\ge 0$ on $K$. (This is since $A$ separates points, we can shift by a constant, and square the function.) Then there is an open neighborhood of $y$ on which $h>0$.  %finitely many cover $F$ by compactness.
An easy compactness argument gives that there exists $g=g_x\in A$ with $g(x)=0$, $g>0$ on $F$, 
%strictly positive on neighborhood, going to be strictly positive on all
%rescale it so that  
$0\le g\le 1$ on $K$. 
Pick $R=R_x\in \N$ such that $g>\fc2R$ on $F$, set $U=U_x=\set{y\in K}{g(y)<\rc{2R}}$. 
%compact, attain inf

Do this for all $x\in E$.
%finitely many will cover.
Compactness gives a finite cover: there exist $x_1,\ldots, x_m$ such that $E\subeq \bigcup_{i=1}^m U_{x_i}$. To simplify notation, set $g_i=g_{x_i}$, $R_i=R_{x_i}$, $U_i=U_{x_i}$, and $i=1,\ldots, m$. For $n\in \N$,  by Bernoulli's inequality,
\begin{align*}
\text{on }U_i&&
(1-g_i^n)^{R_i^n}&\ge 1-(g_iR_i)^n>1-2^{-n}\to 1\text{ as }n\to \iy\\
\text{on }F&&
(1-g_i^n)^{R_i^n}&\le \rc{(1+g_i^n)^{R_i^n}}\le \rc{(g_iR_i)^n}<\rc{2^n}\to 0\text{ as }n\to \iy
\end{align*}
There exists $n_i\in \N$ such that $h_i=1-(1-g_i^{n_i})^{R_i^{n_i}}$ satisties
\begin{itemize}
\item
on $U_i$, $h_i\le \rc 4$
\item
on $F$, $h_i\ge \pf 34^{\rc m}$
\item
on $K$, $0\le h_i\le 1$.
\end{itemize}
Set $h=h_1h_2\cdots h_m$. Then $h\le \rc 4$ on $E$, $h\ge \fc 34$ on $F$, and $0\le h\le 1$ on $K$. Set $f=h-\rc2$.
%corresp is at most a quarter, other at most 1, so at most 1/4
Given $g\in C^{\R}(K)$, $\ve{g}_{\iy}\le 1$, set
\[
E=\set{x\in K}{g(x)\le -\rc 4},\,F=\set{x\in K}{g(x)\ge \rc 4}.
\]
Let $f\in A$ be as above. Then $\ve{f-g}\le \fc 34$, i.e., $d(g,A)\le \fc 34$. By Riesz's Lemma~\ref{lem:riesz}, $A$ is dense in $C^{\R}(K)$.
\end{proof}
%reading math is a sort of computation, info decompression
\begin{rem}
The complex version says that if $A$ is a subalgebra of $C^{\C}(K)$ that separates points of $K$ contains the constant functions, and is closed under complex conjugation ($f\in A\implies \ol f\in A$), then $A$ is dense in $C^{\C}(K)$.
\end{rem}
\section{Open mapping lemma}
We'll assume the Baire category theorem and its consequences: principle of uniform boundedness, open mapping theorem (OMT), closed graph theorem (CGT). 
\begin{df}
Let $A,B$ be subsets of a metric space $(M,d)$ and let $\de\ge 0$. Say $A$ is \textbf{$\de$-dense} in $B$ if for all $b\in B$ there exists $a\in A$ with $d(a,b)\le \de$.
\end{df}
\begin{lem}[Open mapping lemma]\llabel{lem:oml}
Let $X,Y$ be normed spaces, $X$ complete, $T\in \cal B(X,Y)$. Assume for some $M\ge 0$ and $0\le \de<1$ that $T(MB_X)$ is $\de$-dense in $B_Y$. Then $T$ is surjective. 
More precisely, 
%important: have quantitative
for all $y\in Y$ there exists $x\in X$ such that $y=Tx$ and
\[
\ve x\le \fc{M}{1-\de}\ve y,
\]
i.e.,
\[
T\pa{\fc{M}{1-\de}B_X}\supeq B_Y.
\]
Moreover, $Y$ is complete.
\end{lem}
\begin{proof}
The proof involves successive approximations. 
Let $y\in B_Y$. There exists $x_1\in MB_X$ with $\ve{y-Tx_i}\le \de$. Then $\fc{y-Tx_i}{\de}\in B_Y$. There exists $x_2\in MB_X$, with $\ve{\fc{y-Tx_i}{\de}-Tx_2}\le \de$, i.e., $\ve{y-Tx_1-\de Tx_2}\le \de^2$, and so forth. Obtain $(x_n)$ in $MB_X$ such that 
\[
\ve{y-Tx_1-\de Tx_2-\cdots -\de^{n-1} Tx_n}\le \de^n
\]
for all $n$. Set $x=\sum_{n=1}^{\iy} \de^{n-1} x_n$. This converges since $\sum_{n=1}^{\iy}\ve{\de^{n-1}x_n}\le M\sum_{n=1}^{\iy}\de^{n-1}=\fc{M}{1-\de}$, and $X$ is complete.\footnote{This kind of geometric sum argument comes up a lot in functional analysis!} 
So $x\in \fc{M}{1-\de}B_X$ and  by continuity $Tx=\suo \de^{n-1} Tx_n=y$. For the ``moreover" part, let $\hat Y$ be the completion of $Y$, and view 
%unique banach space of which it is a dense subspace
$T$ as a map $X\to \hat Y$. Since $B_Y$ is dense in $B_{\hat Y}$, $T(MB_X)$ is $\de'$-dense in $B_{\hat Y}$ for $\de<\de'<1$. By the first part, $T(X)=\hat Y=Y$, so $Y$ is complete.
\end{proof}
\begin{rem}
Suppose $T\in\cal B(X,Y)$, $X$ is complete, and the image of the ball is dense: $\ol{T(B_X)}\supeq B_Y$. Suppose that for all $\ep>0$, $T((1+\ep)B_X)$ is $1$-dense in $B_Y$. Take $M>1,0< \de<1$ so that $1+\ep=\fc{M}{1-\de}$;  lemma~\ref{lem:oml} shows that  $T\pa{(1+\ep)B_X}\supeq B_Y$. It follows that $T(B_X^{\circ})\supeq B_Y^{\circ}$. (For a subset $A$ of a topological space, $A^{\circ}$ or $\text{int}(A)$ denotes the interior of $A$.)
\end{rem}

{\color{blue} Lecture 3}

\subsection{Applications of the open mapping lemma}
\begin{thm}[Open mapping theorem]\llabel{thm:omt}
Let $X,Y$ be Banach spaces, $T\in \cal B(X,Y)$ be onto. Then $T$ is an open map.
\end{thm}
\begin{proof}
Let $Y=T(X)=\bigcup_{n=1}^{\iy} T(nB_X)$. The Baire category theorem tells us that there exists $N$ with $\text{int}(\ol{T(NB_X)})\ne \phi$. Then there exists $r>0$ with
\[
\ol{T(NB_X)}\supeq rB_Y.
\]
By Lemma~\ref{lem:oml}, for $M=\fc{2N}{r}$, we have $T(MB_X)\supeq B_Y$. Therefore, $U$ is open and $T(U)$ is open.
\end{proof}
%say T is open, same as T^{-1} cont.
\begin{thm}[Banach isomorphism theorem]\llabel{thm-bit}
If in addition $T$ is injective, then $T^{-1}$ is continuous.
\end{thm}
\begin{proof}
An open map that is a bijection is a homeomorphism.
\end{proof}
\begin{thm}[Closed graph theorem]
Let $X,Y$ be Banach spaces and $T:X\to Y$ be linear. Assume that whenever $x_n\to 0$ in $X$, $Tx_n\to y$ in $Y$, then $y=0$. Then $T$ is continuous.
\end{thm}
This is a powerful result. Usually we have to show the sequence converges and show it converges to 0. This says that we only have to check the second part given the first part. %once we show it converges, it automatically converges to 0.
\begin{proof}
The assumption says that the graph of $T$
\[
\Ga(T)=\set{(x,Tx)}{x\in X}
\]
is closed in $X\opl Y=\set{(x,y)}{x\in X,y\in Y}$ with norm, e.g.
\[
\ve{(x,y)}=\ve{x}+\ve y.
\]
So $\Ga(T)$ is a Banach space. Consider $U:\Ga(T)\to X$, $U(x,y)=x$. $U$ is a linear bijection and $\ve{U}\le 1$. From the Banach isomorphism theorem~\ref{thm-bit}, $U^{-1}$ is continuous, i.e., $x\mapsto (x,Tx)$ is continuous.
\end{proof}
We'll give three more applications. The first one is to quotient spaces. Let $X$ be a normed space and $Y$ be a closed subspace. Then $X/Y=\set{x+Y}{x\in X}$ is a normed space with 
\[
\ve{x+Y}=d(x,Y)=d(0,x+Y)=\inf\set{\ve{x+y}}{y\in Y}.
\]
We need $Y$ closed to ensure that if $\ve{z}=0$ then $z=0$ for $z\in X/Y$.
\begin{pr}\llabel{pr:5}%5
Let $X,Y$ be as above. If $X$ is complete, then so is $X/Y$.
\end{pr}
%this can be proved directly too.
\begin{proof}
Consider the quotient map $q:X\to X/Y$, $q(x)=x+Y$. This is a bounded linear map, $q\in \cal B(X,X/Y)$, so
\[
\ve{q(x)}=d(x,Y)\le \ve{x}
\]
and $q(B_X^{\circ})\subeq B_{X/Y}^{\circ}$. If $\ve{x+Y}<1$ then there exists $y\in Y$ with $\ve{x+y}<1$ and $q(x+y)=q(x)=x+Y$ so $q(B_X^{\circ})=B_{X/Y}^{\circ}$. So for any $M>1$, $\ol{q(MB_X)}\supeq B_{X/Y}$. %image of some large ball contains ball. 
By Lemma~\ref{lem:oml}, $X/Y$ is complete.
\end{proof}
\begin{pr}
Every separable Banach space $X$ is (isometrically isomorphic to) a quotient of $\ell_1$.
\end{pr}
\begin{proof}
Let $\set{x_n}{n\in \N}$ be dense in $B_X$. Let $(e_n)$ be the standard basis of $\ell_1$. If $a=(a_n)\in \ell_1$, then $a=\sum_{n=1}^{\iy} a_ne_n$. Define
\[
T:\ell_1\to X,\qquad T\pa{\sum_{n=1}^{\iy} a_ne_n}=\sum_{n=1}^{\iy}a_nx_n.
\]
This is well defined because the sum converges: $\sum_{n=1}^{\iy} \ve{a_nx_n}\le \suo |a_n|=\ve{a}$. So $T\in \cal B(\ell_1,X)$, $\ve{T}\le 1$. Also $T(B_{\ell_1}^{\circ})\subeq B_X^{\circ}$. 
We have $T(B_{\ell_1})\supeq \set{x_n}{n\in \N}$, so $\ol{T(B_{\ell_1})}\supeq B_X$.

%checkme
%This means given $x\in B_X^{\circ}$, there exists a sequence $rx_n\to rx\in B_X$, $rx_n\in  $\ve{rx}=1$. Then $x_n\to x\in B_X^{\circ}$. 
So $T(B_{\ell_1}^{\circ})=B_X^{\circ}$. We have a unique $\wt T:\ell_1/\ker T\to X$ such that
\[
\xymatrix{
\ell_1\ar[rr]^T\ar[rd]_q & & X\\
& \ell_1/\ker T\ar[ru]_{\wt T}&
}
\]
commutes, i.e., $T=\wt Tq$ where $q$ is the quotient map.
%onto from lemma~\ref{lem:oml}
Moreover, $\wt T$ is a linear bijection
\[
\wt T(B_{\ell_1/\ker T}^{\circ})=\wt T(q(B_{\ell_1}^{\circ}))=T(B_{\ell_1}^{\circ})=B_X^{\circ}.
\]
So $\wt T$ is an isometric isomorphism, and $X\cong \ell_1/\ker T$.
\end{proof}
This suggests that to understand all Banach spaces we just have to understand $\ell_1$. But $\ell_1$ is quite complicated, not as innocent as it looks.

Recall the following definition.
\begin{df}
A topological space $K$ is \textbf{normal} if whenever $E,F$ are disjoint closed sets in $K$, there are disjoint open sets $U,V$ in $K$ such that $E\subeq U$, $F\subeq V$. 
\end{df}
\begin{lem}[Urysohn's lemma]\llabel{lem:urysohn}
If $K$ is normal and $E,F$ are disjoint closed subsets of $K$, then $\exists$ continuous $f:K\to [0,1]$ such that $f=0$ on $E$ and $f=1$ on $F$.
\end{lem}
This can be used to construct partitions of unity which we'll use in chapter 3.
\begin{thm}[Tietze's Extension Theorem]
If $K$ is normal and $L$ is a closed subset of $K$, $g:L\to \R$ is bounded and continuous, then there exists a bounded, continuous $f:K\to \R$ such that $f|_L=g$, $\ve{f}_{\iy}=\ve{g}_{\iy}$. 
\end{thm}
\begin{proof}
Let $C_b(K)=\set{h:K\to \R}{h\text{ bounded, continuous}}$, a closed subspace of $\ell_{\iy}(K)$ in $\ved_{\iy}$ (so it is a Banach space). Consider $R:C_b(K)\to C_b(L)$, $R(f)=f|_L$. We need $R(B_{C_b(K)})=B_{C_b(L)}$ (``$\subeq$" is clear, since $\ve{R}\le 1$.) Let $g\in B_{C_b(L)}$ so $-1\le g\le 1$. Let $E=\set{y\in L}{g(y)\le -\rc{3}}$, $F=\set{y\in L}{g(y)\ge \rc 3}$. Urysohn's lemma gives $\exists f\in C_b(K)$ such that $-\rc 3\le f\le \rc 3$, $f=-\rc 3$ on $E$, and $f=\rc 3$ on $F$.

We have
\[
\ve{R(f)-g}_{\iy} \le \fc23 
\]
and $\ve f\le \rc 3$. So $R(\rc 3B_{C_b(K)})$ is $\fc 23$-dense in $B_{C_b(L)}$. By the Open Mapping Lemma~\ref{lem:oml}, $R$ is surjective.
\end{proof}
\begin{rem}
The theorem holds in the complex case too.
\end{rem}

%%%%%%%%%%%%%%%%%%%%%%%%%%%


 
%\bibliographystyle{plain}
%\bibliography{refs}
\end{document}