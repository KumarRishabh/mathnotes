%%%This is a science homework template. Modify the preamble to suit your needs. 

\documentclass[12pt]{article}

\makeatother
%AMS-TeX packages
\usepackage{amsmath}
\usepackage{amssymb}
\usepackage{amsthm}
\usepackage{array}
\usepackage{amsfonts}
\usepackage{cancel}
\usepackage[all,cmtip]{xy}%Commutative Diagrams
\usepackage[pdftex]{graphicx}
\usepackage{float}
%geometry (sets margin) and other useful packages
\usepackage[margin=1in]{geometry}
\usepackage{sidecap}
\usepackage{wrapfig}
\usepackage{verbatim}
\usepackage{mathrsfs}
\usepackage{marvosym}
\usepackage{stmaryrd}
\usepackage{hyperref}
\usepackage{graphicx,ctable,booktabs}

\newtheoremstyle{norm}
{3pt}
{3pt}
{}
{}
{\bf}
{:}
{.5em}
{}

\theoremstyle{norm}
\newtheorem{thm}{Theorem}[section]
\newtheorem{lem}[thm]{Lemma}
\newtheorem{df}{Definition}
\newtheorem{rem}{Remark}
\newtheorem{st}{Step}
\newtheorem{pr}[thm]{Proposition}
\newtheorem{cor}[thm]{Corollary}
\newtheorem{clm}[thm]{Claim}

%Math blackboard, fraktur, and script commonly used letters
\newcommand{\A}[0]{\mathbb{A}}
\newcommand{\C}[0]{\mathbb{C}}
\newcommand{\sC}[0]{\mathcal{C}}
\newcommand{\cE}[0]{\mathscr{E}}
\newcommand{\F}[0]{\mathbb{F}}
\newcommand{\cF}[0]{\mathscr{F}}
\newcommand{\cG}[0]{\mathscr{G}}
\newcommand{\sH}[0]{\mathscr H}
\newcommand{\Hq}[0]{\mathbb{H}}
\newcommand{\cI}[0]{\mathscr{I}}%ideal sheaf
\newcommand{\N}[0]{\mathbb{N}}
\newcommand{\Pj}[0]{\mathbb{P}}
\newcommand{\sO}[0]{\mathcal{O}}
\newcommand{\cO}[0]{\mathscr{O}}
\newcommand{\Q}[0]{\mathbb{Q}}
\newcommand{\R}[0]{\mathbb{R}}
\newcommand{\Z}[0]{\mathbb{Z}}
%Lowercase
\newcommand{\ma}[0]{\mathfrak{a}}
\newcommand{\mb}[0]{\mathfrak{b}}
\newcommand{\fg}[0]{\mathfrak{g}}
\newcommand{\vi}[0]{\mathbf{i}}
\newcommand{\vj}[0]{\mathbf{j}}
\newcommand{\vk}[0]{\mathbf{k}}
\newcommand{\mm}[0]{\mathfrak{m}}
\newcommand{\mfp}[0]{\mathfrak{p}}
\newcommand{\mq}[0]{\mathfrak{q}}
\newcommand{\mr}[0]{\mathfrak{r}}
%Letter-related
\providecommand{\cal}[1]{\mathcal{#1}}
\renewcommand{\cal}[1]{\mathcal{#1}}
\newcommand{\bb}[1]{\mathbb{#1}}
%More sequences of letters
\newcommand{\chom}[0]{\mathscr{H}om}
\newcommand{\fq}[0]{\mathbb{F}_q}
\newcommand{\fqt}[0]{\mathbb{F}_q^{\times}}
\newcommand{\sll}[0]{\mathfrak{sl}}
%Shortcuts for symbols
\newcommand{\nin}[0]{\not\in}
\newcommand{\opl}[0]{\oplus}
\newcommand{\ot}[0]{\otimes}
\newcommand{\rc}[1]{\frac{1}{#1}}
\newcommand{\rra}[0]{\rightrightarrows}
\newcommand{\send}[0]{\mapsto}
\newcommand{\sub}[0]{\subset}
\newcommand{\subeq}[0]{\subseteq}
\newcommand{\supeq}[0]{\supseteq}
\newcommand{\nsubeq}[0]{\not\subseteq}
\newcommand{\nsupeq}[0]{\not\supseteq}
%Shortcuts for greek letters
\newcommand{\al}[0]{\alpha}
\newcommand{\be}[0]{\beta}
\newcommand{\ga}[0]{\gamma}
\newcommand{\Ga}[0]{\Gamma}
\newcommand{\de}[0]{\delta}
\newcommand{\De}[0]{\Delta}
\newcommand{\ep}[0]{\varepsilon}
\newcommand{\eph}[0]{\frac{\varepsilon}{2}}
\newcommand{\ept}[0]{\frac{\varepsilon}{3}}
\newcommand{\la}[0]{\lambda}
\newcommand{\La}[0]{\Lambda}
\newcommand{\ph}[0]{\varphi}
\newcommand{\rh}[0]{\rho}
\newcommand{\te}[0]{\theta}
\newcommand{\om}[0]{\omega}
\newcommand{\Om}[0]{\Omega}
\newcommand{\si}[0]{\sigma}
%Brackets
\newcommand{\ab}[1]{\left| {#1} \right|}
\newcommand{\ba}[1]{\left[ {#1} \right]}
\newcommand{\bc}[1]{\left\{ {#1} \right\}}
\newcommand{\pa}[1]{\left( {#1} \right)}
\newcommand{\an}[1]{\langle {#1}\rangle}
\newcommand{\fl}[1]{\left\lfloor {#1}\right\rfloor}
\newcommand{\ce}[1]{\left\lceil {#1}\right\rceil}
%Text
\newcommand{\btih}[1]{\text{ by the induction hypothesis{#1}}}
\newcommand{\bwoc}[0]{by way of contradiction}
\newcommand{\by}[1]{\text{by~(\ref{#1})}}
\newcommand{\ore}[0]{\text{ or }}
%Arrows
\newcommand{\hr}[0]{\hookrightarrow}
\newcommand{\xr}[1]{\xrightarrow{#1}}
%Formatting
\newcommand{\subprob}[1]{\noindent\textbf{#1}\\}
%Functions, etc.
\newcommand{\Ann}{\operatorname{Ann}}
\newcommand{\Arc}{\operatorname{Arc}}
\newcommand{\Ass}{\operatorname{Ass}}
\newcommand{\Aut}{\operatorname{Aut}}
\newcommand{\chr}{\operatorname{char}}
\newcommand{\cis}{\operatorname{cis}}
\newcommand{\Cl}{\operatorname{Cl}}
\newcommand{\Der}{\operatorname{Der}}
\newcommand{\End}{\operatorname{End}}
\newcommand{\Ext}{\operatorname{Ext}}
\newcommand{\Frac}{\operatorname{Frac}}
\newcommand{\FS}{\operatorname{FS}}
\newcommand{\GL}{\operatorname{GL}}
\newcommand{\Hom}{\operatorname{Hom}}
\newcommand{\Ind}[0]{\text{Ind}}
\newcommand{\im}[0]{\text{im}}
\newcommand{\nil}[0]{\operatorname{nil}}
\newcommand{\ord}[0]{\operatorname{ord}}
\newcommand{\Proj}{\operatorname{Proj}}
\newcommand{\rad}{\operatorname{rad}}
\newcommand{\Rad}{\operatorname{Rad}}
\newcommand{\rank}{\operatorname{rank}}
\newcommand{\Res}[0]{\text{Res}}
\newcommand{\sign}{\operatorname{sign}}
\newcommand{\SL}{\operatorname{SL}}
\newcommand{\Spec}{\operatorname{Spec}}
\newcommand{\Specf}[2]{\Spec\pa{\frac{k[{#1}]}{#2}}}
\newcommand{\spp}{\operatorname{sp}}
\newcommand{\spn}{\operatorname{span}}
\newcommand{\Supp}{\operatorname{Supp}}
\newcommand{\Tor}{\operatorname{Tor}}
\newcommand{\tr}[0]{\text{trace}}
\newcommand{\Var}{\operatorname{Var}}
\newcommand{\vol}[0]{\operatorname{vol}}
%Commutative diagram shortcuts
\newcommand{\fiber}[3]{\xymatrix{#1\times_{#3} #2}\ar[r]\ar[d] #1\ar[d] \\ #2 \ar[r] & #3}
\newcommand{\commsq}[8]{\xymatrix{#1\ar[r]^{#6}\ar[d]^{#5} &#2\ar[d]^{#7} \\ #3 \ar[r]^{#8} & #4}}
%Makes a diagram like this
%1->2
%|    |
%3->4
%Arguments 5, 6, 7, 8 on arrows
%  6
%5  7
%  8
\newcommand{\pull}[9]{
#1\ar@/_/[ddr]_{#2} \ar@{.>}[rd]^{#3} \ar@/^/[rrd]^{#4} & &\\
& #5\ar[r]^{#6}\ar[d]^{#8} &#7\ar[d]^{#9} \\}
\newcommand{\back}[3]{& #1 \ar[r]^{#2} & #3}
%Syntax:\pull 123456789 \back ABC
%1=upper left-hand corner
%2,3,4=arrows from upper LH corner, going down, diagonal, right
%5,6,7=top row (6 on arrow)
%8,9=middle rows (on arrows)
%A,B,C=bottom row
%Other
%Other
\newcommand{\op}{^{\text{op}}}
\newcommand{\fp}[1]{^{\underline{#1}}}
\newcommand{\rp}[1]{^{\overline{#1}}}
\newcommand{\rd}[0]{_{\text{red}}}
\newcommand{\pre}[0]{^{\text{pre}}}
\newcommand{\pf}[2]{\pa{\frac{#1}{#2}}}
\newcommand{\pd}[2]{\frac{\partial #1}{\partial #2}}
\newcommand{\bs}[0]{\backslash}
\newcommand{\sia}[0]{ $\si$-algebra}
\newcommand{\ol}[1]{\overline{#1}}
\newcommand{\iy}[0]{\infty}
\newcommand{\nl}[1]{\left \Vert #1\right \Vert_{L^1}}
%Matrices
\newcommand{\coltwo}[2]{
\left[
\begin{matrix}
{#1}\\
{#2} 
\end{matrix}
\right]}
\newcommand{\matt}[4]{
\left[
\begin{matrix}
{#1}&{#2}\\
{#3}&{#4}
\end{matrix}
\right]}
\newcommand{\smatt}[4]{
\left[
\begin{smallmatrix}
{#1}&{#2}\\
{#3}&{#4}
\end{smallmatrix}
\right]}
\newcommand{\colthree}[3]{
\left[
\begin{matrix}
{#1}\\
{#2}\\
{#3}
\end{matrix}
\right]}
%
%Redefining sections as problems
%
\makeatletter
\newenvironment{problem}{\@startsection
       {section}
       {1}
       {-.2em}
       {-3.5ex plus -1ex minus -.2ex}
       {2.3ex plus .2ex}
       {\pagebreak[3]%forces pagebreak when space is small; use \eject for better results
       \large\bf\noindent{Problem }
       }
       }
       {%\vspace{1ex}\begin{center} \rule{0.3\linewidth}{.3pt}\end{center}}
       }
\makeatother


%
%Fancy-header package to modify header/page numbering 
%
\usepackage{fancyhdr}
\pagestyle{fancy}
%\addtolength{\headwidth}{\marginparsep} %these change header-rule width
%\addtolength{\headwidth}{\marginparwidth}
\lhead{Problem \thesection}
\chead{} 
\rhead{\thepage} 
\lfoot{\small\scshape 18.125 Real and Functional Analysis} 
\cfoot{} 
\rfoot{\scriptsize PS \# 3} % !! Remember to change the problem set number
\renewcommand{\headrulewidth}{.3pt} 
\renewcommand{\footrulewidth}{.3pt}
\setlength\voffset{-0.25in}
\setlength\textheight{648pt}



%%%%%%%%%%%%%%%%%%%%%%%%%%%%%%%%%%%%%%%%%%%%%%%
%
%Contents of problem set
%    
\begin{document}
\title{18.952 Differential Forms PSet \#7}% !! Remember to change the problem set number
\author{Holden Lee}
\date{4/11/11}% !! Remember to change the date
\maketitle
\thispagestyle{empty}
\begin{problem}{\it Cotangent bundle as a manifold}
We associate $\R^{n*}$ with $\R^n$ via the map $\phi:\R^{n*}\to \R^n$ defined by
\[
\phi(\eta)=(\eta(e_1),\ldots, \eta(e_n)),
\]
where $e_1,\ldots, e_n$ are the standard basis vectors. With this identification, $\tilde{\ph}$ can be viewed as a map from $\tilde{U}$ to $V\times \R^n\subeq \R^{2n}$.

We check the following.
\begin{enumerate}
\item 
$\bigcup \tilde{U}=M$: Since the $(U,V,\ph)$ form an atlas for $X$, $\bigcup U=X$. Given $(p,\zeta)\in M$, we have $p\in U$ for some $U$, and hence $(p,\zeta)\in\tilde{U}$.
\item
$\tilde{V}_{1,2}:=\tilde{\ph}_1 (\tilde{U}_1\cap \tilde{U}_2)$ is open in $\tilde{V}_1$, where $(\tilde{U}_1,\tilde{V}_1, \tilde{\ph}_1)$ and $(\tilde{U}_2,\tilde{V}_2, \tilde{\ph}_2)$ are charts: Since the $(U,V,\ph)$ form an atlas, $V_{1,2}=\ph_1(U_1\cap U_2)$ is open in $V_1$. Hence $\tilde{V}_{1,2}=V_{1,2}\times \R^n$ is open in $\tilde{V}_1=V_1\times \R^n$.
\item 
$\tilde{\ph}_{1,2}=\tilde{\ph}_2\circ \tilde{\ph}_1^{-1}$ is a diffeomorphism. Let $(q, (a_1,\ldots, a_n))=\tilde{\ph}_1(p,\zeta)\in V_{1,2}$, i.e. $q=\ph_1(p)$ and %$(a_1,\ldots, a_n)$ corresponds to $\eta\in T^*_p\R^n$ with 
$(a_1,\ldots, a_n)=(\eta(e_1),\ldots, \eta(e_n))$ where $\eta=(d\ph_1^{-1})_q^*\zeta$. Then 
\begin{align*}
\tilde{\ph}_{1,2}(q,(a_1,\ldots, a_n))&=(\ph_2\circ \ph_1^{-1}(q),
\phi[ (d\ph_{2}^{-1})_{\ph_2(p)}^* (d\ph_{1})_p^*\eta])\\
&=(\ph_2\circ \ph_1^{-1}(q),\phi[\eta\circ (d\ph_1)_p\circ (d\ph_{2})_p^{-1}]).
\end{align*}
Since the $(U,V,\ph)$ are an atlas for $X$, $\ph_2\circ \ph_1^{-1}$ is $C^{\iy}$, taking care of the first component above. For the second component, note
\begin{align*}
\phi[\eta\circ (d\ph_1)_p\circ (d\ph_{2})_p^{-1}]
&=\phi[\eta \circ d(\ph_1\circ \ph_{2}^{-1})_{\ph_2(p)}]\\
&=\pa{\eta \circ d(\ph_{2,1})_{\ph_2(p)}(e_i)}_{1\le i\le n}\\
&=\pa{\eta \circ d(\ph_{1,2})_{q}^{-1}(e_i)}_{1\le i\le n}\\
%&=\pa{\eta(D\ph_{1,2}(p)^{-1}(e_i))}_{1\le i\le n}\\
&=[a_1\;a_2\;\cdots\;a_n]D\ph_{1,2}(q)^{-1}
\end{align*}
where $D\ph_{1,2}(q)$ is the matrix of the derivative at $q$. The above 
%The $i$th component of the above is
%\[
%\sum_{j=1}^n %(a_1,\ldots, a_n)\pa{\left.\pd{(\ph_{2,1})_q}{x_i}\right|_{\ph_{1,2}(q)}}_{1\le i\le n}^T.
%a_j\pa{\left.\pd{\ph_{2,1}}{x_i}\right|_{\ph_{1,2}(q)}}_j
%%=
%%\sum_{j=1}^n 
%%a_j\pa{\left.\pd{\ph_{1,2}}{x_i}\right|_{q}}_j^{-1}
%\]
%which 
is a $C^{\iy}$ function of $q, a_1,\ldots, a_n$, since $\ph_{1,2}$ %, and hence $\pd{\ph_{2,1}}{x_i}$, 
is a $C^{\iy}$ diffeomorphism of $q$.

Hence $\tilde{\ph}_{1,2}$ is $C^{\iy}$. Since this is likewise true for $\tilde{\ph}_{2,1}$, it is a diffeomorphism.
\end{enumerate}
\end{problem}
\begin{problem}{\it Canonical one form}
The projection $\pi:M\to X$ defined by $\pi(p,\zeta)=p$ is differentiable, because by using the same atlas as in problem 1, if $(U,V,\ph)$ is a chart at $p$ we have that $\ph\pi\tilde{\ph}^{-1}$ is the projection map $\tilde{V}=V\times \R^{n*}\to V$. Then $d\pi_q$ gives a map $T_qM\to T_pX$, where $\pi(q)=p$, so $d\pi_q^*$ gives a map on dual spaces in the opposite direction, $T_p^*X\to T_q^*M$. Since $\zeta\in T_p^*X$, we get $\al_q=d\pi_q^*\zeta\in T_q^*M$. Thus $q=(p,\zeta)\mapsto \al_q$ is a one-form on $M$.
\end{problem}
\begin{problem}{\it Functoriality}
\subprob{(i)}
Suppose $g(p_1,\zeta_1)=(p_2,\zeta_2)$. Let $(U_1,V_1,\ph_1)$ and $(U_2,V_2,\ph_2)$ be charts at $p_1$ and $p_2$, respectively. To show $g$ is a diffeomorphism, we need to show $\tilde{g}:=\tilde{\ph_2}\circ g \circ\tilde{\ph}_1^{-1}$ is a diffeomorphism:
\[
\xymatrix{
\tilde{U}_1\ar[r]^{g}\ar[d]^{\tilde{\ph}_1}& \tilde{U}_2\ar[d]^{\tilde{\ph}_2}\\
\tilde{V}_1\ar[r]^{\tilde g}&\tilde V_2
}.
\]
Suppose $(p_1',(a_1,\ldots, a_n))=\tilde{\ph}_1(p_1,\zeta_1)\in \tilde V_1$, i.e. $p_1'=\ph_1(p_1)$ and $(a_1,\ldots, a_n)=\phi(\eta_1)$ where $\eta_1=(d\ph_1)_*\zeta_1\in T^*_{p_1'}V_1$. (As in problem 1, $\phi(\eta)=(\eta(e_1),\ldots, \eta(e_n))$.) Suppose furthermore $p_2'=\ph_2(p_2)$.  Then
\[
\tilde{g}(p_1',(a_1,\ldots, a_n))= (\ph_2f\ph_1^{-1}(p_1'), \phi[(d\ph_2^{-1})_{p_2'}^*(df^{-1})_{p_2}^*(d\ph_1)_{p_1}^*\eta_1]).
\]
Since $f$ is $C^{\iy}$, $\ph_2f\ph_1^{-1}$ is $C^{\iy}$, taking care of the first component. For the second component note that
\begin{align*}
\phi[(d\ph_2^{-1})_{p_2'}^*(df^{-1})_{p_2}^*(d\ph_1)_{p_1}^*\eta_1]
&=\phi[\eta_1\circ (d\ph_1)_{p_1}\circ (df^{-1})_{p_2}\circ (d\ph_2^{-1})_{p_2'}]\\
&=\phi[\eta_1\circ d(\ph_1\circ f^{-1}\circ \ph_2^{-1})_{p_2'}]\\
&=\phi[\eta_1\circ d(\ph_2\circ f\circ \ph_1^{-1})_{p_1'}]\\
&=[a_1\;a_2\;\cdots \;a_n]D(\ph_2\circ f\circ \ph_1^{-1})(p_1')^{-1}.
\end{align*}
This is $C^{\iy}$ since $\ph_2\circ f\circ \ph_1^{-1}$ is by assumption a  $C^{\iy}$ diffeomorphism. Since the same argument applies to $f^{-1}$, we get that $g$ is a diffeomorphism.\\

\subprob{(ii)}
Let $q_i=(p_i,\zeta_i)$ and $g(q_1)=q_2$ (so $f(p_1)=p_2$).

The following diagram commutes, by examining its action on $(p,\zeta)$:
\[
\xymatrix{
M_1\ar[r]^{g}\ar[d]^{\pi_1}&M_2\ar[d]^{\pi_2}\\
X_1\ar[r]^f& X_2
}\qquad
\xymatrix{
(p,\zeta)\ar[r]^{g}\ar[d]^{\pi_1}&(g(p_1),\zeta\circ (df^{-1})_{p_2})\ar[d]^{\pi_2}\\
\zeta\ar[r]^f& \zeta\circ (df^{-1})_{p_2}
}.
\]
Passing to maps on tangent spaces,
\[
\xymatrix{
T_{q_1}M_1\ar[r]^{dg_{q_1}}\ar[d]^{(d\pi_1)_{q_1}}&T_{q_2}M_2 \ar[d]^{(d\pi_2)_{q_2}}\\
T_{p_1}X_1\ar[r]^{df_{p_1}}& T_{p_2}X_2
}.
\]
Taking dual spaces and transposes,
\begin{equation}\label{p7-3-1}
\xymatrix{
T_{q_1}^*M_1&T_{q_2}^*M_2\ar[l]_{dg_{q_1}^*}\\
T_{p_1}^*X_1\ar[u]_{(d\pi_{1})_{q_1}^*}& T_{p_2}^*X_2\ar[l]_{df_{p_1}^*}
\ar[u]_{(d\pi_{2})_{q_2}^*}
}.
\end{equation}
Thus
\begin{align*}
\al_1(q_1)&=(d\pi_1)_{q_1}^*\zeta_1\\
&=(d\pi_1)_{q_1}^*df_{p_1}^*\zeta_2\\
&=dg^*_{q_1}(d\pi_2)_{q_2}^* \zeta_2&\text{ by~(\ref{p7-3-1})}\\
&=(dg^* \al_2)(q_1)
\end{align*}
giving $\al_1=dg^* \al_2$.
\end{problem}
\begin{problem} {\it Canonical one-form for $X=\R^n$}
$\pi$ acts by $\pi(x_1,\ldots, x_n,\xi_1,\ldots, \xi_n)=(x_1,\ldots, x_n)$; its derivative with respect to $x_i$ is $e_i$ and its derivative with respect to $\xi_i$ is 0. 
Hence writing $\xi=(\xi_1,\ldots, \xi_n)=\xi_1dx_1+\cdots +\xi_ndx_n$, 
$\al_q=(d\pi)_q^*\xi=\xi\circ d\pi_q$ acts on the basis vectors $\pa{\pd{}{x_i}}_q, \pa{\pd{}{\xi_i}}_q$ by
\begin{align*}
\al_q\pa{\pd{}{x_i}}_q&=\xi\circ (d\pi)_q\pa{\pd{}{x_i}}_q=\xi\pa{\pd{}{x_i}}_p=\xi_i\\
\al_q\pa{\pd{}{\xi_i}}_q&=\xi\circ (d\pi)_q\pa{\pd{}{\xi_i}}_q=\xi(0)=0.
\end{align*}
Hence 
\[\al_q=\sum_{i=1}^n \xi_i(dx_i)_q\]
and 
\[\al=\sum_{i=1}^n\xi_i dx_i.\]
This shows $\al$ is smooth.
\end{problem}
\begin{problem}{\it Symplectic form}
First note $M$ is $2n$-dimensional.

Since $M$ is locally diffeomorphic to an open subset of $T^*\R^n$, and differential forms are functorial by Problem 3, it suffices to prove the statement for $T^*\R^n$. From $\al=\sum_{i=1}^n \xi_idx_i$ we get
\[
\om=-d\al=-\sum_{i=1}^n d\xi_i\wedge dx_i=\sum_{i=1}^ndx_i\wedge d\xi_i.
\]
Since $d^2=0$, we have $d\om=-d^2\al=0$. 

Now
\[
\om^{\wedge n}=\sum_{\pi \in S_n}dx_{\pi(1)}\wedge d\xi_{\pi(1)}\wedge \cdots \wedge dx_{\pi(n)}\wedge d\xi_{\pi(n)}=n!\,dx_1\wedge d\xi_1\wedge\cdots \wedge dx_n\wedge d\xi_n.
\]
To get the first equality, note that upon expanding $\pa{\sum_{i=1}^ndx_i\wedge d\xi_i}^{\wedge n}$, all terms with repeated one-forms equal 0, and all terms without repeated one-forms are given above. To get the second equality, note that if $\pi$ is composed of $k$ transpositions, then it takes $k$ transpositions to permute the $dx_i$'s and $k$ transpositions to permute the $d\xi_i$'s in $dx_{\pi(1)}\wedge d\xi_{\pi(1)}\wedge \cdots \wedge dx_{\pi(n)}\wedge d\xi_{\pi(n)}$, and so it takes $2k$ transpositions, i.e. an even permutation, to permute the term into $dx_1\wedge d\xi_1\wedge\cdots \wedge dx_n\wedge d\xi_n$. Hence all terms equal $dx_1\wedge d\xi_1\wedge\cdots \wedge dx_n\wedge d\xi_n$.

Since $\om$ is closed and nondegenerate, it is symplectic.
\end{problem}
\begin{problem}{\it Hamiltonian vector fields}
We have
\begin{align*}
\iota(v)(\om^{\wedge n})&=\begin{array}{cccccccc}
 &\iota(v)\omega & \wedge & \omega & \wedge & \cdots & \wedge & \omega\\
+ & \omega & \wedge & \iota(v)\omega& \wedge & \cdots & \wedge & \omega\\
 & \vdots &  &  & \vdots &  &  & \vdots\\
+ & \omega & \wedge & \omega & \wedge & \cdots & \wedge & \iota(v)\omega\end{array}\\
&=n(\iota(v)\om)\wedge \om^{\wedge (n-1)}\\
\implies \iota(v_p)(\om_p^{\wedge n})&=n(\iota(v_p)\omega_p)\wedge \om_p^{\wedge(n-1)}.
\end{align*}
The first equality comes from the derivation property $\iota(v)(\om\wedge \mu)=(\iota(v)\om)\wedge\mu+(-1)^k\om\wedge(\iota(v)\mu)$ when $\om$ is a $k$-form, applied multiple times. (Here $\om$ is a 2-form.) The second equality follows from the fact that moving a one-form $\iota(v)\om$ past the $2k$-form $\om^{\wedge k}$ preserves the sign. Indeed, by distributivity it suffices to show  that if $\mu$ is a decomposable $k$-form, then
\[
\mu\wedge dx=(-1)^k dx\wedge\mu.
\]
However this is apparent since the RHS is the LHS permuted by a cycle of length $k+1$, which has sign $(-1)^k$ (Problem 1.4.7).

%Define $\ga_p:T_pM\to \Om^1(M)$ by
%\[
%\ga_p(e)=
%\]
%Note that $[\iota(v)\mu]_p$ depends only on $v(p)$. For $e\in T_pM$, we define $\iota(e)\mu$ to be $[\iota(v)\mu]_p$ where $v$ is any vector field such that $v(p)=e$.

%Since $\om_p^{\wedge n}\neq 0$, we have that if $v\neq 0$, then $\iota(v)(\om_p^{\wedge n})\neq 0$. 
%Note that if $N$ is a $n$-dimensional manifold and $v(p)\neq 0$, then $(\iota(v)\om)_p\ne 0$ for any nonzero $\om\in \Om^n(N)$. Indeed, by functoriality it suffices to check this for $N=\R^n$, and in this case, if $v=\sum_i f_i\pa{\pd{}{x_i}}$, then
%\[
%\iota(v)(dx_1\wedge \cdots \wedge dx_n)=\sum_i (-1)^{i-1} f_i(dx_1\wedge \cdots \wedge \widehat{dx_i}\wedge \cdots \wedge dx_n).
%\]
%This is nonzero at $p$ for $v(p)=(f_1(p),\ldots, f_n(p))\neq 0$. Since $\Om^n(N)$ is spanned by $dx_1\wedge \cdots \wedge dx_n$, the result follows.
Note that if $N$ is a $n$-dimensional manifold and $e\in T_pN,\,e\neq 0$, then $\iota(e)\mu_p\ne 0$ for any $n$-form $\mu$ with $\mu_p\ne 0$. Indeed, by functoriality it suffices to check this for $N=\R^n$, and in this case, if $e=\sum_i a_i\pa{\pd{}{x_i}}_p$, then
\[
\iota(e)(dx_1\wedge \cdots \wedge dx_n)_p=\sum_i (-1)^{i-1} a_i(dx_1\wedge \cdots \wedge \widehat{dx_i}\wedge \cdots \wedge dx_n)_p.
\]
This is nonzero for $v(e)=(a_1,\ldots, a_n)\neq 0$. Since $\wedge^nT_p^*N$ is spanned by $(dx_1\wedge \cdots \wedge dx_n)_p$, the result follows.

Since $\om^{\wedge n}_p\neq 0$ for any $p\in M$, by Problem 5, the above gives that for $e\in T_pM,e\neq 0$, $\iota(e)(\om_p^{\wedge n})\neq 0$. But this equals $n(\iota(e)\om_p)\wedge \om^{\wedge(n-1)}_p$. Hence $\iota(e)\om_p\neq 0$. Thus the map $e\mapsto \iota(e)\om_p$ is injective. Since it is a linear map from the $2n$-dimensional space $T_pM$ to the $2n$-dimensional space $T_p^*M$, it must be bijective.

%Since $\om_p^{\wedge n}\neq 0$, if $v\neq 0$, then by the above $\iota(v)(\om_p^{\wedge n})\neq 0$. But this equals $n[(\iota(v)\om_p)\wedge \om_p^{\wedge (n-1)}]$, showing $\iota(v)\om_p\neq 0$. The map $v\mapsto \iota(v)\om_p$ can be viewed as a linear map from $T_pM$ to $\Om^1(M)$, since only $v_p$ matters above. The above shows this map is injective; since the domain and range have dimension $2n$ it is bijective.
Applying this to each point $p$ of $M$, we get that there is exactly one value of $v_H(p)$ that will make $(\iota(v_H)\om)_p$ equal to $(dH)_p$. Hence
 there exists exactly 1 vector field $v_H$ such that 
\[\iota(v_H)\om=dH.\]
%This follows from the general fact that if $e_1,\ldots, e_m$ is a basis of $T_pM$, $\eta_1,\ldots, \eta_m$ the dual basis of $T_p^*M$, and $v_p=\sum_i f_ie_i$, then \[[\iota (v)(\eta_1\wedge\cdots \wedge \eta_n)]_p=\sum_i (-1)^i f_i(\eta_1\wedge\cdots \wedge\hat{\eta_i}\wedge \cdots \wedge \eta_n).\]
\end{problem}
\begin{problem}{\it Conservation of energy}
First note
\[
L_{v_H}H=\iota (v_H)dH=\iota(v_H)(\iota(v_H)\om)=0.
\]
%Fix $x$ and let $\ga(t)=f_t(x)$. Then, letting $p=\ga(t)$,
Think of $x$ as fixed. Letting $p=f_t(x)$, 
\[
\frac{d}{dt} H(f_t(x))=(dH)_p\pa{\frac{df_t(x)}{dt}}=(dH)_p(v_H)=L_{v_H}H(p)=0.
\]
Hence
\[f_t^*H(x)=H(f_t(x))=H(f_0(x))=H(x).\]
\end{problem}
\begin{problem}{\it Conservation of volume}
By Problem 2.6.10 and Problem 7,
\begin{align*}
\frac{d}{dt} f_t^* \om
&= f_t^*L_{v_H}\om\\
&=f_t^*[d\iota(v_H)\om+\iota(v_H)\underbrace{d\om}_0]\\
&=f_t^*d(dH)=0.
\end{align*}
Hence
\[
f_t^*\om=f_0^*\om=\om
\]
and
\[
f_t^*(\om^{\wedge n})=(f_t^*\om)^{\wedge n}=\om^{\wedge n}.
\]
\end{problem}

\begin{lem}[Problem 2.6.10]
If $f_t:U\to U$ is the one-parameter group of diffeomorphisms of $v$, and $\om\in \Om^k(U)$, then
\[
\frac{d}{dt} f_t^* \om
= f_t^*L_{v}\om
\]
\end{lem}
\begin{proof}
We first show this for $t=0$. We proceed in 4 steps.
\begin{enumerate}
\item Proof for 0-forms: Let $\ph\in C^{\iy}(U)$. Then at $t=0$,
\[
\frac{d}{dt}f_t^*\ph(p)=(d\ph)_p\pa{\frac{df_t}{dt}(p)}=(d\ph)_p(v(p))=L_v\ph(p).
\]
\item If the theorem holds for $\om$ then it holds for $d\om$: We have
\[
\frac{d}{dt} f_t^*(d\om)=\frac{d}{dt}df_t^*\om =d\pa{\frac{d}{dt} f_t^* \om}=d(L_v\om)=L_v(d\om).\]
The last equality follows from $d^2=0$ and
\[
dL_v \om=d(d\iota(v)\om+\iota(v)d\om)=d\iota(v)d\om =(d\iota(v)+\iota(v)d)d\om=L_vd\om.
\]
\item If the theorem holds for $\om_1$ and $\om_2$ then it holds for $\om_1\wedge \om_2$: Differentiating $f_t^*(\om_1\wedge \om_2)=f_t^*\om_1\wedge f_t^*\om_2$ at $t=0$ and using the product rule gives
\begin{align*}
\frac{d}{dt}f_t^*(\om_1\wedge \om_2)&=\frac{d}{dt}(f_t^*\om_1\wedge f_t^*\om_2)\\
&=\pa{\frac{d}{dt} f_t^* \om_1}\wedge f_t^*\om_2=f_t^*\om_1\wedge \pa{\frac{d}{dt}f_t^*\om_2}\\
&=L_v\om_1 \wedge f_t^*\om_2+\om_1\wedge f_t^*\om_2\\
&=L_v\om_1 \wedge \om_2+\om_1\wedge L_v\om_2\\
&=L_v(\om_1\wedge \om_2).
\end{align*}

\item Every $k$-form can be written in the form $\sum_I f_Idx_I$, so the result follows from 1-3.
\end{enumerate}
Now differentiate the identity
\[
f_{s+t}^* \om=f_t^*(f_s^*\om)
\]
with respect to $s$, set $s=0$, and use the result for $s=0$ to get
\[
\frac{d}{dt}f_t^*\om=f_t^*L_v\om
\]
for any $t$.
\end{proof}
%\begin{thebibliography}{9}
%\bibitem{rudin} Rudin, W.: "Principles of Mathematical Analysis," McGraw-Hill, CA, 1976.
%\end{thebibliography}
\end{document}