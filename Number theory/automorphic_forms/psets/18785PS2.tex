%%%This is a science homework template. Modify the preamble to suit your needs. 

\documentclass[12pt]{article}

\makeatother
%AMS-TeX packages
\usepackage{amsmath}
\usepackage{amssymb}
\usepackage{amsthm}
\usepackage{array}
\usepackage{amsfonts}
\usepackage[all,cmtip]{xy}%Commutative Diagrams
\usepackage[pdftex]{graphicx}
\usepackage{float}
%geometry (sets margin) and other useful packages
\usepackage[margin=1in]{geometry}
\usepackage{sidecap}
\usepackage{wrapfig}
\usepackage{verbatim}
\usepackage{mathrsfs}
\usepackage{marvosym}
\usepackage{hyperref}
\usepackage{graphicx,ctable,booktabs}

\newtheoremstyle{norm}
{3pt}
{3pt}
{}
{}
{\bf}
{:}
{.5em}
{}

\theoremstyle{norm}
\newtheorem{thm}{Theorem}[section]
\newtheorem{lem}[thm]{Lemma}
\newtheorem{df}{Definition}
\newtheorem{rem}{Remark}
\newtheorem{st}{Step}
\newtheorem{pr}[thm]{Proposition}
\newtheorem{cor}[thm]{Corollary}
\newtheorem{clm}[thm]{Claim}

%Math blackboard, fraktur, and script commonly used letters
\newcommand{\A}[0]{\mathbb{A}}
\newcommand{\C}[0]{\mathbb{C}}
\newcommand{\sC}[0]{\mathcal{C}}
\newcommand{\cE}[0]{\mathscr{E}}
\newcommand{\F}[0]{\mathbb{F}}
\newcommand{\cF}[0]{\mathscr{F}}
\newcommand{\cG}[0]{\mathscr{G}}
\newcommand{\sH}[0]{\mathscr H}
\newcommand{\Hq}[0]{\mathbb{H}}
\newcommand{\cI}[0]{\mathscr{I}}%ideal sheaf
\newcommand{\N}[0]{\mathbb{N}}
\newcommand{\Pj}[0]{\mathbb{P}}
\newcommand{\sO}[0]{\mathcal{O}}
\newcommand{\cO}[0]{\mathscr{O}}
\newcommand{\Q}[0]{\mathbb{Q}}
\newcommand{\R}[0]{\mathbb{R}}
\newcommand{\Z}[0]{\mathbb{Z}}
%Lowercase
\newcommand{\ma}[0]{\mathfrak{a}}
\newcommand{\mb}[0]{\mathfrak{b}}
\newcommand{\fg}[0]{\mathfrak{g}}
\newcommand{\vi}[0]{\mathbf{i}}
\newcommand{\vj}[0]{\mathbf{j}}
\newcommand{\vk}[0]{\mathbf{k}}
\newcommand{\mm}[0]{\mathfrak{m}}
\newcommand{\mfp}[0]{\mathfrak{p}}
\newcommand{\mq}[0]{\mathfrak{q}}
\newcommand{\mr}[0]{\mathfrak{r}}
%Letter-related
\newcommand{\bb}[1]{\mathbb{#1}}
%More sequences of letters
\newcommand{\chom}[0]{\mathscr{H}om}
\newcommand{\fq}[0]{\mathbb{F}_q}
\newcommand{\fqt}[0]{\mathbb{F}_q^{\times}}
\newcommand{\sll}[0]{\mathfrak{sl}}
%Shortcuts for symbols
\newcommand{\nin}[0]{\not\in}
\newcommand{\opl}[0]{\oplus}
\newcommand{\ot}[0]{\otimes}
\newcommand{\rc}[1]{\frac{1}{#1}}
\newcommand{\rra}[0]{\rightrightarrows}
\newcommand{\send}[0]{\mapsto}
\newcommand{\sub}[0]{\subset}
\newcommand{\subeq}[0]{\subseteq}
\newcommand{\supeq}[0]{\supseteq}
\newcommand{\nsubeq}[0]{\not\subseteq}
\newcommand{\nsupeq}[0]{\not\supseteq}
%Shortcuts for greek letters
\newcommand{\al}[0]{\alpha}
\newcommand{\be}[0]{\beta}
\newcommand{\ga}[0]{\gamma}
\newcommand{\Ga}[0]{\Gamma}
\newcommand{\de}[0]{\delta}
\newcommand{\De}[0]{\Delta}
\newcommand{\ep}[0]{\varepsilon}
\newcommand{\eph}[0]{\frac{\varepsilon}{2}}
\newcommand{\ept}[0]{\frac{\varepsilon}{3}}
\newcommand{\la}[0]{\lambda}
\newcommand{\La}[0]{\Lambda}
\newcommand{\ph}[0]{\varphi}
\newcommand{\rh}[0]{\rho}
\newcommand{\te}[0]{\theta}
\newcommand{\om}[0]{\omega}
%Brackets
\newcommand{\ab}[1]{\left| {#1} \right|}
\newcommand{\ba}[1]{\left[ {#1} \right]}
\newcommand{\bc}[1]{\left\{ {#1} \right\}}
\newcommand{\pa}[1]{\left( {#1} \right)}
\newcommand{\an}[1]{\langle {#1}\rangle}
\newcommand{\fl}[1]{\left\lfloor {#1}\right\rfloor}
\newcommand{\ce}[1]{\left\lceil {#1}\right\rceil}
%Text
\newcommand{\btih}[1]{\text{ by the induction hypothesis{#1}}}
\newcommand{\bwoc}[0]{by way of contradiction}
\newcommand{\by}[1]{\text{by~(\ref{#1})}}
\newcommand{\ore}[0]{\text{ or }}
%Arrows
\newcommand{\hr}[0]{\hookrightarrow}
\newcommand{\xr}[1]{\xrightarrow{#1}}
%Formatting
\newcommand{\subprob}[1]{\noindent\textbf{#1}\\}
%Functions, etc.
\newcommand{\Ann}{\operatorname{Ann}}
\newcommand{\AP}{\operatorname{AP}}
\newcommand{\Ass}{\operatorname{Ass}}
\newcommand{\Aut}{\operatorname{Aut}}
\newcommand{\chr}{\operatorname{char}}
\newcommand{\cis}{\operatorname{cis}}
\newcommand{\Cl}{\operatorname{Cl}}
\newcommand{\Der}{\operatorname{Der}}
\newcommand{\End}{\operatorname{End}}
\newcommand{\Ext}{\operatorname{Ext}}
\newcommand{\Frac}{\operatorname{Frac}}
\newcommand{\FS}{\operatorname{FS}}
\newcommand{\GL}{\operatorname{GL}}
\newcommand{\Hom}{\operatorname{Hom}}
\newcommand{\Ind}[0]{\text{Ind}}
\newcommand{\im}[0]{\text{im}}
\newcommand{\nil}[0]{\operatorname{nil}}
\newcommand{\ord}[0]{\operatorname{ord}}
\newcommand{\Proj}{\operatorname{Proj}}
\newcommand{\Rad}{\operatorname{Rad}}
\newcommand{\rank}{\operatorname{rank}}
\newcommand{\Res}[0]{\text{Res}}
\newcommand{\sign}{\operatorname{sign}}
\newcommand{\SL}{\operatorname{SL}}
\newcommand{\Spec}{\operatorname{Spec}}
\newcommand{\Specf}[2]{\Spec\pa{\frac{k[{#1}]}{#2}}}
\newcommand{\spp}{\operatorname{sp}}
\newcommand{\spn}{\operatorname{span}}
\newcommand{\Supp}{\operatorname{Supp}}
\newcommand{\Tor}{\operatorname{Tor}}
\newcommand{\tr}[0]{\text{trace}}
%Commutative diagram shortcuts
\newcommand{\fiber}[3]{\xymatrix{#1\times_{#3} #2}\ar[r]\ar[d] #1\ar[d] \\ #2 \ar[r] & #3}
\newcommand{\commsq}[8]{\xymatrix{#1\ar[r]^{#6}\ar[d]^{#5} &#2\ar[d]^{#7} \\ #3 \ar[r]^{#8} & #4}}
%Makes a diagram like this
%1->2
%|    |
%3->4
%Arguments 5, 6, 7, 8 on arrows
%  6
%5  7
%  8
\newcommand{\pull}[9]{
#1\ar@/_/[ddr]_{#2} \ar@{.>}[rd]^{#3} \ar@/^/[rrd]^{#4} & &\\
& #5\ar[r]^{#6}\ar[d]^{#8} &#7\ar[d]^{#9} \\}
\newcommand{\back}[3]{& #1 \ar[r]^{#2} & #3}
%Syntax:\pull 123456789 \back ABC
%1=upper left-hand corner
%2,3,4=arrows from upper LH corner, going down, diagonal, right
%5,6,7=top row (6 on arrow)
%8,9=middle rows (on arrows)
%A,B,C=bottom row
%Other
%Other
\newcommand{\op}{^{\text{op}}}
\newcommand{\fp}[1]{^{\underline{#1}}}
\newcommand{\rp}[1]{^{\overline{#1}}}
\newcommand{\rd}[0]{_{\text{red}}}
\newcommand{\pre}[0]{^{\text{pre}}}
\newcommand{\pf}[2]{\pa{\frac{#1}{#2}}}
\newcommand{\pd}[2]{\frac{\partial #1}{\partial #2}}
\newcommand{\bs}[0]{\backslash}
%Matrices
\newcommand{\coltwo}[2]{
\left[
\begin{matrix}
{#1}\\
{#2} 
\end{matrix}
\right]}
\newcommand{\matt}[4]{
\left[
\begin{matrix}
{#1}&{#2}\\
{#3}&{#4}
\end{matrix}
\right]}
\newcommand{\smatt}[4]{
\left[
\begin{smallmatrix}
{#1}&{#2}\\
{#3}&{#4}
\end{smallmatrix}
\right]}
\newcommand{\colthree}[3]{
\left[
\begin{matrix}
{#1}\\
{#2}\\
{#3}
\end{matrix}
\right]}
%
%Redefining sections as problems
%
\makeatletter
\newenvironment{problem}{\@startsection
       {section}
       {1}
       {-.2em}
       {-3.5ex plus -1ex minus -.2ex}
       {2.3ex plus .2ex}
       {\pagebreak[3]%forces pagebreak when space is small; use \eject for better results
       \large\bf\noindent{Problem }
       }
       }
       {%\vspace{1ex}\begin{center} \rule{0.3\linewidth}{.3pt}\end{center}}
       }
\makeatother


%
%Fancy-header package to modify header/page numbering 
%
\usepackage{fancyhdr}
\pagestyle{fancy}
%\addtolength{\headwidth}{\marginparsep} %these change header-rule width
%\addtolength{\headwidth}{\marginparwidth}
\lhead{Problem \thesection}
\chead{} 
\rhead{\thepage} 
\lfoot{\small\scshape 18.785 Analytic Number Theory} 
\cfoot{} 
\rfoot{\footnotesize PS \# 2} % !! Remember to change the problem set number
\renewcommand{\headrulewidth}{.3pt} 
\renewcommand{\footrulewidth}{.3pt}
\setlength\voffset{-0.25in}
\setlength\textheight{648pt}
\allowdisplaybreaks[1]

%%%%%%%%%%%%%%%%%%%%%%%%%%%%%%%%%%%%%%%%%%%%%%%
%
%Contents of problem set
%    
\begin{document}
\title{18.785 Analytic Number Theory Problem Set \#2}% !! Remember to change the problem set number
\author{Holden Lee}
\date{2/13/11}% !! Remember to change the date
\maketitle
\thispagestyle{empty}

%Example problems
\begin{problem}{\it (Commensurability)}
\subprob{(A)}
It follows directly from the definition that commensurability is reflexive and symmetric. We prove it is transitive. Suppose $\Ga\sim \Ga'$ and $\Ga'\sim \Ga''$. Then
\[
[\Ga:\Ga\cap \Ga']<\infty,\quad [\Ga':\Ga'\cap \Ga'']<\infty.
\]
Note that there is an injection (of sets)
\[
\frac{\Ga\cap \Ga'}{\Ga\cap \Ga'\cap \Ga''}\hr \frac{\Ga'}{\Ga'\cap \Ga''}.
\]
Hence
\begin{align*}
[\Ga:\Ga\cap \Ga'']\leq [\Ga:\Ga\cap \Ga'\cap \Ga'']
&=[\Ga:\Ga\cap \Ga'][\Ga\cap \Ga':\Ga\cap \Ga'\cap\Ga'']\\
&\leq [\Ga:\Ga\cap \Ga'][\Ga':\Ga'\cap\Ga'']<\infty.
\end{align*}
By symmetry $[\Ga'':\Ga\cap \Ga'']<\infty$ as well. Hence $\Ga\sim \Ga''$.

Hence commensurability is an equivalence relation.\\

\subprob{(B)}
Suppose $\Ga$ is discrete and commensurable with $\Ga'$; we will show $\Ga'$ is discrete. The subgroup $\Ga\cap \Ga'$ is discrete, so we may replace $\Ga$ with $\Ga\cap \Ga'$ and assume $\Ga\subeq \Ga'$.


%coset representatives of $\Ga$ in $\Ga'$, where $n=[\Ga':\Ga]<\infty$. %Let $x\in \Ga'$. 
Since $\{1\}$ is open in $\Ga$, there exists an open set $U\subeq \Ga'$ such that $U\cap \Ga=\{1\}$.
For each coset of $\Ga$ that $U$ intersects, choose an element $x_i$.
Let $x_1=1,x_2,\ldots, x_n$ be the chosen elements. ($n$ is finite since $[\Ga':\Ga]<\infty$.)
Since multiplication is a homeomorphism, $x_i^{-1}U$ are all open in $\Ga'$. 
Let $V=\bigcap_{i=1}^n x_i^{-1}U$.
If $x\in V$ then $x_ix\in U$ for all $i$. However the $x_ix$ are all in different cosets of $\Ga$ in $\Ga'$, so they must represent all $n$ cosets that intersect $U$, in particular, $\Ga$. Hence $x_ix\in \Ga$ for some $i$. Since $x_ix\in U$, from definition of $\Ga$ we get $x_ix=1$.
%and $x\in x_i^{-1}\Ga \cap U$. Hence $x=x_i^{-1}$, and by replacing $U$ with $\bigcap_{i=1}^n x_i^{-1}U$ we may assume $U\subeq \{x_i^{-1}|1\leq i\leq n\}$.
Hence $x=x_i^{-1}$. This shows that $V\subeq \{x_i^{-1}|1\leq i\leq n\}$. Note  $1\in V$.

Since $G$ is a $T_1$-space, for $2\leq i\leq n$ there exists a neighborhood $W_i$ around $1$ containing $1$ but not $x_i^{-1}$. Then $W=\bigcap_{i=2}^n W_i$ is a neighborhood around $1$ containing $1$ but none of the other $x_i$. Then $V\cap W=\{1\}$ is open in $\Ga'$. All translates of $\{1\}$ in $\Ga'$ are open in $\Ga'$ so $\Ga'$ is discrete.\\

\subprob{(C)}
\begin{lem}\label{p2-1-l1}
Let $X$ be a locally compact topological space on which the group $H$ acts.  Then $H\backslash X$ is locally compact.
\end{lem}
\begin{proof}
Let $\bar x$ be a point in $H\bs X$; suppose it is the image of $x\in X$ under $\pi:X\to H\bs X$. By local compactness of $X$ there exists a compact set $K_x$ around $x$ containing a neighborhood $U_x$ of $x$. Let $V_x=\pi (U_x)$. Then
\begin{equation}\label{p2-1-1}
\pi^{-1}(V_x)=\bigcup_{h\in H} h U_x,
\end{equation}
which is open because the translates $h U_x$ are all open. Since $\pi$ is a quotient map, $V_x$ is open. Now $\pi(K_x)$ is a compact subset of $H\bs X$ containing $V_x$. Hence $H\bs X$ is locally compact.
\end{proof}
\begin{lem}\label{p2-1-l2}
Let $X$ be a topological space on which the group $H$ acts.
Suppose $H\bs X$ is compact, $H'\subeq H$, and $[H:H']$ is finite. 
%\begin{enumerate}
%\item
%Then $H\backslash X$ is locally compact.
%\item
%If $H\backslash X$ is compact, $H'\subeq H$ acts on $X$, and $[H:H']$ is finite, 
Then $H'\bs X$ is compact.
%\end{enumerate}
\end{lem}
\begin{proof}
Let $\bar x$ be a point in $H\bs X$; suppose it is the image of $x\in H'\bs X$ under $\pi:H'\bs X\to H\bs X$. %By local compactness of $H'\bs X$ there exists a compact set $K_x$ around $x$ containing a neighborhood $U_x$ of $x$. Let $V_x=\pi (U_x)$. Then
%\begin{equation}\label{p1-1-2}
%\pi^{-1}(V_x)=\bigcup_{h\in H} h U_x,
%\end{equation}
%which is open because the translates $h U_x$ are all open. Since $\pi$ is a quotient map, $V_x$ is open. %%Now $\pi(K_x)$ is a compact subset of $H\bs X$ containing $V_x$. %Hence $H\bs X$ is locally compact.
Define $U_x,V_x,K_x$ as in Lemma~\ref{p2-1-l1} but with the quotient map $\pi:H'\bs X\to H\bs X$ instead. As shown, $V_x$ is open.

%For the second part, s
Since $H\bs X$ is compact and covered by $V_x$, there exist $x_1,\ldots, x_n$ such that $H\bs X=\bigcup_{m=1}^n V_{x_m}$. 
Let $h_1,\ldots, h_l$ be coset representatives of $H'$ in $H$, where $l=[H:H']<\infty$. Then
\[
\bigcup_{m=1}^n
\bigcup_{k=1}^l
 h_kK_{x_m}\supeq 
\bigcup_{m=1}^n
\bigcup_{k=1}^l
 h_kU_{x_m}\supeq
\bigcup_{m=1}^n 
\pi^{-1}(V_{x_m})
= \pi^{-1}\pa{\bigcup_{m=1}^n V_{x_m}}
=\pi^{-1}(H\bs X)
=H'\bs X.
\]
Now each $K_{x_m}$ (and hence each $h_kK_{x_m}$) is compact, so $H'\bs X$ is a finite union of compact subsets and hence compact.
\end{proof}
Suppose that the conditions in the problem hold and $\Ga\bs G$ is compact.
Since $G$ is locally compact, from Lemma~\ref{p2-1-l1}, $\Ga'\cap \Ga\bs G$ is locally compact.
Thus we can apply Lemma~\ref{p2-1-l2} with $X=G$, $H=\Ga$, and $H'=\Ga'\cap \Ga$, to find that $\Ga\cap \Ga'\bs G$ is compact.
Then the quotient space $\frac{\Ga\cap\Ga' \bs G}{\Ga'}=\Ga'\bs G$ is also compact, as needed. %Hence it suffices to show that if $\Ga\bs G$ is compact, then 

%We may assume $\Ga'\subeq \Ga$, and replace $G$ with $\Ga'\bs G$ and $\Ga$ with $\Ga'\bs \Ga$. Since $[\Ga:\Ga']$ was assumed finite, we now have that $\Ga$ is finite. Note $\Ga\bs G$ is compact by the lemma.
%Then we are left to show that if $\Ga\bs G$ is compact, $G$ is locally compact, and $\Ga$ is finite, then $G$ is compact. 

\end{problem}
\begin{problem}{\it (Geodesics)}
\subprob{(A)}
Suppose $L$ is the half line defined by $\Re (z)=k$. Then the fractional linear transformation $\ga(z)=z-k$, i.e. given by $\ga=\pa{\begin{smallmatrix}1&-k\\0&1\end{smallmatrix}}$ maps $L$ to the positive imaginary axis.

Now suppose $L$ is a semicircle centered at $(x,0)$ with radius $r$. First translate $-x$ units to the origin, and then dilate by $\rc r$. Thus it suffices to find a fractional linear transformation sending the unit semicircle centered at the origin to the positive imaginary axis. We claim that $\ga(z)=\frac{2z+2}{-z+1}$ works. Indeed, for $t=\cos \theta$, $u=\sin\theta$, $t+ui$ on the semicircle,
\begin{align*}
\ga(t+ui)&=\frac{2[(t+1)+ui]}{(-t+1)-ui}\\
&=\frac{2[(t+1)+ui][(-t+1)+ui]}{(t-1)^2+u^2}\\
&=\frac{2[(1-t^2-u^2)]}{(t-1)^2+u^2}\\
&=\frac{4ui}{(t-1)^2+u^2}\\
&=\frac{4\sin\theta}{(\cos\theta-1)^2+\sin^2\theta}i\\
&=\frac{2\sin\theta}{1-\cos\theta}i\\
&=2\pa{\cot\frac{\theta}{2}}i
\end{align*}
so $\ga$ maps the semicircle surjectively to the positive imaginary axis.\\

\subprob{(B)}
Given any two points in $\mathcal H$, there is a half-line or semicircle $L$ going through them and orthogonal to the real axis. Indeed, if $x,y$ have the same real part then the first case holds; in the second case $x,y$ are on the circle centered at the intersection of their perpendicular bisector with the real axis. By (A) an element $\ga\in \SL_2(\R)$ transforms $L$ into the positive imaginary axis. Suppose it sends $x,y$ to $ai,bi$ with $a<b$.

We show that the unique geodesic between $ai$ and $bi$ is the vertical line segment joining them. 
Since fractional linear transformations are isometries under the measure $d\mu=\frac{ds}{y}$, applying $\ga^{-1}$ we may then conclude that the geodesic between $x$ and $y$ is the segment or arc of $L$ in between them.

Let $(x(t),y(t))$ be such that $x(0)+y(0)i=ai$ and $x(1)+y(1)i=bi$. Then
\begin{align*}
\int\, ds&=\int\rc{y} \sqrt{dx^2+dy^2}\\
&=\int_0^1\rc{y(t)}\sqrt{x'(t)^2+y'(t)^2}\,dt\\
&\geq\int_0^1\frac{y'(t)}{y(t)}\,dt\\
&=\left.\ln(y(t))\right|^b_{t=a}\\
&=\ln b-\ln a
\end{align*}
with equality only if $x'(t)=0$ for all $t$, i.e. $x(t)\equiv 0$, and $y'(t)$ is always positive, i.e. $y(t)$ is increasing from $a$ to $b$. This path is just the line segment from $ai$ to $bi$.
\end{problem}
\begin{problem} {\it (Set of elliptic points has no limit point)}
Suppose that $x$ is an elliptic point of $\Ga$. Since $\Ga$ is a discrete subgroup of $G$, by Proposition 3.1.2(b), there exists a neighborhood $U$ of $x$ such that if $\ga\in \Ga$ and $U\cap \ga U\neq \phi$ then $\ga x=x$. Suppose $y$ is an elliptic point of $\Ga$ and $y\in U$. Then there exists $\ga$ so that $\ga y=y$. Then $y\in U\cap \ga U$ so $U\cap\ga U\neq \phi$ and $\ga x=x$. But $\ga$ can only fix one point in $\mathcal H$, so
$y=x$. Thus $U\cap \Ga=\{x\}$, and $x$ is not a limit point.
\end{problem}
\begin{problem} {\it (Fundamental domains have the same volume)}
Let $X$ be the topological space and $D'$ and $D$ be two fundamental domains. We assume the boundaries of $D$ and $D'$ have zero volume. %Let $\text{Bd}(Y)$ denote the boundary $\overline{Y}-Y$ of $Y$.

Note that $d\mu$ is invariant under $\ga$ because it is the Haar measure. Note $\Ga$ is countable as it is a discrete subgroup of $PSL_2(\R)$, which has a countable base. Hence by countable additivity (noting that  $D'\cap \ga(%\bar 
D)$ for different $\ga$ %have empty interior, as pairwise intersections of $D'\cap \ga(\bar D)$ 
are disjoint) %they cannot have a common intersection with $D$ since 
\begin{align*}
\mu(D')&=\sum_{\ga\in \Ga} \mu(D'\cap \ga(%\bar 
D))\\
&=\sum_{\ga\in \Ga}\mu(%\bar 
D\cap \ga^{-1}(D'))&\text{(invariance of }d\mu\text{ under }\ga^{-1})\\
&=\mu\pa{\bigcup_{\ga\in \Ga}%\bar
 D\cap \ga^{-1}(D')},
\end{align*}
the last step following since $%\bar{D}
D \cap \ga^{-1}(D')$ are disjoint (otherwise there would be two $\Ga$-equivalent points in $D'$). 
Now $\bigcup_{\ga\in \Ga} D\cap \ga^{-1}(D')$ is contained in $\bar D$. Hence $\mu(D')\leq \mu(\bar D)=\mu(D)$. Similarly $\mu(D)\leq \mu(D')$. Hence $\mu(D)=\mu(D')$.

%Now we claim that $E=\bigcup_{\ga\in \Ga} D\cap \ga^{-1}(D')$ has closure equal to $\bar D$. Otherwise there exists a point $x$ and an open neighborhood $U\ni x$ such that $U\cap E=\phi$. Let $U'=U\cap D$ (which is open). Then $V=\bigcup_{\ga\in \Ga} \ga(U')$ is open, and hence $X-V$ is closed. But $D'\subeq X-V$ and so $\bigcup_{\ga\in \Ga}\bar{D'}\subeq X-V$, a contradicting that it equals $X$. So $\bar E=\bar D$, and the volume equals $\mu(E)=\mu(\bar E)=\mu(\bar D)=\mu(D)$ as needed.
%Now the boundary of $\bigcup_{\ga\in \Ga}D\cap \ga^{-1}(D')$ is contained in $\text{Bd}(D)\cup \bigcup_{\ga\in \Ga}\ga^{-1}(\text{Bd}(D'))$. But the boundary of $D'$ has measure 0, so $\ga^{-1}(\bar{D'}-D')$, and hence $E$, has measure 0.
%Then the volume equals 
\end{problem}
\begin{problem} {\it (Co-compact iff no parabolic elements)}
Suppose $\Ga$ has no parabolic elements. Then $\mathcal{H}^*=\mathcal{H}\cup P_{\Ga}=\mathcal{H}$. Hence $\Ga\bs \mathcal H=\Ga\bs \mathcal H^*$. But the latter is compact because $\Ga$ is Fuschian of the first kind. (Use Proposition 3.6.2(3), which says every fundamental domain has finite volume, and Siegel's Theorem.)

Now suppose $\Ga$ has a parabolic element. Let $z$ be a cusp of $\Ga$, considered in $\Ga\bs\mathcal H^*$. Since $\Ga\bs\mathcal H^*$ is a Riemann surface we can take a neighborhood $U$ around $z$ homeomorphic to the unit disc $D$ in $\C$, such that $\bar{U}\cong \bar{D}$. %We may take $U$ small enough so that $U$ contains no other cusp; i.e. every other point of $U$  is in $\mathcal H$. 
Now consider $\bar{U}\cap\Ga\bs \mathcal{H}$. If $\Ga\bs\mathcal H$ were compact, then $\bar{U}\cap \Ga\bs \mathcal H$ would be compact (since it is a closed subset). Now
$\bar{U}\cap  \Ga\bs \mathcal H=\bar{U}-P_{\Ga}$. However, there can be at most a countable number of cusps since $\Ga$ is countable. Hence $\bar{U}\cap \mathcal H\cong \bar{D}-S$ where $S$ is a countable set containing 0. Since every disc around 0 contains uncountably many elements, 0 is a limit point of $\bar{D}-S$. Hence $\bar{D}-S\subeq \C$ is not closed and hence not compact, a contradiction. 
Hence $\mathcal H$ is not compact.
\end{problem}
\begin{problem} {\it(Fundamental domain for $\Ga_0(N)$)}
\begin{enumerate}
\item Find coset representatives: let $S=\smatt 0{-1}{1}0$ and $T=\smatt1101$. %and $T'=\smatt 1{-1}01$. %Start with a list containing a single matrix $I=\smatt 1001$. At each stage, place the matrices currently on the list in the set $A$, and multiply those matrices by $S$ and $T$ on the left.
Consider a ``tree" as follows. Let $I=\smatt 1001$ be the root, with a single branch leading to $S$. At the $n$th stage, each leaf gives rise to several more vertices: if the leaf is labeled with the matrix $M$, then let its descendants be labeled with $MS$, $MT$, or $MT^{-1}$. If $M$ was obtained from the previous stage by multiplication by $S$, then we do not include the $MS$ branch (because $S^2=-I$ is the identity transformation). If $M$ was obtained from the previous stage by multiplication by $T, T^{-1}$ then we do not include the $T^{-1},T$ branch, respectively, since these are inverses.

Now choose elements of the tree as follows. Choose $I$, and at stage $n$, look at the marked elements in the $n$th level of the tree. Look at all the descendants of those elements, and mark those whose $\Ga_0(N)$-coset has not been represented by previous marked elements. Continue until we reach a level where no elements are marked. We will stop since $\Ga_0(N)$ has finite index in $\SL_2(\Z)$, and we get all the coset representatives since $S=\smatt 0{-1}{1}0$ and $T=\smatt1101$ generate $\SL_2(\Z)$. (For each coset there's an element that can be written as a word containing a minimal number of $S$, $T$, and $T^{-1}$'s; such an element will be picked, as no subword ending at the rightmost letter will be replaceable with a smaller subword representing the same coset.)

(Note that we do not have a branch from $I$ to $T$ because $T\in \Ga_0(N)$.)

\item Let $D$ be the standard fundamental domain. For each vertex on the graph, consider its matrix $M$ and associate to it the region $M\bar D$.

Note that $T\bar D$ is simply $\bar D$ translated by 1, so is adjacent to $\bar D$ (by the side defined by $\Re(z)=\rc{2}$, $\Im(z)\geq \sqrt 3/2$). Note that $S\bar D$ is the inversion of $\bar D$, which is adjacent to $\bar D$ via the circular arc of radius 1 in $\bar D$. Hence for any matrix $M$, $M\bar D$ and $MS\bar D$ are adjacent, and $M\bar D$ and $MT\bar D$ are adjacent. Thus the fundamental regions corresponding to the matrices that are adjacent in the tree are adjacent, and the regions corresponding to the marked matrices form a connected domain $D'$.

%Let $M_1,\ldots, M_n$ be the coset representatives.

\item Find the bounding geodesics: Each marked matrix in the tree such that its descendants are not both marked has a side that is a side of $D'$. To find these sides, apply the fractional transformation corresponding to that matrix to the vertices of $D$: $\pm\rc{2}+\frac{\sqrt3}{2}i,i,\infty$. Then connect them with geodesics (vertical rays or semicircles orthogonal to the real axis) and pick out the boundary.

(If an element has all its descendants marked, then it is surrounded on all sides by other regions in $D'$.)

\item The interior of $D'$ is a fundamental domain. First we check that no two elements in the interior of $D'$ are related by an element of $\Ga_0(N)$. Let $q_1,q_2$ be two elements in $D'^{\circ}$. %Since $D$ is a fundamental domain for $\SL_2(\Z)$, there exist $\ga_1,\ga_2\in \SL_2(\Z)$ and $p_1,p_2\in \overline{D}$ such that $\ga_1 p_1=q_1,\ga_2 p_2=q_2$. Then $
Then there exist coset representatives $M_1,M_2$ such that $q_1=M_1p_1$ and $q_2=M_2p_2$ where $p_1,p_2\in \overline{D}$. Supposing that $q_1,q_2$ are related by an element of $\Ga_0(N)$, we have that $p_1,p_2$ are related by an element of $\SL_2(\Z)$. 
First assume $p_1$ and $p_2$ are in $D$; then they must be equal since $D$ is a fundamental region for $\SL_2(\Z)$. Then we must have $M_1=M_2$ and hence $q_1=q_2$. Now suppose $p_1,p_2$ are on the boundary of $D$. Then $q_1,q_2$ must be on the boundary of some region in $D'$. Let $B$ be the union of these boundaries; note $B$ has empty interior.

Let $S$ be the set of points of $D'^{\circ}$ that are $\Ga_0(N)$-equivalent to a different point in $D'^{\circ}$. We claim that $S$ is open. Then since $S\subeq B$, it will follow that $S=\phi$. 

Take $p\in S$; suppose $\ga p=q$, $\ga\in \Ga_0(N)$, $p\neq q$. Since $\ga$ is a homeomorphism, there exist disjoint neighborhoods $U,V$ around $p,q$ contained in $D'^{\circ}$ so that $\ga$ is a homeomorphism $U\to V$. Then $p\in U\subeq S$. Hence $S$ is open, as needed.

%However, since $D'$ is open it is easy to see that the set of points of $D'$ that are $\Ga_0(N)$-equivalent to another point in $D'$ is open. Hence the set of such points is $\phi$.

Finally we show $D'$ ``tiles" $\cal H$. Indeed,
\[
\bigcup_{\ga\in \Ga_0(N)}\ga D'=\bigcup_{\ga\in \Ga_0(N)} \bigcup_{M\text{ coset representative}}\ga M\overline{D}
=\bigcup_{\ga\in \SL_2(\Z)} \ga \overline{D}=\mathcal H.
\]
\end{enumerate}
\end{problem}
\end{document}
