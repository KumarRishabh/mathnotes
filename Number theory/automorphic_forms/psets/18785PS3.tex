%%%This is a science homework template. Modify the preamble to suit your needs. 

\documentclass[12pt]{article}

\makeatother
%AMS-TeX packages
\usepackage{amsmath}
\usepackage{amssymb}
\usepackage{amsthm}
\usepackage{array}
\usepackage{amsfonts}
\usepackage{cancel}
\usepackage[all,cmtip]{xy}%Commutative Diagrams
\usepackage[pdftex]{graphicx}
\usepackage{float}
%geometry (sets margin) and other useful packages
\usepackage[margin=1in]{geometry}
\usepackage{sidecap}
\usepackage{wrapfig}
\usepackage{verbatim}
\usepackage{mathrsfs}
\usepackage{marvosym}
\usepackage{hyperref}
\usepackage{graphicx,ctable,booktabs}

\newtheoremstyle{norm}
{3pt}
{3pt}
{}
{}
{\bf}
{:}
{.5em}
{}

\theoremstyle{norm}
\newtheorem{thm}{Theorem}[section]
\newtheorem{lem}[thm]{Lemma}
\newtheorem{df}{Definition}
\newtheorem{rem}{Remark}
\newtheorem{st}{Step}
\newtheorem{pr}[thm]{Proposition}
\newtheorem{cor}[thm]{Corollary}
\newtheorem{clm}[thm]{Claim}

%Math blackboard, fraktur, and script commonly used letters
\newcommand{\A}[0]{\mathbb{A}}
\newcommand{\C}[0]{\mathbb{C}}
\newcommand{\sC}[0]{\mathcal{C}}
\newcommand{\cE}[0]{\mathscr{E}}
\newcommand{\F}[0]{\mathbb{F}}
\newcommand{\cF}[0]{\mathscr{F}}
\newcommand{\cG}[0]{\mathscr{G}}
\newcommand{\sH}[0]{\mathscr H}
\newcommand{\Hq}[0]{\mathbb{H}}
\newcommand{\cI}[0]{\mathscr{I}}%ideal sheaf
\newcommand{\N}[0]{\mathbb{N}}
\newcommand{\Pj}[0]{\mathbb{P}}
\newcommand{\sO}[0]{\mathcal{O}}
\newcommand{\cO}[0]{\mathscr{O}}
\newcommand{\Q}[0]{\mathbb{Q}}
\newcommand{\R}[0]{\mathbb{R}}
\newcommand{\Z}[0]{\mathbb{Z}}
%Lowercase
\newcommand{\ma}[0]{\mathfrak{a}}
\newcommand{\mb}[0]{\mathfrak{b}}
\newcommand{\fg}[0]{\mathfrak{g}}
\newcommand{\vi}[0]{\mathbf{i}}
\newcommand{\vj}[0]{\mathbf{j}}
\newcommand{\vk}[0]{\mathbf{k}}
\newcommand{\mm}[0]{\mathfrak{m}}
\newcommand{\mfp}[0]{\mathfrak{p}}
\newcommand{\mq}[0]{\mathfrak{q}}
\newcommand{\mr}[0]{\mathfrak{r}}
%Letter-related
\newcommand{\bb}[1]{\mathbb{#1}}
\providecommand{\cal}[1]{\mathcal{#1}}
\renewcommand{\cal}[1]{\mathcal{#1}}
%More sequences of letters
\newcommand{\chom}[0]{\mathscr{H}om}
\newcommand{\fq}[0]{\mathbb{F}_q}
\newcommand{\fqt}[0]{\mathbb{F}_q^{\times}}
\newcommand{\sll}[0]{\mathfrak{sl}}
%Shortcuts for symbols
\newcommand{\nin}[0]{\not\in}
\newcommand{\opl}[0]{\oplus}
\newcommand{\ot}[0]{\otimes}
\newcommand{\rc}[1]{\frac{1}{#1}}
\newcommand{\rra}[0]{\rightrightarrows}
\newcommand{\send}[0]{\mapsto}
\newcommand{\sub}[0]{\subset}
\newcommand{\subeq}[0]{\subseteq}
\newcommand{\supeq}[0]{\supseteq}
\newcommand{\nsubeq}[0]{\not\subseteq}
\newcommand{\nsupeq}[0]{\not\supseteq}
%Shortcuts for greek letters
\newcommand{\al}[0]{\alpha}
\newcommand{\be}[0]{\beta}
\newcommand{\ga}[0]{\gamma}
\newcommand{\Ga}[0]{\Gamma}
\newcommand{\de}[0]{\delta}
\newcommand{\De}[0]{\Delta}
\newcommand{\ep}[0]{\varepsilon}
\newcommand{\eph}[0]{\frac{\varepsilon}{2}}
\newcommand{\ept}[0]{\frac{\varepsilon}{3}}
\newcommand{\la}[0]{\lambda}
\newcommand{\La}[0]{\Lambda}
\newcommand{\ph}[0]{\varphi}
\newcommand{\rh}[0]{\rho}
\newcommand{\te}[0]{\theta}
\newcommand{\om}[0]{\omega}
%Brackets
\newcommand{\ab}[1]{\left| {#1} \right|}
\newcommand{\ba}[1]{\left[ {#1} \right]}
\newcommand{\bc}[1]{\left\{ {#1} \right\}}
\newcommand{\pa}[1]{\left( {#1} \right)}
\newcommand{\an}[1]{\langle {#1}\rangle}
\newcommand{\fl}[1]{\left\lfloor {#1}\right\rfloor}
\newcommand{\ce}[1]{\left\lceil {#1}\right\rceil}
%Text
\newcommand{\btih}[1]{\text{ by the induction hypothesis{#1}}}
\newcommand{\bwoc}[0]{by way of contradiction}
\newcommand{\by}[1]{\text{by~(\ref{#1})}}
\newcommand{\ore}[0]{\text{ or }}
%Arrows
\newcommand{\hr}[0]{\hookrightarrow}
\newcommand{\xr}[1]{\xrightarrow{#1}}
%Formatting
\newcommand{\subprob}[1]{\noindent\textbf{#1}\\}
%Functions, etc.
\newcommand{\Ann}{\operatorname{Ann}}
\newcommand{\AP}{\operatorname{AP}}
\newcommand{\Ass}{\operatorname{Ass}}
\newcommand{\Aut}{\operatorname{Aut}}
\newcommand{\chr}{\operatorname{char}}
\newcommand{\cis}{\operatorname{cis}}
\newcommand{\Cl}{\operatorname{Cl}}
\newcommand{\Der}{\operatorname{Der}}
\newcommand{\End}{\operatorname{End}}
\newcommand{\Ext}{\operatorname{Ext}}
\newcommand{\Frac}{\operatorname{Frac}}
\newcommand{\FS}{\operatorname{FS}}
\newcommand{\GL}{\operatorname{GL}}
\newcommand{\Hom}{\operatorname{Hom}}
\newcommand{\Ind}[0]{\text{Ind}}
\newcommand{\im}[0]{\text{im}}
\newcommand{\nil}[0]{\operatorname{nil}}
\newcommand{\ord}[0]{\operatorname{ord}}
\newcommand{\Proj}{\operatorname{Proj}}
\newcommand{\PSL}{\operatorname{PSL}}
\newcommand{\Rad}{\operatorname{Rad}}
\newcommand{\rank}{\operatorname{rank}}
\newcommand{\Res}[0]{\text{Res}}
\newcommand{\sign}{\operatorname{sign}}
\newcommand{\SL}{\operatorname{SL}}
\newcommand{\Spec}{\operatorname{Spec}}
\newcommand{\Specf}[2]{\Spec\pa{\frac{k[{#1}]}{#2}}}
\newcommand{\spp}{\operatorname{sp}}
\newcommand{\spn}{\operatorname{span}}
\newcommand{\Supp}{\operatorname{Supp}}
\newcommand{\Tor}{\operatorname{Tor}}
\newcommand{\tr}[0]{\text{trace}}
%Commutative diagram shortcuts
\newcommand{\fiber}[3]{\xymatrix{#1\times_{#3} #2}\ar[r]\ar[d] #1\ar[d] \\ #2 \ar[r] & #3}
\newcommand{\commsq}[8]{\xymatrix{#1\ar[r]^{#6}\ar[d]^{#5} &#2\ar[d]^{#7} \\ #3 \ar[r]^{#8} & #4}}
%Makes a diagram like this
%1->2
%|    |
%3->4
%Arguments 5, 6, 7, 8 on arrows
%  6
%5  7
%  8
\newcommand{\pull}[9]{
#1\ar@/_/[ddr]_{#2} \ar@{.>}[rd]^{#3} \ar@/^/[rrd]^{#4} & &\\
& #5\ar[r]^{#6}\ar[d]^{#8} &#7\ar[d]^{#9} \\}
\newcommand{\back}[3]{& #1 \ar[r]^{#2} & #3}
%Syntax:\pull 123456789 \back ABC
%1=upper left-hand corner
%2,3,4=arrows from upper LH corner, going down, diagonal, right
%5,6,7=top row (6 on arrow)
%8,9=middle rows (on arrows)
%A,B,C=bottom row
%Other
%Other
\newcommand{\op}{^{\text{op}}}
\newcommand{\fp}[1]{^{\underline{#1}}}
\newcommand{\rp}[1]{^{\overline{#1}}}
\newcommand{\rd}[0]{_{\text{red}}}
\newcommand{\pre}[0]{^{\text{pre}}}
\newcommand{\pf}[2]{\pa{\frac{#1}{#2}}}
\newcommand{\pd}[2]{\frac{\partial #1}{\partial #2}}
\newcommand{\bs}[0]{\backslash}
\newcommand{\ol}[1]{\overline{#1}}
%Matrices
\newcommand{\coltwo}[2]{
\left[
\begin{matrix}
{#1}\\
{#2} 
\end{matrix}
\right]}
\newcommand{\matt}[4]{
\left[
\begin{matrix}
{#1}&{#2}\\
{#3}&{#4}
\end{matrix}
\right]}
\newcommand{\smatt}[4]{
\left[
\begin{smallmatrix}
{#1}&{#2}\\
{#3}&{#4}
\end{smallmatrix}
\right]}
\newcommand{\colthree}[3]{
\left[
\begin{matrix}
{#1}\\
{#2}\\
{#3}
\end{matrix}
\right]}
%
%Redefining sections as problems
%
\makeatletter
\newenvironment{problem}{\@startsection
       {section}
       {1}
       {-.2em}
       {-3.5ex plus -1ex minus -.2ex}
       {2.3ex plus .2ex}
       {\pagebreak[3]%forces pagebreak when space is small; use \eject for better results
       \large\bf\noindent{Problem }
       }
       }
       {%\vspace{1ex}\begin{center} \rule{0.3\linewidth}{.3pt}\end{center}}
       }
\makeatother


%
%Fancy-header package to modify header/page numbering 
%
\usepackage{fancyhdr}
\pagestyle{fancy}
%\addtolength{\headwidth}{\marginparsep} %these change header-rule width
%\addtolength{\headwidth}{\marginparwidth}
\lhead{Problem \thesection}
\chead{} 
\rhead{\thepage} 
\lfoot{\small\scshape 18.785 Analytic Number Theory} 
\cfoot{} 
\rfoot{\footnotesize PS \# 3} % !! Remember to change the problem set number
\renewcommand{\headrulewidth}{.3pt} 
\renewcommand{\footrulewidth}{.3pt}
\setlength\voffset{-0.25in}
\setlength\textheight{648pt}
\allowdisplaybreaks[1]

%%%%%%%%%%%%%%%%%%%%%%%%%%%%%%%%%%%%%%%%%%%%%%%
%
%Contents of problem set
%    
\begin{document}
\title{18.785 Analytic Number Theory Problem Set \#3}% !! Remember to change the problem set number
\author{Holden Lee}
\date{2/21/11}% !! Remember to change the date
\maketitle
\thispagestyle{empty}

%Example problems
\begin{problem}{\it (Invariant measure)}
\subprob{(A) Bruhat decomposition}
Since $S\in \SL_2(\R)$, $B\cup BSB\subeq \SL_2(\R)$.

Now we show that for $M=\smatt abcd\in \SL_2(\R)$, $M\in B$ iff $c= 0$ and $M\in BSB$ iff $c\neq 0$. The first is obvious. For the second, note that the matrices in $BSB$ are in the form
\[
\matt ef0{e^{-1}}\matt 0{-1}10\matt gh0{g^{-1}}
=
\matt {fg}{fh-eg^{-1}}{e^{-1}g}{e^{-1}h},\quad e,g\neq 0.
\]
The lower left entry is hence nonzero. Conversely, if $c=0$, then $M$ can be written in the above form by letting
\begin{align*}
e&=c^{-1}\\
f&=a\\
g&=1\\
h&=dc^{-1}.
\end{align*}
(Note this gives $b=\frac{ad-1}{c}$ which is true since $\det(M)=1$.)
Hence $M\in BSB$. This shows $\SL_2(\R)=B\sqcup BSB$.\\

\subprob{(B)}
\begin{enumerate}
\item $|y|^{-2}dxdy$ is invariant under diagonal matrices: The matrix $\matt a00{a^{-1}}\in \SL_2(\R)$ corresponds to the transformation $z\mapsto c z$ or $x+yi\mapsto cx+cyi$, where $c=a^2$. Then $|y|^{-2}dxdy$ becomes
\[
|cy|^{-2}d(cx)d(cy)=|y|^{-2}dxdy.
\]
\item $|y|^{-2}dxdy$ is invariant under unipotent matrices: The matrix $\matt 1a01$ corresponds to the transformation $z\mapsto z+a$, or $x+yi\mapsto (x+a)+yi$. Then $|y|^{-2}dxdy$ becomes
\[
|y|^{-2}d(x+a)dy=|y|^{-2}dxdy.
\]
\item $|y|^{-2}dxdy$ is invariant under $S$: $S$ corresponds to the transformation $z\mapsto-\rc z$, or $r\cis \theta \mapsto -\rc{r}\cis(-\theta)$. Noting that the Jacobian from rectangular to polar coordinates is $r$,
\[
|y|^{-2}dxdy= \frac{r}{|r\sin\theta|^2}drd\theta.
\]
Under $S$, this gets sent to
\[
\frac{\rc{r}}{\ab{\rc{r^2}\sin\theta}^2}d\pa{-\rc{r}}d(-\theta)
=
\frac{-r^3}{|r\sin\theta|^2}\cdot \frac{dr}{r^2}\cdot (-d\theta)
=
\frac{r}{|r\sin\theta|^2}drd\theta,
\]
which is the same as the original expression.
\end{enumerate}
Now note that $B=DU$, where $D$ is the subgroup of diagonal matrices in $\SL_2(\R)$ and $U$ is the subgroup of unipotent matrices, as
\[
\matt a00{a^{-1}}\matt 1{ba^{-1}}01=\matt ab0{a^{-1}}.
\]
Since $|y|^{-2}$ is invariant under $D$, $U$, and $S$, it is invariant under $B$ and $BSB$. Hence by the Bruhat decomposition it is invariant under $\SL_2(\R)$.
\end{problem}
\begin{problem}{\it (Genus)}
\subprob{(A)}
By the Riemann-Hurwitz formula, for $f:\cal R\to \cal R'$ a holomorphic map of compact Riemann surfaces that is $m$-to-1 at finitely many points,
\begin{equation}\label{rhur}
2g(\cal R)-2=m(2g(\cal R')-2)+\sum_{p\in \cal R} (e_p-1),
\end{equation}
where $e_p$ is the ramification index of $p$. We also have the following: For $p'\in \cal R'$,
\begin{equation}\label{sumram}
m=\sum_{p\in \cal R, f(p)=p'}e_p.
\end{equation}

Let $\Ga'=\SL_2(\Z)$. For a group $G\subeq \SL_2(\Z)$, let $\overline{G}$ denote its image in $\PSL_2(\Z)$. 
Putting in $\cal R'=\Ga'\bs\cal H^*$ and $\cal R=\Ga\bs \cal H^*$, and noting that $g(\SL_2(\Z)\bs \cal H^*)=0$, $m=\mu=[\Ga':\Ga]$,~(\ref{rhur}) becomes
\[
g(\Ga\bs \cal H^*)=1-\mu+\rc{2}\sum_{p\in \Ga\bs \cal H^*} (e_p-1).
\]

Now the only nonequivalent cusp of $\Ga'$ is $\infty$ and the only nonequivalent elliptic points of $\Ga'$ %$\Ga\bs\cal H^*$ all lie over 
are $i$ and $\om=e^{\frac{2\pi i}{3}}$, by Proposition 3.5.3. 
Thus the stabilizer of any elliptic point $z$ of $\overline{\Ga'}$ is a subgroup $\ol{\Ga'}_z$ conjugate to one of the following subgroups.
\begin{align}
\overline{\Ga'}_{\omega}&=\bc{
\pm I,\pm \smatt 0{-1}{1}{-1},\pm \smatt {-1}{1}{-1}{0}
}
\label{stabom}\\
\overline{\Ga'}_{i}&=\bc{
\pm I,\pm \smatt 0{-1}{1}{0}
}\label{stabi}
\end{align}
These groups have order 2 and 3 in $\PSL_2(\Z)$, respectively. If $z\in \cal H^*$ elliptic in $\Ga'=\SL_2(\Z)$ remains elliptic under $\Ga$, we must have $\Ga'_z\subeq \Ga$; in this case $e_z=[\overline{\Ga'}_z:\overline{\Ga}_z]=1$. Otherwise, $e_z=[\overline{\Ga'}_z:\overline{\Ga}_z]=|\ol{\Ga'}_z|$.

First consider the points of $\Ga\bs \cal H^*$ lying over $i$. Note $\nu_2$ is the number of such points with $e_z=1$. Let $a$ be the number of points with $e_z=2$.
 By~(\ref{sumram}),
\[\mu=2a+\nu_2,\]
so $a=\frac{\mu-\nu_2}{2}$. Hence
\begin{equation}\label{v2}
\sum_{p\in \Ga\bs \cal H^*,\, f(p)=i}(e_p-1)=a= \frac{\mu-\nu_2}{2}.
\end{equation}

Next consider the points of $\Ga\bs \cal H^*$ lying over $\omega$. Note $\nu_3$ is the number of points with $e_z=1$. Let $b$ be the number of points with $e_z=3$.
 By~(\ref{sumram}),
\[\mu=3a+\nu_3,\]
so $a=\frac{\mu-\nu_3}{3}$. Hence
\begin{equation}\label{v3}
\sum_{p\in \Ga\bs \cal H^*,\, f(p)=\omega}(e_p-1)=a= \frac{2(\mu-\nu_3)}{3}.
\end{equation}

Finally consider the cusps of $\overline{\Ga}$. We claim that if $p$ is a cusp of $\Ga'$, then it is a cusp of $\Ga$. Indeed, if $\ga$ is a parabolic element of $\Ga'$ fixing $p$, then since $\Ga$ has finite index in $\Ga'$, some nonzero power $\ga^m$ is contained in $\Ga$; it fixes $p$. (Note $\ga^m\neq I$ since $\ga$ has infinite order.) Hence using~(\ref{sumram}),
\begin{equation}\label{vinf}
\sum_{p\in \Ga\bs \cal H^*,\, f(p)=\infty}(e_p-1)=\pa{\sum_{p\in \Ga\bs \cal H^*,\, f(p)=\infty}e_p}-\nu_{\infty}=\mu-\nu_{\infty}.
\end{equation}

We've accounted for all elliptic points and cusps of $\Ga$. Putting ~(\ref{v2}),~(\ref{v3}), and~(\ref{vinf}) into~(\ref{rhur}) gives
\begin{equation}\label{genus}
g=1+\frac{\mu}{12}-\frac{\nu_2}{4}-\frac{\nu_3}{3}-\frac{\nu_{\infty}}{2}.
\end{equation}

\subprob{(B)}
Below, $p$ will always represent a prime.
\begin{lem}
\[
[\PSL_2(\Z):\overline{\Ga_0(N)}]=N\prod_{p|N}\pa{1+\rc p}.
\]
\end{lem}
\begin{proof}
Let $G$ be the group
\[
\{
(a,y)|a\in (\Z/N\Z)^{\times},y\in \Z/N\Z
\}/\{\pm(1,0)\}
\]
with the operation
\[
(a,y)(a',y')=(aa',ay'+a'^{-1}y).
\]
The fact that $G$ is a group can be shown directly, or by noting that the group structure on $G$ is the ``pushforward" of the group structure on $\Ga_0(N)$ by $\pi$ below.
We claim that
\[
1\to \overline{\Ga(N)}\to \overline{\Ga_0(N)} \xr{\pi}G\to 1
\]
is a short exact sequence, where 
\[
\pi\pa{\matt ab{Nc}d}=(a,b)\bmod N.
\]
We verify:
\begin{enumerate}
\item
$\pi$ is surjective: Given $(\ol{a},\ol{b})\in G$, we can choose $b$ so that $a\equiv \ol{a}\pmod N,b\equiv \ol{b}\pmod N$ so that $\gcd(a,b)=1$.
Let $d$ be an integer such that $ad\equiv 1\pmod N$. By B\'ezout's Theorem we can find $k,l$ so that $ak-lb=\frac{1-ad}{N}$. Then $a(d+kN)-Nlb=1$, and the following  matrix is in $\SL_2(\Z)$.
\[
\pi\pa{\matt ab{Nl}{d+kN}}=(a,b).
\]
\item
$\ker(\pi)=\overline{\Ga(N)}$: The inclusion $\overline{\Ga(N)}\subeq \ker (\pi)$ is clear.
Conversely, if $A=\matt ab{Nc}d\in \Ga_0(N)$, $\pi(A)=(1,0)$, then $a\equiv 1\pmod N$ and $b\equiv 0\pmod N$; moreover $ad-(Nc)d=1$ and $a\equiv 1\pmod N$ imply $b\equiv 1 \pmod N$. 
\end{enumerate}
%$G$ is a group because
%\begin{align*}
%[(a,b,y)(a',b',y')](a'',b'',y'')=(aa'a'',bb'b'',aa'y+ab''y+b'b''y)=(a,b,y)(
%\end{align*}
First suppose $N\neq 2$.
Then $|G|=\rc{2}\ph(N)N$, so
\[
[\PSL_2(\Z):\ol{\Ga_0(N)}]=\frac{[\PSL_2(\Z):\ol{\Ga(N)}]}{|G|}=\frac{\frac{N^3}{2}\prod_{p|N}\pa{1-\rc{p^2}}}{N\prod_{p|N}\pa{1-\rc p}}=N\prod_{p|N}\pa{1+\rc p}.
\]
For $N=2$, $[\PSL_2(\Z),\ol{\Ga(N)}]=6$ and $|G|=2$, so $[\PSL_2(\Z):\ol{\Ga_0(N)}]=3$ (and the above formula works as well).
\end{proof}
\begin{lem}
The equivalence classes of elliptic points of order 2 in $\Ga_0(N)$ are in bijection with the solutions to $a^2+1\equiv 0\pmod N$, and the elliptic points of order 3 in $\Ga_0(N)$ are in bijection with the solutions to $a^2+a+1\equiv 0\pmod N$.
\end{lem}
\begin{proof}
Let $z$ be an elliptic point of order 2 in $\Ga_0(N)$. Its stabilizer subgroup in $\PSL_2(\Z)$ is conjugate to~(\ref{stabi}), and must be the same as the stabilizer subgroup in $\overline{\Ga_0(N)}$. 
Let $\ga z=z$ with $\ga\neq \pm 1$. Then $\ga$ is conjugate to $\smatt 0{-1}{1}0$ and has characteristic polynomial $x^2+1$. It must have trace 0 and be in $\overline{\Ga_0(N)}$. Hence modulo $N$ it is in the form
\begin{equation}\label{formofmat}
\matt ab0{a^{-1}},\quad a+a^{-1}\equiv 0\pmod N.
\end{equation}
This gives $a$ so that $a^2+1\equiv 0\pmod N$.
Note that the map $z\mapsto a$ is well-defined because equivalent $z$ get sent to the same $a$: If $z_1$ and $z_2$ are elliptic points with $\ga_jz_j=z_j,\ga_j\neq \pm I$ and $\tau z_1 =z_2,\tau\in \ol{\Ga_0(N)}$, 
then $\tau \ga_1\tau^{-1}z_2=z_2$ so $\tau \ga_1\tau^{-1}=\ga_2$. Working modulo $N$, we write $\ga_1=\smatt ab0{a^{-1}}$, $\tau=\smatt cd0{c^{-1}}$ and hence
\[
\ga_2=\tau \ga_1\tau^{-1}=\matt a{-ad+c^2b+ca^{-1}d}0{a^{-1}}
\]
which has the same upper-left-hand entry.

Let $m$ be the number of solutions to $a^2+1\equiv 0\pmod N$. Note for every such solution, $\smatt a{1}{-a^2-1}{-a}$ is an elliptic matrix with upper lefr corner $a$, so there are at least $m$ $\Ga_0(N)$-inequivalent elliptic points of order 2.
It suffices to show there are at most $m$ distinct elliptic points of order 2.
\begin{lem}
For each
\[(z,t)\in P:=\frac{(\Z/N\Z)^2-\{(0,0)\}}{(\Z/N\Z)^{\times}}\]
take an integer matrix of the form
$\smatt xyzt$. These matrices
form a set of right coset representatives for ${\Ga_0(N)}$ in $\SL_2(\Z)$ (or of $\overline{\Ga_0(N)}$ in $\PSL_2(\Z)$). 
\end{lem}
\begin{proof}
First note that coset representatives for ${\Ga_0(N)}$ in $\SL_2(\Z)$ correspond to coset representatives for ${\Ga_0(N)}/{\Ga(N)}$ in $\SL_2(\Z/N\Z)$. Thus we work modulo $N$.
We show that the map (of sets)
\begin{align*}
\SL_2(\Z/N\Z)\bs (\Ga_0(N)/\Ga(N))&\to P\\
\matt xyzt&\mapsto(z,t)
\end{align*}
is well-defined and bijective. For each $(z,t)\in P$ we can find a matrix of the above form by B\'ezout, so this map is surjective.

First, it is well-defined: If $\smatt {a^{-1}}b0{a}\in \Ga(N)$, then 
\begin{equation}\label{checkcoset}
\matt {a^{-1}}b0{a}
\matt xyzt=\matt {a^{-1}x+bz}{a^{-1}y+bt}{az}{at}.
\end{equation}
whose bottom row is just the original multiplied by $a$.

It remains to show that the map is injective, i.e. every matrix in the form $\smatt {x'}{y'}{az}{at}\in \SL_2(\Z)$ with $a\in (\Z/N\Z)^{\times}$ is in the same coset. Suppose $\smatt{x}{y}{z}{t}\in \SL_2(\Z/N\Z)$ is the coset representative. Assume $z\neq 0$ (if $z=0$, work with $t$ instead of $z$; the argument is similar).
Then given $\smatt {x'}{y'}{az}{at}\in \SL_2(\Z)$, we have $ax't-ay'z=1$ so $x't-y'z=a^{-1}$. Taking this modulo $\gcd(z,N)$ gives $x't\equiv a^{-1}\pmod{\gcd(z,N)}$, which has a unique solution for $x'$ modulo $\gcd(z,N)$. Hence there are $\frac{N}{\gcd(z,N)}$ possible values of $x'$ modulo $N$. The value of $x'$ uniquely determines $y'$, so there are $\frac{N}{\gcd(z,N)}$ matrices with bottom row $az,at$. 
Fixing $a$ and letting $b$ range over the residues modulo $N$ in~(\ref{checkcoset}), $a^{-1}x+bz$ can take $\frac{N}{\gcd(z,N)}$ values. Hence all the matrices with bottom row $az,at$ are in the coset $\Ga_0(N)\matt{x}{y}zt$, as needed.
\end{proof}
Let $\ga_1,\ga_2\neq \pm I$ be stabilizers for elliptic points $p_1,p_2$ in ${\Ga_0(N)}$, and suppose $p_1,p_2$ are $\Ga_0(N)$-inequivalent. By Proposition 3.5.3, we can write $\ga_j=M_jSM_j^{-1}$, where $S=\smatt 0{-1}{1}0$ and $M_j\in \PSL_2(\Z)$. Write
$M_j=A_jR_j$ where $R_j$ is one of the coset representatives above and $A_j\in \ol{\Ga_0(N)}$. Then
\begin{align*}
\ga_1&=A_1R_1SR_1^{-1}A_1^{-1}\\
\ga_2&=A_2R_2SR_2^{-1}A_2^{-1}
\end{align*}

Let $R=\matt xyzt$ be a coset representative chosen above. Then
\[
RS R^{-1}=\matt{yt+xz}{-x^2-y^2}{t^2+z^2}{-yt-zx}.
\]
In order for this to be in $\Ga_0(N)$, we must have 
\begin{equation}\label{sqeq}
t^2+z^2\equiv 0\pmod N.
\end{equation}
We count the number of $(t,z)\in P$ that make this equation true. Note that $\gcd(t,z)=1$. Let $g=\gcd(t,N)$. If~(\ref{sqeq}) holds then $g|z$ giving $g=1$. Thus we can divide the equation above by $z^2$ and let $x=\frac tz$ to get $x^2+1\equiv 0\pmod N$. Each solution $(t,z)$ corresponds to a solution $x$. Thus there are $m$ coset representatives $R$ such that $RSR^{-1}\in \ol{\Ga_0(N)}$. 

Now if $R_1=R_2$, then $\ga_2=A_2A_1^{-1}\ga_1A_1A_2^{-1}$ so $\ga_1,\ga_2$ are conjugate in $\ol{\Ga_0(N)}$ and $p_1,p_2$ are $\Ga_0(N)$-equivalent. The number of $\Ga_0(N)$-inequivalent elliptic points is hence at most the number of distinct coset representatives $R$ such that $RSR^{-1}\in \Ga_0(N)$, which equals $m$. But we've already shown there are at least $m$ distinct elliptic points, so the number must equal exactly $m$.

For the case that $z$ is an elliptic point of order 3, we have $\ga z=z$ for some $\ga$ conjugate to $T=\smatt0{-1}{1}{-1}$ instead. The proof is the same with minor changes.
\begin{enumerate}
\item
The trace is 1 so we have $a^2+a+1\equiv 0\pmod N$ in~(\ref{formofmat}) instead.
\item
The map $z\mapsto a$ is sujective because the elliptic point corresponding to $\smatt {a}{1}{-a^2-a-1}{-1-a}$ maps to $a$.
\item 
Keeping the same notation, 
the bottom-left entry in $RTR^{-1}$ is $z^2+tz+t^2$ instead of $t^2+z^2$.
\end{enumerate}
\end{proof}
It remains to count the number of solutions to $a^2+1\equiv 0\pmod N$. Let $p|N,p\neq 2$. The number of solutions to $a^2\equiv -1\pmod p$ is 2 if $-1$ is a square modulo $p$ and 0 otherwise. By Hensel's Lemma solutions lift uniquely to modulo $p^{v_p(n)}$. The number of solutions to $a^2\equiv -1\pmod{2^{\al}}$ is 1 if $\al=1$ and 0 if $\al>1$. Hence by the Chinese Remainder Theorem the total number of solutions is
\begin{equation}\label{ell2}
\nu_2=\begin{cases}
\prod_{p|N,\,p\neq 2}\pa{1+\pf{-1}{p}},&4\nmid N\\
0,&4| N.
\end{cases}
\end{equation}
%Now for the elliptic points of order 3, we consider
Now $a^2+a+1\equiv 0\pmod N$ has no solutions if $2|N$. If $2\nmid N$, then rewrite as $(2a+1)^2\equiv -3\pmod N$. For $p\neq 2,3$, this equation has 2 solutions if $-3$ is a square mod $p$ and 0 otherwise; solutions mod $p$ lift to solutions mod $p^{v_p(n)}$. For $p=3$, we see that there is 1 solution mod 3 but none mod 9. Hence
\begin{equation}\label{ell3}
\nu_3=\begin{cases}
\prod_{p|N,\,p\neq 2,3}\pa{1+\pf{-3}{p}},&2\nmid N,\,9\nmid N\\
0,&\text{else}.
\end{cases}
\end{equation}

Finally, we count the cusps.
\begin{lem}
Suppose $z\in \cal H$, $\Ga'$ is a discrete subgroup of $\SL_2(\R)$, $\Ga\subeq \Ga'$ is a subgroup of finite index, and $\sigma_1,\ldots, \sigma_k\in \Ga'$ are such that $\sigma_j(z)$ are all the $\Ga$-inequivalent points. Then
\[
\Ga'=\bigsqcup_{j=1}^k\Ga\sigma_j\Ga'_z.
\]
\end{lem}
\begin{proof}
Given $\ga\in \Ga'$, there exists $\sigma_j$ such that $\ga(z)$ is $\Ga$-equivalent to $\sigma_j(z)$. This means there exists $A\in \Ga$ such that $\ga(z)=A\sigma_j(z)$. Then $\sigma_j^{-1}A^{-1}\ga(z)=z$ so there exists $\tau\in \Ga'_z$ such that $\sigma_j^{-1}A^{-1}\ga=\tau$. Rearranging gives
\[
\ga=A\sigma_j\tau\in \Ga\sigma_j\Ga'_z.
\]
These double cosets are disjoint because if $\ga\in \Ga\sigma_j\Ga'_z$, then $\ga(z)$ is $\Ga$-equivalent to $\sigma_j(z)$, and by assumption different $\sigma_j(z)$ are $\Ga$-inequivalent.
\end{proof}
Take $\Ga'=\SL_2(\Z)$, $\Ga=\Ga_0(\Z)$, and $z=0=\smatt 0{-1}{1}0\infty$; then $\sigma_1(z),\ldots, \sigma_k(z)$ are all the cusps, and $\nu_{\infty}$ is the number of double cosets $\Ga_0(N)\bs \Ga'/\Ga'_z$. Considering our coset representatives $\smatt **zt$, two of them are in the same double coset if they are related on the right by an element of $\Ga'_z=\bc{\smatt 10n1}$. Now
\[
\matt **zt\matt 10n1=\matt **{nt+z}{t}.
\]
Thus the number of double cosets is the number of pairs $(z,t)$ under the equivalence relation $(z,t)\sim (z',t')$ if $z'=nt+z,t=t'$ for some $n$. Fixing $t$, there are $\ph(\gcd(n/t,t))$ inequivalent choices for $z$. Hence
\begin{equation}\label{cusp}
\nu_{\infty}=\sum_{d|N}\ph(\gcd(n,n/d)).
\end{equation}
Now we can put~(\ref{ell2}),~(\ref{ell3}), and~(\ref{cusp}) into~(\ref{genus}) to get $g(X_0(N))$.
\end{problem}
\begin{problem} {\it (Picard's Little Theorem)}
Suppose $f$ is an entire function omitting two values $y_1,y_2$.

Note $\mu_2=6$ so
\[
g(X(2))=\left.1+\mu_N\cdot \frac{N-6}{12N}\right|_{N=2}=0.
\]
The number of cusps is
\[
\frac{\mu_N}{N}=3.
\]
%Let $a,b,\infty$ be the cusps. 
There is one cusp at $\infty$ and two inequivalent cusps on $\R$.
Note that $\Ga(2)\bs \cal H$ is analytically isomorphic to $\C-\{y_1,y_2\}$ (say, via $\ph$) since they both have genus 0 and two finite points omitted.%\footnote{As shown in~cite{Ahl}, an explicity isomorphism is given by $\la\pa{\frac{\om_2}{\om_1}}
(From~\cite{mir}, any compact Riemann surface of genus 0 and no cusps is analytically isomorphic to the Riemann sphere.)

Thus $f$ induces a holomorphic map $g: \C\to \Ga(2)\bs \cal H$.
Now $\cal H$ is a covering space of $\Ga(2)\bs \cal H$ so $g$ induces a analytic map $h$ so that the following diagram commutes. (Here $\pi$ is the projection map.)
\[
\xymatrix{
&\cal H\ar^{\pi}[d]\\
\C\ar^{g}[r]\ar^{h}[ru]\ar^{f}[rd]& \Ga(2)\bs \cal H\ar^{\ph}_{\cong}[d]\\ &\C-\{y_1,y_2\}.
}
\]
Now $u(z)=e^{iz}$ is an analytic map from $\cal H$ to $D-\{0\}$ ($D$ being the unit disc centered at 0). Hence $u(h(z))$ is an entire function with image contained in $D$. Then $u(h(z))$ is bounded so constant by Liouville's Theorem. But the inverse image of any point under $u$ is discrete, so this means that $h(z)$ is constant, and $f(z)=\ph(\pi(h(z))$ is constant.
\end{problem}
\begin{problem} {\it (An automorphic form)}
For $\ga=\matt abcd\in \SL_2(\Z)$,
\begin{align}
\nonumber  \ga(z)&=\frac{az+b}{cz+d}\\
\label{gd1}
\ga'(z)&=\frac{\cancelto{1}{ad-bc}}{(cz+d)^2}
%\ga''(z)&=\frac{-2}{(cz+d)^3}
\end{align}

We differentiate the equation 
\begin{equation}\label{fgak}
f|[\ga]_k(x)=f(\ga(x))(cx+d)^{-k}=f(x)
\end{equation}
and use~(\ref{gd1}) and~(\ref{fgak}) to obtain
\begin{align}
\nonumber f'(x)&=f'(\ga(x))\ga'(x)(cx+d)^{-k}-kf(\ga(x))(cx+d)^{-k-1}\\
 &=f'(\ga(x))(cx+d)^{-2-k}-kf(x)(cx+d)^{-1}
\label{fd1}\\
\label{fdg1} f'(\ga(x))&=(cx+d)^{k+1}(f'(x)(cx+d)+kf(x)).
\end{align}
Now differentiate $f'(x)$ again and use~(\ref{gd1}),~(\ref{fgak}), and~(\ref{fdg1}) to obtain
\begin{align}
\nonumber f''(x)&=f''(\ga(x))\ga'(x)(cx+d)^{-2-k}-f'(\ga(x))(k+2)(cx+d)^{-3-k}\\
\nonumber &\quad -kf'(x)(cx+d)^{-1}+kf(x)(cx+d)^{-2}\\
\nonumber &= f''(\ga(x))(cx+d)^{-4-k}-(2+k)(cx+d)^{-2}(f'(x)(cx+d)+kf(x))\\
\nonumber &\quad -kf'(x)(cx+d)^{-1}+kf(x)(cx+d)^{-2}\\
\nonumber
f''(\ga(x))&=f''(x)(cx+d)^{k+4}+(k+2)(cx+d)^{k+2}(f'(x)(cx+d)+kf(x))\\
&\quad +kf'(x)(cx+d)^{k+3}-kf(x)(cx+d)^{k+2}.\label{fdg2}
\end{align}
Use~(\ref{fgak}), ~(\ref{fdg1}), and~(\ref{fdg2}) to write
\begin{align*}
g(\ga(x))&=(k+1)f'(\ga(x))^2-kf(\ga(x))f''(\ga(x))\\
&=(k+1)(cx+d)^{2k+2}\ba{
f'(x)^2(cx+d)^2+2kf(x)f'(x)(cx+d)+k^2f(x)^2
}\\
&\quad-kf(x)(cx+d)^k
[
f''(x)(cx+d)^{k+4}+(k+2)(cx+d)^{k+2}(f'(x)(cx+d)+kf(x))\\
&\quad
+kf'(x)(cx+d)^{k+3}-kf(x)(cx+d)^{k+2}
]\\
&=(cx+d)^{2k+4}\ba{(k+1)f'(x)^2-kf''(x)}\\
&\quad +(cx+d)^{2k+3}\ba{
2k(k+1)f(x)f'(x)-k(k+2)f(x)f'(x)-k^2f(x)f'(x)}\\
&\quad +(cx+d)^{2k+2}\ba{
k^2(k+1)f(x)^2-k^2(k+2)f(x)^2+k^2f(x)^2
}\\
&=(cx+d)^{2k+4} [(k+1)f'(x)^2-kf''(x)]\\
&=(cx+d)^{2k+4} g(x)
\end{align*}
Hence $g|[\ga]_{2k+4}=g$, and $g$ is a weight $(2k+4)$-modular form. (Note that since $f$ is holomorphic, so are its derivatives, and so $g$ is holomorphic.)

If $f$ is a modular form, then translating a cusp to $\infty$ we can write the Fourier expansion as
\[
f(z)=\sum_{n\geq 0} a_ne^{2\pi inz}.
\]
Note $f'(z)=\sum_{n\geq 1} 2\pi ina_ne^{2\pi inz}$ has no constant term, and neither does $f''(z)$. Hence $g(z)=(k+1)f'(z)^2-kf(z)f''(z)$ has no constant term, and is a cusp form.
\end{problem}
\end{document}
