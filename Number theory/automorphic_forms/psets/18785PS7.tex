%%%This is a science homework template. Modify the preamble to suit your needs. 

\documentclass[12pt]{article}

\makeatother
%AMS-TeX packages
\usepackage{amsmath}
\usepackage{amssymb}
\usepackage{amsthm}
\usepackage{array}
\usepackage{amsfonts}
\usepackage{cancel}
\usepackage[all,cmtip]{xy}%Commutative Diagrams
\usepackage[pdftex]{graphicx}
\usepackage{float}
%geometry (sets margin) and other useful packages
\usepackage[margin=1in]{geometry}
\usepackage{sidecap}
\usepackage{wrapfig}
\usepackage{verbatim}
\usepackage{mathrsfs}
\usepackage{marvosym}
\usepackage{hyperref}
\usepackage{graphicx,ctable,booktabs}

\newtheoremstyle{norm}
{6pt}
{6pt}
{}
{}
{\bf}
{:}
{.5em}
{}

\theoremstyle{norm}
\newtheorem{thm}{Theorem}[section]
\newtheorem{lem}[thm]{Lemma}
\newtheorem{df}{Definition}
\newtheorem{rem}{Remark}
\newtheorem{st}{Step}
\newtheorem{pr}[thm]{Proposition}
\newtheorem{cor}[thm]{Corollary}
\newtheorem{clm}[thm]{Claim}

%Math blackboard, fraktur, and script commonly used letters
\newcommand{\A}[0]{\mathbb{A}}
\newcommand{\C}[0]{\mathbb{C}}
\newcommand{\sC}[0]{\mathcal{C}}
\newcommand{\cE}[0]{\mathscr{E}}
\newcommand{\F}[0]{\mathbb{F}}
\newcommand{\cF}[0]{\mathscr{F}}
\newcommand{\cG}[0]{\mathscr{G}}
\newcommand{\sH}[0]{\mathscr H}
\newcommand{\Hq}[0]{\mathbb{H}}
\newcommand{\cI}[0]{\mathscr{I}}%ideal sheaf
\newcommand{\N}[0]{\mathbb{N}}
\newcommand{\Pj}[0]{\mathbb{P}}
\newcommand{\sO}[0]{\mathcal{O}}
\newcommand{\cO}[0]{\mathscr{O}}
\newcommand{\Q}[0]{\mathbb{Q}}
\newcommand{\R}[0]{\mathbb{R}}
\newcommand{\Z}[0]{\mathbb{Z}}
%Lowercase
\newcommand{\ma}[0]{\mathfrak{a}}
\newcommand{\mb}[0]{\mathfrak{b}}
\newcommand{\fg}[0]{\mathfrak{g}}
\newcommand{\vi}[0]{\mathbf{i}}
\newcommand{\vj}[0]{\mathbf{j}}
\newcommand{\vk}[0]{\mathbf{k}}
\newcommand{\mm}[0]{\mathfrak{m}}
\newcommand{\mfp}[0]{\mathfrak{p}}
\newcommand{\mq}[0]{\mathfrak{q}}
\newcommand{\mr}[0]{\mathfrak{r}}
%Letter-related
\newcommand{\bb}[1]{\mathbb{#1}}
\providecommand{\cal}[1]{\mathcal{#1}}
\renewcommand{\cal}[1]{\mathcal{#1}}
%More sequences of letters
\newcommand{\chom}[0]{\mathscr{H}om}
\newcommand{\fq}[0]{\mathbb{F}_q}
\newcommand{\fqt}[0]{\mathbb{F}_q^{\times}}
\newcommand{\sll}[0]{\mathfrak{sl}}
%Shortcuts for symbols
\newcommand{\nin}[0]{\not\in}
\newcommand{\opl}[0]{\oplus}
\newcommand{\ot}[0]{\otimes}
\newcommand{\rc}[1]{\frac{1}{#1}}
\newcommand{\rra}[0]{\rightrightarrows}
\newcommand{\send}[0]{\mapsto}
\newcommand{\sub}[0]{\subset}
\newcommand{\subeq}[0]{\subseteq}
\newcommand{\supeq}[0]{\supseteq}
\newcommand{\nsubeq}[0]{\not\subseteq}
\newcommand{\nsupeq}[0]{\not\supseteq}
%Shortcuts for greek letters
\newcommand{\al}[0]{\alpha}
\newcommand{\be}[0]{\beta}
\newcommand{\ga}[0]{\gamma}
\newcommand{\Ga}[0]{\Gamma}
\newcommand{\de}[0]{\delta}
\newcommand{\De}[0]{\Delta}
\newcommand{\ep}[0]{\varepsilon}
\newcommand{\eph}[0]{\frac{\varepsilon}{2}}
\newcommand{\ept}[0]{\frac{\varepsilon}{3}}
\newcommand{\la}[0]{\lambda}
\newcommand{\La}[0]{\Lambda}
\newcommand{\ph}[0]{\varphi}
\newcommand{\rh}[0]{\rho}
\newcommand{\te}[0]{\theta}
\newcommand{\om}[0]{\omega}
\newcommand{\Om}[0]{\Omega}
\newcommand{\si}[0]{\sigma}
%Brackets
\newcommand{\ab}[1]{\left| {#1} \right|}
\newcommand{\ba}[1]{\left[ {#1} \right]}
\newcommand{\bc}[1]{\left\{ {#1} \right\}}
\newcommand{\pa}[1]{\left( {#1} \right)}
\newcommand{\an}[1]{\left\langle {#1}\right\rangle}
\newcommand{\fl}[1]{\left\lfloor {#1}\right\rfloor}
\newcommand{\ce}[1]{\left\lceil {#1}\right\rceil}
%Text
\newcommand{\btih}[1]{\text{ by the induction hypothesis{#1}}}
\newcommand{\bwoc}[0]{by way of contradiction}
\newcommand{\by}[1]{\text{by~(\ref{#1})}}
\newcommand{\ore}[0]{\text{ or }}
%Arrows
\newcommand{\hr}[0]{\hookrightarrow}
\newcommand{\xr}[1]{\xrightarrow{#1}}
%Formatting
\newcommand{\subprob}[1]{\noindent\textbf{#1}\\}
%Functions, etc.
\newcommand{\Ann}{\operatorname{Ann}}
\newcommand{\AP}{\operatorname{AP}}
\newcommand{\Ass}{\operatorname{Ass}}
\newcommand{\Aut}{\operatorname{Aut}}
\newcommand{\chr}{\operatorname{char}}
\newcommand{\cis}{\operatorname{cis}}
\newcommand{\Cl}{\operatorname{Cl}}
\newcommand{\Der}{\operatorname{Der}}
\newcommand{\End}{\operatorname{End}}
\newcommand{\Ext}{\operatorname{Ext}}
\newcommand{\Frac}{\operatorname{Frac}}
\newcommand{\FS}{\operatorname{FS}}
\newcommand{\GL}{\operatorname{GL}}
\newcommand{\Hom}{\operatorname{Hom}}
\newcommand{\Ind}[0]{\text{Ind}}
\newcommand{\im}[0]{\text{im}}
\newcommand{\nil}[0]{\operatorname{nil}}
\newcommand{\ord}[0]{\operatorname{ord}}
\newcommand{\Proj}{\operatorname{Proj}}
\newcommand{\PSL}{\operatorname{PSL}}
\newcommand{\Rad}{\operatorname{Rad}}
\newcommand{\rank}{\operatorname{rank}}
\newcommand{\Res}[0]{\text{Res}}
\newcommand{\sign}{\operatorname{sign}}
\newcommand{\SL}{\operatorname{SL}}
\newcommand{\Spec}{\operatorname{Spec}}
\newcommand{\Specf}[2]{\Spec\pa{\frac{k[{#1}]}{#2}}}
\newcommand{\spp}{\operatorname{sp}}
\newcommand{\spn}{\operatorname{span}}
\newcommand{\Supp}{\operatorname{Supp}}
\newcommand{\Tor}{\operatorname{Tor}}
\newcommand{\tr}[0]{\text{trace}}
%Commutative diagram shortcuts
\newcommand{\fiber}[3]{\xymatrix{#1\times_{#3} #2}\ar[r]\ar[d] #1\ar[d] \\ #2 \ar[r] & #3}
\newcommand{\commsq}[8]{\xymatrix{#1\ar[r]^{#6}\ar[d]^{#5} &#2\ar[d]^{#7} \\ #3 \ar[r]^{#8} & #4}}
%Makes a diagram like this
%1->2
%|    |
%3->4
%Arguments 5, 6, 7, 8 on arrows
%  6
%5  7
%  8
\newcommand{\pull}[9]{
#1\ar@/_/[ddr]_{#2} \ar@{.>}[rd]^{#3} \ar@/^/[rrd]^{#4} & &\\
& #5\ar[r]^{#6}\ar[d]^{#8} &#7\ar[d]^{#9} \\}
\newcommand{\back}[3]{& #1 \ar[r]^{#2} & #3}
%Syntax:\pull 123456789 \back ABC
%1=upper left-hand corner
%2,3,4=arrows from upper LH corner, going down, diagonal, right
%5,6,7=top row (6 on arrow)
%8,9=middle rows (on arrows)
%A,B,C=bottom row
%Other
%Other
\newcommand{\op}{^{\text{op}}}
\newcommand{\fp}[1]{^{\underline{#1}}}
\newcommand{\rp}[1]{^{\overline{#1}}}
\newcommand{\rd}[0]{_{\text{red}}}
\newcommand{\pre}[0]{^{\text{pre}}}
\newcommand{\pf}[2]{\pa{\frac{#1}{#2}}}
\newcommand{\pd}[2]{\frac{\partial #1}{\partial #2}}
\newcommand{\bs}[0]{\backslash}
\newcommand{\ol}[1]{\overline{#1}}
\newcommand{\mmod}[1]{\,(\text{mod}^{\times} #1)}
\newcommand{\nmod}[1]{\,(\text{mod}\, #1)}
%Matrices
\newcommand{\coltwo}[2]{
\left[
\begin{matrix}
{#1}\\
{#2} 
\end{matrix}
\right]}
\newcommand{\matt}[4]{
\left[
\begin{matrix}
{#1}&{#2}\\
{#3}&{#4}
\end{matrix}
\right]}
\newcommand{\smatt}[4]{
\left[
\begin{smallmatrix}
{#1}&{#2}\\
{#3}&{#4}
\end{smallmatrix}
\right]}
\newcommand{\colthree}[3]{
\left[
\begin{matrix}
{#1}\\
{#2}\\
{#3}
\end{matrix}
\right]}
\newcommand{\iy}[0]{\infty}
%
%Redefining sections as problems
%
\makeatletter
\newenvironment{problem}{\@startsection
       {section}
       {1}
       {-.2em}
       {-3.5ex plus -1ex minus -.2ex}
       {2.3ex plus .2ex}
       {\pagebreak[3]%forces pagebreak when space is small; use \eject for better results
       \large\bf\noindent{Problem }
       }
       }
       {%\vspace{1ex}\begin{center} \rule{0.3\linewidth}{.3pt}\end{center}}
       }
\makeatother


%
%Fancy-header package to modify header/page numbering 
%
\usepackage{fancyhdr}
\pagestyle{fancy}
%\addtolength{\headwidth}{\marginparsep} %these change header-rule width
%\addtolength{\headwidth}{\marginparwidth}
\lhead{Problem \thesection}
\chead{} 
\rhead{\thepage} 
\lfoot{\small\scshape 18.785 Analytic Number Theory} 
\cfoot{} 
\rfoot{\footnotesize PS \# 6} % !! Remember to change the problem set number
\renewcommand{\headrulewidth}{.3pt} 
\renewcommand{\footrulewidth}{.3pt}
\setlength\voffset{-0.25in}
\setlength\textheight{648pt}
\allowdisplaybreaks[1]

%%%%%%%%%%%%%%%%%%%%%%%%%%%%%%%%%%%%%%%%%%%%%%%
%
%Contents of problem set
%    
\begin{document}
\title{18.785 Analytic Number Theory Problem Set \#7}% !! Remember to change the problem set number
\author{Holden Lee}
\date{3/22/11}% !! Remember to change the date
\maketitle
\thispagestyle{empty}

%Example problems
\begin{problem}{\it (Describing $Y_0(N)$)}
\subprob{(A)}
\begin{thm}\cite[VI.5.1.1, VI.5.3]{silv}
The following categories are equivalent:
\begin{enumerate}
\item
Elliptic curves over $\C$ and isogenies.
\item
$\C/\La$ where $\La$ is a lattice, and analytic maps (which are in the form multiplication by a complex number, taking $\La$ to $\La'$)
%\item
%Lattices $\La\sub \C$ up to homothety, with
%\[
%\text{Map}(\La_1,\La_2)=\{\al\in \C:\al\La_1\subeq \La_2\}.
%\]
%(Composition of maps is multiplication.)
\end{enumerate}
Moreover, if $\La$ is the lattice corresponding to $E$, then $E\cong \C/\La$ as complex Lie groups.
\end{thm}
Let $S$ be the set of pairs $(E,C)$ modulo equivalence. Define $\theta:\cal H\to S$ by
\[
\theta(z)=(\C/\La(z,1),\La(z,\rc N)/\La(z,1)).
\]
(By the theorem $\C/\La(z,1)$ corresponds to a unique elliptic curve.)
We show that $z\sim z'$ in $Y_0(N)$ iff $\theta(z)\sim \theta(z')$ in $S$. This will show that $\theta$ is in fact a map $Y_0(N)\to S$ and is injective.

%We have $z\sim z'$ iff $z'=
Suppose $\theta(z)=\theta(z')$. Then %there must be an isomorphism between the elliptic curves corresponding to $\C/\La(z,1)$ and $\C/\La(z',1)$, i.e. %by the equivalence of categories, there is $\al\in \C$ so that $\al \La(z,1)\subeq \La(z',1)$ and $\rc{\al}\La(z',1)=\La(z,1)$, i.e. $\al\La(z,1)=\La(z',1)$.
we must have %$\La(z,1)=\La(z',1)$ up to isomorphism, i.e. up to multiplication by a complex number
$\al \La(z,1)=\La(z',1)$ for some $\al$. Let $z'=\ga z$ where $\ga\in\SL_2(\Z)$.  Then
\[
\La(z',1)=\La\pa{\frac{az+b}{cz+d},1}=\rc{cz+d}\La\pa{az+b,cz+d}.
\]
Thus $\al=\rc{cz+d}$. Then $\theta(z)=\theta(z')$ iff $\rc{cz+d}\La(z,\rc{N})/\La(z,1)=\La(z',\rc{N})/\La(z',1)$. Since the image of $\La(z,\rc{N})/\La(z,1)$ under $\rc{cz+d}$ will consist of $N$ points, this will be true iff $\rc{N}$ is in the image, i.e. there exist integers $u,j$ such that 
\[
\rc{cz+d}\pa{uz+\frac{j}{N}}=\rc{N}.
\]
Rearranging gives this equivalent to
\[
\pa{u-\frac cN}z-\frac{d-j}{N}=0.
\]
Noting $z,1$ are $\R$-linearly independent, $u,j$ exist iff $N|c$, in which case we can take $u=\frac cN$ and $j=d$. Thus $\theta(z)=\theta(z')$ iff $z'=\ga z$ with $\ga\in \Ga_0(N)$.

%$(z,1)$ and $(z',1)$ are related by an element $\ga=\smatt abcd\in\Ga(1)$. Now $\ga(z,1)=(az+b,cz+d)\sim\pa{\frac{az+b}{cz+d},1}$ so $z'=\frac{az+b}{cz+d}=\ga z$.
%
%We also must have $\La(z,\rc N)=\La(z',\rc N)$ up to homothety. Say $(z,\rc N)$ and $(z',\rc N)$ are related by $\ga'=\smatt {a'}{b'}{c'}{d'} \in\Ga(1)$, up to multiplication by a complex number. Then
%%we must have
%%\[
%%\ga\coltwo z{1/N}=\coltwo{z'}{1/N}
%%\]
%%for some $\ga\in \Ga(1)$. This gives
%%\[
%%\ga
%%\]
%\[ 
%\ga\pa{z,\rc N}=\pa{a'z+b'\rc N,c'z+d'\rc N}\sim\pa{\frac{a'z+b'/N}{Nc'z+d'},\rc N}.
%\]
%so $z'=\frac{az+b/N}{Ncz+d}$, and $z,z'$ are related by an element in $\Ga_0(N)$. (Why?) 
%%So $z,z'$ are related by an element in $\Ga_0(N)$. 
%Conversely, if $\ga z=z'$ with $\ga=\smatt ab{Nc'}{d}\in \Ga_0(N)$, then  $\La(z,\rc N)$ and $\La(z',\rc N)$ are related by $\smatt a{Nb}cd$.
%reversing the above we can find $\ga$ and $\ga'$ that take $(z,1)$ to $(z',1)$ and $(z,\rc N)$ to $(z', \rc N)$.

%Note $\theta$ is surjective because any elliptic curve corresponds to $\La(z,1)$ for some $z\in \cal H$, and why?
Now we prove surjectivity. Any elliptic curve is associated to some $\C/\La(z,1)$ with $z\in \cal H$; any cyclic subgroup of size $N$ is generated by some $\frac{az+b}{N}$ with $\gcd(a,b,N)=1$. By adding a multiple of $N$ to $a$, we may assume $\gcd(a,b)=1$. By B\'ezout there exists $\smatt ab{c}d\in SL_2(\Z)$. Now
\begin{align*}
\pa{\C/\La(z,1),\bc{k\frac{az+b}{N},0\le k< N}}&
= \pa{\C/\La(az+b,cz+d),\bc{k\frac{az+b}{N},0\le k< N}}\\
&=\rc{az+b}\pa{\C/\La\pa{\frac{cz+d}{az+b},1},\bc{\frac{k}{N},0\le k<N}}\\
&\sim \pa{\C/\La\pa{\frac{cz+d}{az+b},1},\La\pa{\frac{cz+d}{az+b},\rc N}/\La\pa{\frac{cz+d}{az+b},1}}\\
&=\theta\pa{\frac{cz+d}{az+b}}.
\end{align*}
\subprob{(B)}
We claim that $Y(N)$ is in bijection with the set $S$ of triplets $(E,x_1,x_2)$, where $x_1$ and $x_2$ generate $E[n]$ and such that $e_m(x_2,x_1)=e^{\frac{2\pi i}{n}}$, modded out by equivalence (isomorphism of elliptic curves $E\to E'$ taking $x_1,x_2$ to $x_1',x_2'$).
Define the map by
\[
\theta(z)=\pa{\C/\La(z,1), \frac{z}{N}, \rc N}.
\]

First we show $z\sim z'$ iff $\theta(z)\sim \theta(z')$. This will show $\theta$ is well-defined and injective. If $\theta(z)\sim \theta(z')$, then writing $z'=\ga z$, $\ga=\smatt abcd$ as before, the isomorphism $\C/\La(z,1)\to \C/\La(z',1)$ must be multiplication by $\rc{cz+d}$. 
%Now $\frac{z}{N}, \rc{N}$ gets sent to $\frac{z'}{N},\rc{N}$ iff there exist integers $u,v,w,x$ such that
%\begin{align*}
%\frac{z/N}{cz+d}&=uz'+v+\frac{z'}{N}\\
%\iff \frac{z}{N}&=\underbrace{\pa{ua+vc+\frac{a}{N}}}_{=0} z+\underbrace{\pa{b+\frac{b}{N}+vd}}_{=0}\\
%\frac{1/N}{cz+d}&=wz'+x+\rc{N}\\
%\iff \rc{N}&=\underbrace{\pa{wa+xc+\frac{c}{N}}}_{=0}z+\underbrace{\pa{wb+wd+\frac{d}{N}}}_{=0}.
%\end{align*}
%For the first to hold we need $a\equiv 1\pmod N$ and $N|b$ in which case we can take... For the second to hold we need $N|c$ and $d\equiv 1\pmod N$ in which case we can take...
Now
\begin{align*}
\frac{1/N}{cz+d}&=\rc{N}\pa{a-c\pf{az+b}{cz+d}}\\
&=\frac{a-cz'}{N}\\
\frac{z/N}{cz+d}&=\rc{N}\pa{-b+d\pf{az+b}{cz+d}}\\
&=\frac{-b+dz'}{N}.
\end{align*}
We need 
\begin{align*}
\frac{z/N}{cz+d}&\equiv \frac{z'}{N}\pmod{\La(z',1)}\\
\frac{1/N}{cz+d}&\equiv \frac{1}{N}\pmod{\La(z',1)}.
\end{align*}
By the above this is true iff $a\equiv d\equiv 1\pmod N$ and $b\equiv c\equiv 0\pmod N$, i.e. $\ga\in \Ga(N)$, i.e. $z\sim z'$.

For surjectivity, suppose $(\C/\Ga(z,1), \frac{az+b}{N},\frac{cz+d}{N})\in S$. Now since the Weil pairing is alternating and bilinear, and $e_m(\rc{N},\frac{z}{N})=e^{\frac{2\pi i}{N}}$. 
\[
e_m\pa{
\frac{cz+d}{N}
,\frac{az+b}{N}}=e^{\frac{2\pi i}{N}\det\smatt abcd}.
\]
Hence $ad-bc\equiv 1\pmod N$. Since $\{\frac{az+b}{N},\frac{cz+d}{N}\}$ is a basis, $\gcd(a,b,N)=1$. By adding a constant multiple of $N$ to $a$ we may assume $\gcd(a,b)=1$. Now
\[
\det\matt  ab{c+rN}{d+sN}=ad-bc+(sa-rb)N.
\]
By B\'ezout we can choose $r,s$ so that the determinant is 1. Replacing $a,b,c,d$ by $a,b,c+rN, d+sN$, we assume $\ga\in \SL_2(\Z)$. Now
\begin{align*}
\pa{\C/\Ga(z,1), \frac{az+b}{N},\frac{cz+d}{N}}&= 
\pa{\C/\Ga\pa{az+b,cz+d},\frac{az+b}{N},\frac{cz+d}{N}}\\
&\sim \pa{\C/\Ga\pa{\frac{az+b}{cz+d},1},\frac{az+b}{N(cz+d)},\frac{1}{N}}\\
&=\theta\pa{\frac{az+b}{cz+d}}.
\end{align*}

\end{problem}
\begin{problem}{\it (Two definitions of Hecke operator)}

Note $z\in X_0(N)$ corresponds to $(\C/\La(z,1),\La(z,\rc N)/\La(z,1))$ via the isomorphism in problem 1, and this corresponds to the map
\[
\C/\La(z,1)\to \C/\La(z,\rc N).
\]

Assume $p$ does not divide $N$.
Then
\[
%\Ga_0(N)
%\matt{1}00{p}
%\Ga_0(N)=
\Ga_0(N)
\matt{p}00{1}
\Ga_0(N)=
\Ga_0(N)\matt p001 \sqcup \bigsqcup_{k=0}^{p-1}\Ga_0(N)\matt 1k0p.
\]
Hence as a correspondence, $T(p)$ takes $z$ to $\{pz\}\cup \{\frac{z+k}{p}:0\leq k<p\}$, which by our bijection above, corresponds to the maps
\begin{align*}
\C/\La(pz,1)&\to \C/\La(pz,\rc N)\\
\C/\La(\frac{z+k}{p},1)&\to \C/\La(\frac{z+k}{p},\rc N).
\end{align*}

Now we calculate the Hecke operator as a correspondence on the moduli space for $X_0(N)$. 
There are $p+1$ subgroups of order $p$ in $\C/\La(z,1)$; they are $\La(z,\rc p)$ and $\La(\frac{z+k}{p},1),0\le k<p$. These correspond to the maps
\begin{align*}
\C/\La(z,\rc p)&\to \C/\an{\La(z,\rc p),\La(z, \rc N)}\\
&\quad = \C/\La(z,\rc{pN}).\\
\C/\La(\frac{z+k}{p}, 1)&\to \C/\an{\La(\frac{z+k}{p},1),\La(z,\rc N)}\\
&\quad = \C/\La(\frac{z+k}{p},\rc N).
\end{align*}
The second map is the same as the one calculated above; the first maps match after scaling by $p$.
\end{problem}
\begin{problem}{\it (Weil conjectures for $\Pj^N$)}
Note
\[
|\Pj^N(\F_{q^n})|=\ab{\frac{\F_{q^n}^{N+1}-\{0\}}{\F_{q^n}^{\times}}}
=\frac{(q^n)^{N+1}-1}{q^n-1}
=1+q^n+\cdots +q^{nN}.
\]
Hence
\begin{align*}
e^{\sum_{n=1}^{\iy} |\Pj^N(\F_{q^n})|\frac{T^n}{n}}
&=e^{\sum_{n=1}^{\iy} (1+q^n+\cdots +q^{nN})\frac{T^n}{n}}\\
&=e^{-\ln(1-t)-\ln(1-qt)-\cdots -\ln(1-q^Nt)}\\
&=\frac{1}{(1-t)(1-qt)\cdots (1-q^Nt)}.
\end{align*}
We check the Weil conjectures.
\begin{enumerate}
\item
Rationality: $Z(V;T)\in \Q(T)$.
\item
Functional equation:
\begin{align*}
Z\pa{\Pj^N;\rc{q^NT}}&=\rc{\pa{1-\rc{q^NT}}\cdots \pa{1-\rc T}}\\
&=q^{1+2+\cdots +N} T^{N+1}\rc{(q^NT-1)\cdots (t-1)}\\
&=(-1)^{N+1}q^{n\ep/2} T^{\ep}Z(\Pj^N;T)
\end{align*}
where $\ep=N+1$.
\item 
Riemann hypothesis:
Take $P_1(T)=\cdots =P_{2n-1}(T)=1$, $P_{2k}(T)=1-q^{k}T$ for $0\le k\le N$. Then
\[
Z(\Pj^N;T)=\frac{P_1(T)\cdots P_{2N-1}(T)}{P_0(T)P_2(T)\cdots P_{2N}(T)}.
\]
\end{enumerate}
\end{problem}
\begin{problem}{\it (Isogenous elliptic curves have same number of points over finite field)}
\subprob{(A)}
For an elliptic curve $E$, let $E^{(q)}$ denote the elliptic curve whose defining equation is the same as that for $E$ but with all coefficients raised to the $q$th power. Let $\phi_E:E\to E^{(q)}$ be the $q$th power (Frobenius) map. Let $\psi:E_1\to E_2$ be an isogeny over $\F_q$; note that it induces an isogeny $\psi^{(q)}:E_1^{(q)}\to E_2^{(q)}$ such that the following commute:
\[
\xymatrix{
E_1\ar[r]^{\psi}\ar[d]^{\phi_{E_1}} &E_2\ar[d]^{\phi_{E_2}}\\
E_1^{(q)}\ar[r]^{\psi^{(q)}}&E_2^{(q)}
}\quad
\xymatrix{
E_1\ar[r]^{\psi}\ar[d]^{1-\phi_{E_1}} &E_2\ar[d]^{1-\phi_{E_2}}\\
E_1^{(q)}\ar[r]^{\psi^{(q)}}&E_2^{(q)}
}
\]
This is since %Because if $\psi((x,y))=(x',y')$ then $\psi(x^p)
$\phi$ is not only a morphism $E\to E^{(q)}$ but also an automorphism on $\F_q$. The above gives
\begin{equation}\label{p7-4-1}
\deg_s(\phi_{E_2})\deg_s(\psi)=\deg_s(\phi_{E_2}\psi)
=\deg_s(\psi^{(q)}\phi_{E_1})=\deg_s(\psi^{(q)})\deg_s(\phi_{E_1}).
\end{equation}
and similarly
\begin{equation}\label{p7-4-2}
\deg_s(1-\phi_{E_2})\deg_s(\psi)=\deg_s((1-\phi_{E_2})\psi)
=\deg_s(\psi^{(q)}(1-\phi_{E_1}))=\deg_s(\psi^{(q)})\deg(1-\phi_{E_1}).
\end{equation}
Since $\deg_s(\phi_{E_1})=\deg_s(\phi_{E_2})=1$, from~(\ref{p7-4-1}) we get $\deg_s(\psi)=\deg_s(\psi^{(q)})$. Putting this in~(\ref{p7-4-2}) we get $\deg_s(1-\phi_{E_1})=\deg_s(1-\phi_{E_2})$. However the separable degree of a morphism is the size of the kernel, and $\ker(1-\phi_E)$ is simply $E(\F_{q})$, since $\phi$ fixes exactly the points of $\F_q$. Hence 
we get 
$|E_1(\F_q)|=|E_2(\F_q)|$ as needed.\\
%the above becomes
%\[
%\ker(\psi)|E_1(\F_q)|=\ker(\psi^{(q)})|E_2(\F_q)|
%\]
%and it suffices to show %$|E_1(\F_q)|=|E_2(\F_q)|$ it suffices to show that
%\[
%\ker(\psi)=\ker(\psi^{(q)}).
%\]
%However, this is clear since why?

\subprob{(B) Converse}
The converse holds as well.
\begin{thm}\cite[III.7.7]{silv}
If $K$ is a finite field, then
\[
\Hom_K(E_1,E_2)\otimes \Q_l\cong \Hom_K(V_{\ell}(E_1),V_{\ell}(E_2))
\]
via the natural map.
\end{thm}
Given that $E_1$ and $E_2$ have the same number of points over $K$, we want to show $E_1$ and $E_2$ are isogenous, i.e. $\Hom_K(E_1,E_2)\neq 0$. By this theorem it suffices to show that $\Hom_K(V_{\ell}(E_1),V_{\ell}(E_2))\neq 0$.

Let $\phi$ be the Frobenius morphism on $E$.
Note that $\deg(1-\phi)$ is the number of points of the elliptic curve $E$ in $K$, and that $\tr(\phi_{\ell})=1+\deg(\phi)-\deg(1-\phi)$. Moreover, it can be shown that $V_{\ell}(E)$ is a semisimple representation of $G(\ol{K}/K)=\an{\phi}$.

Now let $\phi,\phi'$ denote the Frobenius morphisms on $E,E'$. From the above considerations and the assumption, $\tr(\phi_{\ell})=\tr(\phi'_{\ell})$, i.e. the characters corresponding to the Galois representations $V_{\ell}(E_1)$ and $V_{\ell}(E_2)$ are equal.
By semisimplicity we can write $V_{\ell}(E_1)=\bigoplus n_iV_i$ and $V_{\ell}(E_2)=\bigoplus n_i'V_i$ where $V_i$ are the irreducible representations. Equality of characters says that $n_i=n_i'$, so $V_{\ell}(E_1)\cong V_{\ell}(E_2)$ and $\Hom_K(V_{\ell}(E_1),V_{\ell}(E_2))\neq 0$, as needed.
\end{problem}
\begin{thebibliography}{9}
\bibitem{silv} Silverman, J.: "The Arithmetic of Elliptic Curves," Springer, 1986.
\end{thebibliography}
\end{document}