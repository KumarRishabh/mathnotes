%%%This is a science homework template. Modify the preamble to suit your needs. 

\documentclass[12pt]{article}

\makeatother
%AMS-TeX packages
\usepackage{amsmath}
\usepackage{amssymb}
\usepackage{amsthm}
\usepackage{array}
\usepackage{amsfonts}
\usepackage{cancel}
\usepackage[all,cmtip]{xy}%Commutative Diagrams
\usepackage[pdftex]{graphicx}
\usepackage{float}
%geometry (sets margin) and other useful packages
\usepackage[margin=1in]{geometry}
\usepackage{sidecap}
\usepackage{wrapfig}
\usepackage{verbatim}
\usepackage{mathrsfs}
\usepackage{marvosym}
\usepackage{hyperref}
\usepackage{graphicx,ctable,booktabs}

\newtheoremstyle{norm}
{3pt}
{3pt}
{}
{}
{\bf}
{:}
{.5em}
{}

\theoremstyle{norm}
\newtheorem{thm}{Theorem}[section]
\newtheorem{lem}[thm]{Lemma}
\newtheorem{df}{Definition}
\newtheorem{rem}{Remark}
\newtheorem{st}{Step}
\newtheorem{pr}[thm]{Proposition}
\newtheorem{cor}[thm]{Corollary}
\newtheorem{clm}[thm]{Claim}

%Math blackboard, fraktur, and script commonly used letters
\newcommand{\A}[0]{\mathbb{A}}
\newcommand{\C}[0]{\mathbb{C}}
\newcommand{\sC}[0]{\mathcal{C}}
\newcommand{\cE}[0]{\mathscr{E}}
\newcommand{\F}[0]{\mathbb{F}}
\newcommand{\cF}[0]{\mathscr{F}}
\newcommand{\cG}[0]{\mathscr{G}}
\newcommand{\sH}[0]{\mathscr H}
\newcommand{\Hq}[0]{\mathbb{H}}
\newcommand{\cI}[0]{\mathscr{I}}%ideal sheaf
\newcommand{\N}[0]{\mathbb{N}}
\newcommand{\Pj}[0]{\mathbb{P}}
\newcommand{\sO}[0]{\mathcal{O}}
\newcommand{\cO}[0]{\mathscr{O}}
\newcommand{\Q}[0]{\mathbb{Q}}
\newcommand{\R}[0]{\mathbb{R}}
\newcommand{\Z}[0]{\mathbb{Z}}
%Lowercase
\newcommand{\ma}[0]{\mathfrak{a}}
\newcommand{\mb}[0]{\mathfrak{b}}
\newcommand{\fg}[0]{\mathfrak{g}}
\newcommand{\vi}[0]{\mathbf{i}}
\newcommand{\vj}[0]{\mathbf{j}}
\newcommand{\vk}[0]{\mathbf{k}}
\newcommand{\mm}[0]{\mathfrak{m}}
\newcommand{\mfp}[0]{\mathfrak{p}}
\newcommand{\mq}[0]{\mathfrak{q}}
\newcommand{\mr}[0]{\mathfrak{r}}
%Letter-related
%\newcommand{\cal}[1]{\mathcal{#1}}
\newcommand{\bb}[1]{\mathbb{#1}}
%More sequences of letters
\newcommand{\chom}[0]{\mathscr{H}om}
\newcommand{\fq}[0]{\mathbb{F}_q}
\newcommand{\fqt}[0]{\mathbb{F}_q^{\times}}
\newcommand{\sll}[0]{\mathfrak{sl}}
%Shortcuts for symbols
\newcommand{\nin}[0]{\not\in}
\newcommand{\opl}[0]{\oplus}
\newcommand{\ot}[0]{\otimes}
\newcommand{\rc}[1]{\frac{1}{#1}}
\newcommand{\rra}[0]{\rightrightarrows}
\newcommand{\send}[0]{\mapsto}
\newcommand{\sub}[0]{\subset}
\newcommand{\subeq}[0]{\subseteq}
\newcommand{\supeq}[0]{\supseteq}
\newcommand{\nsubeq}[0]{\not\subseteq}
\newcommand{\nsupeq}[0]{\not\supseteq}
\newcommand{\nequiv}[0]{\not\equiv}
%Shortcuts for greek letters
\newcommand{\al}[0]{\alpha}
\newcommand{\be}[0]{\beta}
\newcommand{\ga}[0]{\gamma}
\newcommand{\Ga}[0]{\Gamma}
\newcommand{\de}[0]{\delta}
\newcommand{\De}[0]{\Delta}
\newcommand{\ep}[0]{\varepsilon}
\newcommand{\eph}[0]{\frac{\varepsilon}{2}}
\newcommand{\ept}[0]{\frac{\varepsilon}{3}}
\newcommand{\la}[0]{\lambda}
\newcommand{\La}[0]{\Lambda}
\newcommand{\ph}[0]{\varphi}
\newcommand{\rh}[0]{\rho}
\newcommand{\te}[0]{\theta}
\newcommand{\om}[0]{\omega}
%Brackets
\newcommand{\ab}[1]{\left| {#1} \right|}
\newcommand{\ba}[1]{\left[ {#1} \right]}
\newcommand{\bc}[1]{\left\{ {#1} \right\}}
\newcommand{\pa}[1]{\left( {#1} \right)}
\newcommand{\an}[1]{\langle {#1}\rangle}
\newcommand{\fl}[1]{\left\lfloor {#1}\right\rfloor}
\newcommand{\ce}[1]{\left\lceil {#1}\right\rceil}
%Text
\newcommand{\btih}[1]{\text{ by the induction hypothesis{#1}}}
\newcommand{\bwoc}[0]{by way of contradiction}
\newcommand{\by}[1]{\text{by~(\ref{#1})}}
\newcommand{\ore}[0]{\text{ or }}
%Arrows
\newcommand{\hr}[0]{\hookrightarrow}
\newcommand{\xr}[1]{\xrightarrow{#1}}
%Formatting
\newcommand{\subprob}[1]{\noindent\textbf{#1}\\}
%Functions, etc.
\newcommand{\Ann}{\operatorname{Ann}}
\newcommand{\AP}{\operatorname{AP}}
\newcommand{\Ass}{\operatorname{Ass}}
\newcommand{\Aut}{\operatorname{Aut}}
\newcommand{\chr}{\operatorname{char}}
\newcommand{\cis}{\operatorname{cis}}
\newcommand{\Cl}{\operatorname{Cl}}
\newcommand{\Der}{\operatorname{Der}}
\newcommand{\End}{\operatorname{End}}
\newcommand{\Ext}{\operatorname{Ext}}
\newcommand{\Frac}{\operatorname{Frac}}
\newcommand{\FS}{\operatorname{FS}}
\newcommand{\GL}{\operatorname{GL}}
\newcommand{\Hom}{\operatorname{Hom}}
\newcommand{\Ind}[0]{\text{Ind}}
\newcommand{\im}[0]{\text{im}}
\newcommand{\nil}[0]{\operatorname{nil}}
\newcommand{\ord}[0]{\operatorname{ord}}
\newcommand{\Proj}{\operatorname{Proj}}
\newcommand{\Rad}{\operatorname{Rad}}
\newcommand{\rank}{\operatorname{rank}}
\newcommand{\Res}[0]{\text{Res}}
\newcommand{\sign}{\operatorname{sign}}
\newcommand{\SL}{\operatorname{SL}}
\newcommand{\Spec}{\operatorname{Spec}}
\newcommand{\Specf}[2]{\Spec\pa{\frac{k[{#1}]}{#2}}}
\newcommand{\spp}{\operatorname{sp}}
\newcommand{\spn}{\operatorname{span}}
\newcommand{\Supp}{\operatorname{Supp}}
\newcommand{\Tor}{\operatorname{Tor}}
\newcommand{\tr}[0]{\text{trace}}
%Commutative diagram shortcuts
\newcommand{\fiber}[3]{\xymatrix{#1\times_{#3} #2}\ar[r]\ar[d] #1\ar[d] \\ #2 \ar[r] & #3}
\newcommand{\commsq}[8]{\xymatrix{#1\ar[r]^{#6}\ar[d]^{#5} &#2\ar[d]^{#7} \\ #3 \ar[r]^{#8} & #4}}
%Makes a diagram like this
%1->2
%|    |
%3->4
%Arguments 5, 6, 7, 8 on arrows
%  6
%5  7
%  8
\newcommand{\pull}[9]{
#1\ar@/_/[ddr]_{#2} \ar@{.>}[rd]^{#3} \ar@/^/[rrd]^{#4} & &\\
& #5\ar[r]^{#6}\ar[d]^{#8} &#7\ar[d]^{#9} \\}
\newcommand{\back}[3]{& #1 \ar[r]^{#2} & #3}
%Syntax:\pull 123456789 \back ABC
%1=upper left-hand corner
%2,3,4=arrows from upper LH corner, going down, diagonal, right
%5,6,7=top row (6 on arrow)
%8,9=middle rows (on arrows)
%A,B,C=bottom row
%Other
%Other
\newcommand{\op}{^{\text{op}}}
\newcommand{\fp}[1]{^{\underline{#1}}}
\newcommand{\rp}[1]{^{\overline{#1}}}
\newcommand{\rd}[0]{_{\text{red}}}
\newcommand{\pre}[0]{^{\text{pre}}}
\newcommand{\pf}[2]{\pa{\frac{#1}{#2}}}
\newcommand{\pd}[2]{\frac{\partial #1}{\partial #2}}
%Matrices
\newcommand{\coltwo}[2]{
\left[
\begin{matrix}
{#1}\\
{#2} 
\end{matrix}
\right]}
\newcommand{\matt}[4]{
\left[
\begin{matrix}
{#1}&{#2}\\
{#3}&{#4}
\end{matrix}
\right]}
\newcommand{\smatt}[4]{
\left[
\begin{smallmatrix}
{#1}&{#2}\\
{#3}&{#4}
\end{smallmatrix}
\right]}
\newcommand{\colthree}[3]{
\left[
\begin{matrix}
{#1}\\
{#2}\\
{#3}
\end{matrix}
\right]}
%
%Redefining sections as problems
%
\makeatletter
\newenvironment{problem}{\@startsection
       {section}
       {1}
       {-.2em}
       {-3.5ex plus -1ex minus -.2ex}
       {2.3ex plus .2ex}
       {\pagebreak[3]%forces pagebreak when space is small; use \eject for better results
       \large\bf\noindent{Problem }
       }
       }
       {%\vspace{1ex}\begin{center} \rule{0.3\linewidth}{.3pt}\end{center}}
       }
\makeatother


%
%Fancy-header package to modify header/page numbering 
%
\usepackage{fancyhdr}
\pagestyle{fancy}
%\addtolength{\headwidth}{\marginparsep} %these change header-rule width
%\addtolength{\headwidth}{\marginparwidth}
\lhead{Problem \thesection}
\chead{} 
\rhead{\thepage} 
\lfoot{\small\scshape 18.785 Analytic Number Theory} 
\cfoot{} 
\rfoot{\footnotesize PS \# 1} % !! Remember to change the problem set number
\renewcommand{\headrulewidth}{.3pt} 
\renewcommand{\footrulewidth}{.3pt}
\setlength\voffset{-0.25in}
\setlength\textheight{648pt}



%%%%%%%%%%%%%%%%%%%%%%%%%%%%%%%%%%%%%%%%%%%%%%%
%
%Contents of problem set
%    
\begin{document}
\title{18.785 Analytic Number Theory Problem Set \#1}% !! Remember to change the problem set number
\author{Holden Lee}
\date{2/4/11}% !! Remember to change the date
\maketitle
\thispagestyle{empty}
%\begin{align*}
%4(d_1(n)-d_3(n))
%&=\left(\frac{c_j}{2}+1\right)d_1\left(\frac n{q_m^{c_m}}\right)+\frac{c_j}{2}d_3\left(\frac n{q_m^{c_m}}\right)\\
%&\quad -\left(\frac{c_j}{2}+1\right)d_3\left(\frac n{q_m^{c_m}}\right)-\frac{c_j}{2}d_1\left(\frac n{q_m^{c_m}}\right)\\
%&=d_1\left(\frac n{q_m^{c_m}}\right)-d_3\left(\frac n{q_m^{c_m}}\right)\\
%&=(b_1+1)\cdots (b_k+1)\end{align*}
%
%$4(d_1(n)-d_3(n))=\left(\frac{c_j}{2}+1\right)d_1\left(\frac n{q_m^{c_m}}\right)+\frac{c_j}{2}d_3\left(\frac n{q_m^{c_m}}\right)-\left(\frac{c_j}{2}+1\right)d_3\left(\frac n{q_m^{c_m}}\right)-\frac{c_j}{2}d_1\left(\frac n{q_m^{c_m}}\right)\\=d_1\left(\frac n{q_m^{c_m}}\right)-d_3\left(\frac n{q_m^{c_m}}\right)=(b_1+1)\cdots (b_k+1)$
%$1-\frac13+\frac 15-\cdots=\lim_{N\to \infty} \frac 1N\left(\left\lfloor \frac N1\right\rfloor-\left\lfloor \frac N3\right\rfloor+\left\lfloor \frac N5\right\rfloor-\cdots\right)= \lim_{N\to \infty} \sum_{n\equiv 1\pmod 4} \left\lfloor \frac Nn\right\rfloor -\left\lfloor \frac N{n+2}\right\rfloor.$
%\begin{align}
%\nonumber
%&\quad\frac{1}{N}\left(\left\lfloor \frac N1\right \rfloor -\left\lfloor\frac N3\right \rfloor +\left\lfloor\frac  N5\right \rfloor -\cdots\right)-\left(1-\frac 13+\frac 15-\cdots\right)\\
%\nonumber
%&= \left|\frac 1N\sum_{n\equiv 1\pmod 4} \left(\left\lfloor \frac Nn \right\rfloor-\left\lfloor \frac N{n+2}\right\rfloor\right)
%-\frac 1N\sum_{n\equiv 1\pmod 4} \left(\frac Nn-\frac N{n+2}\right)\right|\\
%\nonumber
%&\le\frac 1N \left(\sum_{\scriptsize\begin{array}{c} n\equiv 1\pmod 4\\ n\le \sqrt N\end{array}}\left| \left\{\frac Nn\right\}-\left\{ \frac N{n+2}\right\}\right|\right.\\
%&\quad \left.+\sum_{\scriptsize\begin{array}{c} n\equiv 1\pmod 4\\ n> \sqrt N\end{array}} \left(\frac Nn-\frac N{n+2}+\left\lfloor \frac Nn\right\rfloor-\left\lfloor \frac N{n+2}\right\rfloor\right)\right)\\
%\nonumber
%&\le\frac 1N \left(\sum_{\scriptsize\begin{array}{c} n\equiv 1\pmod 4\\ n\le \sqrt N\end{array}}\left| \left\{\frac Nn\right\}-\left\{ \frac N{n+2}\right\}\right| \right.\\
%&\quad\left.+\sum_{\scriptsize\begin{array}{c} n\equiv 1\pmod 4\\ n> \sqrt N\end{array}} \left(\frac Nn-\frac N{n+4}+\left\lfloor \frac Nn\right\rfloor-\left\lfloor \frac N{n+4}\right\rfloor\right)\right)\\
%\nonumber
%&\le\frac 1N\left( 2\sqrt N+\frac{N}{\sqrt N}+\frac{N}{\sqrt N}\right)\\
%\nonumber
%&\le\frac {4N}{\sqrt N}\to 0 \text{ as }N\to \infty.
%\end{align}

\begin{problem}{}
Label the edges of the fundamental parallelogram as follows.
\[
\xymatrix{
& \al+\omega_2 \ar[rr]^{C_2} & & \al+\omega_1+\omega_2\ar[ld]^{C_3} \\
\al\ar[ur]^{C_1} & &\al+\omega_1 \ar[ll]^{C_4}&
}
\]
We calculate $\int_{\partial P} \frac{zf'(z)}{f(z)}\,dz$ in two ways.

\textbf{Way 1:} 
\[
\int_{\partial P} \frac{zf'(z)}{f(z)} \,dz=\ba{
\int_{C_1} \frac{zf'(z)}{f(z)}\,dz
+\int_{C_3} \frac{zf'(z)}{f(z)}\,dz
}
+\ba{
\int_{C_2} \frac{zf'(z)}{f(z)}\,dz
+\int_{C_4} \frac{zf'(z)}{f(z)}\,dz
}.
\]
Noting that $C_3$ is just $C_1$ shifted by $\omega_1$ and reversed, and that $C_2$ is just $C_4$ shifted by $\omega_2$ and reversed, this equals
\[
\int_{\partial P} \frac{zf'(z)}{f(z)} \,dz=
\int_{C_1} \ba{\frac{zf'(z)}{f(z)}
-\frac{(z+\omega_1)f'(z+\omega_1)}{f(z+\omega_1 )}
}\,dz
+
\int_{C_4} \ba{\frac{zf'(z)}{f(z)}- \frac{(z+\omega_2)f'(z+\omega_2)}{f(z+\omega_2)}
}\,dz.
\]
Since $f$ is elliptic, $f(z)=f(z+\omega_1)=f(z+\omega_2)$, giving
\[
\int_{\partial P} \frac{zf'(z)}{f(z)} \,dz=
-\omega_1\int_{C_1} \frac{f'(z)}{f(z)} \,dz-\omega_2\int_{C_4} \frac{f'(z)}{f(z)}\,dz.
\]
Now $\ln(f(z))$ can be defined in a neighborhood around $C_1$ and $C_4$, since $f$ has no poles or zeros on $\partial P$. Since $f(\al)=f(\al+\omega_1)=f(\al+\omega_2)$, we have $\ln(f(\al+\omega_1))-\ln(f(\al))=2\pi i c_1$ and $\ln(f(\al))-\ln(f(\al+\omega_2))=2\pi i c_2$ for some integers $c_1$ and $c_2$. But these equal the above integrals by definition of $\ln f(z)$, so
\begin{equation}\label{p1-1-1}
\int_{\partial P} \frac{zf'(z)}{f(z)} \,dz=
-2\pi i(\omega_1 c_1+\omega_2 c_2).
\end{equation}

\textbf{Way 2:}
Note $\Res_a \frac{f'(z)}{f(z)}=\ord_a f$ so $\Res_a \frac{zf'(z)}{f(z)}=a \ord_a f$. Letting $a_k$ be the poles and zeros of $f$ in $P$, we get by Cauchy's Theorem that
\begin{equation}\label{p1-1-2}
\int_{\partial P} \frac{zf'(z)}{f(z)}=2\pi i\sum_{k} \Res_{a_k} \frac{f'(z)}{f(z)}=2\pi i \sum_{k}m_k a_k.
\end{equation}
Equating~(\ref{p1-1-1}) and~(\ref{p1-1-2}) give
\[
\sum_{k}m_ka_k=-\omega_1c_1-\omega_2c_2\equiv 0\pmod{\La}.
\]
\end{problem}
\begin{problem}{\it }
For $\al\in \La$, let $P_{\al}=\{\al+t_1\omega_1+t_2\omega_2|t_i\in [0,1)\}$. Let $d$ be the diameter of $P$. Let $C_{m,n}=\{x:m\leq |x|< n\}$. Let
\[
\mathcal R_n=\bigcup_{\al\in \La\cap C_{n-1,n}}P_{\al}.
\]
The area of ${\cal R}_n$ is related to the number of points of $\La$ in the annulus
by a constant:
\begin{equation}\label{p1-2-1}
[{\cal R}_n]=|\La\cap C_{n-1,n}|[P]
\end{equation}
($[\cdot ]$ denotes area.) 
Next note that no point of ${\cal{R}}_n$ can be more than distance $d$ away from $C_{n-1,n}$, since each $P_{\al}$ has diameter $d$ and contains the point $\al\in C_{n-1,n}$. Hence $R_n\subeq C_{n-d-1,n+d}$, and for $n\geq d+1$,
\begin{equation}\label{p1-2-2}
[{\cal R}_n]\leq [C_{n-d-1,n+d}]=\pi((n+d)^2-%\min((n-m-1)^2,0)\leq \pi[(4m+2)n+(m+1)^2].
(n-d-1)^2)\leq\pi(4d+2)n.
\end{equation}
From~(\ref{p1-2-1}) and~(\ref{p1-2-2}) we get
\[
|\La\cap C_{n-1,n}|\leq \frac{\pi(4d+2)}{[P]}n
\]
%(For $n\geq m+1$ check this by expanding; for $n<m$, 
Hence for $s>2$, letting $m=\ce{d}$,
\[
\sum_{\la\in \La-\{0\}} \frac{1}{|\la|^s}=\sum_{\la\in (\La-\{0\})\cap C_{0,m}}\rc{|\la|^s} +\sum_{n=m+1}^{\infty}\sum_{\la\in \La\cap C_{n-1,n}} \rc{|\la|^s}.
\]
The first sum is finite since a lattice is discrete, while
\[
\sum_{\la\in \La\cap C_{n-1,n}} \rc{|\la|^s}\leq 
|\La\cap C_{n-1,n}|\rc{(n-1)^s}
\leq \frac{\pi(4d+2)}{[P]}\cdot \frac{n}{(n-1)^s}.
\]
Hence the second sum converges by comparison to the convergent series $\sum_{n\geq 1} \rc{n^{s-1}}$ (since $s-1>1$).

%Thus $\sum_{\la\in \La-\{0\}} \frac{1}{\la^p}$ converges absolutely.
\end{problem}
\begin{problem} {\it }
\subprob{(A)}
There exists $\al$ such that $\al\La_1=\La_2$ (the lattices are homothetic) if and only if their corresponding elliptic curves are isomorphic, $E(\La_1)\cong E(\La_2)$. The $j$-invariant of $\La$ is equal to the $j$-invariant of $E(\La)$ so it suffices to show that two elliptic curves in the form $y^2=x^3+ax+b$ are isomorphic iff their $j$-invariants are equal.

(The equation of $E(\La)$ is $y^2=4x^3-g_2x-g_3=0$ which under change of coordinates becomes $y^2=x^3-4g_2x-16g_3$. By definition the $j$-invariant of this elliptic curve is $j(E)=\frac{1728(4(-4g_2)^3)}{16(4(-4g_2)^3+27(-16g_3)^2)}=\frac{1728g_2^3}{g_2^3-27g_3^2}=j(\La)$.)

Let $y^2=x^3+ax+b$ be an elliptic curve. %and $a'y^2=b'x^3+c'x+d'$ be the equation of two elliptic curves. 
Two elliptic curves are isomorphic over $\C$ iff they are related by a change of coordinates; the only possible change of coordinates keeping this form of the equation are $x=u^2x',y=u^3y'$
which transform the equation to
\begin{equation}\label{p1-3-1}
y'^2=x'^3+a'x'+b',\quad a'=\frac{a}{u^4},b'=\frac{b}{u^6}.
\end{equation}
The new $j$-invariant is
\[
j'=\frac{1728(4a')^3}{16(4a'^3+27b'^2)}=\frac{1728(4a)^3}{16(4a^3+27b^2)}=j.
\]
Hence if two elliptic curves are isomorphic then their $j$-invariants are equal. Conversely, suppose the $j$-invariants of $y^2=x^3+ax+b$ and $y^2=x^3+a'x+b'$ are equal. Then $\frac{1728(4a')^3}{16(4a'^3+27b'^2)}=\frac{1728(4a)^3}{16(4a^3+27b^2)}$, giving $a^3b'^2=a'^3b^2$. Either none of $a,a',b,b'$ are zero, or $a,a'$ are zero, or $b,b'$ are zero (since $a,b$ can't both be zero, and neither can $a',b'$). Hence one of $\pf{b}{b'}^{\rc 6}$ or $\pf{a}{a'}^{\rc 4}$ is defined (or they're both defined and equal). Taking $u$ in~(\ref{p1-3-1}) to be this value transforms the first equation into the second. Hence if two elliptic curves have the same $j$-invariant then they are isomorphic.\\

\subprob{(B)(i)}
Let $\La_1$ be the set of all points in $\La$ in the first quadrant or on the positive real axis. Then $\La^*=\La_1\cup i\La_1\cup i^2\La_1\cup i^3\La_1$. Hence
\begin{align*}
G_6(\La)&=\sum_{\la\in \La^*}\rc{\la^6}\\
&=\sum_{\la\in \La_1}\pa{\rc{\la^6}+\rc{(i\la)^6}+\rc{(-\la)^6}+\rc{(-i\la)^6}}\\
&=\sum_{\la\in \La_1}\pa{\rc{\la^6}-\rc{\la^6}+\rc{\la^6}-\rc{\la^6}}\\
&=0.
\end{align*}
Then $g_3=0$ and hence 
\[j(\La)=\frac{1728g_2^3}{g_2^3-27\cancelto{0}{g_3}^2}=1728.\]

\subprob{(ii)}
Let $\La_1$ be all the points of $\La$ between the positive real axis and the ray with angle $\frac{2\pi}{3}$, including the positive real axis but not the other ray. Let $\omega=e^{\frac{2\pi i}{3}}$. Then $\La^*=\La_1\cup \omega\La_1\cup \omega^2\La_1$ so
\begin{align*}
G_4(\La)&=\sum_{\la\in \La^*}\rc{\la^4}\\
&=\sum_{\la\in \La_1}\pa{\rc{\la^4}+\rc{(\omega \la)^4}+\rc{(\omega^2)^4}}\\
&=\sum_{\la\in \La_1}(1+\omega+\omega^2)\rc{\la^4}\\
&=0.
\end{align*}
Thus $g_2=0$ and $j(\La)=0$.
\end{problem}
\begin{problem} {\it }
\subprob{(A)}
Taking logs,
\[
\ln \sigma(z)=\ln(z)+\sum_{\la\in \La^*} \ln\pa{1-\frac{z}{\la}}+\frac{z}{\la} +\rc{2} \pf{z}{\la}^2.
\]
%This converges absolutely because expanding in Taylor series,...

Then
\begin{align*}
\frac{d}{dz}\ln \sigma(z)&=\rc z+\sum_{\la\in \La^*} \frac{-1/\la}{1-\frac z{\la}}+\rc{\la}+\frac{z}{\la^2}\\
\frac{d^2}{dz^2}\ln \sigma(z)&=-\rc{z^2}+\sum_{\la\in \La^*} -\rc{\la^2}\pf{1}{1-\frac{z}{\la}}^2+\frac{1}{\la^2}=-\wp(z).
\end{align*}

\subprob{(B)}
By periodicity $\wp(z)=\wp(z+\la)$. Integrating twice and using (A) gives
\[
\ln\sigma(z+\la)=\ln\sigma(z)+az+b
\]
for some constants $a,b$. 
Exponentiating gives
\[
\sigma(z+\la)=e^{az+b}\sigma(z).
\]

\subprob{(C)(i)}
If $\sum_{k=1}^r n_i(z_i)=\la$, then replace it with $\pa{\sum_{k=1}^r n_i(z_i)} - (\la)+(0)$, to make it equal to 0. (This is okay since modulo $\La$, $(\la)=(0)$.)

From the infinite product, $\sigma(z)$ has a simple zero at each $z\in \La$ and no other zeros or poles. 
Let $f(z)=\prod_{k=1}^r \sigma(z-z_k)^{n_k}$; note $f$ is meromorphic. From the above observation, $f(z)$ has a zero/pole of  order $n_k$ at each $z_k$,  %(actually, at each $z_k+\la,\la\in \La$), 
and no other zeros modulo $\La$, so $\text{div}(f)=\sum_{k=1}^r n_i(z_i)$.

Next we show that $f$ is elliptic. From (B), for $\la\in \La$ we have
\begin{align*}
f(z+\la)&=\prod_{k=1}^r \sigma(z+\la-z_k)^{n_k}\\
&=\prod_{k=1}^r e^{n_k(a(z-z_k)+b)}\sigma(z-z_k)^{n_k}\\
&=e^{(az+b)\sum_{k=1}^r n_k+(-a)\sum_{k=1}^rn_kz_k} f(z)=f(z).
\end{align*}
Hence $f$ is elliptic.\\

\subprob{(ii)}
First we show that $f$ is an analytic isomorphism. Let $(x,y)$ on the elliptic curve be given. Now $\wp(z)-x$ is a non-constant elliptic function, so it has a zero, say $z=a$. Since $\wp$ is even $z=-a$ is also a zero, and $\wp'(z)=\pm \wp(a)$ depending on whether $z=a$ or $-a$. Now $(x,\pm\wp'(a))$ are exactly the points on $E$ with first coordinate $x$ (from the equation $y^2=ax^3+bx+c$). Hence $f$ is surjective. (Note $0$ gets sent to the point at infinity.)

Now suppose $\phi(z_1)=\phi(z_2)$. %for $z_1\nequiv z_2\pmod \La$. 
Then $\wp(z)-\wp(z_1)$ has zeros $z_1,z_2$. Since $\wp(z)-\wp(z_1)$ has exactly two zeros (since it has one pole of order 2), and $\wp$ is even, $z_1=\pm z_2$. (If $2z_1\equiv 0\pmod{\La}$ then $\wp(z)-\wp(z_1)$ has a double zero at $z_1$, and that is the only zero.) But $\wp'(z_1)$ is odd so
\[
\wp'(z_1)=\wp'(z_2)=\wp'(\pm z_1)=\pm\wp'(z_1)
\]
so either $z_1\equiv z_2\pmod{\La}$ or $\wp'(z_1)=0,z_1\nequiv z_2\pmod{\La}$.
Let $\omega_1,\omega_2$ generate $\La$ and let $\omega_3=\omega_1+\omega_2$. 
 Note $\wp'(\omega_m/2)=0$ because $\wp'$ is odd and $\omega_m/2\equiv -\omega_m/2\pmod{\La}$.
Since the equation of the elliptic curve is cubic in $x$, and $(\wp(z),\wp'(z))\in E$, these three are the only possible values of $\wp(z)$ given that $\wp'(z)=0$. 
Since $\wp(z)-\wp(\omega_m/2)$ is even, it has a double zero at $\omega_m/2$; since its only pole is of order 2 this is its only zero modulo $\La$. Hence $\wp(z)\neq \wp(\omega_m)$ for $z\nequiv \omega_m\pmod{\La}$. This shows $\phi$ is injective.

Next to show that $\phi$ is analytically an isomorphism, note $\frac{dx}{y}$ is a nonvanishing holomorphic differential on $E$, and
\[\phi^*(dx/y)=d\wp(z)/\wp'(z)=dz\]
is nonvanishing holomorphic on $\C/\La$.

%, because the equation for $E$ has two solutions $(x,\pm y)$.

From part (i), there exists $f$ such that
\[\text{div}(f)=(z_1+z_2)-(z_1)-(z_2)+(0).\]
Since every elliptic function is a rational combination of $\wp$ and $\wp'$, we can write $f=F(\wp,\wp')$ for some $F\in \C(x,y)$. The function $f$ on $\C/\La$ corresponds to $F$ on $E$, since $f=F\circ \phi$. Since $\phi$ is an analytic isomorphism, it preserves zeros and poles:
\[
\text{div}(F)=(\phi(z_1+z_2))-(\phi(z_1))-(\phi(z_2))+(0).
\]
Hence $(\phi(z_1+z_2))-(\phi(z_1))-(\phi(z_2))+(0)$ is a principal divisor. 
From~\cite[III.3.5]{s}, $\phi(z_1+z_2)-\phi(z_1)-\phi(z_2)+0=0$, so $\phi(z_1+z_2)=\phi(z_1)+\phi(z_2)$. \\

\subprob{(D)}
Note $\wp(z)$, and hence $\wp(z)-\wp(a)$, has a pole of multiplicity 2 at $z=0$ and no other poles. Hence its zeros have total multiplicity 2. Now $\wp(z)-\wp(a)$ has a zero at $z=a$; since it is even it has a zero at $z=-a$. (If $a\equiv -a\pmod{\La}$ then this is a zero of multiplicity 2.) By the construction in (C), $\frac{\sigma(z+a)\sigma(z-a)}{\sigma(z)^2}$ has zeros and poles with the same orders as $\wp(z)-\wp(a)$. Now the quotient between these two functions is entire and bounded (since its maximum and minimum are attained on the (closed) fundamental parallelogram, which is compact), so a constant, by Liouville's Theorem.

To find the constant, we note that $z^2(\wp(z)-\wp(a))$ equals 1 at $z=0$ (the coefficient of $\rc{z^2}$ in the Laurent expansion is 1), while (using the fact that the expansion of $\sigma(z)$ is $z+\cdots$, and $\sigma$ is odd as $\La^*=-\La^*$)
\[z^2\left.\frac{\sigma(z+a)\sigma(z-a)}{\sigma(z)^2}\right|_{z=0}
=
\sigma(a)\sigma(-a)=-\sigma(a)^2.
%a\prod_{\la\in \La^*} \pa{1-\frac{a}{\la}}\pa{1+\frac{a}{\la}}e^{a^2/\la}
%=
%-a^2\prod_{\la\in \La^*} \pa{1-\frac{a^2}{\la}}e^{a^2/\la}
%=-\sigma(a)^2
\]
Hence the constant is $-\rc{\sigma(a)^2}$, and
\[
\wp(z)-\wp(z)=-\frac{\sigma(z+a)\sigma(z-a)}{\sigma(z)^2\sigma(a)^2}.
\]

\end{problem}
\begin{thebibliography}{9}
\bibitem{s} Silverman, J.: "The Arithmetic of Elliptic Curves," Springer, 1986.
\end{thebibliography}

\end{document}
