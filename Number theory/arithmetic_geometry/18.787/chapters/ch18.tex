\lecture{Tue. 11/13/12}

Last time we showed that for $A\in \abk$, an ample line bundle $L$ on $A$ induces a Rosati involution
\[
\ri_L:\phi\mapsto \la_L^{-1}\phi^{\vee}\la_L
\]
on $\End^0(A)$. This is a positive involution. We will see that the Rosati involution leads to the Riemann Hypothesis for $A$. 

First we prove a fact about the Frobenius.
\begin{thm}\llabel{thm:frob-ri-frob}
Let $A\in (\text{Ab}/\fq)$. Then
\[
\Frob_q^{\ri_L}\Frob_q=[q].
\]
\end{thm}
\begin{proof}
We have
\bal
\la_L^{-1}\Frob_q^{\vee} \la_L \Frob_q&=[q]\\
\iff \Frob_q^{\vee} \la_L \Frob_q &= [q]\la_L.
\end{align*}
Now the LHS is (using $f^{\vee} \la_L f=\la_{f^*L}$ and noting $\Frob_q^* L\cong L^{\ot q}$) $\la_{\Frob_q^* L}=\la_{L^{\ot q}}=q\la_L$. 
Construction of Mumford line bundle commutes with tensor.

$\La:\Pic A\to \Hom(A,\av)$, $L\mapsto \la_L$ is  group homomorphism. 
\end{proof}
While we won't prove it here, the Riemann hypothesis for abelian varieties relies on Theorem~\ref{thm:frob-ri-frob}.

\subsection{Complex abelian varieties: An overview}

We'll talk about complex abelian varieties. We start with elliptic curves over $\C$.

Let 
\[
\tau\in \cal H:=\set{z\in \C}{\Im z>0}.
\]
Define the lattice
\[
\La_{\tau}:=\Z\ot \Z\tau
\]
and
\[
E_{\tau}:=\C/\La_{\tau}  
\]
as a complex manifold of dimension 1 with group structure (call it an ``abelian manifold," if you'd like).

Note although $E_{\tau}$ is defined as a complex manifold, it comes from an algebraic variety in the following sense. We have a map
\bal
E_Z\cong \C/\La_{\tau} &\xrc \{y^2=4x^3-g_2(\tau)x-g_3(\tau)\}\stackrel{\text{closed}}{\sub} \Pj^2(\C)\\
z\mapsto (\wp(z),\wp'(z))
\end{align*}
where $\wp$ is the Weierstrass $\wp$-function. We have that $E_{\tau}$ is a projective algebraic variety with group structure.

 
Higher dimensional complex torus, algebraic structure, better handle on abelian variety. Difficult construct. Lattice, magincally an abelian varity comes out. Doesn't work exactly that way.

In fact, we have bijections \fixme{compile}
\begin{gather*}
%\xymatrix@R-24pt{
\SL_2(\Z)\bs \cal H %\ar@{<->}[r]^{1-1} & 
\xleftrightarrow{1-1}
\bt{$\C$-tori}{of dimension 1}/\cong %\ar@{<->}[r]^{1-1} &
\xleftrightarrow{1-1}
 \bt{elliptic}{curves/$\C$}/\cong\\
\tau \mapsto %\mt{r} & 
\C/\La_{\tau} \mapsto %\mt{r} & 
\text{Projective curve }y^2=4x^3+\cdots %proj curve
%}
\end{gather*}
We have the middle is projective complex manifold, the RHS is a projective curve. 

We have $x\mapsto \pa{\om\mapsto \int_0^x \om}$, $x\in E_{\text{an}}$. $E(\C)\cong H^0(E_{\text{an}},\Om^1)^*/H_1(E_{\text{an}},\Z)$, where the embedding is given by $\ga\mapsto (\om\mapsto \int_{\ga} \om)$, and $E\mapsto E(\C)$.

nonvanish global diffl form
tangent space cotangent to that guy, think of as exponential map.

We're seeking a generalization. Not all $\C$-tori of dimension $g$ correspond to abelian varities. We need an extra condition, and then we get
%\fixme{DOESN'T COMPILE}
\[
\bth{polarizable}{$\C$-tori}{of dimension $g$}/\cong \xleftrightarrow{1-1}\bt{abelian variety/$\C$}{of dimension $g$}/\cong.
\]
The generalization of the upper half plane is the Siegel upper half space $\cal H_g$ of dimension $\fc{g(g+1)}{2}$.
%\fixme{DOESN'T COMPILE}
\[
\text{Sp}_{2g}(\Z)\bs \cal H_g\xleftrightarrow{1-1} \bt{principally polarized}{$\C$-torus of dimension $g$}/\cong.
\]
Polarizable vs. polarized. Admits one polarization, vs. actually a pair: complex torus and principal polarization.

Elliptic curve has only 1 possible polarization up to congruence.

For torus a polarization will be the same as a Riemann form. A polarization is $(A,\text{Riemann form})$. 
%Basically we consider 

\subsection{$\C$-tori}
\begin{df}
A $\C$-torus is a complex manifold isomorphic to $V/\La$ where 
\begin{itemize}
\item
$V$ is a $\C$-vector space of dimension $g<\iy$
\item
$\La$ is a free $\Z$-module generated by some $\R$-basis of $V$ (i.e., a lattice).
\end{itemize}

Note a $\C$-torus is a compact complex manifold with group structure, i.e. a compact Lie group. 
\end{df}
We can show that 
\[
\Hom(V/\La,V'/\La')=\bt{$\phi\in \Hom_{\C-\text{linear}}(V,V')$ such that}{$\phi(\La)\subeq \La'$}.
\]

A different perspective is given by the following.
\begin{df}
A \textbf{Riemann pair} $(\La,J)$ where $\La$ a free $\Z$ module of finite rank and $J\in \End_{\R}(\La\ot_{\Z} \R)$ such that $J^2=-1$. (Hence there is a ``complex structure" on $\La_{\R}:=\La\ot \R$.)
\end{df}
The reason for introducing this is that given a complex torus we can associate it with a Riemann pair.
\bal
\pat{$\C$-torus}& \to \pat{Riemann pairs}\\
V/\La & \mapsto (\La, J=\text{multiplication by i}\cir V=\La_{\Z}\ot \R)\\
(\La\ot_{\Z}\ot \R)/\Ga \mapsfrom & (\La,J).
\end{align*}
Here the complex structure on $\La\ot_{\Z}\ot \R$ is given by the isomorphism $\R[J]/(J^2+1)\cong \C$, with $J\lra i$.

%these are central obj we'd like to look at.
%so far nothing much is going on
The key questions are
\begin{itemize}
\item
is the manifold projective?
\item
algebraic?
\end{itemize}
%proper not auto proj
As we will see, these questions are related. In algebraic geometry, to study this question you study ample line bundles. It would be nice to have a classification of line bundles. That's what we'd like to do. 

\subsubsection{Line bundles on $X_{\text{an}}=V/\La$}
We'll soon consider algebraic varieties over $\C$; we'd like to distinguish algebraic and analytic varieties.

We follow~\cite[\S2]{Mu70}. We have a correspondence (Appell-Humbert Theorem)
\[
\Pic X\xrc \bth{$(E,\al)$, $E:\La\times \La\to \Z$}{bilinear alternating, $E(Jx,Jy)=E(x,y)\forall x,y\in \La$}{$\exists \al:\La\to \C_1^*=\{|z|=1\}\ldots$}.
\]
We'll give the details. 
Correspondence $L(E,\al)\mapsfrom (E,\al)$. 

Pull back from $V\times \La$ to $V$. Line bundles correspond to certain $\La$-action on $V$. Action on trivial line bundle over $V$. 

%E conn component to Picard manifold Pic^0 rep conn component in Pic manifold

We can make the above correspondence explicit.

The correspondence restricts to
\[
\Pic^0(X_{\text{an}}) \xrc \set{(0,\al)}{\al\in \Hom_{\text{cont}}(\La,\C_1^*)}
\]
\begin{thm}[Lefschetz, \S3]
$L(E,\al)$ is ample iff
\[
\text{for all }x\in \La, x\ne 0, \qquad E(x,Jx)>0.
\]
(condition 3)
%extend R-lin to V. Underlie is Herm form on V.
%both Riem forms, one l-adic, one integral.
(Different conventions. Here, follow Milne's notation.)
%stats on which use more commonly
\end{thm}

\begin{df}
A form satisfying the following is a \textbf{Riemann form} (a polarization).
\begin{enumerate}
\item
$\Z$-bilinear and alternating.
\item $E(Jx,Jy)=E(x,y)$ for all $x,y\in \La$.
\item
For all $x\in \La, x\ne 0, \qquad E(x,Jx)>0$.
\end{enumerate}
\end{df}
Consequently, say $X_{\text{an}}\lra (\La,J)$. If there exists a Riemann form $E$ for $(\La,J)$, then there exists $X_{\text{an}}\hra \Pj^N(\C)_{\text{an}}$ a closed imbedding. 

Analytic topology finer than Zariski, but the amazing thing here is that analytic closed subsets are projectively closed in Zariski topology.

We'll use Chow's Theorem as a black box, and get if $X_{\text{an}}$ is Zariski closed so $X_{\text{an}}$ is a projective variety over $\C$.

sat lin cond, then torus is alg variety, even proj variety.

%gaga vs agag
For the converse, let's get some general facts, using Chow's Theorem and Serre's GAGA. %AGAG.
We have a faithful map
\bal
(x,\cO_X)&\mapsto (X_{\text{an}},\\
\pat{algebraic varieties/$\C$} &\mapsto  \pat{$\C$-analytic spaces}
\end{align*}
If we have alg variety, then we can cover it with open affine varieties, locally closed sets in affine spaces. Complex analytic structure on $S\subeq \C^N$ locally closed. This correspondence restricts to the following.
\bal
\pat{algebraic varieties/$\C$} &\mapsto  \pat{$\C$-analytic spaces}\\
\pat{smooth algebraic varieties/$\C$} &\mapsto  \pat{smooth $\C$-analytic spaces}
\end{align*}
where the bottom map is a bijection, actually
Smooth projective, get equivalence of categories. Don't expect 
fully faithful because exponential map cannot be imitated by polynomial or rational map. But faithful because happening on level of points, if 2 maps same on points, then same algebraic maps.

In the bottom case, 
\[
\text{Coh}(X)\xrc \text{Coh}(X_{\text{an}}).
\]

Algebraic Picard group same as analytic Picard group.

In equiv of categories, proj manifold has unique alg structure. If two alg varieties is analy, then already is as alg varieties. 

\begin{thm}
Let $X_{\text{an}}$ be a $\C$-torus, corresponding to $(\La, J)$. Then the following are equivalent.
\begin{enumerate}
\item
$(\La,J)$ admits a Riemann form.
\item
$X_{\text{an}}$ is a projective manifold.
\item
There exists $Y$ algebraic variety over $\C$, such that $Y_{\text{an}}\cong X_{\text{an}}$. 
\item
There exist $f_1,\ldots, f_g$ algebraically independent meromorphic functions on $X_{\text{an}}$. 
\end{enumerate}•
\end{thm}
\begin{proof}
For (1)$\implies$(2) use Lefschetz Theorem, for (2)$\implies$(3) use Chow's Theorem, for (3)$\implies$(4) use for smooth projective, set of meromorphic functions are set of rational function, $\C(X)$, transcendence degree $g$. 
%proper can show projective. A priori alebraic variety.
For (4)$\implies$(1) the idea is that if we have $f_1,\ldots, f_g$, then we can form a divisor $(D_i)=\pat{poles of $f_i$}$. Take linear combination with coefficient order of poles. Then take $D=\sum D_i$ and find that $L(D)=L(E,\al)$ with $E$ a Riemann form.
\end{proof}
The upshot of this is that the set of Riemann pairs $(\La,J)$ admitting a Riemann form (``polarizable") correspond to $\C$-tori which are algebraic, which corresponds to abelian varieties over $\C$.
\[
\bt{Riemann pairs $(\La,J)$}{admitting a Riemann form} \xleftrightarrow{1-1}
\bt{$\C$-tori which}{are algebraic} %
\xleftrightarrow{1-1} 
\pat{Ab$/\C$}.
\]

Not just pos semidef, but pos def/
(Hermitian form). No isotropic subspace where trivial. Mod out by subspace, down to smaller dimension, make work.
\subsection{Duals}
Let $X$ be an abelian variety over $\C$, $X_{\text{an}}=V/\La$. We have a dual abelian variety over $\C$, $X^{\vee}$. What is 
$(X^{\vee})_{\text{an}}=V'/\La'$? We want to describe $V',\La'$ in terms of $V$ and $\La$.

We know that as groups 
\[\Pic^0(X)\cong \Hom(\La,\C_1^*)\stackrel{?}{\cong} V'/\La'.
\]
How to put complex torus structure such that also abelian variety?

The answer is that the Poincar\'e line bundle $\cal P\to X\times X^{\vee}$ goes to $\cal P_{\text{an}}=L(\cE, \al)\to X_{\text{an}}\times X_{\text{an}}$. Eventually we get a Hermitian pairing 
\[
V\times V'\to \C
\]
and restricting,
\[
\La\times \La'\to %\im\Psi
\Z.
\]
So we let $V'$ be the Hermitian dual,
\[
V'=V^*=\Hom_{\C\text{-antilinear}}(V,\C)\cong
\set{\phi:V\to \C}{\phi(az)=\ol a\phi(z),a\in \C}
\]
and 
\[
\La'=\set{\phi\in V}{\im \phi(\la)\in \Z\text{ for all }\la\in \La}
\]
we can check there exists a Riemann form $V'/\La'$. We have as a group,
\bal
V'/\La' &\xrc \Hom(\La,\C_1^*)\\
\phi & \mapsto (\la \mapsto e^{-2\pi i \im\phi(\la)}).
\end{align*}
Given $\Pic^0$ the structure of complex polarizable torus.

Why are Riemann forms polarizations?
\subsection{Polarization}
Given $X$ a complex abelian variety corresponding to $(\La,J)$, we have
\bal
\bt{Riemann forms}{for $(\La,J)$}&\xleftrightarrow{1-1} \bt{polarization}{$X\to X^{\vee}$}\\
E&\mapsto \la_L(E,\al).
\end{align*}
A priori this construction is analytic. Can view as algebraic, associate...? Polarization is by definition a morphism $X\to X^{\vee}$ which have the form $\la_L$ for some $L$. Since line bundles exhausted by $L(E,\al)$, surj.If two things are same, Riemann form same, well-defined (doesn't depend $\al$). 

Principal pol. are those inducing isomorphisms. Riemann forms give perfect pairing on $\la$, equivalently determinant 1. In the above, those $E$ with $\det E=1$ correspond to principal polarizations $X\xrc X^{\vee}$. This Riemann form really is compatible with the $\ell$-adic Riemann form, formula relating them, just 
tensor this with $\zl$. Naturally, share very similar characteristics. All formulas here sim to formulas there.

Thursday, CM abelian varieties.