\lecture{Tue. 10/23/12}
\llabel{sec:cartier}

(I was absent today. Notes are borrowed from John Binder.)

Last time we proved duality results for abelian schemes over $S$. Given a morphism $f:A\to B$ in $\Abs$, we had morphisms
\[
\xymatrix{
& A\times B^{\vee} \ar[ld]_{1\times f^{\vee}} \ar[rd]^{f\times1}&\\
A\times \av && B\times B^{\vee}
}
\]
and we saw that the Poincar\'e line bundles on $A\times \av$ and $B\times B^{\vee}$ are related by the following (Lemma~\ref{lem:2poincare-bundle}):
\[
(1\times f^{\vee})^* \cal P_A\cong (f\times 1)^*\cal P_B.
\]

\begin{thm}[Poincar\'e reducibility]\llabel{thm:poincare-reducibility}
Suppose $X$ is an abelian variety and $Y$ is an abelian subvariety, $Y\hra X$. Then there is an abelian subvariety $Z\hra X$ such that the multiplication map %$Y\cap Z$ is finite and  
\[
\mu_X:Y\times Z\to X
\] 
is an isogeny.
\end{thm}
To prove this, we need the following lemma.
\begin{lem}\llabel{lem:laf*l}
Let $L\in \Line(X)$ be a line bundle on an abelian variety $X$. Let $f:Y\to X$ be a morphism in $\abk$. Then 
\[
\la_{f^*L}=f^{\vee}\circ \la_L \circ f.
\]
\end{lem}
In other words, the following diagram commutes.
\[
\xymatrix{
Y\ar[r]^f\ar@/_2pc/[rrr]^{\la_{f^*L}} & X \ar[r]^{\la_L} & X^{\vee}\ar[r]^{f^{\vee}} & Y^{\vee}.
}
\]
\begin{proof}
The morphism $\la_{f^* L}$ is characterized by
\[
(1\times \la_{f^*L})^* \cal P_Y \cong M(f^*L):=\mu^*(f^*L)\ot p_1(f^*L)^{-1}\ot p_2(f^*L)^{-1}.
\]
We need to show that
\[
(1\times \la_{f^*L})^*\cal P_Y = (1\times(f^{\vee}\circ \la_L\circ f))^* \cal P_Y,
\]
which by Lemma~\ref{lem:2poincare-bundle} is equivalent to
\[
(1\times f)^* (1\times \la_L)^*%\underbrace{
(1\times f^*)^*%}_{=(f^*\times 1)^* \cal P_X%\text{ (Lemma~\ref{lem:2poincare-bundle})}}
= \cal P_Y\cong M(f^*L).
\]
The left-hand side is 
\[
(1\times f)^*\underbrace{(1\times \la_L)^* \cal P_X}_{M(L)}.
\]
and this is isomorphic to the RHS (exercise). Hint: The following commutes.
\[
\commsq{Y\times Y}{X\times X}{Y}{X.}{\mu_Y}{(f,f)}{\mu_X}{f}
\]
\end{proof}
Now we prove Theorem~\ref{thm:poincare-reducibility}.
\begin{proof}[Proof of Theorem~\ref{thm:poincare-reducibility}]
Given $i:Y\hra X$ choose an ample line bundle $L\in \Line(X)$. Let 
\[
Z':=\ker(X\xra{\la_L}X^{\vee} \xra{i^{\vee}} Y^{\vee}) \subeq X.
\]
Now $Z'\cap Y$ is finite over $k$ since by Lemma~\ref{lem:laf*l},
\[
Z'\cap Y%=\ker(Y\xhookrightarrow{i})
=\ker (i^{\vee}\circ \la_L\circ i)=\ker(\la_{i^*L})=K_{i^*L}
\]
which is finite over $k$.  %illegible? (if pull... line bundle is a ...)
Take $Z=(Z'\rd)^{\circ}$. 
\begin{fct}\cite[5.31]{GGBM}
Let $Y$ be an abelian variety over a field $k$. If $Z\hra Y$ is a closed subgroup scheme, then $Z^{\circ}$ is an open and closed subgroup scheme of $Y$ that is geometrically irreducible. The reduced underlying scheme $Z^{\circ}_{\text{red}}$ is an abelian subvariety of $Y$. 
$Z$ is an abelian subvariety of $X$ and $\dim Z'=\dim X-\dim Y$.
\end{fct}
Observe that  $\mu_X:Y\times Z\to X$ is an isogeny:
\begin{enumerate}
\item $\ker \mu_X\subeq (Y\cap Z)\times (Y\cap Z)$ is finite, and 
\item $\dim(Y\times Z)=\dim Y-\dim Z=\dim Y-\dim Z'\ge \dim X$.
\end{enumerate}
\begin{df}
$A\in \abk$ is \textbf{simple} if there is no proper nontrivial abelian subvariety $B\subeq A$.
\end{df}
\begin{pr}
 If $A\to B$ is an isogeny and $A$ is simple, then $B$ is simple. 
\end{pr}
\begin{cor}
Given $A\in \abk$, there exist simple $A_1,\ldots, A_r\in \abk$ and $n_1,\ldots, n_r\in \N$ and an isogeny
\[
A_1^{n_1}\times \cdots \times A_r^{n_r}\to A.
\]
Moreover the $A_i,n_i$ are unique.
\end{cor}
\begin{rem}
$A$ is $k$-simple does not imply that $A\times_k k'$ is $k'$-simple. Simplicity is not preserved under base change (corresponding to tensor product of algebras).
\end{rem}
%\begin{proof}
%Mumford says to use the ``standard" idea: $A$, $B$ simple means $A\to B$ is finite (?) or an isogeny.
%\end{proof}
\end{proof}
\subsection{Cartier duality}
We studied the dual functor $A\mapsto A^{\vee}$ on $\Abs$. Now we define Cartier duality, which is a functor on the category of finite locally free group schemes over $S$. We give two approaches.
\subsubsection{Concrete approach}
Suppose we are in the affine case, so we can write $S=\Spec(Q)$ and $G=\Spec R$. A morphism $G\to S$ corresponds to a algebra homomorphism $Q\to R$.

%We first give some preparation.
First, we give a basic definition.
\begin{df}
A \textbf{Hopf algebra}  $R$ over $Q$ is an algebra equipped with the following functions. (Note items 1 and 2 are part of the definition for an algebra.)
\begin{enumerate}
\item
Ring multiplication $m:R\ot_Q R\to R$
\item
Structure map (unit): $\de:Q\to R$ (corresponding to $G\to S$)
\item $i\sh: R\to R$ corresponding to $i:G\to G$.
\item Counity $e\sh: R\to Q$ corresponding to $e:S \to G$.
\item Comultiplication $\De:R\to R\ot_Q R$ corresponding to $G\times G\to G$.
\end{enumerate}
\end{df}
What happens when we dualize? Define $R^{\vee}=\Hom_{Q\text{-alg}}(R,Q)$. This is equipped with maps dual to the ones in the definition: (we drop the $\sharp$ for notational convenience)
\begin{enumerate}
\item $R^{\vee}\to R^{\vee}\ot_QR^{\vee}$
\item $R^{\vee}\to Q^{\vee}\cong Q$ (canonically; $\Hom_Q(Q,Q)=Q$)
\item $R^{\vee}\xra{i^{\vee}} R^{\vee}$
\item $Q^{\vee}\xra{e^{\vee}} R^{\vee}$
\item $R^{\vee}\ot R^{\vee}\xra{m^{\vee}} R^{\vee}$.
\end{enumerate}
Now $G^{\vee}=(\Spec R^{\vee}, \mu^{\vee}, e^{\vee},i^{\vee})$ is a group scheme over $S$; in fact it is a finite locally free commutative group scheme. 

We see that $G\to G'$ ($\Spec R\to \Spec R'$) gives $G'^{\vee}\to G^{\vee}$ ($\Spec R'\to \Spec R$). You can check that this is a morphism of group schemes.\\

\prbbox{Show that $(G^{\vee})^{\vee}\xrc G$ functorially in $G$.}
\vskip0.15in

For general schemes $S$, we glue from the affine case.

\begin{ex}
We have $(\Z/n\Z)^{\vee} \cong \mu_n$ when $S=\Spec k$. (Recall that $\ul{\Z/n\Z} =\Spec\pa{\bigopl_{x\in \Z/n\Z} k}$.) Here the multiplication map is 
\begin{align*}
\mu\sh:\bigopl_{\ga} k &\to \bigopl_{(\de,\de')\in (\Z/n\Z)^2}k\\
(a_{\ga})&\mapsto (a_{\de\ne \de'}).
\end{align*}
We claim that $\mu_n=\Spec(k[t]/(t^n-1))$ and $\mu\sh:t\to t\ot t$. Note $\bigopl_{\ga}k^{\vee} \cong k[\Z/n\Z]\cong k[t]/(t^n-1)$  via the maps
\[
((a_n)\mapsto a_m)\mapsfrom \underbrace{m}_{\Z/n\Z} \mapsfrom t^m.
\]
as $k$ vector-spaces, as $k$-algebras, and as Hopf algebras.
\end{ex}
\subsubsection{Second approach to $G^{\vee}$}
\begin{thm}
Let $G$ be a finite locally free abelian group scheme over $S$. Then the functor
\begin{align*}
\ul{G^{\vee}}:\schs &\to \pat{Gps}\\
T&\mapsto \Hom_{(\text{Gp}/T)} (G\times T, \G_{m,T})
\end{align*}
is representable by $\ul{G^{\vee}}$.
\end{thm}
\begin{proof}
Observe  that $\ul{G^{\vee}}$ is a Zariski sheaf %(?? for $S=\Spec Q$). 
so we can reduce to $S=\Spec Q$, $G=\Spec R$, $G'=\Spec R'$, and $T=\Spec Q'$. It suffices to show 
\[
\ul{G^{\vee}}(\Spec Q')\cong G^{\vee}(\Spec Q')
\]
for all $Q$-algebras $Q'$, and that this is functorial in $Q'$. Note 
\begin{align*}
\ul{G^{\vee}}(R')&\subeq \Hom_{(\text{Sch}/S)} (\Spec R\ot Q', \Spec Q'[t,t^{-1}])\\
&=\Hom_{Q\text{-alg}}(Q'[t,t^{-1}], R\ot_{Q}Q') \cong R_{Q'}^{\times}
\end{align*}
the last via $\ph\mapsto \ph(t)$. (We use $R_{Q'}$ to denote $R\ot_Q Q'$.)
Consider
\beq{eq:787-12-1}
G^{\vee}(R')=\set{\al \in R^{\times}_{Q'}}{\mu\sh (\al)\equiv \al\ot \al}
\eeq
\begin{clm}
\beq{eq:787-12-2}
G^{\vee}(Q')=\set{\al\in R_{Q'}}{\mu\sh =\al\ot \al,\, e\sh(\al)\equiv 1}.
\eeq
\end{clm}
\begin{proof}
\eqref{eq:787-12-1}$\subeq $\eqref{eq:787-12-2}: The following commute.
\[
\cmp{G\times S}{G\times G}{G}{(\id,e)}{}{\mu}{}{}{\id}
\qquad
\xymatrix{
G\ar[r]^{(\id,i)} \ar[rd] & G\times G \ar[r]^{\mu} & G\\
& S \ar[ru]_e&
}
\]
We get $R\to R\ot_Q R= R$, $\al\mapsto (\id,e)\sh (\mu\sh \al)=\al$. $(\id,e)\sh (\mu\sh \al)=(\id,e)\sh (\al \ot \al)=\al \ot e\sh \al$ by (1).(?) Hence \eqref{eq:787-12-1}$\subeq $\eqref{eq:787-12-2}.

\eqref{eq:787-12-2}$\subeq $\eqref{eq:787-12-1}: We have the diagram
\[
\xymatrix{
G\ar[r]^{(\id,i)}\ar[rd] & G\times G\ar[r]^{\mu}& G\\
& S\ar[ru]_e &}
\]
This gives $\al i\sh \al = e\sh \al=1 $ using (1), so $\al\in R_Q^{\times}$.
\end{proof}
\fixme{Now we prove the theorem. We want to show $\ul{G^{\vee}}(Q')=G^{\vee}(Q')$. The LHS is in $R_{Q'}$ and the RHS is in $\Hom(R^{\vee}_{Q'},Q')$, [under arrow] duality of $Q'$ modules. Claim the isomorphism restricts to the bijection. The LHS (i) corresponds to multiplication preserving, the RHS (ii) identity preserving... See Mumford.} 
\end{proof}