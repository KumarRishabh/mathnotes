\lecture{Tue. 11/27/12}

We finish the proof of Tate's Theorem and give some applications.

First, recall our conditions. We fix a polarization $\la$ on $A$; then we have the Weil pairing $E^{\la}$, which is skew-symmetric.
\begin{enumerate}
\item[(a1)]
$\text{Hyp}(k,A,d,\ell)$:
\[
\bth{$B\in\abk$}
{$\exists$ polarization of degree $d^2$}
{$\exists$ isogeny $B\to A$ of degree $\ell$-power}/\cong
\text{ is finite.}
\]
\item[(a2)]
If $W\subeq \vl A$ is a maximal $E^{\la}$-isotropic $\ql$-subspace, which is $\Ga$-stable. Then there exists $u\in E_{\ell}$ such that $u(\vl A)=W$.
\item[(a3)]
If $\ell\ne \chr k$ and $F_{\ell}\cong \ql\times \cdots \times \ql$, then $(A_{\ell}''-2)$.
\item[(a4)]
$F:=\Q(\Frob_q)\sub E=\End^0(A)$ is a semisimple $\Q$-algebra and as $\ql$-algebras, $F\ot_{\Q} \Q_{\ell}\cong F_{\ell}$. 
\end{enumerate}•
Last time we we in the middle of proving the following.
\begin{thm*}[Theorem~\ref{thm:tate}]
Let $A,B\in \abk$, let $k=\fq$, and let $\Ga=G(\ol k/k)$. Let $\ell\ne p$ be prime, where $p=\chr k$. Then the natural map
\[
\Hom(A,B)\ot_{\Z}\zl \xrc \Hom_{\zl \Ga}(\tl A,\tl B)
\]
is an isomorphism.
\end{thm*}
Last time we reduced to showing that 
\begin{enumerate}
\item[$(A_{\ell}')$] For one $\ell$,
\[
\End(A)\ot_{\Z} \ql \xrc \End_{\ql \Ga} (\vl A).
\]
\item[(B)] $\dim_{\ql} \Hom_{\ql \Ga}(\vl A,\vl B)$ is independent of $\ell$, for $\ell\ne p$. 
\end{enumerate}
We further reduced $(A_{\ell}')$ to the following two conditions.
\begin{enumerate}
\item[$(A_{\ell}''-1)$] $F_{\ell}$ is a semisimple $\ql$-algebra.
\item[$(A_{\ell}''-2)$] $F_{\ell}=Z_{\End_{\ql}(\vl A)}(E_{\ell})$.  %(the commutant).
\end{enumerate}
Here $E_{\ell}$, $F_{\ell}$ are the images in the following diagram.
\[
\xymatrix{
\End (A)\ot \ql \ha{rd}^{\cal \vl} & \\
& \End_{\ql}(\vl A)\\
\zl[\Ga]\ar[ru]_{} & 
}
\]
Recall that our plan is to show (a1)$\implies$(a2)$\implies$(a3) and (a3)$+$(a4)$\implies$(A). %Then applying Chebotarev we get $(A_{\ell}''-1)$ and $(A_{\ell}''-2)$. Choose $\ell$ splits completely in each of the fields contained in $F$.
We then choose $\ell$ splitting completely in $F$ using Chebotarev to get that the conditions of (a3) hold.

We've seen that (a1) holds, and that (a1)$\implies$(a2). %, (a1)$\implies$(a4). 


First, we'll review (a1)$\implies$(a2). (a2) is very technical; its use is apparent in the proof of (a3).

\subsection{Proof of (A)}
\subsubsection{Review: Proof of (a1)$\implies$(a2)}

We defined an intermediate lattice 
\[\ell^n\tl A\sub X_n:=(\tl A\cap W)+\ell^n \tl A\sub \tl A.\]
We constructed $f_n:B_n\to A$ as an isogeny of $\ell$-power degree with $f(\tl B_n)=X_n$. %\tl A_n, image precisely X_n.
We exhibited a polarization $\la_n:=\ell^{-n}f_n^{\vee} \la f_n$ on $B_n$ of degree $d^2$ (degree independent of $n$). Then we applied (a1) to show there are finitely many isomorphism classes of $B_n$. Thus we can find a subsequence $\{B_{i_j}\}$ all isomorphic over $k$. Then we obtained maps $u_j$ making the following commute:
\[
\xymatrix{
B_{i_1} \ar[d]_{f_{i_1}}\ar[r]^{\cong} & B_{i_j}\ar[d]_{f_{i_j}}\\
A\ard{r}_{u_j\in \End^0A} & A.
}
\]
We found an infinite subsequence $\{u_{j_k}\}$ converging to $u\in E_{\ell}$ %l-adic limit. may not be in end^0 A.
and we checked $u(\vl A)=W$. %larger larger trivial
%take quotient of \tl A by one of the group schemes. Quotient of A by some l-power order finite group scheme.

\subsubsection{Proof of (a2)$\implies$(a3)}
Let $D_{\ell}:=Z_{\End_{\ql}(\vl A)}(E_{\ell})$. %centralizer
We claim that $F_{\ell}\subeq D_{\ell}$. Indeed, by definition $F_{\ell}$ is the image of $\ql \Ga$, and all endomorphisms in $E_{\ell}$ are defined over $k$ and hence commute with the Galois action.
%$F_{\ell}\sub D_{\ell}$ because $\Ga$ commutes with $E_{\ell}$ on $\vl A$. Endomorphism defined over $k$, then action on Tate module commutes with Galois action.

We want to show $D_{\ell}\subeq F_{\ell}$.
The key claim is the following. 
\begin{clm}\llabel{clm:787-21}
Let $W\sub \vl A$ be an isotropic subspace with respect to $E^{\la}$. If $W$ is $F_{\ell}$-stable, then $W$ is $D_{\ell}$-stable.
\end{clm}
Let's assume this for now.

We know that $\vl$ is a module over $F_{\ell}\cong \ql\times \cdots \times \ql$. Then we can decompose
\[
\vl A=V_1\opl V_2\opl\cdots \opl V_r
\]
where the $i$th $\ql$ in $F_{\ell}$ operates on $V_i$. The decomposition is obtained from multiplying by idempotents:  $V_1=(1,0,\ldots,0)V_{\ell}A$, and so forth.

For all $i$, for all nonzero $v\in V_i$, consider $W=\ql v_i$, the line generated by $v$. Because the action of each $\ql$ is just scalar multiplication, $W$ is $F_{\ell}$-stable. because $E^{\la}$ is skew-symmetric and the line is 1-dimensional, $W$ is isotropic.

Because $W$ is skew-symmetric and isotropic, applying Claim~\ref{clm:787-21} gives that $W$ is $D_{\ell}$-stable. Because every vector in $V_i$ is an eigenvector, $D_{\ell}|_{V_i}$ is a scalar operator. But $F_{\ell}$ contains all possible combinations of scalar operators for each $V_i$, so $D_{\ell}\subeq F_{\ell}$. This shows that (a2)$\implies$(a3). %modulo the proof of the key claim. 

We now prove the key claim. 
\begin{proof}[Proof of Claim~\ref{clm:787-21}]
We use a decreasing induction on $\dim W$. Recall that if $A$ has dimension $g$,  $\vl A$ has dimension $2g$. Because $E$ is skew-symmetric, the maximal isotropic subspace is of dimension $g$. 
\begin{enumerate}
\item
Base case: Suppose $W$ is maximal isotropic: $\dim W=g=\dim A$. This is where we use (a2). By (a2), there exists $u\in E_{\ell}$ with $u(\vl A)=W$. Then
\[
D_{\ell}(W)=D_{\ell}[u(\vl A)]=u[D_{\ell}(\vl A)]\subeq u(\vl A)=W.
\]
\item
Let $d\le g$ be given. We give the induction step from dimension $d$ to dimension $d-1$. Suppose $W$ is $F_{\ell}$-stable and isotropic as in the claim, and $\dim W=d-1$. We create an isotropic subspace which is $F_{\ell}$-stable and has dimension one larger. %w/o tampering with these conditions. concoct
Consider the orthogonal space
\[
W^{\perp}:=\set{v\in \vl A}{E^{\la}(v,w)=0}.
\]
We can check
\begin{itemize}
\item
$W\subeq W^{\perp}$: This follows since $W$ is isotropic.
\item
$W^{\perp}$ is $F_{\ell}$-stable: 
%the pairing is more or less equivariant with respect to Galois action.
%reduce to canonical pairing
We have a canonical pairing $e_{\ell}:T_{\ell}A\times \tl \av \to \zl(1)$. It suffices to show $e_{\ell}(\ga x,\ga,y)=0$ iff $e_{\ell}(x,y)=0$ for all $\ga\in \Ga$. But this is clear: because the pairing comes from Cartier duality, $e_{\ell}(\ga x,\ga y)=\ga e_{\ell}(x,y)$. %, so this works. %galois equivariance of the pairing.
\end{itemize}
The idea in the following is that since $W^{\perp}$ is strictly larger, we can attach one more dimension to $W$ to use the induction hypothesis.

Since $F_{\ell}\cong \ql \times \cdots \times \ql$, the $F_{\ell}$-module injection $W\hra W^{\perp}$ splits,. As the irreducible $\ql$-modules are 1-dimensional, we can write % If we have an inclusion of vector spaces we can split such a sequence. %W^{\perp} to W and complementary subspace. decomp everything.
%We can write
\[
W^{\perp}=W\opl \pa{
\bigoplus_{i=1}^{2(g-\dim W)} L_i
}
\]
where $\dim_{\ql}L_i=1$ and $L_i$ is $F_{\ell}$-stable.
\end{enumerate}

Note that $2(g-\dim W)\ge 2$. Consider $W\opl L_i$, $i=1,2$. They are 
\begin{itemize}
\item
$F_{\ell}$-stable (clear), and
\item
isotropic. Indeed, we have $E^{\la}(W,W)=0$, and and $L_i\sub W^{\perp}$ implies $E^{\la}(L_i , W)=0$. Finally, $E^{\la}(L_i,L_i)=0$ by skew-symmetry. 
\end{itemize}
By the induction hypothesis, $W\opl L_i$ are $D_{\ell}$-stable. Hence $W=(W\opl L_1)\cap (W\opl L_2)$ is $D_{\ell}$-stable.
\end{proof}

\subsubsection{Proof of (a4)}

It remains to prove (a4). The key claim is a comparison between the geometric action $\Frob_q$ and arithmetic (Galois) action $F_q$. Here $F_q$ sends $X$ to $X^q$; it is a topological generator on $\Ga$. This induces some action on $\vl A$ which we'll denote by $F'_q$.

%\begin{proof}
%It suffices to prove $\Frob_q=F_q$ on $A(\ol \fq)$. Then the maps will be the same on $A[\ell^n](\F_q)$, and hence the same on $\tl A$ and $\vl A$. 
%We have a diagram
By definition of the Frobenius maps, we have
\[
\commsq{\Spec \ol \fq}{\Spec \ol \fq}{A}{A.}{f}{F_q}{f}{\Frob_q}
\]
%Statement about
Now we have an injection 
\[
\cal \vl:
E:=\End^0(A)\hra \End_{\ql}(\vl A).
\]
%We have $F=\Q(\Frob_q)\sub \End^0(A)$. We have in $F_{\ell}\twoheadleftarrow \ql \Ga$ mapping $\im(F_q)\mapsfrom F_q$. The claim gives, 
%Since $\cal \vl $ is injective, 
This shows that $\vl (\Frob_q)=F_q'\in F_{\ell}$. Then $\cal \vl|_F$ factors through $F_{\ell}$:
\[
\xymatrix{
E=\End^0A\ha{r}^{\cal \vl } & \End_{\ql}(\vl A) \\
F=\Q(\Frob_q) \ard{r}^{\exists !} \ha{u} & F_{\ell} \ha{u}
}
\]
%Thus $\cal \vl|_F$ factors through $F_{\ell}$. We deduce %(dim over Q same. equality actually holds. can show once have main theorem.
$F\subeq Z(E)$. (Because $\Ga$ centralizes $E$, so does $F_{\ell}$ and hence $F$.) Hence $F$ is a semisimple $\Q$-algebra. 

Next we show that in the diagram above, $F\ot_{\Q}\ql\cong F_{\ell}$. Injectivity is clear because $\cal \vl$ is injective. Surjectivity follows from the fact that $F_q^{\Z}\sub \Ga$ is dense, hence $\cal V_{\ell}(F\ot \ql)\subeq F_{\ell}$ is dense. %(These are finite dimensional over $\ql$.)
%Since $F_{\ell}$ is a finite-dimens
%center of semisimple alg semisimple.sub of semsimp conn
%frownface \Frob_q=\smatt 1110\in M_2(\Q)$.

%all elts right
%left: integral powers.
%integral powers, all galois powers.
%From algebraic number theory, 
A finite-dimensional vector space over $\ql$ is complete. Thus $\cal V_{\ell}(F\ot \ql)=F_{\ell}$. This proves (a4).

Note that also $F_{\ell}$ is semisimple (and ($A_{\ell}''-1$)), because it is in the center of $\End_{\ql}(A)$.

To see (A), note by the Chebotarev density theorem we can take $\ell$ splitting completely in $F$. Then $F\ot_{\Q}\cong \ql\times \cdots \times \ql$, so by (a4), $F_{\ell}\cong \ql\times \cdots \times \ql$. By (a3), ($A_{\ell}''-2$) holds. Hence (A) holds. %the conclusion of (a3) holds.) %is now taken care of. The proof of (A) is complete.
%\end{proof}

We used the fact that $k$ is a finite field in (a4). The other place we used it was in the finiteness hypothesis in (a1). %The reduction steps for number fields work the same (recycled?), but 
%finiteness hypothesis work more, much more difficult.

It remains to justify part (B). 
\subsection{Proof of (B)}
We need the following lemma from linear algebra.
\begin{lem}[Rational canonical form]\llabel{thm:rcf}
Let $V$ be a finite-dimensional $k$-vector space, where $k$ is any field. Let $\phi\in \End_k(V)$ be semisimple (i.e., diagonalizable over $\ol k$). Let 
\[
P_{\phi}:=\prod_{f\in k[x]}f^{a(f)}.
\]
 be the characteristic polynomial of $\phi$. 
Then $V\cong \bigoplus_{f\text{ irreducible}} (k[x]/f)^{a(f)}$ as $k[x]$-modules, where $x$ acts on $V$ as $\phi$.
\end{lem}
%(Characteristic polynomial gives information about what's on the diagonal.)

\begin{df}
Let $A,B\in \abk$. Let $P_{\Frob,A},P_{\Frob,B}\in \Q[X]$ be the characteristic polynomials of $\Frob_q$  on $\vl A$, $\vl B$. For any extension $K/\Q$, write
\[
P_{\Frob_q,A}=\prod_{f\in K[X]\text{ irreducible}} f^{a_K(f)},\qquad P_{\Frob_q,B} =\prod_{f\in K[X]\text{ irreducible}} f^{b_K(f)}.
\]
Define
\[
r_k(A,B)=\sum_f a_k(f)b_k(f)\deg f\in \Z.
\]
\end{df}
By Lemma~\ref{thm:rcf} it suffices to prove the following claim.
\begin{clm}
Keep the setup. For $\ell\ne p$, we have
\begin{enumerate}
\item%want prove indep of $\ell$
$\dim_{\ql}\Hom_{\ql \Ga}(\vl A, \vl B)=r_{\ql}(A,B)$.
\item
$r_{\ql}(A,B)=r_{\Q}(A,B)$.
\end{enumerate}
\end{clm}
\begin{proof}
(1) Using the fact that $\Frob_q^{\Z}\sub \Ga$ is dense, the LHS is $\dim_{\ql} \Hom_{\ql(\Frob_q)} (\vl A, \vl B)$.  By the lemma, this equals
\[\dim_{\ql}\Hom_{\ql(\Frob_q)} \pa{
\bigoplus_f (\ql[X]/f)^{a_{\ql}(f)}, \bigoplus_g (\ql[X]/g)^{b_{\ql}(g)}.
}\]
(From (a4), $\Frob_q$ is semisimple.) %\fixme{Why need semisimple?} 
The only nontrivial homomorphisms between simple modules are endomorphisms, so this equals 
\[
\dim\pa{\bigoplus_f  \cal M_{a_{\ql}(f)\times b_{\ql}(f)} \pa{
\End_{\ql\text{-module}}\pa{\ql[X]/f}
}}
\]
Note that the $\End$ has dimension equal to the degree of $f$. Thus the total dimension is
\[
\sum_f a_{\ql}(f)b_{\ql}(f) \deg f=r_{\ql}(A,B).
\]

%f separable no rept root
(2) Write $P_{\Frob_q,A}=\prod_{f\in \Q[X]\text{ irreducible}}f^{a_{\Q}(f)}$. 
%Each factor $f\in \ql[X]$ is irreducible. 
Decompose $f=\prod_{f_i\in \ql[X]\text{ irreducible}} f_i$. Note because $f$ is separable there is no repeated factor. This shows $a_{\Q}(f)=a_{\ql}(f_i)$, and $\deg(f)=\sum_i \deg(f_i)$. We also have $b_{\Q}(f)=b_{\Q}(f_i)$. This shows $r_{\Q}(A,B)=r_{\ql}(A,B)$.
\end{proof}
This completes the proof of Tate's Theorem.
\subsection{Application}
We highlight one application which we'll use in Honda-Tate Theory.
\begin{thm}[Tate, \S3]$\,$
Let $A$ be abelian varieties over a finite field $k$.
\begin{enumerate}
\item[(a)]
We have
\[
\rank_{\Z}\Hom(A,B)=r_{\Q}(A,B).
\]
\item[(b)]
Omitted.
\item[(c)]
The following are equivalent. 
\begin{enumerate}
\item[(c1)]
$B\sim A$, i.e., $B, A$ are $\fq$-isogenous.
\item[(c2)]
$P_{\Frob_q,A}=P_{\Frob_q,B}$.
\item[(c3)]
$\ze(s,A)=\ze(s,B)$.
\item[(c4)]
$\#A(\F_{q^r})=\#B(\F_{q^r})$ for all $r\ge 1$.
\end{enumerate}•
\end{enumerate}•
\end{thm}
\begin{proof}
Use the main theorem and the proof of (B).

Note $(c2)\iff (c3)\iff (c4)$ is standard. In fact, these equivalences hold for
smooth proper varieties over a finite field, if we require (c2) to be true for all cohomological degrees. %\'Etale cohomology on every degree. 
(c1) is specific to abelian varieties. 

(c1)$\iff$(c2): (c1) implies $\vl A\cong \vl B$ as $\ql\Ga$-modules. This gives (c2). To go backwards use Tate's Theorem~\ref{thm:tate}. If $\ph_{\ell}:\vl A\to \vl B$ is an injection, then there is $\ph:A\to B$ in $\Hom(A,B)\ot \Q$ approximating $\ph_{\ell}$ that is injective. %(By ``approximate" we mean that thinking of $\ph,\ph_{\ell}$ as a matrix acting on the Tate module, make all matrix entries close.)
%may have to multiply to make sure send Tate in Tate.
%Dim. Injective hom, has to be isogeny. 
Since it is an injective homomorphism between abelian varieties of the same dimension, it has to be an isogeny.
\end{proof}