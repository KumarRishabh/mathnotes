\lecture{Tue. 12/11/12}

Today we'll discuss Honda-Tate Theory. 
%\fixme{which does...}
We'll start by recalling some facts about Brauer groups.

\subsection{Brauer groups}

Let $F$ be any field. We're particularly interested in the $p$-adic case.
\begin{df}
Define the \textbf{Brauer group} as the group of similarity classes of central simple algebras over $F$, 
\[
\Br(F)=(\text{CSA}/F)/\sim
\]
with multiplication given by tensor product. We impose that $A\sim \cal M_n(A)$ for any algebra $A$.
\end{df}
The Brauer group is generated by division algebras $D$. %\sim \cal M_n(D)$.

For $F$ a number field, we have
\begin{enumerate}
\item
for $v$ non-archimedean,\footnote{This is one of the ingredients that goes into saying we have a {\it class formation} in local class field theory. Note $\Br(F_v)=H^2(\ol{F_v}/F_v)$.}
\[
\inv_v:\Br(F_v)\xra{\cong} \Q/\Z
\]
\item
for $v$ real,
\[
\inv_v:\Br(F_v)\xrc \rc2\Z/\Z
\]
\item
for $v$ complex,
\[
\Br(F_v)=0.
\]
\end{enumerate}

In the number field case we have the exact sequence (See Theorem 27.3.5 and $\S27.4$ in my number theory text.)
\beq{eq:787-brauer-seq}
\xymatrix{
0\ar[r] & \Br(F) \ar[r] & \bigopl_v \Br(F_v) \ar[r]^-{\sum \inv_v} & \Q/\Z\ar[r] & 0\\
 & D\mt{r} & \{D\ot_F F_v\}& &
}
\eeq
In other words, given $\bc{\fc{r_v}{s_v}\in \Q/\Z}_v$ that is zero for almost all $v$ and $\sum_v \fc{r_v}{s_v}\equiv 0\pmod 1$, there exists, up to equivalence, a unique central division algebra $D/F$ such that $\inv_v(D_v)=\fc{r_v}{s_v}$. We have $[D:F]^{\rc 2}=\lcm_v s_v$. %examine the proof 

\subsection{More on $p$-divisible groups}
Let $k$ be perfect of characteristic $p>0$. Recall that there is an anti-equivalence of categories
\[
\mathbb D:\pat{BT/$k$}:=\pat{$p$-divisible group/$k$}\xrc \bth{$W[F,V]$-modules}{free of finite rank}{as $W$-modules}.
\]
(We can get a covariant version by dualizing.)
%confused, inconvenient, dual, reverse arrows.
In particular, this sends
\begin{align*}
\mu_{p^n}=\lim \mu_{p^n} &\mapsto (W, F=p, V=1)\\
\Q_p/\Z_p=\lim \rc{p^n} \Z_p/\Z_p & \mapsto (W, F=1,V=p).
\end{align*}
%dualize flip roles F, V
These are $W$-modules rather than vector spaces over the fraction field of $W$. We introduce a rational version, which is more convenient to work with.
Let 
%\si=\Frob operating on :=?
\[
L:=\Frac(W)=W\ba{\rc p}.
\]
Let $(\text{BT}^0/k)$ be the category with the same objects but with 
\[
\Hom^0(\ura G,\ura H):=\Hom(\underrightarrow{G},\underrightarrow{H})\ot_{\Z_p} \Q_p.
\]
Then we get the following anti-equivalence of categories, induced by $\mathbb D$:
\[\mathbb V:
\pat{BT${}^0$/$k$}\xrc
\bt{finite-dimensional $L$-vector space $\cal V$}{$F:\cal V\xrc \cal V\quad \si$-linear}=:\pat{Isoc/$k$}.
\]
(The RHS is called \textbf{isocrystals} over $k$, ``iso" standing for ``isogeny.")
By $\si$-linear we mean $F(\la v)=\si(\la)F(v)$. Note we only need data on $F$ because $V=p\circ F^{-1}=F^{-1}\circ p$. When we rationalize, $F$ and $V$ has to be bijections, because multiplication by $p$ is bijection (it wasn't a bijection on the $W[F,V]$-module).

$\Hom^0$ is useful in classifying abelian varieties up to isogeny. By Poincar\'e reducibility, abelian varieties decompose into simple varieties. %Here, we have a category and 
We want to identify and classify the simple objects. The way to do this is look at the linear algebraic category $\pat{Isoc/$k$}$ and do linear algebra.

If $k=\ol k$, there are no interesting extension classes between objects; every object in $\pat{Isoc/$k$}$ can be decomposed as a finite direct sum of simple objects:
\[
\bigopl_{i} L[F]/(F^{s_i}-p^{r_i})^{\opl e_i}
\]
where $\fc{r_i}{s_i}\in \Q$ is called the \textbf{slope}.\footnote{In fact it is the slope of a side of the Newton polygon of the characteristic polynomial of $F$.}  
On the simple objects, $F$ can act like $p$ to some rational power. 

When $k=\ol k$, these are all the objects.
\begin{thm}
$\mathbb V$ induces an equivalence of categories
\[
(\text{BT}^0/k)\xrc (\Isoc/k)%^{[0,1]}
\]
where the RHS is generated by $\fc{r_i}{s_i}\in \Q\cap [0,1]$.
\end{thm}

An application is the following. If we have a decomposition on the RHS, this corresponds to a decomposition on the LHS.

Let $\Si\in (\text{BT}/k)$. We have an isogeny
\[
\Si\to\prod_{i}(\Si_{\fc{r_i}{s_i}})^{e_i}
\]
%($p$-divisible group version of decomp?)
%semisimpl by construct functor and showing semisimp on RHS
%fppf snese
%Related to Brauer group computation.
%end of this guy, end of that guy.
where $\Si_{r_i/s_i} =\mathbb V^{-1} (L[F]/(F^{s_i}-p^{r_i}))$. 
We can now compute the endomorphisms of $\Si$:
\begin{align*}
\End^0(\Si)&=\prod_i \cal M_{e_i} (\End^0(\Si_{r_i/s_i}))\\
&=\prod_i \cal M_{e_i}(D_{r_i/s_i}).
\end{align*}
Here $D_{\fc{r_i}{s_i}}$ is a central division algebra with $\inv_vD =\fc{r_i}{s_i}$. To see this, we can compute this in linear algebraic category $\pat{Isoc/$k$}$.

\begin{ex}
For instance, if we have an elliptic curve $E$ over $\Fp$, $E[p^{\iy}]$, ordinary elliptic curves correspond to $\Q_p\times \Q_p$, with slope 0 or 1. By semisimplicity, $D$ is a CDA with $\inv$ 1 (?).
\end{ex}
\subsection{Tate's Theorem for $\ell=p$}
We give an analogue of Tate's Theorem~\ref{thm:tate} for when $\ell= p$.
\begin{thm}[Tate]
Let $A\in (\text{Ab}/\fq)$. Then
\[
\End(A)\ot_{\Z}\Z_p\cong \End_{W[F,V]}(\mathbb D(A[p^{\iy}]))^{\text{op}}
\]
\end{thm}
\begin{proof}
This follows directly from the equivalence of categories.
\end{proof}
We take the opposite because $\mathbb D$ is contravariant.

The theory seems analogous to that for Tate modules. Is there some big theory bringing them together? We expect some grand theory of unification. $\mathbb D$ generalizes to the $p$-adic crystalline cohomology theory and for $\ell\ne p$ we have the \'etale cohomology theory. Hopefully, there is some motivic cohomology theory that feeds both crystalline and \'etale cohomology theory depending on whether $\ell=p$.

%Betti, de Rham cohom
\subsection{Honda-Tate Theory}
\subsubsection{Weil numbers}
\begin{df}
Let $q=p^a$, $a\in \N$. A \textbf{Weil $q$-integer ($q$-number)} of weight $a$ is $\al\in \ol{\Z}$ (or $\ol{\Q}$)
%weight p-integer of weight a
such that
\[
\iota(\al)\ol{\iota(\al)}=q
\]
for all $\Q(\al)\hra \C$.
\end{df}
In the weight 0 case, the Weil $q$-integers (numbers) are exactly the roots of unity.

\begin{rem}
If $\al$ is a Weil $q$-integer, then for all $\ga\in \Ga_{\Q}:=G(\ol{\Q}/\Q)$,  $\ga\al$ is also a Weil $q$-integer. 
\end{rem}

We can rephrase the Riemann hypothesis succintly using abelian varieties.
\begin{thm}[Riemann hypothesis for abelian varieties]
The roots of $P_{\Frob_q,A}\in \Z[X]$ are Weil $q$-integers.
\end{thm}
We didn't prove this, but basically the fact that %the
%Frobenius times the Rosati involution of Frobenius is $q$
$\Frob_q^{\ri_L}\Frob_q=[q]$ (Theorem~\ref{thm:frob-ri-frob}) underlies the proof of the theorem. 

\subsubsection{Honda-Tate Theory}

Write $\pi_A=\Frob_{q,A}\in \End(A)$. Suppose $A$ is $\fq$-simple. Then $\pi_A\in \Q(\pi_A)\subeq \End(A)$, $\End(A)$ is a division algebra, and its center $\Q(\pi_A)$ is a field. We can choose an embedding $\Q(\pi_A)\hra \ol{\Q}$, well defined up to the Galois action $\Ga_{\Q}$ of $\ol{\Q}$. This assigns a well-defined element $\pi_A\in \ol{\Q}$ for any abelian variety, which is a Weil $q$-integer by the theorem.

Summarizing, we have a map
\begin{align*}
\bt{$\F_q$-simple}{abelian variety/$\fq$}/\fq\text{-isogeny} &\to \pat{Weil $q$-integer}/\Ga_{\Q}\text{-action}\\
A&\mapsto \pi_A.
\end{align*}
Note that $P_{\Frob,A}$ is a power of an irreducible polynomial.
%power of irreducible polynomial
%2 vect coprime, can use decomp to decomp
%divides twice dimension when simple variety. not in general

\begin{thm}[Honda-Tate, 1967]\llabel{thm:honda-tate}
The map above is a bijection.
\end{thm}
Tate showed this map is injective, and Honda showed it is surjective in his Ph.D. thesis in 1968.

By Honda-Tate, we can define an inverse map $\pi\mapsto A_{\pi}$.
\begin{thm}\llabel{thm:honda-tate2}
We have the following.
\begin{enumerate}
\item 
$E_{\pi}:=\End^0(A_{\pi})$ is the unique CDA over $F_{\pi}:=\Q(\pi)$ such that for $v$ a place of $F_{\pi}$, 
\[
\inv_v(E_{\pi})=\begin{cases}
0,&v=\C\\
\rc2,& v=\R\\
0,&v\nmid p,\,v\nmid \iy\\
\fc{v(\pi)}{v(q)}[F_{\pi,v}:\Q_p], & v\mid p,\,v\nmid \iy.
\end{cases}
\]
%(Since $v=\R$, it's always a quaternion algebra. Always get imaginary quad field so real case does not occur.)
%specify all but one, then last one determined.
\item
Moreover,
\[
2\dim(A_{\pi})=[E_{\pi}:F_{\pi}]^{\rc2}[F_{\pi}:\Q].
\]
%sum nums is 0 in \Q/\Z, you can show
\end{enumerate}
\end{thm}
As a sanity check, we can check that the sum of the invariants claimed above is equal to 0 in $\Q/\Z$ (exercise). 
Note $v=\R$ occurs rarely, so you can argue that case by case. Use $\pi\ol{\pi}=q$, i.e., $v(\pi)+v(\ol{\pi})=v(q)$. 
%argue similar when inert, ramified. 1/2, even. Still.

Note $F_{\pi}=\Q(\pi)$ has a well-defined complex conjugation. We have $\Q(\pi)\hra \C$, and regardless of the embedding, conjugation in $\C$ induces the same automorphism in $\Q(\pi)$. This is simply because complex conjugation sends $\pi\mapsto \fc{q}{\pi}$, and $\fc{q}{\pi}$ is well-defined before embedding.

This implies $\pi$ is totally real or CM. The totally real case occurs rarely; in fact we can give a complete list of cases where it occurs. We'll sketch one case of this in the following example.
\begin{ex}
First note that $F_{\pi}=\Q(\pi)$ is totally real iff complex conjugation is trivial on $\Q(\pi)$, i.e., $\pi^2=q$. Let $q=p^a$.

First suppose $a$ is even. Then $F_{\pi}=\Q$. %something 1/2
Then $E_{\pi}$ is a CDA over $\Q$ ramified at exactly $p$, $\iy$. This is quaternionic, and it corresponds to a supersingular elliptic curve, unique up to isogeny over $\ol{\Fp}$ (in fact, isogeny over a quadratic extension). The dimension is 1 by part 2 of the theorem: %replace Frob by Frob^2, become same.
%\pm \sqrt{p}.
%isomorphic after quadratic extnsion.
\[
2\ub{\dim(A_{\pi})}1={\ub{[E_{\pi}:F_{\pi}]}{4}}^{\rc2}\ub{[F_{\pi}:\Q]}{1}.
\]

Now suppose $a$ is odd. %what you will see is a little odd
Then $F_{\pi}=\Q(\sqrt p)$, and $E_{\pi}$ is a CDA over $\Q(\sqrt{p})$ ramified at two real places. 
The dimension is 2.
\[
2\ub{\dim(A_{\pi})}2={\ub{[E_{\pi}:F_{\pi}]}{4}}^{\rc2}\ub{[F_{\pi}:\Q]}{2}.
\]
%Supersingular can define over $\F_{p^2}$ but not $\F_p$. 

For instance, when $a=1$, there is no supersingular elliptic curve over $\F_p$, but there is a supersingular elliptic curve $E$ over $\F_{p^2}$. Take the product of $E$ and its Galois twist; it will be defined over $\F_p$ and be an abelian variety of dimension 2. By construction, when we basechange to a quadratic extension it splits into two supersingular elliptic curves.
\end{ex}
Now we understand the totally real case, so we understand half of the theory.\\

\prbbox{
Let $0\le \fc rs<\rc2$. Suppose
\[
x^2-p^rx+p^s=0
\]
has roots $\{\pi,\pi'\}$. Show the following.
\begin{itemize}
\item
$\pi$ and $\pi'$ are Weil $q$-integers where $q=p^s$.
\item
The corresponding places $v,v'$ in $\Q(\pi)$ extend $v_p$. %\fixme{unramified?}
\item
$v(\pi)=\fc rs$, $v(\pi')=v'(\pi)=1-\fc rs$.
\end{itemize}•
}
\vskip0.15in

In this case $\fc{v(\pi)}{v(p)}=$\fixme{?} We get a division algebra split outside $p$, and at $p$, the invariants can be $\fc rs$ or $1-\fc rs$. It's not quaternionic but it can be arbitrarily large. The reason that Tate mentions this example is because this construction gives an answer to a question posed by Yuri Manin: Under what conditions does a $p$-divisble group arise as the $p$-divisible group of some abelian variety? We can construct any $p$-divisible group as the $p$-divisible group of an  abelian variety, given it is symmetric, which roughly means  isogenous to its dual.

We sketch a proof of Theorem~\ref{thm:honda-tate2}.
\begin{proof}[Proof of Theorem~\ref{thm:honda-tate2}]
The proof for the totally real case comes from basic facts about supersingular elliptic curves and the above example.

We focus on when $F_{\pi}$ is a CM field. We know $\Br(\C)=\{0\}$, so there is nothing to prove about invariants at archimedean places. %There is no $\R$ case. 
Now suppose $v\nmid \iy$, $v\nmid p$. By Tate's Theorem~\ref{thm:tate} for $\ell\ne p$ we have
\[
E_{\pi}\ot_{\Q}\ql \cong \End_{\ql(\pi)}(\vl A_{\pi}).
\]
%take centralizer, still
The RHS is semisimple by Tate's Theorem and hence the product of matrix algebras over finite extensions of $\ql$.
The LHS is
\[
\prod_{v\mid \ell} E_{\pi}\ot_F F_{\pi,v}.
\]
By matching, we get 
\[
\inv_v(E_{\pi,v})=0.
\]
%This uses Tate's Theorem when $\ell\ne p$.

For $v\mid p$ we do something similar except we use Dieudonn\'e Theory and Tate's Theorem for $\ell=p$.
\end{proof}
(The key fact at play here is that by~\eqref{eq:787-brauer-seq}, to understand the endomorphism algebra it's sufficient to understand it over every $\ql$ and $\Q_p$, and Tate's Theorem tells us we can just understand the associated $\ell$-adic Tate module or $p$-divisible group.)

We've converted proving facts about abelian varieties over finite fields to proving facts about rational Dieudonn\'e modules. To do so, we had to go through p-divisible groups as an intermediate step. %to understand end alg, it's sufficient to understand LOCALLY because of the exact sequence.
By looking at the endomorphism algebra of Dieudonn\'e modules associated to $p$-divisible groups, we compute everything explicitly using linear algebra. There, the Frobenius (given by $\pi$) acts 
%Frobenius 
on simple objects like $p$ to a rational number. We can compute the endomorphism algebra from that rational number. 



The theorem is not easy, but we have the right way think about, it is ``elementary" (a linear algebra problem).
\begin{proof}[Proof of Theorem~\ref{thm:honda-tate}]
To show injectivity, suppose $\pi_A\sim \pi_B$ are $\Ga_{\Q}$-conjugate.

We want to show $\Hom(A,B)\ne 0$. 
It suffices to prove that $\Hom^0(A,B)\ne 0$ (because $\Hom(A,B)$ is torsion-free). Since $\ell\ne p$, it is enough to show $\Hom^0(A,B)\ot \ql$ is nonzero. But by Tate's Theorem,
\[
\Hom^0(A,B)\ot \ql\cong \Hom_{\ql(\Frob)}(\vl A, \vl B).
\]
Given that $\pi_A$ and $\pi_B$ are conjugate, can we find $\ql$-linear map from $\vl A$ to $\vl B$ intertwining $\Frob$? %Relatively elementary when . 
Here $\pi_A\in \vl A$ and $\pi_B\in \vl B$ are the embeddings of $\Frob$. 
We can use linear algebraic information to produce an actual map of algebraic varieties. %This morphism represents a key step.


%The map is $\pi_A:\ql(\Frob)\to \vl A$ and $\pi_B:\

For surjectivity, we say that given a CM-pair $(L,\Phi)$, we can define a CM abelian variety $(A,i)$ over a number field. %construct wrt Weil number $\pi$. 
%We can't just do this: arb data, arb reduction, cannot realize given Weil number in the reduction. 
We need some control of the data in order to realize the given Weil number. Taking $A$ modulo $w$, $w\mid p$, we get $A$ defined over a finite field with Frobenius $\pi_A$.

But how do we relate $\pi_A$ and $\pi$? We keep track of our data: how $L$ and $\Phi$ can be used to describe $\pi_A$. The key is to relate $(L,\Phi)$ to $\pi_A$. Additionally we have to choose $L$ carefully to start with. We compute the reduction of the CM abelian variety in terms of the CM pair explicitly. We can compute the $p$-adic valuations $v(\pi_A)$ in terms of $L,\Phi$. This is the key point.

These are other relatively minor issues: 
We have to choose $\ell$ carefully. We aer not in the $\F_q$ we started with, and have to adjust there. Even if we control the data, cannot arrange $\pi_A$ to be exactly $\pi$. 

We skip the proof of part 2.
\end{proof}
%q-weil num-> dim: part 2 of thm. E_pi, F_pi=Q(pi). 