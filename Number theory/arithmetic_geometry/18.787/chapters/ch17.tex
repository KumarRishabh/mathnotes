\lecture{Thu. 11/8/12}

Today we'll talk about the Rosati involution.

Recall that if $A\in \abk$ is an abelian variety, $(\ell,\chr k)$, we defined a pairing %cartier duality
\[
A[\ell^n]\times A[\ell^n]^{\vee} \to \mu_{\ell}^n.
\]
Using $A[\ell^n]^{\vee}\cong \av[\ell^n]$, and using the functorial nature of this isomorphism, we were able to patch these map together, taking the inverse limit over $\ol k$-points to get
%no can basis.
\[
\tl A\times \tl \av \to \Z_{\ell}(1),
\]
where $\Z_{\ell}(1)$ is a free $\Z_{\ell}$ module of rank 1 (without a canonical basis).

Given $\la\in \Hom(A,\av)$, we defined $E^{\la}$ using the following diagram.
%p pull back to m
\[
\xymatrix{
T_{\ell}A \ar@{}[r]|-{\times} \aq{d}& T_{\ell}\av \ar[r]^e & \Z_{\ell}[1]\aq{d}\\
T_{\ell}A\ar@{}[r]|-{\times}& T_{\ell}A  \ar[r]_{E^{\la}}\ar[u]_{\la} & \Z_{\ell}(1).
}
\]
A line bundle $L$ gives a pairing $E^L:=E^{\la_L}$.
$e$ is perfect $\Z_{\ell}$-linear; $E^L$ is $\Z_{\ell}$-linear but may not be perfect %worst case 0.
We showed that
\begin{itemize}
\item $E^L$ is skew-symmetric.
\item $E^{\la}$ is skew-symmetric. This implies $\la=2\la_L$ and $L=(1\times \la)^* \cal P$.
\end{itemize}
A variant of this is that letting
\[
V_{\ell}A:=T_{\ell} A\ot_{\Z_{\ell}} \Q_{\ell}
\]
ane $\la\in \Hom^0(A,A^{\vee})$ we get a $\Q_{\ell}$-bilinear pairing:
\[
E^{\la}:V_{\ell}\times \vl A\to \ql(1):=\zl(1)\ot_{\zl} \ql.
\]
We have that $E^{\la}$ is skew-symmetric iff $\la=c\la_L$ for some $c\in \Q$ and line bundle $L$. After killing denominator you know $\la=2\la_L$ for some $L$.

\subsection{Rosati involution}
\begin{df}
Let $B$ be a finite dimensional noncommutative semisimple $\Q$-algebra. A map $*:B\to B$ is an \textbf{involution} if $*$ is a $\Q$-algebra map $B\xrc B\op$ with $*^2=\id$.

The involution $*$ is \textbf{positive} $\tr_{B/\Q} (bb^*)>0$ for all $b\ne 0$. 
\end{df}
By this we mean the following. For $B=\prod B_i$, $Z(B)=\prod Z(B_i)$ and
\[
\tr_{B/\Q}(b_i)=\prod_i \tr_{Z(B_i)/\Q} (\tr_{B_i/Z(B_i)}^0(b_i))
\] 
for all $b\ne 0$. 
It doesn't hurt to think of the case of division algebras.

If you're over $p$-adic field, Hasse invariant changes from $\la$ to $1-\la$. But number field, can do at every place, classifying invariant for opposite algebra.

\begin{ex}\llabel{ex:qi-inv}
Suppose $[E:\Q]=2$. Let $1\ne c\in G(E/\Q)$. Then
\begin{itemize}
\item
$c$ is an involution.
\item 
$c$ is positive iff $E$ is imaginary.
%a^2+db^2, a^2-db^2
\end{itemize}
\end{ex}
\begin{ex}
Let $B=\cal M_n(\Q)$. Define $*:g\mapsto g^T$. It is not an automorphism because it changes the order; it is an homomorphism to the opposite algebra. This is a positive involution because $g^Tg$ is a sum of squares.

Now let $B=\cal M_n(\Q)\times \cal M_n(\Q)$. Then let $*:(g,h)\mapsto (h^T,g^T)$. This is an involution, but is not positive. For instance $(0,1)$ goes to $(1,0)$, and the trace is 0.

Now let $B=\cal M_n(E)$ as in Example~\ref{ex:qi-inv}. Then $*:g\mapsto (g^c)^T$ is an involution; it is positive if $R$ is imaginary.

This is still true if we replace $\Q$ by a totally real field and $E$ by a purely imaginary quadratic extension.
\end{ex}
Let's now define the Rosati involution.
\begin{df}
Let $L$ be a fixed ample line bundle. The \textbf{Rosati involution} 
\[\ri_{L}:\End^0\to \End^0\]
associated to $L$ is defined by 
\[
\ri_L(\phi)=\la_L^{-1}\circ \phi^{\vee}\circ \la_L. 
\]
\end{df}
\begin{pr}\llabel{pr:ri-inv}
We have the following.
\begin{enumerate}
\item
$\ri_L$ is an involution.
\item
($\ri_L$ behaves nicely under Weil pairing) $E^L(\phi x,y)=E^L(x,\phi^{\ri_L}y)$.
\end{enumerate}
\end{pr}
\begin{proof}
We have 
\bal
\ri_L^2&=\la_L^{-1} (\la_L^{-1} \phi^{\vee} \la_L)^{\vee} \la_L\\
&=(\la_L^{-1} \la_L^{\vee}) \phi (\la_L^{-1} \la_L^{\vee}).
\end{align*}
We've shown before that $\la_L=\la_L^{\vee}$; this is from the commutative triangle
\[
\ctrr{A}{(\av)^{\vee}}{\av}{\ka_A}{}{}{\la_L}{}{\la_{L}^{\vee}}
\]

For part 2, note the RHS equals
\bal
E^L(x,\la_L^{-1}\phi^{\vee}\la_Ly)
&=e(x,\la_L(\la_L^{-1} \phi^{\vee}\la_L)y)\\
&=e(\phi x,\la_Ly)\\
&=E^L(\phi x, y).
\end{align*}
This equals the LHS.
%proof pres, isolated
\end{proof}
Here's our first deep fact.
\begin{thm}
$\ri_L$ is a positive involution.
\end{thm}
\begin{proof}
See~\cite[\S21]{Mu70} for $k=\ol k$ and~\cite{GGBM} for $k\ne \ol k$.
\end{proof}
\subsection{Symmetric elements}
Consider the group homomorphism
\bal
\La:\Pic A&\to \Hom(A,\av)\\
\cL & \mapsto \la_{\cL}.
\end{align*}
To see this is a group homomorphism, note that the correspondence below associates $\la_{\cL}$ and $M(L)$ by definition:
\bal
\av(A)&\cong \Pic^0(A\times A)\\
\la_{\cL} & \lra M(\cL)=\mu^*\cL \ot p_1^* \cL^{-1} \ot p_2^*\cL^{-1}.
\end{align*}
Under this correspondence,
\bal
\la_{\cL_1}-\la_{\cL_2}&\lra M(\cL_1)\ot M(\cL_2)^{-1}\\
\la_{\cL_1\ot \cL_2^{-1}} & \lra M(\cL_1\ot \cL_2^{-1}).
\end{align*}
The RHS's are congruent, just by unraveling the definition of $M(\cL)$. Hence $\La$ is a group homomorphism.

\begin{pr}
Let $\cL\in \Pic A$. Then $\la_{\cL}=0$ iff $\cL\in \Pic^0A$. %defn different from Mumford. 
\end{pr}
Note our definition of $\Pic^0$ is a bit different from Mumford's. From Mumford's definition this is trivial but from our definition it's not.
%prove one thing by one def and another by other def.
\begin{proof}
$\Leftarrow$: See~\cite[\S8, Theorem 1]{Mu70}.

$\Rightarrow$: Exercise.
\end{proof}
%Saul: This is related to Chern classes. (?)

Choose an ample line bundle $L$. This gives an isogeny $\la_L\in \Hom^0(\av,A)$. Now consider the following diagram. 
\bal
\fc{\Pic A}{\Pic^0}\ot_{\Z}\Q\xra{\La\ot_{\Z}\Q,\cong} 
\Hom^0(A,\av) & \xra{\la_L^{-1}} \End^0(A)\\
\psi&\mapsto \la_L^{-1} \psi.
\end{align*}
\begin{df}
The \textbf{Neron-Severi} group of $A$ is
\[
\NS(A):=\fc{\Pic A}{\Pic^0A}.
\]
\end{df}
First map doesn't depend on $\ell$ but the second does.
\begin{pr}
We have 
\[
\im(\la_L^{-1}\circ \La_{\Q})=(\End^0(A))^{\ri_L=1}:=\set{\phi\in \End^0(A)}{\phi^{\ri_L}=\phi}.
\]
\end{pr}
We call the elements in the right set \textbf{symmetric elements}.
\begin{proof}
Let $\phi\in \End^0(A)$. We have 
the following chain of equivalences.
\begin{enumerate}
\item
$\phi\in \im(\la_L^{-1})$
\item
$\la_L\phi\in \im(\La_{\Q})$
\item
$\la_L\phi =c\la_{\cL}$ for some $\cL$ and $c\in \Q$.
\item %skew-sym
$e(x,\la_L\phi y)=-e(y,\la_L\phi x)$ for all $x,y$.
\item
$E^L(x,\phi y)=-E^L(y,\phi x)=E^L(\phi x,y)$.
\item
$E^L(x,\phi y)=E^L(x,\phi^{\ri_L}y)$ for all $x,y$. (To get here, we used Proposition~\ref{pr:ri-inv}.)
\item
$\phi y=\phi^{\ri y}$ for all $y\in \vl A$.
(by perfectness)
\item
$\phi=\phi^{\ri}$ in $\End^0(\tl A)$.
\item
$\phi=\phi^{\ri}$ in $\End^0(A)$ (because it injects into $\End^0(\tl A)$). 
\end{enumerate}
\end{proof}
\begin{cor}\llabel{cor:Qsubalg|g}
Let $A\in\abk$ of dimension $g$. For $K\sub (\End^0 Q)^{\ri_L=1}$ be a $\Q$-subalgebra  of $\End^0(A)$ (a priori non-commutative). Then $[K:\Q]\mid g$.
\end{cor}
This imposes a condition on what the center of the endomorphism algebra can be. Roughly speaking the center is preserved by the Rosati involution. If it's nontrivial it's an automorphism of order 2.

Index 1 or 2 subfield of the center is contained in here. For that field you can apply this corollary.
\begin{ex}
Suppose $A$ is simple. 
Then $K=(Z(\End^0(A)))^{\ri_L=1}$.
\end{ex}
%Saul: rank of NS divides g? SW: May not be subalg. A priori group hom \la_L^{-1}. Padma: image may not be...? 
%?%for the matrix algebra: really symmetric elements $\matt accb$.
%It's not necessarily preserved under mult. If closed under mult then commutative. May not be closed under mult. 
%not closed under mult!
\begin{proof}[Proof of Corollary~\ref{cor:Qsubalg|g}]
\step{1} $K$ is a field. Let $\phi,\psi\in K$. Because $K\subeq (\End^0A)^{\ri_L=1}$. 
%no chance closed under mult bc not comm.
\[\phi\psi=(\phi\psi)^{\ri=1}=\psi^{\ri_L}\phi^{\ri_L}=\psi\phi.
\]
\step{2} The function
\begin{align*}
\chi:\Pic A&\to \Z\\
M&\mapsto \chi(M)
\end{align*}
induces a map 
\[
\chi_{\Q}:\fc{\Pic A}{\Pic^0(A)}\ot_{\Z}\Q\to \Q
\]
%trivial on Pic^0. Not grp hom. Disturb by Pic^0 doesn't change E char. Now quotient and take tensor product, now show homog poly
We won't prove the first fact.
\begin{fct}(\cite[\S16]{Mu70})
$\chi(M\ot M')=\chi(M)$ for all $M'\in \Pic^0(A)$.
\end{fct}
Here is an intuitive explanation. Discrete locally constant valued function. Moduli spce $\Pic^0$. $M'$ defines point in conn scheme. Line bund corresp ident. Alg family connecting them. Euler char doesn't change. 
Constant in whole conn component. You can make this rigorous.

%general phenomenon for cohomology. flat? 
%oth semicont.
We know that $\chi$ is a polynomial function on $\Pic A\ot_{\Z}\Q$, homogeneous of degree $g$ (\S 16, $\chi(L)=\fc{(D^g)}{g!}$. Then $\chi(L^{\ot n})=\fc{n^g (D^g)}{g!}$); we already used this in Section 19. (calculating $\chi(M^n)$).\\

Let's consider the diagram from a slightly different perspective. %On the space of symmetric elements
We have the following maps
\[
\xymatrix{
(\End^0A)^{\ri_L=1} \ar@/_1pc/[rrd]_{\xi'} & \fc{\Pic A}{\Pic^0 A} \ot \Q\ar[l]^{\cong}  \ar[r]^{\chi_{\Q}}\ar[rd]_{\chi_{\Q}'} & \Q\\
 & & \Q
}
\]
where teh top left map is $\la^{-1}\la_M\mapsfrom M$. We have $\chi'=\rc{\chi(L)}\chi$ ($L$ is fixed).\\

\step{3} $\xi'|_K$ is a norm function.
%not norm on whole space. S.t. not closed under multiplication. We only talk about on subalgebra $K$. Otherwise something similar should happen on the whole space.

We have $\xi'(1)=\chi_{\Q}'(L)=\fc{\chi(L)}{\chi(L)}=1$ (this is why we scaled it). 

\begin{lem}
Let $B$ be a finite-dimensional vector space and $f:V\to \Q$ be polynomial with $f(1)=1$. Then $f(xy)=f(x)f(y)$ iff $f(xy)^2=f(x)^2f(y)^2$.
%elem exercise in alg
%not neg.
\end{lem}
\begin{proof}
Exercise.
\end{proof}
If suffices to prove that 
%check 2 elements, satisfy norm property for them.
\[
\xi'((\la_L^{-1}\la_M)(\la_L^{-1} \la_{M'}))^2=\xi'(\la_L^{-1}\la_M)^2\xi'(\la_L^{-1}\la_{M'})^2
\]
Plugging in another result from \S16, $\xi'(\la_L^{-1}\la_M)=\fc{\chi(M)^2}{\chi(L)^2}$. 

Let $K$ be closed under multiplication.  We find $K\ni \la_L^{-1}\la_{M''}=\la_L^{-1}\la_M \circ \la_L^{-1}\la_{M'}$. Using another fact from \S16, $\chi(L)^2=\deg \la_L$, unraveling, we get the RHS is 
\[
\fc{\deg \la_M}{\deg \la_L}\cdot \fc{\deg \la_{M'}}{\deg \la_L}.
\] 
Using the fact that the degree is multiplicative (two isogenies, think of the extension degree of function field, or generic number of fibers), even when extended to $\Hom^0$, we get the LHS is
\[
\fc{\deg\la_{M''}}{\deg\la_L}=\fc{\deg (\la_M\circ \la_L^{-1}\circ \la_{M'})}{\deg\la_L}=\fc{\deg \la_M(\deg \la_L)^{-1} \deg\la_{M'}}{\deg \la_L}
\]
which is the RHS. \\

\step 4 By Step 3, we conclude that 
\[
\xi'|_{\Q}=(N_{K/\Q}^{\min})^r,\qquad r\in \Z.
\]
%cf. deg of endo alg
%cf. endo alg divides 2g
The LHS is a norm of degree $g$ and $\N_{K/\Q}^{\min}$ has degree $[K:\Q]$. Thus $[K:\Q]\mid g$.
\end{proof}
Let's summarize what we have found out about the endomorphism algebra. 
\subsection{Summary on $\End^0A$}
Let $A\in \abk$, and let the center be $F=Z(\End^0A)$; this is
finite dimensional semisimple over $\Q$.

Then $\End^0 A$ is a finite-dimensional semisimple $\Q$-algebra. Suppose $A$ is simple and $F$ is a field. Then 
\begin{itemize}
\item
$\End^0(A)$ is a finite division algebra over $\Q$.
\item
$[\End^0:F]^{\rc 2}[F:\Q]=2g$. Using complex theory, we actually have $[\End^0 A:\Q]\mid 2g$ is $\chr k=0$.
%if ell curve char 0, then deg divides 2, so cannot be quaterion algebra.
\item
$\End^0(A)$ has a positive involution. %$\ri_L$.
\item
$F^{\ri_L=1}$ is a totally real subfield, of index 1 or 2 in $F$. $F$ is a ``CM field" or a totally real field.
\item
If $K\subeq (\End^0A)^{\ri_L}$ then $[K:\Q]\mid g$.
\end{itemize}•
Now we can go to the purely algebra side and try classify all division algebra with positive involution with all these properties, and  make a list of candidates,~\cite[\S21]{Mu70}.

%\R\times \cdots \times \C\times 
%trivial or switch R's, conj, or switch C's. But positive so preserve components. $F\ot_{\Q}\R$.
%fixed field has $\R$. Real embedding in that place. F itself can be CM. fixed part totally real. 

Converse statement: given algebra can we realize as endomorphism algebra of ab var?

finfield: 
Honda-Tate theory

Char 0: complex theory.

Complete info or not? %bjorn

next time complex abelian var.