\lecture{Tue. 10/2/12}

\subsection{Line bundles on abelian varieties}
Recall last time we used Corollary~\ref{cor:787-7-2} to prove Corollary~\ref{cor:787-7-3}, which tells us how the multiplication-by-$n$ map acts on line bundles in $\Pic(A)$.

\begin{df}
We say that $L$ is \textbf{symmetric} if $[-1]^*L\cong L$, and \textbf{anti-symmetric} if $[-1]^*L\cong L^{-1}$.
\end{df}
Note that Corollary~\ref{cor:787-7-3} says that
\[
[n]^*L=\begin{cases}
n^2 L, &L\text{ is symmetric}\\
n L, &L\text{ is anti-symmetric}\\
\end{cases}
\]
If $L$ is any line bundle, then we can easily produce a symmetric and antisymmetric line bundle:
\bal
L\ot [-1]^*L&\text{ is symmetric}\\
L\ot [-1]^*L^{-1}&\text{ is anti-symmetric}
\end{align*}
\begin{cor}\llabel{cor:[n]=isogeny}
Let $A\in (\text{Ab}/k)$ satisfy the following hypothesis.
\begin{itemize}
\item
$A$ has a ample line bundle.\footnote{Equivalently, $A$ is projective 
over $k$, rather than just proper.}
\end{itemize}
Then $[n]:A\to A$ is an isogeny.
\end{cor}
Later, we will show that the hypothesis is always satisfied. Here, projectivity helps us because it allows us to use facts about ample line bundles.
\begin{proof}
%qf surj
Since $[n]$ is a map between abelian schemes of the same dimension, it suffices to show either $[n]$ is quasi-finite or $[n]$ is surjective (Lemma~\ref{lem:787-7-1}). We will show that $\ker[n]:=A[n]$ is quasi-finite over $k$.

We make the following observations.
\begin{enumerate}
\item
$[n]^*$ is ample: Because $L$ is ample, $[-1]^*L$ is ample (automorphisms preserve line bundles). Because the tensor product of ample line bundles is ample, using Corollary~\ref{cor:787-7-3} we get $[n]^*L$ is ample.
\item
$[n]^*L|_{A[n]}$ is trivial: We have that following commutes
\[
\xymatrix{
A[n]\ha{r} \ar[rd] & A \ar[r]^{[n]} & A\\
& \Spec k\ar[ru]_{e_n}&
}
\]
So pulling back by $[n]$ is the same as pulling back to $\Spec k$ and then to $A[n]$. $\Spec k$ has trivial Picard group, so the pullback of $L$ to $\Spec k$ is trivial.
\item
$[n]^*L|_{A[n]}$ is ample: The pullback of an ample line bundle via an immersion is ample. %REF
%We use the following fact.
%ample=lots of sections
\item
We use the following fact.
\begin{fct}[Line bundle criterion for finiteness]\llabel{fct:787-8-1}
Let $X$ be a proper variety over $k$. Let $L$ be an ample and trivial line bundle on $X$. Then $\dim (X)=0$.
\end{fct}
\begin{proof}
If the dimension is positive, can choose line bundle with trivial sections. Since $L$ is ample, tensoring a high power of it should give a lot of sections. But becase $L$ is trivial, doesn't do anything. Can't produce more sections, contradiction.
\end{proof}
Now apply this to $[n]^*L|_{A[n]}$.
\end{enumerate}
\end{proof}
\begin{cor}[Theorem of the Square]\llabel{cor:square}%theorem of square?
Let $A\in (\text{Ab}/k)$ and let $L$ be a line bundle over $A$. Let $T_x:A\to A$ be given by $y\mapsto x+y$. Then the following hold.
%T-valued points, say s.t. similar
\begin{enumerate}
\item Let $x,y\in A(k)$. Then
\[
T_{x+y}^* L\ot L\cong T_x^*L\ot T_y^* L.
\]
\item Suppose $x,y\in A(T)$, and $T\in \schk$. Then
\[
T_{x+y}^* L\ot L\cong T_x^* L\ot T_y^* L\ot p_2^*(\text{stuff}).
\]
where $T_x$ is given by
\begin{align*}
A\times_k T&\xra{T_x} A\times_k T\\
(a,t)&\mapsto (a+x(t),t)
\end{align*}
and $p_2$ is the projection $A\times_k T\to T$.
\end{enumerate}
\end{cor}
Note (1) is just a special case of (2).
\begin{proof}
Apply Corollary~\ref{cor:787-7-2} to the maps
\[
\xymatrix@R-24pt{
A \ar[r] & A\\
a \ar@{|->}[r]^{f_{1}} 
\ar@{|->}[rd]^{f_2}
\ar@{|->}[rdd]_{f_3}  
& x\\
& y\\
& a.
}
\]
where $f_{123}=T_{x+y}$, $f_{13}=T_x$, $f_{23}=T_y$, and $f_3=\id_A$, i.e., $f_1$ is the map such that 
\[\ctri{A}{A}{\Spec k}{f_1}{}{}{str}{}{x}\]
commutes, and similarly for $f_2$. The corollary gives us that
\[
f_{123}^*L\ot f_3^* L\cong f_{13}^* L\ot f_{23}^* L\ot (\cdots)
\]
In the case of (1), the last term is trivial. (2) is left as an exercise. (In this case the last term is not necessarily trivial.)
\end{proof}

As a consequence of Corollary~\ref{cor:square}, 
%L^{-2}$. 
the map $\phi_L:A(k)\to \Pic(A)$ given by $x\mapsto T_xL\ot L^{-1}$ is a group homomorphism. It is easy to check $\phi_{L_1\ot L_2}=\phi_{L_1}+\phi_{L_2}$ where the addition on the RHS is in $\Pic A$.

Taking $y\in A(k)$ we have $\phi_{T_y^*L}=\phi_L$ by Corollary~\ref{cor:square}. This is because 
\begin{align*}
\phi_{T_y^*L}(x)&=T_x^*(T_y^* L)\ot (T_y^*L)^{-1}\\
&=(\underbrace{T_y\circ T_x}_{T_{x+y}})^*L\ot (T_y^*L)^{-1}\\
&=(T_x^*L \ot T_y^*L \ot L^{-1})\ot (T_y^*L)^{-1}
&\text{by Theorem of the Square~\ref{cor:square}}\\
&=T_x^*L \ot L^{-1} =\phi_L(x).
\end{align*}
Next time we'll upgrade $\phi_L$ from a group homomorphism to a morphism of group schemes (on the left-hand side we would consider $A$ as a scheme, and on the right-hand side we can give $\Pic(A)$ the structure of a scheme, called the Picard scheme of $A$). We will make sense of the kernel $K(L)$ as a closed subgroup scheme of $A$.

\subsection{Seesaw Theorem and $K(L)$}
%\fixme{This section has not been edited.}
We will need the Seesaw Theorem  to make sense of $K(L)$ as a subgroup scheme.

In Mumford, the Seesaw Theorem is used to prove the Theorem of the Cube. It is easier to understand the Seesaw Theorem using the theory of Picard schemes (which we'll take as a black box for now).

\begin{thm}[Seesaw Theorem]\llabel{thm:seesaw}
Let $X$ be a proper variety over $k$, $Y$ a scheme over $k$, and $\cL$ a line bundle over $X\times_k Y$. Then there exists a unique closed subscheme $Y_0\subeq Y$ such that 
\begin{enumerate}
\item
$\cL|_{X\times Y_0}\cong p_2^*M$ for some $M\in \Pic(Y_0)$. 
(I.e., the restriction is trivial on $X$.)
%moduli-theoretic df of picard schemes
\item
For all $f:T\to Y$ such that $(1_X\times f)^*\cL\cong p_2^* K$ for some $K\in \Pic(T)$, 
\[
\xymatrix{
X\times_k T\ar[r]^{1_X\times f} \ar[rd]_{p_2} & X\times_k Y\\
& T,
}
\]
there exists a unique map $T\to Y_0$ making the following commute
\[
\xymatrix{T\ar[r]\ard{rd}_{\exists!} & Y\\ & Y_0\ha{u}}
\]
%restriction nothing, trivial in (i) for S=Speck k
%Here we have
\end{enumerate}
Moreover, for any scheme $T$ over $k$,
\beq{eq:787-8-1}
Y_0(T)=\set{f\in Y(T)}{(1_X\times f)^* \cL\cong p_2^*M\text{ for some }M\text{ over }T}.
\eeq
\end{thm}
\fixme{Why is this called Seesaw?}

%This is called the Seesaw Theorem. %Restriction of 
%Collection $Y$ such that restriction to $X\times_k \{y\}$ is closed.
%In proof, go between points...?
Think of item 2 as saying that $Y_0$ is the largest closed subscheme of $Y$ such that $\cL|_{X\times Y_0}$ is $p^*_2(M)$ for some $M\in \Pic(Y_0)$.

There are two interesting applications of the Seesaw Theorem.
\begin{enumerate}
\item
Proof of Theorem of the Cube. (See~\cite[\S 6]{Mu70}.)
\item
Definition of $K(L)\subeq A$ as a closed subgroup scheme, such that
\[
K(L)(k)=\ker\phi_L.
\]
\end{enumerate}
%One is to the proof of Theorem of the Cube; the other is the definition of $K(L)$ as a group scheme.
%theorem of cube over general base scheme.
\begin{rem}
The Seesaw Theorem gives the Theorem of the Cube with its corollaries. The Theorem of the Cube and the Seesaw Theorem  give the definition $K(L)\subeq A$ as a closed subgroup scheme. The definition of $K(L)$ then means that for $L$ ample we can define the dual abelian scheme
\[
A^{\vee}:=A/K(L).
\]
In this way we can avoid the big theory of Picard schemes; we can show that with this definition $A^{\vee}$ satisfies properties that we expect a moduli space to satisfy. 

Then $\phi_L:A\to A^{\vee}$ is just the quotient map. We do have to check we get the same object for different choices of $L$.

Alternatively, if you accept the theory of Picard schemes as in~\cite{GGBM}, we get the Seesaw Theorem and the Theorem of the Cube as before. Now directly from the theory of Picard schemes, we get $A^{\vee}=\Pic^0(A)$, $\phi_L:A\to A^{\vee}$, and then we define $K(L):=\ker \phi_L$. 

We'll talk about both approaches.
\end{rem}
We can apply the Seesaw Theorem~\ref{thm:seesaw} to $X=Y=A\in \abk$ and the ``Mumford" line bundle $\cL=\mu^*L\ot p_1^*L^{-1}\ot p_2^* L^{-1}$, where $p_1,p_2,\mu:A\times A\to A$ are the projection and multiplication maps. We get a closed subscheme $K(L)\subeq A$ satisfying the property
\[
K(L)(T)=\set{x\in A(T)}{(1_A\times x)^* \cL}=p_2(\cL).
\]
What's not clear yet is that this is a sub{\it group} scheme. We will use the Theorem of the Cube to show this.
\begin{lem}
$K(L)\subeq A$ is a closed subgroup scheme.
\end{lem}
\begin{proof}
There is another way to describle the $T$-points of $K(L)$. 
\begin{clm}\llabel{clm:787-8-1}
\[
K(L)(T)=\set{x\in A(T)}{T_x^*L_T\ot L_T^{-1}\cong p_2^*(M)\text{ for some line bundle $M$ over $T$}}.
\]
%where $\cdots $ is a line bundle over $T$.
\end{clm}
Here, $L_T:=p_1^*L$ where $p_1:A\times T\to A$ is the projection.

%This condition is equivalent to the previous condition. 

We now show that if the claim is true then $K(L)(T)$ is a {\it group} scheme by the Theorem of the Square, so $K(L)(T)$ is a sub{\it group} scheme.

For $x,y\in K(L)(T)$, we have by the Theorem of the Square~\ref{cor:square} that
\begin{align*}
T_{x+y}^*L_T\ot L_T^{-1} &\cong (T_x^*L_T\ot L_T^{-1})
\ot 
(T_y^* L_T\ot L_T^{-1})\\
&=p_2^*(M_1)\ot p_2^*(M_2)\cong p_2^*(M_1\ot M_2)
\end{align*}
for some $M_1,M_2$.
Thus $x+y\in K(L)(T)$. That $x\in K(L)(T)$ implies $-x\in K(L)(T)$ is left as an exercise.

It remains to show that the condition on $\cL=\mu^*L\ot p_1^*L^{-1}\ot p_2^* L^{-1}$ in~\ref{eq:787-8-1} gives the condition on $L$ in Claim~\ref{clm:787-8-1}.
\begin{proof}[Proof of Claim~\ref{clm:787-8-1}]
It suffices to prove that $(1_A\times x)^*\cL$ and $T_x^* L_T\ot L_T^{-1}$ differ by $p_2^*(M)$ for some $M$. 

We first verify that $(1_A\times x)^*\mu^*\cL\cong T_x^*L_T$. We have
\[
A\times T\xra{1_A\times x}A\times A\xra{\mu}A
\]
\begin{align*}
(1_A\times x)^* \mu^*L &\cong (\mu(1_A\times x))^* L\\
&\cong (p_1\circ T_x)^*L&\mu\circ (1_A\times x)=p_1\circ T_x\\
&\cong T_x^*p_1^* L\\
&\cong T_x^* L_T.
\end{align*}
Then
\begin{align*}
(1_A\times x)^* \cL&=(1_A\times X)^* (\mu^*L\ot p_1^*L^{-1}\ot p_2^* L^{-1})\\
&=(T_x^*L_T)\ot \underbrace{(1_A\times x)^*p_1^*L^{-1}}_{L_T^{-1}} \ot \underbrace{(1_A\times x)^*p_2^*L^{-1}}_{p_2^*(x^*L)^{-1}}.
\end{align*}
\end{proof}
\end{proof}
%given functorial desc, show subgroup scheme. Use crutch to admit ample line bundle, hence projective. K(L) finite if and only if L ample. Find L such that K(L) finite. Equivalent condition, check that. Next time.
