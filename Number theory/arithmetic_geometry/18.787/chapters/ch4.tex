\lecture{Tue. 9/18/12}

Suppose we have a strictly free group action $G\cir X$, i.e. a group action such that the following map is closed:
\[
\xymatrix@R-24pt{
G\times_S X\ha{r}^{\text{closed}} & X\times_S X\\
(g,x)\ar@{|->}[r] & gx .
}
\]
Let the quotient map be $X\xra{\pi} Y$. Our goal today is to relate quasi-coherent sheaves on $X$ and on $Y$:
\[\xymatrix{\QCoh(X)\ar@/^/[r]^{\pi_*}& \QCoh(Y)\ar@/^/[l]^{\pi^*}}.\]
In actuality rather than have $\QCoh(X)$ on the left-hand side, we have $\QCoh^G(X)$, the category of {\it $G$-equivariant sheaves} on $X$. We'll define what this means, and then show how $G$-invariant quasi-coherent sheaves\footnote{Recall that a sheaf of $\sO_X$-modules is \textbf{quasi-coherent} if $X$ can be covered by open affine subsets $U_i=\Spec A_i$ such that for each $i$ there is an $A_i$-module $M_i$ with $\cF|_{U_i}=\wt{M_i}$. See Hartshorne~\cite[\S II.5]{Ha77}} on $X$ correspond to quasi-coherent sheaves on $Y$.

%Construct a quotient sheaf, descend sheaf to downstairs. 
How will this be useful? Later on, when we construct dual abelian varieties, we will have to mod out by a group scheme, and we'll want to consider line bundles over the quotient scheme.

\subsection{$G$-equivariance for vector bundles}
\llabel{sec:g-eq-vb}

%large space upstairs, large space. 
%
%Mod out by group. 
%
%Remember certain line bundles
%
%Want projectivity of moduli space.
%
%Application: construct dual abelian varieties, moduli spaces of AV. (GIT) 
To motivate the definition of $G$-invariant sheaves, we first give several equivalent definitions of $G$-invariant vector bundles.  This will be a kind of ``toy model" for use to play with.
%This will be a ``toy model"
\begin{df}
Let $G\cir X$ be a group acting on discrete $G$-set. A \textbf{vector bundle} on $X$ is a vector space $V_x$ for each $x\in X$. Let $V=\bigcup_{x\in X} V_x$. A \textbf{$G$-equivariant} vector bundle is $V$ along with any one of the following equivalent structures, called $G$-invariant structures.
% a map $\phi:V\to X$ sending $V_x$ to $x$, and satisfying %and $G$-equivariant str on $v$ (?)
%%$G  $\phi:V=(v_x)_{x\in X}\to X$, a 
\begin{enumerate}
\item A group action $G\cir V$, and
a map $\phi:V\to X$ sending $V_x$ to $x$ such that $\phi$ is $G$-equivariant:
\[\phi(gv)=g\phi(v).\]
\item
A collection of ``multiplication by $g$ maps" $\set{\la_x(g)}{V_x\xra{\sim} V_{gx}}_{g\in G,x\in X}$ such that 
\begin{align*}
\la_x(1)&=\id\\
\la_x(g_1g_2)&=\la_{g_2x}(g_1)\circ \la_x(g_2) %&\text{(group action)}
\end{align*}
where the second condition is says that $\la_x(g)$ is a group action, and comes from the following composition:
\begin{equation}\llabel{eq:787-4-cocycle}
\xymatrix{
V_x\ar[r]^{\la_{x}(g_2)}\ar@/_2pc/[rr]^{\sim}_{\la_x(g_1g_2)} & V_{g_2x} \ar[r]^-{\la_{g_2x}(g_1)} & V_{g_1g_2x}.
}
\end{equation}
%cocyc cond
\end{enumerate}
%move around fibers in general
We want to couch this in the language of the whole space, not individual fibers. To do this, we need to collect all the maps $\la_x(g)$ into a single map. This motivates our next equivalent definition.

%Equivalently, consider 
Consider the two maps
\[\xymatrix@R-24pt{G\times X%\ar@/^/[r]^{p_2}\ar@/_/[r]_{\mu}
\ar[r]& X\\ (g,x)\ar@{|->}[r]^{p_2} \ar@{|->}[rd]_{\mu}  &x\\&gx.}\]
%fiber at gx is vx.

Using these two maps, we construct two vector bundles over $G\times X$, $p_2^*V$ and $\mu^*V$. The fiber over $(g,x)$ in $p_2^*V$ is $V_x$, and the fiber over $(g,x)$ in $\mu^*V$ is $V_{gx}$.

We now ``assemble" the map $\la_x(g)$ into a single map $\la$ on vector bundles over $G\times X$:
\[\xymatrix{p_2^* V\ar[r]^{\la}_{\cong}& \mu^* V}.\]


To express the cocycle condition~\eqref{eq:787-4-cocycle}, we consider 3 maps
\[\xymatrix@R-24pt{G\times G\times  X\ar[r]& X\\ (g_1,g_2,x)\ar@{|->}[r]^{p_3}
\ar@{|->}[rd]^-{\xi}
\ar@{|->}[rdd]_-{\eta}& x\\
&g_2x\\
&g_1g_2x.}\]
%We want to state compatible diagram
\begin{enumerate}[resume]
\item
An $G$-equivariant structure is a map $\la$ on vector bundle over $G\times X$,
\[\xymatrix{p_2^* V\ar[r]^{\la}_{\cong}& \mu^* V}.\]
such that the following commutes (the cocycle condition)
\[
\xymatrix{
p_3^*V\ar[r]^{p_{23}^*\la}\ar@/_2pc/[rr]^{\cong}_{(\mu\times \id_X)^*\la} & \xi^*V \ar[r]^{(\id_G\times \mu)^*\la} & \eta^*V.
}
\]
where $p_{23}$ is passing to the second and third components.
\end{enumerate}
Taking the fiber at $(g_1,g_2,x)$, we recover~\eqref{eq:787-4-cocycle}.

Now we pull back the definition above to $\{g\}\times X\subeq G\times X$.
%Definition in scheme-theoretic case
\begin{enumerate}[resume]
\item %Pull back 1'' to $\{g\}\times X\subeq G\times X$. $\set{\la_g}{V\xra{\cong} T_g^* V}_{g\in G}$, $T_g:X\xra{\cong} X$, $x\mapsto gx$. + cocycle condition,
%A $G$-equivariant structure is given as follows. 
Define $T_g:X\xra{\cong} X$ by $T_g(x)=gx$. A $G$-invariant structure on $G$ is a collection of maps $\set{\la_g}{V\xra{\cong} T_g^* V}_{g\in G}$ satisfying the following cocycle condition
\beq{eq:df4-cocyle}
\xymatrix{
V\ar[r]^{\la_{g_2}}\ar[rrd]_{\la_{g_1g_2}} & T_{g_2}^* V\ar[r]^{T_{g_2}^*\la_{g_1}}& T_{g_2}^*T_{g_1}^* V\ar@{=}[d]\\
&& T_{g_1g_2}^*V.
}
\eeq
%where rhs is canonical.
\end{enumerate}
\end{df}
Why do we need $G$ to be free?

Suppose $G\cir X$ is a free action of a group on a set, i.e. $\Stab_G(x)=\{1\}$. (We also say that $X$ is a principal homogeneous space under $G$.) Let $\pi:X\to X/G$ be the quotient map. Then we have an equivalence
\begin{align}
\llabel{eq:787-4-vb}
\bt{vector bundles over}{$X/G$}&\xra{\cong} \bt{$G$-equivariant vector bundle}{over $X$}.\\
\nonumber
V&\mapsto \pi^*V
\end{align}
This is an exercise. %you can reduce to just looking at the orbit of one point.
%$\pi:X\to X/G$. Exercise!
%Principal homogeneous space over $G$. Come down to one point.
The idea is that by knowing the fiber over point, we can use the action of $G$ on $X$ to copy that fiber on the whole orbit of $x$. 
\begin{rem}
If $G$ is not free and $|G|>1$, then we wouldn't expect such an equivalence. For instance, take $X=\{\cdot \}$ and let $G$ act on $X$ trivially. We still have a map as in~\eqref{eq:787-4-vb}, but it is no longer surjective.

On the left-hand side, a vector bundle over $X/G=\{\cdot \}$ is just a vector space. On the right-hand side, a $G$-equivariant vector bundle over $\{\cdot\}$ is a vector space with $G$-action. %If you pull back you get trivial action, but there are non-trivial $G$-actions on vector spaces. Not equivalent.
The map is no longer surjective, because $V$ and hence $\pi^*V$ can only have the trivial action, but there are non-trivial $G$-actions on vector spaces.
\end{rem}

In general, to get around the ``strictly free" condition, we use stacks instead of schemes. However we will always restrict to free case, so that we can stay within scheme theory.

Our main examples of interest are closed group subschemes acting on schemes. The action is always strictly free in this case.

We've made a toy model; now let's go back to reality.
\subsection{$G$-equivariance for sheaves}

\begin{df}
Let $G\cir X$ be a group scheme over $S$ acting on a scheme $X$ over $S$, and $\cF\in \QCoh(X)$.\footnote{We could work with other sheaves too; however, our proofs will only work for quasicoherent sheaves.}
A \textbf{$G$-equivariant sheaf} is a sheaf $\cF$ with either of the following equivalent structures (called $G$-equivariant structures):
\begin{enumerate}
\item
a map $p_2\cF\xra{\la,\cong} \mu^*\cF$ such that the following commutes
\[\xymatrix{
p_3^*\cF\ar[r]^{p_{23^*}}_{\cong}\ar@/_2pc/[rr]^{\cong}_{ (\mu\times\id_G)^*\la} & \xi^*\cF \ar[r]^{(\id_G\times\mu)^*\la} & \eta^*\cF.
}\]
in $\QCoh(G\times_SG\times_SX)$, where the maps are the analogues of the maps in Section~\ref{sec:g-eq-vb}, for schemes instead of vector bundles.
\item a collection 
\[
\set{\cF_T}{\cF_T\xra{\cong} T_g^*\cF_T}_{T\in \schs,\,g\in G(T)}
\]
in $\QCoh(X_T)$ where $X_T$ is the basechange $X\times_ST$, $T_g:X_T\xra{\cong}X_T$ is the translation map $x\mapsto gx$, and the maps are functorial in $T$. Moreover, this collection satisfies the cocycle condition (we require the same kind of diagram as in~\ref{eq:df4-cocyle}).
\end{enumerate}
\end{df}
A structure in item 1 gives a structure in item 2 just by pulling back. To see a structure in item 2 gives a structure in item 1, we have to use descent theory (omitted).

Compare these two definitions with definitions 3 and 4 in Section~\ref{sec:g-eq-vb}.

Now that we've made the basic definitions, we'd like to see how it applies to quotient schemes. Actually, we'll look at a more general case.
\subsubsection{Construction of $G$-equivariant structure}

Let $\pi:G\cir X\to Y$ be a $G$-invariant map, i.e. a map such that if $p_2,\mu:G\times_SX\to X$ are the projection and multiplication maps, the $G$-equivariance condition 
\[\pi\circ p_2=\pi\circ \mu\]
holds. (For instance, $\pi$ could be the quotient map.) We know that there is a pullback map from quasicoherent sheaves on $Y$ to quasicoherent sheaves on $X$, 
\begin{align*}
\pi^*:\QCoh(Y)&\to \QCoh(X)\\
g&\mapsto \pi^*g.
\end{align*}
However, we also want to equip $\pi^*g$ with a natural $G$-equivariant structure, %Is there a natrual $G$-equivariant structure that we can equip $\pi^*g$ with? Yes. 
so that $\pi^*g$ is in the category of $G$-equivariant sheaves
\[
\QCoh^G(X)=\bt{$G$-equivariant sheaves}{on $X$}.
\]
Is there a natrual $G$-equivariant structure that we can equip $\pi^*g$ with? Yes.


Note that if $X\xra{f}Y\xra{g}{Z}$ are scheme morphisms, $\cF\in \QCoh(X)$ and $\cH\in \QCoh(Z)$, then
\[
(g\circ f)^* \cH\cong f^*(g^*\cH)
\]
and
\[
(g\circ f)_*\cF=g_*(f_*\cF).
\]
For $W\subeq Z$, the sections on $(g\circ f)_*\cF$ are  $\cF(f^{-1}(g^{-1}(W)))$. We will define the $G$-equivariant structure by examining the sections of $(g\circ f)_*\cF$.

By assumption on $\pi$, we have $\pi\circ p_2=\pi\circ \mu$. Thus we can $\la$ as the map making the following diagram commute:
\[
\xymatrix{
p_2^*(\pi^*\cG)\aq{d}\ar[r]^{\la} &\mu^*(\pi^*\cG)\aq{d}\\
\cong (\pi\circ p_2)^* \cG \aq{r} & (\pi\circ \mu)^*\cG.
}
\]
A tedious check shows that $\la:p_2^*(\pi^*\cG)\to \mu^*(\pi^*\cG)$ satisfies the cocycle condition and is functorial in $\cG$, so we do have a $G$-invariant structure on $\pi^*\cG$.
%p^* not jst functor but
%can show functorial in $$

Going in the opposite direction, suppose we have a $G$-equivariant sheaf on $X$, $\cF\in \QCoh^G(X)$. What happens to the $G$-equivariant structure when we push forwards, using $\pi_*$?

In general, $\pi^*\pi_*$ is not the identity; the rank may increase when we pull back and push forward. We'd like to modify $\pi_*$ so that we {\it do} get the identity.

Thus, we don't just push forwards: we push forward {\it and} take the {\it $G$-invariant sections}. We will define the map
\begin{align*}
\pi_*(\cdot)^G:\QCoh^G(X)&\to \QCoh(Y)\\
(\cF, \la)&\mapsto (\pi_*\cF)^G.
\end{align*}
This takes the sheaf $\cF$ on $X$ with $G$-invariant structure to a $\cO_Y$-submodule of $\pi_*\cF$ consisting of $G$-invariant sections, defined formally below.

\begin{df}
Let $\pi:G\cir X\to Y$ be a \textbf{$G$-invariant map}. Let
$V\subeq Y$ be open. We say that a section $s\in \pi_*\cF(V)=\cF(\pi^{-1}(V))$ is \textbf{$G$-invariant} if
\[
\la(p_2^*(s))=\mu^*(s),
\]
i.e. $s$ gets mapped to the same element when we follow the two maps below:
\beq{eq:787-4-4}
\xymatrix{
& (p_2^*\cF)({\color{blue}p_2^{-1}(\pi^{-1}(V))})\ar@{=}[dd]^{\la}_{\cong}\\%convince self!
\cF(\pi^{-1}(V))\ar[ru]^{p_2^*}\ar[rd]_{\mu^*} & \\
& (\mu^*\cF) ({\color{blue}\mu^{-1}(\pi^{-1}(V))}).
}%show on level of rings and modules
\eeq
Define $(\pi_*\cF)^G$ by having $(\pi_*\cF)^G(V)$ consist of the $G$-invariant sections on $V$.
\end{df}
Note that  the blue sets above are the same because of $G$-invariance. Moreover, $(\pi_*\cF)^G$ is actually a $\cO_Y$-submodule because the condition  $\la(p_2^*(s))=\mu^*(s)$ is stable under $\cO_Y$ action.
\begin{ex}
The sheaves $\cO_G,\Om_{G/S}^1$ carry natural $G$-invariant structures. This is easiest to see on the level of $T$-points. For all $g\in G(T)$, are there maps $\cO_G\xra{\cong} T_g^*\cO_G$ and $\Om_{G/S}^1\cong T_g^*\Om_{G/S}^1$ that satisfy the cocycle condition? Yes, because we know the translation $T_g$ induces an isomorphism on these spaces. 
%$G$-invariant functions. Formalism to this guy, $G$-invariant functions. Differentials invariant under $G$-action.

\fixme{For instance, (?)}
\[
\{\text{invariant global differentials on $G$}\}\cong \{
\text{$G$-invariant section of $\Om_{G/S}^1$ over $G$}
\}
\]

In the case of elliptic curves, we get the invariant differential on elliptic curves.
\end{ex}

\begin{rem}
Given any vector bundle, in general there is not necessarily a $G$-invariant structure.
%g-equivariance is add'l structure rather than a condition to satisfy. 
The same is true of sheaves: given a quasi-coherent sheaf, there may not be $G$-equivariant structure, or there may be more than 1.
\end{rem}

\begin{thm}
Let $X$ be a scheme locally of finite type over $S$. Let $G$ be a locally free group scheme over $S$, and suppose we have a strictly free action $G\cir X$.
Then there are 2 maps $\pi^*$ and $\pi_{*}(\cdot )^G$ which induce categorical equivalences (top of diagram below)
%nat transf
\[
\xymatrix{
\QCoh(Y)\ar@/^/[r]^{\pi^*} & \QCoh(X)^G\ar[l]^{\pi_*(\cdot)^G}\\
\{\text{locally free of rank }a\} \ha{u} \ar@{<->}[r] & \{\text{locally free of rank }a\} \ha{u}.}
\]
Moreover, they also induce a bijection on the subcategories of locally free sheaves of a given rank $a$.
%cf. for finintely presented sheaves (flat iff locally free iff projective) finite generators and relations
\end{thm}

\begin{rem}
The $a=1$ case is already important; they give line bundles.
\end{rem}

We will be specially interested in the case where $Y$ is a quotient scheme, $Y=X/G$.

\begin{proof}[Proof sketch]
For the complete proof, see~\cite[p. 70 and p. 114-118]{Mu70}.

Reduce to affine case $G=\Spec(R)$. Because $G$ is locally free, we can assume by localizing further that $R$ is free of rank $r$ over $Q$.
\begin{align*}
G&=\Spec(R)& X&=\Spec(A)\\
S&=\Spec(Q)& Y&=\Spec(B).
\end{align*}
Then $\cF$ corresponds to an $A$-module $M$, i.e. $\cF=\wt{M}$.

We can rewrite~\eqref{eq:787-4-4} in terms of modules (recall that defining pullback involves tensor product), to get the triangle on the LHS below:
%We'll write down something simiar to the triangle diagram (4.2). 
\beq{eq:787-4-5}
\xymatrix{
& M\ot_A(R\ot_Q A)\ar@{=}[dd]^{\la}_{\cong}& M\ot_A(A\ot_BA) \ar[l]_{\cong}^{\psi}\ar@{=}[rr]^-{m\ot a\ot 1\mapsfrom a\ot m}\ar@{=}[dd]_{\cong} &&A\ot_B M\ar@{=}[dd]^{\la'}_{\cong}
\\%convince self!%cancel out a w/ first component, cancel w/ second component. M in the position of A.
M\ar[rd]%_{\mu^*}
\ar[ru] & &&\\
& M\ot_A(R\ot_Q A)& M\ot_A(A\ot_BA) \ar[l]_{\cong}^{\psi} \ar@{=}[rr]^-{m\ot 1\ot a\mapsfrom m\ot a} &&
M\ot_B A.
}%show on level of rings and modules
\eeq
%(1,3): a\mapsto 1\ot a$, (3,3) a\mapsto a\ot 1
The maps in the left triangle carry $m\mapsto m\ot 1$. Note that the two tensor products $M\ot_A(R\ot_Q A)$ are formed differently, though: the top one is formed via the map $p^*:A\to R\ot_Q A$ sending $a\mapsto 1\ot a$ and bottom one is formed via the map $\mu^* :A\to R\ot_Q A$. The sheaf $(\pi_*\cF)^G$ corresponds to the module
\[
M^G=\set{m\in M}{\la(m\ot 1)=m\ot 1}.
\]
We will use two facts from last time.
\begin{enumerate}
\item By Theorem~\ref{thm:gsx=xyx}, we have an isomorphism
$G\times_SX\xra{\cong}X\times_YX$ sending $(g,x)\mapsto (gx,x)$. This gives a ring isomorphism
\bal
A\ot_B A&\xra{\psi, \cong} R\ot_Q A\\
a_1\ot a_2&\mapsto \mu^*(a_1)p_2^*(a_2)=\mu^*(a_1)(1\ot a_2).
\end{align*}
This gives us the middle square in~\eqref{eq:787-4-5}. Here, the tensor product in the upper $M\ot_A (A\ot_B A)$ is formed by the map $A\to A\ot_B A$ given by $a\mapsto 1\ot a$, and the tensor product in the lower $M\ot_A(A\ot_B A)$ is given by $a\mapsto a\ot 1$. This is because $1\ot a$ is sent to $p_2^*(a)$ and $a\ot 1$ is sent to $\mu^*(a)$.

%NOW ADD IN RIGHT SIDE OF COMPLICATED DIAGRM (1,4) to (1,3): 
%geo quotient and 
%So $B$-module
%\[
%M^G=\set{m\in M}{\la(1\ot m)=m\ot 1}\mapsfrom (\pi_*\cF)^G
%\]
%for outer diagram.
\item
If $A$ is projective of rank $r$ over $B$ and integral over $B$, then $A$ is faithfully flat over $B$. %(Flat and map on Spec is surjective.)
\fixme{reference}
\end{enumerate}
We need to show two things.
\begin{enumerate}
\item
%We are more or less ready to start the proof.
We have an isomorphism
%By adjunction, move $\pi_*$ to $\pi^*$, 
\[
\pi^*((\pi_*\cF)^G)\xra{\cong} \cF.
\]
%come from adjunction from $(\pi_*\cF)^G\subeq \pi_*\cF$. i.e.
This is equivalent to having an isomorphism
\begin{align*}
M^G\ot_BA&\xrc M\\
m\ot a&\mapsto am
\end{align*}
\item 
For $\cG$ on $Y$, we have an isomorphism
\[
\cG\xrc (\pi_*(\pi^*\cG))^G.
\]
If $\cG=\wt{N}$, this is equivalent to having an isomorphism
\[
N\xrc (N\ot_B A)^G.
\]
\end{enumerate}

\fixme{THIS IS UNEDITED. Fix after I understand the proof in Mumford. 
$M$ projective over $A$ iff $M^G$ is projective over $B$
If we show these are true we will be done very soon.

How to deduce theorem from 3 and 4.

(i) $\pi_*(\pi^*\cG)^G\cong \cG$ canonically using adjunction. On modules, $(N\ot_B A)^G\cong N$ for all $B$-modules $N$.
%qc sheadf on spec corresp $B$-mod $N$.

$N\ot_BA\to (N\ot_B A)^G\ot_Q A\xrc N\ot_B A$. isomorphism by 3. composition is identity [XYMATRIX] Thus LHS canonical map is isomorphism.
\[
N\ot_BA\xrc (N\ot_B A)^G\ot_BA.
\]
implies $A$ faithfully flat over $B$. It must be an isomorphism to begin with, 
\[
N\xrc (N\ot_B A)^G.
\]

(ii) Use part 4+
locally free=projective. Projective corresponds to projective. + check equality of rank, easy given part 3.

Refer to Mumford section 12. He works over algebraically closed field. $S=\Spec(k)$, $k$ alg closed, but works in generality. 
}

\end{proof}
Next time we can will finally talk about abelian varieties and schemes.